% changelog: "0.0.1, 2024-08-05, Linda Barbiero, Stesura verbale. Verificato da Pietro Busato e Veronica Tecchiati"

\documentclass[8pt]{article}
\usepackage[italian]{babel}
\usepackage[utf8]{inputenc}
\usepackage[letterpaper, left=1in, right=1in, bottom=0.75in, top=0.75in]{geometry}
\usepackage{amsmath}
\usepackage{subfiles}
\usepackage{lipsum}
\usepackage{csquotes}
\usepackage{amsfonts}
\usepackage[sfdefault]{plex-sans}
\usepackage{float}
\usepackage{pifont}
\usepackage{mathabx}
\usepackage[euler]{textgreek}
\usepackage{makecell}
\usepackage{tikz}
\usepackage{wrapfig}
\usepackage{siunitx}
\usepackage{amssymb} 
\usepackage{tabularx}
\usepackage{threeparttable}
\usepackage{adjustbox}
\usepackage[document]{ragged2e}
\usepackage{floatflt}
\usepackage[hidelinks]{hyperref}
\usepackage{graphicx}
\usepackage{hyperref}
\setcounter{tocdepth}{4}
\usepackage{caption}
\usepackage{multicol}
\usepackage{tikz}
\setlength\parindent{0pt}
\captionsetup{font=footnotesize}
\usepackage{fancyhdr} 
\usepackage{graphicx}
\usepackage{capt-of}% 
\usepackage{booktabs}
\usepackage{varwidth}

% -- IMPORTANTE -- %
% utilizzare `` " per le virgolette (``esempio") altrimenit non si chiudono
% non utilizzare \glossterm nei verbali

% -- TITOLO INTESTAZIONE -- %
\newcommand{\customtitle}{VERBALE INTERNO DEL 2024-07-16} % o ESTERNO

% -- STILE INTESTAZIONE -- %
\fancypagestyle{mystyle}{
	\fancyhf{} 
  \fancyhead[R]{\includegraphics[height=1cm]{../../../template/images/logos/NaN1fy_logo.png}} 
	\fancyhead[L]{\leftmark} 
	\renewcommand{\headrulewidth}{1pt} 
	\fancyhead[L]{\customtitle} 
	\renewcommand{\headsep}{1.3cm} 
	\fancyfoot[C]{\thepage} 
}

% -- PER LA FIRMA -- %
\newcommand{\signatureline}[1]{%
	 \par\vspace{0.5cm}
	\noindent\makebox[\linewidth][r]{\rule{0.2\textwidth}{0.5pt}\hspace{3cm}\makebox[0pt][r]{\vspace{3pt}\footnotesize #1}}%
}

% -- PER IL GLOSSARIO -- %
\newcommand{\glossterm}[1]{#1\textsuperscript{G}} % inserisci \glossterm{termine}

% -- per abilitare 4x sottosezioni es 2.1.1.1
% quando lo utilizi metti un // esterno al comando subsubsubsection{esempio} altrimenti è attacatto al titoletto, non modificare il comando (se metti //// nel comando è troppo spaziato quando ci sono table e figure
\setcounter{secnumdepth}{4}
\newcommand{\subsubsubsection}[1]{\paragraph{#1}\mbox{}\\}

\begin{document}
\definecolor{myblue}{RGB}{23,103,162}
\begin{titlepage}
	\begin{tikzpicture}[remember picture, overlay]
		\node[anchor=south east, opacity=0.2, yshift = -4cm, xshift= 2em] at (current page.south east)
      {\includegraphics[width=0.7\textwidth, trim=0cm 0cm 5cm 0cm, clip]{../../../template/images/logos/Universita_Padova_transparent.png}}; 
		\node[anchor=north west, opacity=1, yshift = 4.2cm, xshift= 1.4cm, scale=1.6] at (current
      page.south west) {\includegraphics[width=4cm]{../../../template/images/logos/NaN1fy_logo.png}};
	\end{tikzpicture}
	
	\begin{minipage}[t]{0.47\textwidth}
		{\large{\textsc{Destinatari}}
			\vspace{3mm}
			\\ \large{\textsc{Prof. Tullio Vardanega}}
			\\ \large{\textsc{Prof. Riccardo Cardin}}
		}
	\end{minipage}
	\hfill
	\begin{minipage}[t]{0.47\textwidth}\raggedleft
		{\large{\textsc{Redattori}}
			\vspace{3mm}
			{\\\large{\textsc{Linda Barbiero}}}
			
			
		}
		\vspace{8mm}
		
		{\large{\textsc{Verificatori}}
			\vspace{3mm}
			{\\\large{\textsc{Veronica Tecchiati}\\}} % massimo due 
			{\large{\textsc{Pietro Busato}}}
			
		}
		\vspace{4mm}\vspace{4mm}
	\end{minipage}
	\vspace{4cm}
	\begin{center}
		\begin{flushright}
			{\fontsize{30pt}{52pt}\selectfont \textbf{Verbale Interno Del\\2024-07-16\\}} % o ESTERNO
		\end{flushright}
		\vspace{3cm}
	\end{center}
	\vspace{8.5 cm}
	{\small \textsc{\href{mailto: nan1fyteam.unipd@gmail.com}{nan1fyteam.unipd@gmail.com}}}
\end{titlepage}
\pagestyle{mystyle}
\section*{Registro delle Modifiche}
\begin{table}[ht!]	
	\centering
	\begin{tabular}{p{1.2cm} p{2cm} p{5cm} p{3cm} p{3cm}}
		\toprule
		\textbf{Versione}& \textbf{Data} & \textbf{Descrizione} & \textbf{Redattore} & \textbf{Verificatore} \\
		\midrule
			0.0.1 & 2024-08-05 & Stesura verbale. & Linda Barbiero & Pietro Busato, \quad Veronica Tecchiati \\
 			\\ % spazio tra le righe

		\bottomrule
		% Ruolo Redattore o Verificatore
	\end{tabular}
	\caption{Registro delle modifiche.}
	\label{table:Registro delle modifiche}
\end{table}
\newpage
\tableofcontents
\clearpage
\newpage
\justifying
\section{Contenuti del Verbale}
\subsection{Informazioni sulla riunione}
\begin{itemize}
	\setlength\itemsep{0em}
	\item\textbf{Luogo:} Chiamata Discord;
	\item\textbf{Ora di inizio:} 15:30;
	\item\textbf{Ora di fine:}  16:30.
\end{itemize}
\begin{table}[ht!]
	\begin{minipage}[t]{0.5\linewidth}
		\centering
		\begin{tabular}{p{3cm} p{3cm}}
			\toprule
			\textbf{Partecipante} & \textbf{Durata presenza} \\
			\midrule
			Guglielmo Barison & 0 h \\
			Linda Barbiero &  1 h \\
			Pietro Busato & 1 h \\
			Oscar Konieczny & 1 h \\
			Davide Donanzan & 1 h \\
			Veronica Tecchiati & 1 h \\
			\bottomrule
		\end{tabular}
		\caption{Partecipanti NaN1fy.}
		\label{table:Partecipanti NaN1fy}
	\end{minipage} 
\end{table}
\subsection{Ordine del giorno}
\begin{itemize}
	\setlength\itemsep{0em}
    \item Discussione post SAL avvenuto in data 2024-07-16;
    \item Retrospettiva e consuntivo ore del settimo sprint appena concluso;
    \item Preventivo ore dell'ottavo sprint;
\end{itemize}
\subsection{Sintesi dell'incontro}
La riunione si è tenuta dopo l'incontro con l'azienda SyncLab con il SAL successivo fissato per il 2024-07-30. \\ 
La riunione è iniziata con il calcolo del consuntivo del periodo precedente, valutando gli avvenimenti accaduti e le considerazioni della Proponente sui prossimi step. \\
Come d'accordo con l'azienda, il team si occuperà di ultimare l'implementazione del sistema di stream processing tramite Apache Flink, di sistemare le dashboard relative ai casi d'uso non ancora implementati e di concludere la personalizzazione dell'interfaccia grafica di Grafana.\\
In base agli obiettivi fissati, dunque, si è stipulato il preventivo del successivo sprint della durata di due settimane, eleggendo come rappresentante per la prima Veronica Tecchiati, mentre per la seconda Linda Barbiero.
\subsection{Decisioni prese}
\begin{itemize}
	\setlength\itemsep{0em}
    \item Elezione del nuovo responsabile: Veronica Tecchiati per la prima settimana, Linda Barbiero per la seconda;
	\item Implementazione del sistema di stream processing con Apache Flink;
	\item Conclusione dell'implementazione delle dashboard;
	\item Personalizzazione dell'interfaccia grafica di Grafana.
	% consigliata la forma \textit{Viene adottato} quando viene adottato un certo modo di fare/strumento
	% per nomi di aziene e capitolati usare \texttt{}, e.g \texttt{Easy meal} \texttt{C6} 
\end{itemize}
\newpage
\section{Attività da svolgere}
\begin{table}[ht!]
	\centering
	\begin{tabular}{p{7cm}cp{7cm}}
		\toprule
		\textbf{Titolo} & \textbf{\# Issue} & \textbf{Redattori} \\
		\midrule
        \href{https://github.com/NaN1fy/SyncCity/issues/35}{\underline{Personalizzazione grafica di Grafana}} & 35*\tnote{*} & Linda Barbiero \\\\
        \href{https://github.com/NaN1fy/SyncCity/issues/36}{\underline{Implementazione Apache Flink}} & 36*\tnote{*} & Guglielmo Barison, Pietro Busato, Davide Donanzan \\\\
        \href{https://github.com/NaN1fy/SyncCity/issues/37}{\underline{Rifinitura dashboard e sensori}} & 37*\tnote{*} & Oscar Konieczny, Veronica Tecchiati \\\\
		\bottomrule
	\end{tabular}
	\begin{tablenotes}
		\vspace{1em}
		\item * indica che la issue fa parte della repository SyncCity
	\end{tablenotes}
	\caption{Attività da svolgere.}
	\label{table:Attivita da svolgere}
\end{table}
\end{document}
