% changelog: "2.0.0, 2024-08-19, Approvazione per PB."

\documentclass[8pt]{article}
\usepackage[italian]{babel}
\usepackage[utf8]{inputenc}
\usepackage[letterpaper, left=1in, right=1in, bottom=0.75in, top=0.75in]{geometry}
\usepackage{amsmath}
\usepackage{subfiles}
\usepackage{lipsum}
\usepackage{csquotes}
\usepackage{amsfonts}
\usepackage[sfdefault]{plex-sans}
\usepackage{float}
\usepackage{pifont}
\usepackage{mathabx}
\usepackage[euler]{textgreek}
\usepackage{makecell}
\usepackage{tikz}
\usepackage{wrapfig}
\usepackage{siunitx}
\usepackage{amssymb} 
\usepackage{tabularx}
\usepackage{adjustbox}
\usepackage[document]{ragged2e}
\usepackage{floatflt}
\usepackage[hidelinks]{hyperref}
\usepackage{graphicx}
\usepackage{hyperref}
\setcounter{tocdepth}{4}
\usepackage{caption}
\usepackage{multicol}
\usepackage{tikz}
\setlength\parindent{0pt}
\captionsetup{font=footnotesize}
\usepackage{fancyhdr} 
\usepackage{graphicx}
\usepackage{capt-of}% 
\usepackage{booktabs}
\usepackage{varwidth}
\usepackage{datetime2}
\usepackage{xcolor}
\usepackage{longtable}
\usepackage{array}
\usepackage{ragged2e}
\usepackage{colortbl}
\usepackage{verbatim}
\usepackage{enumitem}

\newcommand{\customtitle}{PIANO DI QUALIFICA}% o ESTERNO

% -- STILE COLONNA CENTRATA PER TABELLE -- %
\newcolumntype{M}[1]{>{\centering\arraybackslash}m{#1}}

% -- STILE INTESTAZIONE -- %
\fancypagestyle{mystyle}{
	\fancyhf{} 
	\fancyhead[R]{\includegraphics[height=1cm]{../../template/images/logos/NaN1fy_logo.png}} 
	\fancyhead[L]{\leftmark} 
	\renewcommand{\headrulewidth}{1pt} 
	\fancyhead[L]{\customtitle} 
	\renewcommand{\headsep}{1.3cm} 
	\fancyfoot[C]{\thepage} 
}

% -- PER LA FIRMA -- %
\newcommand{\signatureline}[1]{%
	 \par\vspace{0.5cm}
	\noindent\makebox[\linewidth][r]{\rule{0.2\textwidth}{0.5pt}\hspace{3cm}\makebox[0pt][r]{\vspace{3pt}\footnotesize #1}}%
}

% -- PER IL GLOSSARIO -- %
\newcommand{\glossterm}[1]{#1\textsuperscript{G}} % inserisci \glossterm{termine}

% -- per abilitare 4x sottosezioni es 2.1.1.1
\setcounter{secnumdepth}{4}
\newcommand{\subsubsubsection}[1]{\paragraph{#1}\mbox{}\\\\}

\begin{document}
\definecolor{myblue}{RGB}{23,103,162}
\begin{titlepage}
	\begin{tikzpicture}[remember picture, overlay]
		\node[anchor=south east, opacity=0.2, yshift = -4cm, xshift= 2em] at (current page.south east) {\includegraphics[width=0.7\textwidth, trim=0cm 0cm 5cm 0cm, clip]{../../template/images/logos/Universita_Padova_transparent.png}}; 
		\node[anchor=north west, opacity=1, yshift = 4.2cm, xshift= 1.4cm, scale=1.6] at (current page.south west) {\includegraphics[width=4cm]{../../template/images/logos/NaN1fy_logo.png}};
	\end{tikzpicture}
	
	\begin{minipage}[t]{0.47\textwidth}
		{\large{\textsc{Destinatari}}
			\vspace{3mm}
			\\ \large{\textsc{Prof. Tullio Vardanega}}
			\\ \large{\textsc{Prof. Riccardo Cardin}}
		}
	\end{minipage}
	\hfill
	\begin{minipage}[t]{0.47\textwidth}\raggedleft
		{\large{\textsc{Redattori}}
			\vspace{3mm}
			{\\\large{\textsc{Guglielmo Barison}\\}} % massimo due 
			{\large{\textsc{Davide Donanzan}}}
			
			
		}
		\vspace{8mm}
		
		{\large{\textsc{Verificatori}}
			\vspace{3mm}
			{\\\large{\textsc{Oscar Konieczny}\\}} 
			{\large{\textsc{Linda Barbiero}\\}} 
			{\large{\textsc{Pietro Busato}\\}}
			
		}
		\vspace{2mm}\vspace{2mm}
	\end{minipage}
	\vspace{4cm}
	\begin{center}
		\begin{flushright}
			{\fontsize{30pt}{52pt}\selectfont \textbf{Piano di Qualifica}} % o ESTERNO
		\end{flushright}
		\vspace{3cm}
	\end{center}
	\vspace{10 cm}
	{\small \textsc{\href{mailto: nan1fyteam.unipd@gmail.com}{nan1fyteam.unipd@gmail.com}}}
\end{titlepage}
\pagestyle{mystyle}
\section*{Registro delle Modifiche}
\begin{table}[ht!]	
	\centering
	\begin{tabular}{p{1.2cm} p{2cm} p{6cm} p{3cm} p{2cm}}
		\toprule
		\textbf{Versione}& \textbf{Data} & \textbf{Descrizione} & \textbf{Redattore} & \textbf{Verificatore} \\
		\midrule
        2.0.0 & 2024-08-19 & \textbf{Approvazione per PB} & & \\\\
        1.4.1 & 2024-08-18 & Completamento e aggiornamento metriche. & Davide Donanzan & Pietro Busato \\\\
        1.3.1 & 2024-08-15 & Completamento della stesura dei paragrafi metriche di qualità di prodotto. & Linda Barbiero & Oscar Konieczny \\\\
        1.3.0 & 2024-08-15 & Stesura iniziale dei paragrafi metriche di qualità di prodotto. & Linda Barbiero & Pietro Busato \\\\
        1.2.0 & 2024-08-12 & Aggiornamento del tracciamento dei requisiti. & Davide Donanzan & Oscar Konieczny \\\\
        1.1.0 & 2024-08-09 & Aggiornamento metriche e test d'accettazione. & Davide Donanzan & Pietro Busato \\
		\bottomrule
		% Ruolo Redattore o Verificatore
	\end{tabular}
	\caption*{Tabella: Registro delle modifiche.}
	\label{table:Registro delle modifiche}
\end{table}
\newpage
\begin{table}[ht!]	
	\centering
	\begin{tabular}{p{1.2cm} p{2cm} p{6cm} p{3cm} p{2cm}}
		\toprule
		\textbf{Versione}& \textbf{Data} & \textbf{Descrizione} & \textbf{Autore} & \textbf{Ruolo} \\
		\midrule
  		1.0.0 & 2024-06-23 & \textbf{Approvazione per RTB} & & \\\\
        0.9.4 & 2024-06-23 & Aggiornamento metriche. & Davide Donanzan & Redattore \\\\
        0.9.3 & 2024-06-03 & Verifica completa con piccole modifiche. & Pietro Busato & Verificatore \\\\
        0.9.2 & 2024-06-01 & Verifica completa con piccole modifiche. & Linda Barbiero & Verificatore \\\\
        0.9.1 & 2024-06-01 & Verifica completa con piccole modifiche. & Oscar Konieczny & Verificatore \\\\
        0.9.0 & 2024-05-31 & Stesura sezione \ref{sec:tracciamento test sistema}. & Davide Donanzan & Redattore \\\\
        0.8.0 & 2024-05-30 & Completamento della sezione \ref{sec:cruscotto della qualita}. & Davide Donanzan & Redattore \\\\
        0.7.0 & 2024-05-29 & Aggiunta di ulteriori \glossterm{test} di unità, integrazione, \glossterm{sistema} e accettazione. & Davide Donanzan & Redattore \\\\
        0.6.0 & 2024-05-28 & Stesura sezione \ref{sec:liste controllo}. & Guglielmo Barison & Redattore \\\\
        0.5.0 & 2024-05-24 & Stesura sezione \ref{sec:cruscotto della qualita}. & Davide Donanzan & Redattore \\\\
        0.4.0 & 2024-05-21 & Stesura sezione \ref{sec:test accettazione}. & Guglielmo Barison & Redattore \\\\
        0.3.1 & 2024-05-15 & Modifiche minori alla struttura di \ref{sec:test unita}. & Davide Donanzan & Redattore \\\\
		0.3.0 & 2024-05-14 & Inizio stesura \ref{sec:test unita} e \ref{sec:test integrazione}. & Davide Donanzan & Redattore \\\\
		0.2.0 & 2024-04-28 & Inizio stesura \ref{sec:strategie testing}. & Guglielmo Barison & Redattore \\\\
		0.1.0 & 2024-04-18 & Stesura \ref{sec:obiettivi qualita}. & Guglielmo Barison & Redattore \\\\
		0.0.0 & 2024-04-09 & Struttura di base ed introduzione.  & Guglielmo Barison & Redattore \\
		\bottomrule
		% Ruolo Redattore o Verificatore
	\end{tabular}
	\caption*{Tabella: Registro delle modifiche.}
	\label{table:Registro delle modifiche}
\end{table}
\newpage
\tableofcontents
\newpage
\listoffigures
\newpage
\listoftables
\newpage
\justifying
\section{Introduzione}
\subsection{Scopo del documento}
Questo documento offre una panoramica dettagliata delle strategie di verifica e validazione adottate per assicurare la \glossterm{qualità} del prodotto e dei processi nel contesto del progetto in questione. Sarà costantemente aggiornato per riflettere l'evoluzione del progetto e concentrerà l'attenzione sui risultati delle verifiche per risolvere tempestivamente eventuali criticità.
\\\\
Il \textit{Piano di Qualifica}, dinamico e incrementale, illustra le pratiche per il controllo di qualità degli artefatti e dei processi, con particolare enfasi sulle metriche di valutazione del prodotto. È progettato per guidare l'adozione di processi mirati al miglioramento continuo, fornendo misure quantitative per valutare il progresso del progetto. Questo impegno costante per la qualità si riflette nel regolare aggiornamento del documento per adattarsi alle esigenze mutevoli del progetto, garantendo così la crescita e l'evoluzione sia del processo che del prodotto nel tempo.
\subsection{Scopo del \glossterm{capitolato}}
Lo scopo del capitolato C6 è proporre una soluzione per la creazione di una piattaforma di monitoraggio per una \glossterm{smart city}. Tale piattaforma deve raccogliere e analizzare in tempo reale una vasta gamma di dati, provenienti da sensori distribuiti nella città, riguardanti aspetti quali traffico, qualità dell'aria, consumi energetici e altro ancora. L'obiettivo è fornire alle autorità locali informazioni dettagliate per prendere decisioni informate sulla gestione delle risorse e l'implementazione dei servizi, coinvolgendo anche i cittadini attraverso la condivisione di dati e la partecipazione attiva. La soluzione proposta prevede l'utilizzo di tecnologie per il data streaming processing e la simulazione dei dati dei sensori, con l'obiettivo di fornire \glossterm{dashboard} intuitive per la visualizzazione dei dati raccolti e l'analisi delle condizioni della città.\\
L’applicativo sviluppato richiede una copertura dei \glossterm{test} di almeno l’80\%.

\subsection{Glossario}
Per garantire chiarezza nel linguaggio utilizzato nei documenti, è stato redatto un Glossario contenente le definizioni dei termini con significato specifico da disambiguare. Tali termini sono contrassegnati con una G ad apice. L'inserimento di un termine nel Glossario è considerato completo solo dopo averne fornito la definizione.
\subsection{Riferimenti}
\subsubsection{Normativi}
\begin{itemize}
	\item \textit{Norme di Progetto v1.0.0};
	\item Presentazione e documentazione del \glossterm{capitolato} d’appalto C6 - SyncCity:
	\begin{itemize}
		\item \href{https://www.math.unipd.it/~tullio/IS-1/2023/Progetto/C6p.pdf}{\color{myblue}https://www.math.unipd.it\textasciitilde{}tullio/IS-1/2023/Progetto/C6p.pdf} (Ultimo accesso: \today)
		\item \href{https://www.math.unipd.it/~tullio/IS-1/2023/Progetto/C6.pdf}{\color{myblue}https://www.math.unipd.it/\textasciitilde{}tullio/IS-1/2023/Progetto/C6.pdf} (Ultimo accesso: \today)
	\end{itemize}
	\item Regolamento di progetto:
	\begin{itemize}
		\item \href{https://www.math.unipd.it/~tullio/IS-1/2023/Dispense/PD2.pdf}{\color{myblue}https://www.math.unipd.it/\textasciitilde{}tullio/IS-1/2023/Dispense/PD2.pdf} (Ultimo accesso: \today)
	\end{itemize}
\end{itemize}
\clearpage
\subsubsection{Informativi}
\begin{itemize}
	\item Dispense T7 - Qualità del software:
	\begin{itemize}
		\item \href{https://www.math.unipd.it/~tullio/IS-1/2023/Dispense/T7.pdf}{\color{myblue}https://www.math.unipd.it/\textasciitilde{}tullio/IS-1/2023/Dispense/T7.pdf} (Ultimo accesso: \today)
	\end{itemize}
	\item Dispense T8 - Qualità di processo:
	\begin{itemize}
		\item \href{https://www.math.unipd.it/~tullio/IS-1/2023/Dispense/T8.pdf}{\color{myblue}https://www.math.unipd.it/\textasciitilde{}tullio/IS-1/2023/Dispense/T8.pdf} (Ultimo accesso: \today)
	\end{itemize}
	\item Dispense T9 - Verifica e validazione:
	\begin{itemize}
		\item \href{https://www.math.unipd.it/~tullio/IS-1/2023/Dispense/T9.pdf}{\color{myblue}https://www.math.unipd.it/\textasciitilde{}tullio/IS-1/2023/Dispense/T9.pdf} (Ultimo accesso: \today)
	\end{itemize}
	\item \glossterm{ISO} / \glossterm{IEC} 9126:
	\begin{itemize}
		\item \href{https://it.wikipedia.org/wiki/ISO/IEC_9126}{\color{myblue}https://it.wikipedia.org/wiki/ISO/IEC\_9126} (Ultimo accesso: \today)
	\end{itemize}
	\item \glossterm{ISO} / \glossterm{IEC} 12207-1995:
	\begin{itemize}
		\item \href{https://www.math.unipd.it/~tullio/IS-1/2009/Approfondimenti/ISO_12207-1995.pdf}{\color{myblue}https://www.math.unipd.it/\textasciitilde{}tullio/IS-1/2009/Approfondimenti/ISO\_12207-1995.pdf} \\ (Ultimo accesso: \today)
	\end{itemize}
\end{itemize}
\clearpage
\section{Obiettivi di qualità}\label{sec:obiettivi qualita}
Ogni \glossterm{processo} viene valutato tramite l'utilizzo di metriche specifiche, le cui definizioni sono dettagliate nelle sezioni Metriche di qualità del processo e Metriche di qualità del prodotto del documento \textit{Norme di Progetto v1.0.0}. Queste sezioni delineano i criteri che le metriche devono rispettare per essere valutate come accettabili o eccellenti. La sigla MPC sta ad indicare le metriche di processo.
% ricordarsi di verficare effettivamente il nome della sezione in NdP, nome in corsivo per riferimenti nel testo a documentazione del gruppo?
\subsection{\glossterm{Qualità} di \glossterm{processo}}
La base della qualità del processo risiede nell'idea che, per ottenere un prodotto conforme a determinati standard di qualità, sia fondamentale sottoporre i processi che lo supportano a controlli regolari, al fine di ottimizzarli. Il concetto di qualità del processo viene quindi applicato a tutte le attività, pratiche e metodologie utilizzate durante l'intero ciclo di vita del software. In breve, la qualità del processo mira a integrare la qualità nel prodotto stesso, garantendo che sia intrinseca al processo e non solo un obiettivo secondario.
\subsubsection{Processi primari}
\subsubsubsection{Fornitura} 
\begin{table}[h]	
	\centering
	\begin{tabular}{lccc}
		\toprule
		\textbf{Metrica}& \textbf{Descrizione} & \textbf{Valore accettabile} & \textbf{Valore ottimo} \\
		\midrule
		MPC-EV & Earned value (EV) & $\geq$ 0 & $\leq$ EAC \\\\
		MPC-PV & Planned Value (PV) & $\geq$ 0 & $\leq$ \glossterm{BAC}\\\\
		MPC-AC & Actual costo (AC) & $\geq$ 0 & $\leq$ EAC\\\\
		MPC-CPI & Cost Performance Index (CPI) & tra 0.95 e 1.05 & $\leq$ 1\\\\
		MPC-EAC & Estimate At Completion (EAC) & deviazione del $\pm$ 5\% dal \glossterm{BAC} & \glossterm{BAC}\\\\
		MPC-ETC & Estimate To Completion (ETC) & $\geq $ 0 & $\leq$ EAC\\\\
		MPC-VAC & Variance At Completion (VAC) & deviazione del $\pm$ 10\% dal \glossterm{BAC} & 0\%\\\\
		MPC-SV & Schedule Variance (SV) & deviazione del $\pm$ 10\% dal \glossterm{BAC} & 0\%\\\\
		MPC-BV & Budget Variance (BV) & deviazione del $\pm$ 10\% dal \glossterm{BAC}  & 0\%\\
		\bottomrule
		% Ruolo Redattore o Verificatore
	\end{tabular}
	\caption{Metriche per il \glossterm{processo} di fornitura.}
	\label{table:Tabella metriche per il processo di fornitura.}
\end{table}
\clearpage
\subsubsubsection{Sviluppo}
\begin{table}[h]	
	\centering
	\begin{tabular}{lccc}
		\toprule
		\textbf{Metrica}& \textbf{Descrizione} & \textbf{Valore accettabile} & \textbf{Valore ottimo} \\
		\midrule
		MPC-RSI & Requirements Stability Index (RSI) & $\geq $ 75\%  & $\leq$ 100\% \\
		% MPC-SFIN & Structural Fan-In (SFIN) & - & Va massimizzato\\\\
		% MPC-SFOUT & Structural Fan-Out (SFOUT) & - & Va minimizzato\\
		\bottomrule
		% Ruolo Redattore o Verificatore
	\end{tabular}
	\caption{Valori accettabili e ottimi per ogni metrica riguardante il \glossterm{processo} di sviluppo.}
	\label{table:Valori accettabili e ottimi per ogni metrica riguardante il processo di sviluppo.}
\end{table}
\subsubsection{Processi di supporto}
\subsubsubsection{Documentazione}
\begin{table}[h]	
	\centering
	\begin{tabular}{lccc}
		\toprule
		\textbf{Metrica}& \textbf{Descrizione} & \textbf{Valore accettabile} & \textbf{Valore ottimo} \\
		\midrule
		MPC-IG & Indice Gulpease (IG) & $\geq$ 60\% & 100 \\\\
		MPC-CO & Correttezza Ortografica (CO) & 0 & 0 \\
		\bottomrule
		% Ruolo Redattore o Verificatore
	\end{tabular}
	\caption{Metriche per il \glossterm{processo} di documentazione.}
	\label{table:Tabella delle metriche per il processo di documentazione}
\end{table}
\subsubsubsection{Verifica}
\begin{table}[h]	
	\centering
	\begin{tabular}{lccc}
		\toprule
		\textbf{Metrica}& \textbf{Descrizione} & \textbf{Valore accettabile} & \textbf{Valore ottimo} \\
		\midrule
		MPC-CC & Code Coverage (CC) & $\geq$ 80\% & 100\% \\
		\bottomrule
	\end{tabular}
	\caption{Metriche per il \glossterm{processo} di verifica.}
	\label{table:Tabella delle metriche per il processo di verifica}
\end{table}
\subsubsubsection{Gestione della qualità}
\begin{table}[H]	
	\centering
	\begin{tabular}{lccc}
		\toprule
		\textbf{Metrica}& \textbf{Descrizione} & \textbf{Valore accettabile} & \textbf{Valore ottimo} \\
		\midrule
		MPC-QMS & Quality Metrics Satisfied (QMS) & $\geq$ 75\%& 100\%\\
		\bottomrule
	\end{tabular}
	\caption{Valori accettabili e ottimi per ogni metrica riguardante il \glossterm{processo} di gestione della qualità.}
	\label{table:Valori accettabili e ottimi per ogni metrica riguardante il processo di gestione della qualità.}
\end{table}
\clearpage
\subsubsection{Processi organizzativi}
\subsubsubsection{Gestione dei processi}
\begin{table}[H]	
	\centering
	\begin{tabular}{lccc}
		\toprule
		\textbf{Metrica}& \textbf{Descrizione} & \textbf{Valore accettabile} & \textbf{Valore ottimo} \\
		\midrule
		MPC-NR & Non-calculated Risk (NR) & $\leq$ 3 & 0\\
		\bottomrule
	\end{tabular}
	\caption{Valori accettabili e ottimi per ogni metrica riguardante il \glossterm{processo} di gestione dei processi.}
	\label{table:Valori accettabili e ottimi per ogni metrica riguardante il processo di gestione dei processi.}
\end{table}
\subsection{\glossterm{Qualità} di prodotto}
Si riferisce alle caratteristiche di un'entità risultante dallo sviluppo software, che influenzano la sua capacità di soddisfare sia le esigenze esplicite che implicite. In altre parole, è quanto il prodotto si adatta alle aspettative del cliente o agli standard predefiniti. Questo implica una valutazione completa del software realizzato, concentrandosi su attributi come usabilità, \glossterm{funzionalità}, affidabilità e manutenibilità, oltre alle prestazioni generali. L'obiettivo è garantire che il software non solo soddisfi le richieste del cliente e funzioni correttamente, ma che lo faccia conformemente ai rigidi standard di \glossterm{qualità} stabiliti. Per raggiungere questo obiettivo, il team si impegna a seguire le metriche di prodotto, indicate con la sigla MPD, come specificato nel documento \textit{Norme di Progetto v1.0.0}.
% ricordarsi di verficare effettivamente il nome della sezione in NdP, nome in corsivo per riferimenti nel testo a documentazione del gruppo?
\subsubsection{Funzionalità}
\begin{table}[H]	
	\centering
	\begin{tabular}{lccc}
		\toprule
		\textbf{Metrica}& \textbf{Descrizione} & \textbf{Valore accettabile} & \textbf{Valore ottimo} \\
		\midrule
		MPD-ROS& Requisiti Obbligatori Soddisfatti (ROS) & 100\% & 100\%\\\\
		MPD-RDS & Requisiti Desiderabili Soddisfatti (RDS) & $\geq$ 0\% & $\geq$ 75\% \\\\
		MPD-ROPS & Requisiti Opzionali Soddisfatti (ROPS) & $\geq$ 0\% & $\geq$ 75\% \\
		\bottomrule
	\end{tabular}
	\caption{Valori accettabili e ottimi per ogni metrica riguardante la funzionalità del prodotto.}
	\label{table:Valori accettabili e ottimi per ogni metrica riguardante la funzionalità del prodotto.}
\end{table}
\subsubsection{Affidabilità}
\begin{table}[H]	
	\centering
	\begin{tabular}{lccc}
		\toprule
		\textbf{Metrica}& \textbf{Descrizione} & \textbf{Valore accettabile} & \textbf{Valore ottimo} \\
		\midrule
		MPD-BC & Branch Coverage (BC) & $\geq$ 80\% & 100\%\\\\
		MPD-SC & Statement Coverage (SC) & $\geq$ 80\% & 100\% \\\\
		MPD-FD & Failure Density (FD) & $\leq$ 20\% & 0\% \\\\
		MPD-PTCP & Passed Test Cases Percentage (PTCP) & 100\%  & 100\% \\
		\bottomrule
	\end{tabular}
	\caption{Valori accettabili e ottimi per ogni metrica riguardante l’affidabilità del prodotto.}
	\label{table:Valori accettabili e ottimi per ogni metrica riguardante l’affidabilità del prodotto.}
\end{table}
\subsubsection{Usabilità}
\begin{table}[H]	
	\centering
	\begin{tabular}{lccc}
		\toprule
		\textbf{Metrica}& \textbf{Descrizione} & \textbf{Valore accettabile} & \textbf{Valore ottimo} \\
		\midrule
		MPD-FU & Facilità di Utilizzo (FU) & $\geq$ 9 click & $\geq$ 5 click \\\\
		MPD-TA & Tempo di Apprendimento (TA) & $\leq$ 15 minuti & $\leq$ 5 minuti \\
		\bottomrule
	\end{tabular}
	\caption{Valori accettabili e ottimi per ogni metrica riguardante l’usabilità del prodotto.}
	\label{table:Valori accettabili e ottimi per ogni metrica riguardante l’usabilità del prodotto.}
\end{table}
\subsubsection{Efficienza}
\begin{table}[H]	
	\centering
	\begin{tabular}{lccc}
		\toprule
		\textbf{Metrica}& \textbf{Descrizione} & \textbf{Valore accettabile} & \textbf{Valore ottimo} \\
		\midrule
		MPD-UR & Utilizzo risorse (UR) & $\geq$ 75\% & 100\% \\
		\bottomrule
	\end{tabular}
	\caption{Valori accettabili e ottimi per ogni metrica riguardante l’efficienza del prodotto.}
	\label{table:Valori accettabili e ottimi per ogni metrica riguardante l’efficienza del prodotto.}
\end{table}
\subsubsection{Manutenibilità}
\begin{table}[H]	
	\centering
	\begin{tabular}{lccc}
		\toprule
		\textbf{Metrica}& \textbf{Descrizione} & \textbf{Valore accettabile} & \textbf{Valore ottimo} \\
		\midrule
		MPD-CC & Complessità Ciclomatica (CC) & 11-20 & 1-10 \\
		% MPD-CS & Code Smell (CS) & 0 & 0 \\
		\bottomrule
	\end{tabular}
	\caption{Valori accettabili e ottimi per ogni metrica riguardante la manutenibilità del prodotto.}
	\label{table:Valori accettabili e ottimi per ogni metrica riguardante la manutenibilità del prodotto.}
\end{table}
\clearpage
\section{Strategie di testing}\label{sec:strategie testing}
In questa sezione viene esposto il piano di testing che verrà utilizzato per garantire la correttezza finale del prodotto. Come enunciato nel documento \textit{Norme di Progetto v1.0.0}, il piano segue il \glossterm{modello a V}, il quale associa ad ogni fase di sviluppo una corrispondente tipologia di testing. Tali tipologie sono le seguenti:
\begin{itemize}
	\item \textbf{Test di unità:} si verifica il corretto funzionamento delle unità componenti il \glossterm{sistema}. Un'unità rappresenta un elemento indivisibile e indipendente del sistema; 
	\item \textbf{Test di integrazione:} si verifica il corretto funzionamento di più unità che cooperano per svolgere uno specifico compito (tali unità devono aver superato i loro \glossterm{test} di unità precedentemente);
	\item \textbf{Test di sistema:} si verifica il corretto funzionamento del sistema nella sua interezza. I requisiti funzionali obbligatori, di vincolo, di \glossterm{qualità} e di prestazione, precedentemente concordati con la \glossterm{Proponente} mediante stipulazione del contratto, devono essere soddisfatti per intero;
	\item \textbf{Test di accettazione:} si verifica il soddisfacimento della \glossterm{Proponente} rispetto al prodotto software. Il loro superamento permette di procedere con il rilascio del prodotto.
\end{itemize}
Per le procedure necessarie all’esecuzione di \glossterm{test} di unità e di integrazione si rimanda al documento \textit{Norme di Progetto v1.0.0} nella sezione relativa al processo di verifica.
% verificare effettivamente poi norme di progetto
\subsection{Codice dei \glossterm{test}}
Ogni \glossterm{test} è associato ad un codice univoco definito nel seguente formato:
\begin{center}
	\textbf{T[Tipologia]-[Numero]}
\end{center}
Dove \textbf{Tipologia} indica la tipologia del test: 
\begin{itemize}
    \item \textbf{U:} di unit\`{a};
	\item \textbf{I:} di integrazione;
	\item \textbf{S:} di sistema;
	\item \textbf{A:} di accettazione.
\end{itemize}
Ogni test ha uno \textbf{Stato}, che puo essere:
\begin{itemize}
	\item \textbf{V:} verificato. Il test ha esito positivo;
	\item \textbf{NV:} non verificato. Il test ha esito negativo; 
	\item \textbf{NI:} non implementato.
\end{itemize}
\clearpage
\subsection{\glossterm{Test} di unità}\label{sec:test unita}
\renewcommand{\arraystretch}{2.5}
\rowcolors{2}{gray!20}{white}
\begin{longtable}{|>{\centering}p{2cm}|>{\RaggedRight}m{12cm}|>{\centering\arraybackslash}p{2cm}|}
    \hline
    \rowcolor{white}
    \textbf{Codice Test} & \textbf{Descrizione} & \textbf{Stato Test} \\
    \hline
    \endfirsthead 
    \rowcolor{white}
    \caption{Tabella dei \glossterm{test} di unità.} 
    \label{table:Tabella dei test di unità}
    \endlastfoot 
    
    TU-01 & Verificare che la classe \verb|SimulatorControllerFactory| crei correttamente un'istanza
    della classe \verb|SimulatorController|.  & V \\
    \hline

    TU-02 & Verificare che la classe \verb|SimulatorController| funzioni correttamente. Nello
    specifico, si esaminano i metodi \verb|start_all| e \verb|stop_all|.  & V \\
    \hline

    TU-03 & Verificare che la classe \verb|SimulatorThread| adoperi correttamente. & V \\
    \hline

    TU-04 & Verificare che la classe \verb|TemperatureSensor| restituisca la stringa in formato \glossterm{JSON} attesa. & V \\
    \hline

    TU-05 & Verificare che la classe \verb|HumiditySensor| restituisca la stringa in formato \glossterm{JSON} attesa. & V \\
    \hline

    TU-06 & Verificare che la classe \verb|PrecipitationIntesitySensor| restituisca la stringa in formato \glossterm{JSON} attesa. & V \\
    \hline

    TU-07 & Verificare che la classe \verb|AirPollutionSensor| restituisca la stringa in formato \glossterm{JSON} attesa. & V \\
    \hline

    TU-08 & Verificare che la classe \verb|WaterLevelSensor| restituisca la stringa in formato \glossterm{JSON} attesa. & V \\
    \hline

    TU-09 & Verificare che la classe \verb|ParkingSensor| restituisca la stringa in formato \glossterm{JSON} attesa. & V \\
    \hline

    TU-10 & Verificare che la classe \verb|PaymentParkingSensor| restituisca la stringa in formato \glossterm{JSON} attesa. & V \\
    \hline

    TU-11 & Verificare che la classe \verb|ElectricalFailureSensor| restituisca la stringa in formato \glossterm{JSON} attesa. & V \\
    \hline
    
    TU-12 & Verificare che la classe \verb|WasteFillingSensor| restituisca la stringa in formato \glossterm{JSON} attesa. & V \\
    \hline

    TU-13 & Verificare che la classe \verb|ChargingStationSensor| restituisca la stringa in formato \glossterm{JSON} attesa. & V \\
    \hline

    TU-14 & Verificare che la classe \verb|ChargeConsumptionSensor| restituisca la stringa in formato \glossterm{JSON} attesa. & V \\
    \hline

    TU-15 & Verificare che la classe \verb|PaymentStationSensor| restituisca la stringa in formato \glossterm{JSON} attesa. & V \\
    \hline
    
    TU-16 & Verificare che le istanze della classe \verb|KafkaProducer| adoperino nelle modalità attese nella produzione di dati, sia nel caso di success che nel caso di failure. & V \\
    \hline

    TU-17 & Verificare che la funzione \verb|write| della classe \verb|KafkaStreamWriter| funzioni correttamente. & V \\
    \hline

    TU-18 & Verificare che la funzione \verb|write| della classe
    \verb|StdoutStreamWriter| funzioni correttamente. & V \\
    \hline

    TU-19 & Verificare che la funzione \verb|acked| funzioni correttamente sia nel caso di success che nel caso di failure. & V \\
    \hline

    TU-20 & Verificare che la classe \verb|Coordinates| funzioni correttamente. & V \\
    \hline

    TU-21 & Verificare che la funzione \verb|jsonfy| funzioni correttamente. & V \\
    \hline

\end{longtable}
\clearpage
\subsection{\glossterm{Test} di integrazione}\label{sec:test integrazione}
\renewcommand{\arraystretch}{2.5}
\rowcolors{2}{gray!20}{white}
\begin{longtable}{|>{\centering}p{2cm}|>{\RaggedRight}m{12cm}|>{\centering\arraybackslash}p{2cm}|}
    \hline
    \rowcolor{white}
    \textbf{Codice Test} & \textbf{Descrizione} & \textbf{Stato Test} \\
    \hline
    \endfirsthead 
    \rowcolor{white}
    \caption{Tabella dei \glossterm{test} di integrazione.} 
    \label{table:Tabella dei test di integrazione}
    \endlastfoot 
        
    TI-01 & Verificare la persistenza dei dati della classe \verb|TemperatureSensor| nel \glossterm{database} \glossterm{ClickHouse}.  & V \\
    \hline

    TI-02 & Verificare la persistenza dei dati della classe \verb|HumiditySensor| nel \glossterm{database} \glossterm{ClickHouse}.  & V \\
    \hline
    
    TI-03 & Verificare la persistenza dei dati della classe \verb|PrecipitationIntensitySensor| nel \glossterm{database} \glossterm{ClickHouse}. & V \\
    \hline

    TI-04 & Verificare la persistenza dei dati della classe \verb|AirPollutionSensor| nel \glossterm{database} \glossterm{ClickHouse}. & V \\
    \hline

    TI-05 & Verificare la persistenza dei dati della classe \verb|WaterLevelSensor| nel \glossterm{database} \glossterm{ClickHouse}. & V \\
    \hline

    TI-07 & Verificare la persistenza dei dati della classe \verb|ParkingSensor| nel \glossterm{database} \glossterm{ClickHouse}. & V \\
    \hline 

    TI-08 & Verificare la persistenza dei dati della classe \verb|PaymentParkingSensor| nel \glossterm{database} \glossterm{ClickHouse}. & V \\
    \hline
    
    TI-09 & Verificare la persistenza dei dati della classe \verb|ElectricalFailureSensor| nel \glossterm{database} \glossterm{ClickHouse}. & V \\
    \hline

    TI-10 & Verificare la persistenza dei dati della classe \verb|WasteFillingSensor| nel \glossterm{database} \glossterm{ClickHouse}. & V \\
    \hline
    
    TI-11 & Verificare la persistenza dei dati della classe \verb|ChargingStationSensor| nel \glossterm{database} \glossterm{ClickHouse}. & V \\
    \hline

    TI-12 & Verificare la persistenza dei dati della classe \verb|ChargeConsumptionSensor| nel \glossterm{database} \glossterm{ClickHouse}. & V \\
    \hline

    TI-13 & Verificare la persistenza dei dati della classe \verb|PaymentStationSensor| nel \glossterm{database} \glossterm{ClickHouse}. & V \\
    \hline
        
\end{longtable}
\clearpage
\subsection{\glossterm{Test} di \glossterm{sistema}}
\renewcommand{\arraystretch}{2.5}
\rowcolors{2}{gray!20}{white}
\begin{longtable}{|>{\centering}p{2cm}|>{\RaggedRight}m{12cm}|>{\centering\arraybackslash}p{2cm}|}
    \hline
    \rowcolor{white}
    \textbf{Codice Test} & \textbf{Descrizione} & \textbf{Stato Test} \\
    \hline
    \endfirsthead 
    \rowcolor{white}
    \caption{Tabella dei \glossterm{test} di \glossterm{sistema}.} 
    \label{table:Tabella dei test di sistema}
    \endlastfoot 
    TS-01 & Verificare che l'amministratore pubblico possa utilizzare l'applicativo solamente previa autenticazione. Tale procedura avviene attraverso l'inserimento di username e password. & V\\
    \hline
    TS-02 & Verificare che l'amministratore pubblico visualizzi un messaggio di errore qualora le credenziali inserite non siano valide. & V\\
    \hline
    TS-03 & Verificare che l'amministratore pubblico possa visualizzare un menù di selezione delle
    \glossterm{dashboard}, che permetta a sua volta di selezionare una \glossterm{dashboard} tra: Sensori, Ambientale, Urbanistica e Soglie. & V\\
    \hline
    TS-04 & Verificare che l'amministratore pubblico possa monitorare i dati
    provenienti dai sensori relativi alla \glossterm{dashboard} Sensori.
    & V \\
    \hline
    TS-05 & Verificare che l'amministratore pubblico possa visualizzare un pannello
    contenente una mappa che mostri le posizioni dei sensori, mediante icone in base al tipo di \glossterm{sensore} nella \glossterm{dashboard} Sensori.
    & V \\
    \hline
    TS-06 & Verificare che l'amministratore pubblico possa monitorare i dati ambientali, provenienti
    dai sensori nella \glossterm{dashboard} Ambientale.
    & V \\
    \hline
    TS-07 & Verificare che l'amministratore pubblico possa visualizzare un pannello
    contenente un grafico, in formato \glossterm{time series}, rappresentante l'andamento della temperatura,
    espressa in gradi Celsius (°C), per ciascun \glossterm{sensore} nella \glossterm{dashboard} Ambientale.
    & V \\
    \hline
    TS-08 & Verificare che l'amministratore pubblico possa visualizzare un pannello
    contenente un grafico, in formato \glossterm{time series}, rappresentante l'andamento dell'umidità, espressa
    in percentuale, per ciascun \glossterm{sensore} nella \glossterm{dashboard} Ambientale.
    & V \\
    \hline 
    TS-09 & Verificare che l'amministratore pubblico possa visualizzare un pannello
    contenente un grafico, in formato \glossterm{time series}, rappresentante l'andamento della temperatura
    percepita,
    espressa in gradi Celsius (°C), per ciascuna coppia di sensori temperatura-umidità, nella
    \glossterm{dashboard} Ambientale.
    & V \\
    \hline
    TS-10 & Verificare che l'amministratore pubblico possa visualizzare un pannello contenente
    un grafico \glossterm{Gauge} rappresentante la media aritmetica della temperatura,
    espressa in gradi Celsius (°C), basato sulle ultime rilevazioni effettuate dai singoli sensori, che aggreghi i dati nella \glossterm{dashboard} Ambientale.
    & V \\
    \hline
    TS-11 & Verificare che l'amministratore pubblico possa visualizzare un pannello contenente
    un grafico \glossterm{Gauge} rappresentante la media aritmetica dell'umidità,
    espressa in percentuale, basato sulle ultime rilevazioni effettuate dai singoli sensori, che aggreghi i dati nella \glossterm{dashboard} Ambientale.
    & V \\
    \hline
    TS-12 & Verificare che l'amministratore pubblico possa visualizzare un pannello contenente
    un grafico \glossterm{Gauge} rappresentante la media aritmetica della temperatura percepita,
    espressa in gradi Celsius (°C), basato sulle ultime rilevazioni effettuate dalle coppie di sensori temperatura-umidità, che aggreghi i dati nella \glossterm{dashboard} Ambientale.
    & V \\
    \hline
    TS-13 & Verificare che l'amministratore pubblico possa visualizzare un pannello contenente
    un grafico bar chart rappresentante valori statistici della temperatura,
    espressa in gradi Celsius (°C), per ciascun \glossterm{sensore} nella \glossterm{dashboard} Ambientale.
    & V \\
    \hline
    TS-14 & Verificare che l'amministratore pubblico possa visualizzare un pannello contenente
    un grafico bar chart rappresentante valori statistici della temperatura percepita,
    espressa in gradi Celsius (°C), per ciascuna coppia di sensori temperatura-umidità, nella \glossterm{dashboard} Ambientale.
    & V \\
    \hline
    TS-15 & Verificare che l'amministratore pubblico possa visualizzare un pannello contenente
    un grafico in formato \glossterm{time series} rappresentante l'intensità delle precipitazioni,
    espressa in millimetri orari (mm/h), per ciascun \glossterm{sensore} nella \glossterm{dashboard} Ambientale.
    & V \\
    \hline
    TS-16 & Verificare che l'amministratore pubblico possa visualizzare un pannello contenente
    un grafico in formato \glossterm{time series} rappresentante la quantità di polveri sottili presenti nell'aria,
    espressa in microgrammi per metro cubo (µg / $\mbox{m}^{\mbox{3}}$), per ciascun \glossterm{sensore} nella \glossterm{dashboard} Ambientale.
    & V \\
    \hline
    TS-17 & Verificare che l'amministratore pubblico possa visualizzare un pannello contenente
    un grafico in formato \glossterm{time series} rappresentante le misurazioni relative al livello dell'acqua,
    espressa in percentuale, per ciascun \glossterm{sensore} nella \glossterm{dashboard} Ambientale.
    & V \\
    \hline
    TS-18 & Verificare che l'amministratore pubblico possa visualizzare un pannello contenente
    un grafico \glossterm{Gauge} rappresentante l'attuale intensità delle precipitazioni,
    espressa in millimetri orari (mm/h), basata sulle ultime rilevazioni effettuate dai singoli sensori, che aggreghi i dati nella \glossterm{dashboard} Ambientale.
    & V \\
    \hline
    TS-19 & Verificare che l'amministratore pubblico possa visualizzare un pannello contenente
    un grafico \glossterm{Gauge} rappresentante l'attuale inquinamento dell'aria,
    espresso in microgrammi per metro cubo (µg / $\mbox{m}^{\mbox{3}}$), basato sulle ultime rilevazioni effettuate dai singoli sensori nella \glossterm{dashboard} Ambientale.
    & V \\
    \hline
    TS-20 & Verificare che l'amministratore pubblico possa visualizzare un pannello contenente
    un grafico \glossterm{Gauge} rappresentante l'attuale livello dell'acqua,
    espresso in percentuale, basato sulle ultime rilevazioni effettuate dai singoli sensori nella \glossterm{dashboard} Ambientale.
    & V \\
    \hline
    TS-21 & Verificare che l'amministratore pubblico possa monitorare i dati provenienti
    dai sensori relativi ai dati urbanistici in una \glossterm{dashboard} apposita.
    & V \\
    \hline
    TS-22 & Verificare che l'amministratore pubblico possa visualizzare un pannello \glossterm{geomap} che evidenzi lo stato aggiornato della disponibilità dei posti nei vari parcheggi nella
    \glossterm{dashboard} Urbanistica. & V \\
    \hline
    TS-23 & Verificare che l'amministratore pubblico possa visualizzare un pannello in formato tabellare che evidenzi lo stato aggiornato dei posti nei vari parcheggi nella
    \glossterm{dashboard} Urbanistica. & V \\
    \hline
    TS-24 & 
    Verificare che l'amministratore pubblico possa visualizzare un pannello nel formato di un registro che mostri l'elenco delle notifiche di pagamento nei vari parcheggi nella \glossterm{dashboard} Urbanistica.& V \\
    \hline
    TS-25 & 
    Verificare che l'amministratore pubblico possa visualizzare un pannello \glossterm{geomap} che evidenzi i guasti nella rete eletrica nella \glossterm{dashboard} Urbanistica.& V \\
    \hline
    TS-26 & 
    Verificare che l'amministratore pubblico possa visualizzare un pannello in formato \glossterm{time series} rappresentante il riempimento di un'isola ecologica, espresso in percentuale nella \glossterm{dashboard} Urbanistica.& V \\
    \hline
    TS-27 & 
    Verificare che l'amministratore pubblico possa visualizzare un pannello \glossterm{geomap} che evidenzi lo stato di disponibilità delle colonnine di ricarica nella \glossterm{dashboard} Urbanistica.& V \\
    \hline
    TS-28 & 
    Verificare che l'amministratore pubblico possa visualizzare un pannello in formato tabellare che evidenzi lo stato delle colonnine di ricarica nella \glossterm{dashboard} Urbanistica.& V \\
    \hline
    TS-29 & 
    Verificare che l'amministratore pubblico possa visualizzare un pannello nel formato \glossterm{time series} rappresentante l'andamento dei consumi delle colonnine di ricarica, espresso in kilowattora (kWh), nella \glossterm{dashboard} Urbanistica.& V \\
    \hline
    TS-30 & 
    Verificare che l'amministratore pubblico possa visualizzare un pannello nel formato di un registro che mostri l'elenco delle notifiche di pagamento, dovute all'utilizzo delle colonnine di ricarica, nella \glossterm{dashboard} Urbanistica.& V \\
    \hline
    TS-31 & Verificare che l'amministratore pubblico possa visualizzare un pannello contenente
    un grafico bar chart rappresentante valori statistici riguardanti i pagamenti effettuati in un parcheggio nella \glossterm{dashboard} Urbanistica.
    & V \\
    \hline
    TS-32 & Verificare che l'amministratore pubblico possa visualizzare un pannello contenente
    un grafico bar chart rappresentante valori statistici riguardanti i pagamenti dovuti all'utilizzo di una colonnina di ricarica nella \glossterm{dashboard} Urbanistica.
    & V \\
    TS-33 & Verificare che l'amministratore pubblico possa visualizzare un pannello contenente
    un grafico \glossterm{Gauge} rappresentante l'efficienza monetaria relativa ad un parcheggio nella \glossterm{dashboard} Urbanistica.
    & V \\
    \hline
    TS-34 & Verificare che l'amministratore pubblico possa controllare il superamento di determinate soglie in una relativa \glossterm{dashboard}. &
    V \\
    \hline
    TS-35 & Verificare che l'amministratore pubblico possa visualizzare un pannello che notifica il superamento delle soglie di temperatura accettabili nella \glossterm{dashboard} Soglie. &
    V \\
    \hline
    TS-36 &Verificare che l'amministratore pubblico possa visualizzare un pannello che notifica il superamento della soglia d'intensità delle precipitazioni accettabili nella \glossterm{dashboard} Soglie. &
    V \\
    \hline
    TS-37 & Verificare che l'amministratore pubblico possa visualizzare un pannello che notifica il superamento della soglia d'inquinamento dell'aria ritenuto accettabile nella \glossterm{dashboard} Soglie. &
    V \\
    \hline
    TS-38 & Verificare che l'amministratore pubblico possa visualizzare un pannello che notifica il superamento della soglia del livello dell'acqua ritenuto accettabile nella \glossterm{dashboard} Soglie. &
    V \\
    \hline
    TS-39 & Verificare che l'amministratore pubblico possa visualizzare un pannello che notifica il superamento della soglia di riempimento delle isole ecologiche ritenuto accettabile nella \glossterm{dashboard} Soglie. &
    V \\
    \hline
    TS-40 & Verificare che l'amministratore pubblico riceva un messaggio di errore qualora il
    \glossterm{sistema} di visualizzazione non riesca a reperire i dati necessari per un determinato pannello. &
    V \\
    \hline
    TS-41 & Verificare che l'amministratore pubblico possa filtrare i dati, visualizzati
    all’interno di un grafico, in base ad un sottoinsieme di sensori da lui
    selezionato. & V \\
    \hline
    TS-42 & Verificare che l'amministratore pubblico possa filtrare i dati in base ad un intervallo temporale. La \glossterm{dashboard} di interesse deve mostrare solamente i dati aventi un timestamp in tale intervallo.
    & V\\
    \hline
    TS-43 & Verificare che l'amministratore pubblico possa rimuovere un filtro precedentemente applicato ai dati.
    & V\\
    \hline
    TS-44 & Verificare che l'amministratore pubblico possa modificare il layout dei pannelli modificandone le dimensioni a proprio piacimento.
    & V\\
    \hline
    TS-45 & Verificare che l'amministratore pubblico possa modificare il layout dei pannelli modificandone la posizione, a proprio piacimento, all'interno della \glossterm{dashboard}.
    & V\\
    \hline
    TS-46 & Verificare che un \glossterm{sensore} possa inserire le rilevazioni della temperatura, espresse in
    gradi Celsius (°C), con annesse coordinate e timestamp della rilevazione. & V \\
    \hline
    TS-47 & Verificare che un \glossterm{sensore} possa inserire le rilevazioni dell'umidità, espresse in
    percentuale, con annesse coordinate e timestamp della rilevazione. & V \\
    \hline
    TS-48 & Verificare che un \glossterm{sensore} possa inserire le rilevazioni dell'intensità delle precipitazioni, espresse in
    millimetri orari (mm/h), con annesse coordinate e timestamp della rilevazione. & V \\
    \hline
    TS-49 & Verificare che un \glossterm{sensore} possa inserire le rilevazioni dell'inquinamento dell'aria, espresse in
    microgrammi per metro cubo (µg / $\mbox{m}^{\mbox{3}}$), con annesse coordinate e timestamp della rilevazione. & V \\
    \hline
    TS-50 & Verificare che un \glossterm{sensore} possa inserire le rilevazioni del livello dell'acqua, espresse in
    percentuale, con annesse coordinate e timestamp della rilevazione. & V \\
    \hline
    TS-51 & Verificare che un \glossterm{sensore} possa inserire le rilevazioni dello stato dei parcheggi,
    espresse tramite valore \glossterm{boolean}, con annesse coordinate e timestamp della rilevazione. & V \\
    \hline 
    TS-52 & Verificare che un \glossterm{sensore} possa inserire le rilevazioni dello stato dei pagamenti
    dei parcheggi, con annesse coordinate e timestamp della rilevazione. & V \\
    \hline
    TS-53 & Verificare che un \glossterm{sensore} possa inserire le rilevazioni di guasti alla rete elettrica espresse tramite valore \glossterm{boolean}, con annesse coordinate e timestamp della rilevazione. & V \\
    \hline
    TS-54 & Verificare che un \glossterm{sensore} possa inserire le rilevazioni dello stato di riempimento delle isole ecologiche,
    espresse in percentuale, con annesse coordinate e timestamp della rilevazione. & V \\
    \hline
    TS-55 & Verificare che un \glossterm{sensore} possa inserire le rilevazioni dello stato di disponibilità delle colonnine di ricarica,
    espresse tramite valore \glossterm{boolean}, con annesse coordinate e timestamp della rilevazione. & V \\
    \hline
    TS-56 & Verificare che un \glossterm{sensore} possa inserire le rilevazioni dei consumi delle colonnine di ricarica,
    espresse in kilowattora (kWh), con annesse coordinate e timestamp della rilevazione. & V \\
    \hline
    TS-57 & Verificare che un \glossterm{sensore} possa inserire le rilevazioni dello stato dei pagamenti dovuti all'utilizzo delle colonnine di ricarica, con annesse coordinate e timestamp della rilevazione. & V \\
    \hline
    TS-58 & Verificare che sia stato implementato almeno un simulatore per ogni tipologia di \glossterm{sensore}
    & V \\
    \hline
    TS-59 & Verificare che i dati prodotti dalle simulazioni siano realistici. & V \\
    \hline
    TS-60 & Verificare che il \glossterm{sistema} possa rilevare eventuali relazioni tra sorgenti di dati
    diversi. & V \\
    \hline
    TS-61 & Verificare che il \glossterm{sistema} possa effettuare previsioni di eventi futuri sulla base di dati storici e attuali. & NI \\
    \hline
\end{longtable}
\clearpage
\subsubsection{Tracciamento dei \glossterm{test} di \glossterm{sistema}}\label{sec:tracciamento test sistema}
\renewcommand{\arraystretch}{2.5}
\rowcolors{2}{gray!20}{white}
\begin{longtable}{|>{\centering}p{4cm}|>{\centering\arraybackslash}p{4cm}|}
\hline
\rowcolor{white}
\textbf{Codice Test} & \textbf{Codice Requisito} \\
\hline
\endfirsthead
\rowcolor{white}
\caption{Tracciamento dei \glossterm{test} di \glossterm{sistema}.}
\label{table:Tracciamento dei test di sistema}
\endlastfoot
    TS-01 & RF-1  \newline
            RF-16 \newline
            RF-17 \\
    \hline 
    TS-02 & RF-18 \\
    \hline 
    TS-03 & RF-19 \\
    \hline 
    TS-04 & RF-20 \\
    \hline
    TS-05 & RF-21 \newline
            RF-22 \\
    \hline
    TS-06 & RF-23 \\
    \hline
    TS-07 & RF-24 \\
    \hline
    TS-08 & RF-25 \\
    \hline
    TS-09 & RF-26 \\
    \hline
    TS-10 & RF-27 \\
    \hline
    TS-11 & RF-28 \\
    \hline
    TS-12 & RF-29 \\
    \hline
    TS-13 & RF-30 \\
    \hline
    TS-14 & RF-31 \\
    \hline
    TS-15 & RF-32 \\
    \hline
    TS-16 & RF-33 \\
    \hline
    TS-17 & RF-34 \\
    \hline
    TS-18 & RF-35 \\
    \hline
    TS-19 & RF-36 \\
    \hline
    TS-20 & RF-37 \\
    \hline
    TS-21 & RF-38 \\
    \hline
    TS-22 & RF-39 \\
    \hline
    TS-23 & RF-40 \\
    \hline
    TS-24 & RF-41 \\
    \hline
    TS-25 & RF-42 \\
    \hline
    TS-26 & RF-43 \\
    \hline
    TS-27 & RF-44 \\
    \hline
    TS-28 & RF-45 \\
    \hline
    TS-29 & RF-46 \\
    \hline
    TS-30 & RF-47 \\
    \hline
    TS-31 & RF-48 \\
    \hline
    TS-32 & RF-49 \\
    \hline
    TS-33 & RF-50 \\
    \hline 
    TS-34 & RF-51 \\
    \hline
    TS-35 & RF-52 \\
    \hline
    TS-36 & RF-53 \\
    \hline
    TS-37 & RF-54 \\
    \hline
    TS-38 & RF-55 \\
    \hline
    TS-39 & RF-56 \\
    \hline
    TS-40 & RF-57 \\
    \hline
    TS-41 & RF-58 \newline
            RF-59 \\
    \hline
    TS-42 & RF-58 \newline
            RF-60 \\
    \hline
    TS-43 & RF-61 \\
    \hline
    TS-44 & RF-62 \newline
            RF-63 \\
    \hline
    TS-45 & RF-62 \newline
            RF-64 \\
    \hline
    TS-46 & RF-65 \newline 
            RF-66 \\
    \hline
    TS-47 & RF-65 \newline
            RF-67 \\
    \hline
    TS-48 & RF-65 \newline
            RF-68 \\
    \hline
    TS-49 & RF-65 \newline
            RF-69 \\
    \hline
    TS-50 & RF-65 \newline
            RF-70 \\
    \hline
    TS-51 & RF-65 \newline
            RF-71 \\
    \hline
    TS-52 & RF-65 \newline
            RF-72 \\
    \hline
    TS-53 & RF-65 \newline
            RF-73 \\
    \hline
    TS-54 & RF-65 \newline
            RF-74 \\
    \hline
    TS-55 & RF-65 \newline
            RF-75 \\
    \hline
    TS-56 & RF-65 \newline
            RF-76 \\
    \hline
    TS-57 & RF-65 \newline
            RF-77 \\
    \hline
    TS-58 & RF-4 \newline
            RF-5 \newline
            RF-6 \newline
            RF-7 \newline
            RF-8 \newline
            RF-9 \newline
            RF-10 \newline
            RF-11 \newline
            RF-12 \newline
            RF-13 \newline
            RF-14 \newline
            RF-15 \\
    \hline
    TS-59 & RF-3 \\
    \hline
    TS-60 & RF-78 \\
    \hline
    TS-61 & RF-79 \\
    \hline
\end{longtable}
\clearpage
\subsection{\glossterm{Test} di accettazione}\label{sec:test accettazione}
Questa sezione presenta i test di accettazione del software, condotti da NaN1fy e dalla
\glossterm{Proponente} sotto la supervisione del gruppo. Questi test mirano a validare il prodotto prima del suo rilascio, garantendone la \glossterm{qualità} e l'aderenza ai requisiti.

\renewcommand{\arraystretch}{2.5}
\rowcolors{2}{gray!20}{white}
\begin{longtable}{|>{\centering}p{2cm}|>{\RaggedRight}m{12cm}|>{\centering\arraybackslash}p{2cm}|}
    \hline
    \rowcolor{white}
    \textbf{Codice Test} & \textbf{Descrizione} & \textbf{Stato Test} \\
    \hline
    \endfirsthead 
    \rowcolor{white}
    \caption{Tabella dei \glossterm{test} di accettazione.} 
    \label{table:Tabella dei test di accettazione}
    \endlastfoot  
    TA-01 & Verificare che l'amministratore pubblico possa:
    \begin{enumerate}
        \setlength\itemsep{0em}
        \item Usufruire dell’applicativo solo previa autenticazione;
        \item Visualizzare messaggio di errore qualore le credenziali inserite non siano valide.
    \end{enumerate} & V \\
    \hline
    TA-02 & Verificare che l'amministratore pubblico, una volta entrato nell'applicativo, possa:
    \begin{enumerate}
        \setlength\itemsep{0em}
        \item Aprire il menu di selezione delle \glossterm{dashboard};
        \item Selezionare la \glossterm{dashboard} dei sensori;
        \item Visualizzare la dashboard;
        \item Visualizzare un pannello con una mappa che indichi, mediante icone collocate presso le coordinate di ciascun \glossterm{sensore}, la loro posizione;
        \item Visualizzare un messaggio di avvertenza di dati mancanti, all’interno del pannello, nel caso il \glossterm{sistema} non riesca a reperire i dati.
    \end{enumerate}
    & V \\
    \hline
    TA-03 & Verificare che l'amministratore pubblico, una volta entrato nell'applicativo, possa:
    \begin{enumerate}
        \setlength\itemsep{0em}
        \item Aprire il menu di selezione delle \glossterm{dashboard};
        \item Selezionare la \glossterm{dashboard} relativa ai dati ambientali;
        \item Visualizzare la dashboard;
        \item Visualizzare un pannello contenente un grafico, in formato \glossterm{time series}, che mostri i
            risultati delle rilevazioni delle temperature, espresse in gradi Celsius (°C),
            effettuate dai singoli sensori;
        \item Visualizzare un pannello contenente un grafico, in formato \glossterm{time series}, che mostri i
            risultati delle rilevazioni dell’umidità, espresse in percentuale, effettuate dai
            singoli sensori;
        \item Visualizzare un pannello contenente un grafico, in formato \glossterm{time series}, che mostri i
            risultati del calcolo della temperatura percepita, espresso in gradi Celsius (°C),
            effettuato dalla coppia di sensori temperatura e umidità;
        \item Visualizzare un pannello contenente un grafico di tipo \glossterm{Gauge} che mostri il risultato
            del calcolo della temperatura media, espresso in gradi Celsius (°C), ottenuto tramite la
            media aritmetica basato sulle ultime rilevazioni effettuate dai singoli sensori;
    \end{enumerate}
    & V \\
        TA-03 &
        \begin{enumerate}[start=8]
        \item Visualizzare un pannello contenente un grafico di tipo \glossterm{Gauge} che mostri il risultato
            del calcolo dell'umidità media, espresso in percentuale, ottenuto tramite la
            media aritmetica basato sulle ultime rilevazioni effettuate dai singoli sensori;
        \item Visualizzare un pannello contenente un grafico di tipo \glossterm{Gauge} che mostri il risultato
            del calcolo della temperatura percepita media, espresso in gradi Celsius (°C), ottenuto tramite la
            media aritmetica basato sulle ultime rilevazioni effettuate dalle coppie di sensori temperatura-umidità;
        \item Visualizzare un pannello contenente un grafico a barre che mostri il risultato
            del calcolo dei valori statistici di temperatura, espresso in gradi Celsius (°C), ottenuti dalle rilevazioni effettuate dai singoli sensori;
        \item Visualizzare un pannello contenente un grafico a barre che mostri il risultato
            del calcolo dei valori statistici di temperatura percepita, espresso in gradi Celsius (°C), ottenuto dalle rilevazioni di coppie di sensori temperatura-umidità;
        \item Visualizzare un pannello contenente un grafico, in formato \glossterm{time series}, che mostri i
            risultati delle rilevazioni dell’intensità delle precipitazioni, espresse in millimetri orari (mm/h), effettuate dai
            singoli sensori;
        \item Visualizzare un pannello contenente un grafico, in formato \glossterm{time series}, che mostri i
            risultati delle rilevazioni dell’inquinamento dell'aria, espresse in microgrammi per metro cubo (µg / $\mbox{m}^{\mbox{3}}$), effettuate dai
            singoli sensori;
        \item Visualizzare un pannello contenente un grafico, in formato \glossterm{time series}, che mostri i
            risultati delle rilevazioni del livello dell'acqua, espresse in percentuale, effettuate dai
            singoli sensori;
        \item Visualizzare un pannello contenente un grafico di tipo \glossterm{Gauge} che mostri il risultato
            del calcolo dell'attuale intensità delle precipitazioni, espresso in millimetri orari (mm/h), ottenuta dalle rilevazioni effettuate dai singoli sensori;
        \item Visualizzare un pannello contenente un grafico di tipo \glossterm{Gauge} che mostri il risultato
            del calcolo dell'attuale inquinamento dell'aria, espresso in microgrammi per metro cubo (µg / $\mbox{m}^{\mbox{3}}$), ottenuto dalle rilevazioni effettuate dai singoli sensori;
        \item Visualizzare un pannello contenente un grafico di tipo \glossterm{Gauge} che mostri il risultato
            del calcolo dell'attuale livello dell'acqua, espresso in percentuale, ottenuto dalle rilevazioni effettuate dai singoli sensori;
        \item Visualizzare un messaggio di avvertenza di dati mancanti, all’interno del pannello, nel caso il \glossterm{sistema} non riesca a reperire i dati.
    \end{enumerate}
    & V \\
    \hline
    TA-04 & Verificare che l'amministratore pubblico, una volta entrato nell'applicativo, possa:
    \begin{enumerate}
        \setlength\itemsep{0em}
        \item Aprire il menu di selezione delle \glossterm{dashboard};
        \item Selezionare la \glossterm{dashboard} relativa ai dati urbanistici;
        \item Visualizzare la dashboard;
        \item Visualizzare un pannello \glossterm{geomap} che mostri lo stato di disponibilità aggiornato dei
            parcheggi, espresso tramite valori \glossterm{boolean} rilevati dai singoli sensori;
        \item Visualizzare un pannello in formato tabellare che mostri l'elenco aggiornato delle informazioni dei
            parcheggi, rilevate dai singoli sensori;
        \item Visualizzare un pannello nel formato di un registro che mostri un elenco dettagliato di
            avvisi di pagamento dovuti all'occupazione di un parcheggio, includendo informazioni come la data di pagamento e il costo
            associato a ciascuna rilevazione effettuata dai sensori;
        \item Visualizzare un pannello \glossterm{geomap} che mostri l'elenco aggiornato di possbili guasti alla linea elettrica, rilevati dai singoli sensori;
        \item Visualizzare un pannello contenente un grafico, in formato \glossterm{time series}, che mostri i
            risultati delle rilevazioni dello stato di riempimento delle isole ecologiche, espresse in percentuale, effettuate dai singoli sensori;
        \item Visualizzare un pannello \glossterm{geomap} che mostri lo stato aggiornato di disponibilità delle colonnine di ricarica,
            rilevato dai singoli sensori;
        \item Visualizzare un pannello in formato tabellare che mostri l'elenco aggiornato delle informazioni delle colonnine di ricarica,
            rilevate dai singoli sensori;
        \item Visualizzare un pannello contenente un grafico, in formato \glossterm{time series}, che mostri i
            risultati delle rilevazioni del consumo delle colonnine di ricarica, espresse in kilowattora (kWh), rilevate dai singoli sensori;
        \item Visualizzare un pannello nel formato di un registro che mostri un elenco dettagliato di
            avvisi di pagamento dovuti all'utilizzo di una colonnina di ricarica, includendo informazioni come la data di pagamento e il costo associato a ciascuna rilevazione effettuata dai sensori;
        \item Visualizzare un pannello contenente un grafico a barre che mostri valori statistici riguardanti i pagamenti effettuati in un parcheggio;
        \item Visualizzare un pannello contenente un grafico a barre che mostri valori statistici riguardanti i pagamenti dovuti all'utilizzo di una colonnina di ricarica;
        \item Visualizzare un pannello contenente un grafico di tipo \glossterm{Gauge} che mostri valori l'efficienza monetaria relativa ad un parcheggio;
        \item Visualizzare un messaggio di avvertenza di dati mancanti, all’interno del pannello, nel caso il \glossterm{sistema} non riesca a reperire i dati.
    \end{enumerate}
    & V \\
    \hline
    TA-05 & Verificare che l'amministratore pubblico, una volta entrato nell'applicativo, possa:
    \begin{enumerate}
        \setlength\itemsep{0em}
        \item Aprire il menu di selezione delle \glossterm{dashboard};
        \item Selezionare la \glossterm{dashboard} relativa al superamento delle soglie;
        \item Visualizzare la dashboard;
        \item Visualizzare un pannello che notifica l'eventuale superamento delle soglie di temperatura entro le quali la temperatura è ritenuta accettabile;
        \item Visualizzare un pannello che notifica l'eventuale superamento della soglia d'intensità delle precipitazioni entro la quale l'intensità è ritenuta accettabile;
        \item Visualizzare un pannello che notifica l'eventuale superamento della soglia d'inquinamento dell'aria entro la quale l'inquinamento è ritenuto accettabile;
        \item Visualizzare un pannello che notifica l'eventuale superamento della soglia del livello dell'acqua entro la quale il livello è ritenuto accettabile;
        \item Visualizzare un pannello che notifica l'eventuale superamento della soglia di riempimento di un'isola ecologica entro la quale il riempimento è ritenuto accettabile;
        \item Visualizzare un messaggio di avvertenza di dati mancanti, all’interno del pannello, nel caso il \glossterm{sistema} non riesca a reperire i dati.
    \end{enumerate}
    & V \\
    \hline
    TA-06 & Verificare che l'amministratore pubblico, una volta entrato nell’applicativo, possa:
    \begin{enumerate}
        \item Scegliere una \glossterm{dashboard} da visualizzare;
        \item Applicare dei filtri per visualizzare solo i dati provenienti dal sottoinsieme di sensori selezionato, nel caso di pannelli di tipo \glossterm{time series}; 
        \item Applicare dei filtri per visualizzare solo i dati provenienti dal sottoinsieme di sensori selezionato, nel caso di pannelli di tipo registro;
        \item Applicare dei filtri per visualizzare solo i dati provenienti dal sottoinsieme di sensori selezionato, nel caso di pannelli di tipo table;
        \item Applicare dei filtri per visualizzare solo i dati provenienti dal sottoinsieme di sensori selezionato, nel caso di pannelli di tipo bar chart;  
        \item Applicare dei filtri per selezionare solo i dati relativi ad un definito intervallo di tempo, all’interno di un’intera dashboard;
        \item Rimuovere filtri precedentemente applicati all'interno di un pannello o di una dashboard.
    \end{enumerate}
    & V \\
    \hline
    TA-07 & Verificare che l'amministratore pubblico, una volta entrato
    nell’applicativo, possa:
    \begin{enumerate}
        \item Scegliere una \glossterm{dashboard} di cui modificare il layout;
        \item Modificare il layout dei pannelli in termini di posizione di tali pannelli e della loro dimensione.
    \end{enumerate}
    & V \\
    \hline
    TA-08 &
    Verificare che un \glossterm{sensore}, una volta connesso al \glossterm{sistema}, possa:
    \begin{enumerate}
        \item Inserire il risultato della rilevazione della temperatura, espressa in gradi Celsius
            (°C), con annesso il timestamp di rilevazione e le proprie coordinate geografiche.        
    \end{enumerate}
    & V \\
    \hline
    TA-09 &
    Verificare che un \glossterm{sensore}, una volta connesso al \glossterm{sistema}, possa:
    \begin{enumerate}
    \item Inserire il risultato della rilevazione dell’umidità, espressa in percentuale, con annesso il timestamp di rilevazione e le proprie coordinate geografiche.
    \end{enumerate}
    & V \\
    \hline
    TA-10 &
    Verificare che un \glossterm{sensore}, una volta connesso al \glossterm{sistema}, possa:
    \begin{enumerate}
    \item Inserire il risultato della rilevazione dell'intensità delle precipitazioni, espressa in millimetri orari (mm/h), con annesso il timestamp di rilevazione e le proprie coordinate geografiche.
    \end{enumerate}
    & V \\
    \hline
    TA-11 &
    Verificare che un \glossterm{sensore}, una volta connesso al \glossterm{sistema}, possa:
    \begin{enumerate}
    \item Inserire il risultato della rilevazione dell’inquinamento dell'aria, espressa in microgrammi per metro cubo (µg / $\mbox{m}^{\mbox{3}}$), con annesso il timestamp di rilevazione e le proprie coordinate geografiche.
    \end{enumerate}
    & V \\
    \hline
    TA-12 &
    Verificare che un \glossterm{sensore}, una volta connesso al \glossterm{sistema}, possa:
    \begin{enumerate}
    \item Inserire il risultato della rilevazione del livello dell'acqua, espressa in percentuale, con annesso il timestamp di rilevazione e le proprie coordinate geografiche.
    \end{enumerate}
    & V \\
    \hline
    TA-13 &
    Verificare che un \glossterm{sensore}, una volta connesso al \glossterm{sistema}, possa:
    \begin{enumerate}
    \item  Inserire il risultato della rilevazione della presenza di auto all’interno del
        parcheggio, con annesso il timestamp di rilevazione, la targa dell'auto e le proprie coordinate geografiche. 
    \end{enumerate}
    & V \\
    \hline
    TA-14 &
    Verificare che un \glossterm{sensore}, una volta connesso al \glossterm{sistema}, possa:
    \begin{enumerate}
    \item Inserire il risultato della rilevazione di avvenuto pagamento all’interno del
        parcheggio, con annesso il timestamp di rilevazione, il valore del pagamento e le proprie coordinate geografiche. 
    \end{enumerate}
    & V \\
    \hline
    TA-15 &
    Verificare che un \glossterm{sensore}, una volta connesso al \glossterm{sistema}, possa:
    \begin{enumerate}
    \item Inserire il risultato della rilevazione di un guasto alla linea elettrica, con annesso il timestamp di rilevazione e le proprie coordinate geografiche. 
    \end{enumerate}
    & V \\
    \hline
    TA-16 &
    Verificare che un \glossterm{sensore}, una volta connesso al \glossterm{sistema}, possa:
    \begin{enumerate}
    \item Inserire il risultato della rilevazione dello stato riempimento di un'isola ecologica, espresso in percentuale, con annesso il timestamp di rilevazione e le proprie coordinate geografiche. 
    \end{enumerate}
    & V \\
    \hline
    TA-17 &
    Verificare che un \glossterm{sensore}, una volta connesso al \glossterm{sistema}, possa:
    \begin{enumerate}
    \item Inserire il risultato della rilevazione della presenza di un'auto nella colonnina di ricarica, con annesso il timestamp di rilevazione, la targa dell'auto e le proprie coordinate geografiche.  
    \end{enumerate}
    & V \\
    \hline
    TA-18 &
    Verificare che un \glossterm{sensore}, una volta connesso al \glossterm{sistema}, possa:
    \begin{enumerate}
    \item Inserire il risultato della rilevazione del consumo di una colonnina di ricarica, espresso in kilowattora (kWh) con annesso il timestamp di rilevazione e le proprie coordinate geografiche.  
    \end{enumerate}
    & V \\
    \hline
    TA-19 &
    Verificare che un \glossterm{sensore}, una volta connesso al \glossterm{sistema}, possa:
    \begin{enumerate}
    \item Inserire il risultato della rilevazione di avvenuto pagamento dovuto all'utilizzo di una colonnina di ricarica, con annesso il timestamp di rilevazione, il valore del pagamento e le proprie coordinate geografiche. 
    \end{enumerate}
    & V \\
    \hline
\end{longtable}
\clearpage
\subsubsubsection{Tracciamento dei \glossterm{test} di accettazione}
\renewcommand{\arraystretch}{2.5}
\rowcolors{2}{gray!20}{white}
\begin{longtable}{|>{\centering}p{4cm}|>{\centering\arraybackslash}p{4cm}|}
\hline
\rowcolor{white}
\textbf{Codice Test} & \textbf{Codice caso d'uso} \\
\hline
\endfirsthead
\rowcolor{white}
\caption{Tracciamento dei \glossterm{test} di accettazione.}
\label{table:Tracciamento dei test di accettazione}
\endlastfoot
    TA-01 & UC-0 \newline
            UC-0.1 \newline
            UC-0.2 \newline
            UC-1  
    \\
    \hline
    TA-02 & UC-0 \newline
            UC-0.1 \newline
            UC-0.2 \newline
            UC-2 \newline
            UC-2.1 \newline
            UC-7
    \\
    \hline 
    TA-03 & UC-0 \newline
            UC-0.1 \newline
            UC-0.2 \newline
            UC-4 \newline
            UC-4.1 \newline
            UC-4.2 \newline
            UC-4.3 \newline
            UC-4.4 \newline
            UC-4.5 \newline
            UC-4.6 \newline
            UC-4.7 \newline
            UC-4.8 \newline
            UC-4.9 \newline
            UC-4.10 \newline
            UC-4.11 \newline
            UC-4.12 \newline
            UC-4.13 \newline
            UC-4.14 \newline
            UC-7
    \\
    \hline
    TA-04 & UC-0 \newline
            UC-0.1 \newline
            UC-0.2 \newline
            UC-5 \newline 
            UC-5.1 \newline
            UC-5.2 \newline
            UC-5.3 \newline
            UC-5.4 \newline
            UC-5.5 \newline
            UC-5.6 \newline
            UC-5.7 \newline
            UC-5.8 \newline
            UC-5.9 \newline
            UC-5.10 \newline
            UC-5.11 \newline
            UC-5.12 \newline
            UC-7
    \\
    \hline
    TA-05 & UC-0 \newline
            UC-0.1 \newline
            UC-0.2 \newline
            UC-6 \newline 
            UC-6.1 \newline
            UC-6.2 \newline
            UC-6.3 \newline
            UC-6.4 \newline
            UC-6.5 \newline
            UC-7
    \\
    \hline
    TA-06 & UC-8 \newline
            UC-8.1 \newline
            UC-8.2 \newline
            UC-9
    \\
    \hline
    TA-07 & UC-10 \newline
            UC-10.1 \newline
            UC-10.2
    \\
    \hline
    TA-08 & UC-11 \newline
            UC-11.1
    \\
    \hline
    TA-09 & UC-11 \newline
            UC-11.2
    \\
    \hline
    TA-10 & UC-11 \newline
            UC-11.3
    \\
    \hline
    TA-11 & UC-11 \newline
            UC-11.4
    \\
    \hline
    TA-12 & UC-11 \newline
            UC-11.5
    \\
    \hline
    TA-13 & UC-11 \newline
            UC-11.6
    \\
    \hline
    TA-14 & UC-11 \newline
            UC-11.7
    \\
    \hline
    TA-15 & UC-11 \newline
            UC-11.8
    \\
    \hline
    TA-16 & UC-11 \newline
            UC-11.9
    \\
    \hline
    TA-17 & UC-11 \newline
            UC-11.10
    \\
    \hline
    TA-18 & UC-11 \newline
            UC-11.11
    \\
    \hline
    TA-18 & UC-11 \newline
            UC-11.12
    \\
    \hline
\end{longtable}
\clearpage
\subsection{Liste di controllo}\label{sec:liste controllo}
Le liste di controllo rappresentano un prezioso strumento a disposizione del
Verificatore per l'identificazione di errori ricorrenti nella documentazione o nel codice. Integrando la descrizione del problema, facilitano la comprensione delle modifiche richieste durante la fase di revisione.
\\
Inoltre, il Verificatore ha la possibilità di aggiornare le liste di controllo nel corso del progetto, man mano che emergono nuovi errori ricorrenti.
\subsubsection{Struttura dei documenti}
\renewcommand{\arraystretch}{2.5}
\rowcolors{2}{gray!20}{white}
\begin{longtable}{|>{\centering}p{5cm}|>{\centering\arraybackslash}p{10cm}|}
\hline
\rowcolor{white}
    \textbf{Aspetto} & \textbf{Spiegazione} \\
\hline
\endfirsthead
\rowcolor{white}
\caption{Lista di controllo per la struttura dei documenti.}
\label{table:Lista di controllo per la struttura dei documenti}
\endlastfoot
    Vuoti documentativi & Non devono essere presenti sezioni senza contenuto. \\
\hline
    Didascalia assente & Tutte le tabelle e le immagini devono avere una didascalia descrittiva. \\
\hline    
Titolo principale & Tutti i titoli principali devono iniziare la pagina nella quale vengono
    inseriti. \\
\hline
    Aggiornamento fantasma & Ad ogni insieme di modifiche deve corrispondere una riga nella tabella
    del changelog. \\


    \hline
\end{longtable}
\subsubsection{Errori ortografici, di lingua italiana e di forma}
\renewcommand{\arraystretch}{2.5}
\rowcolors{2}{gray!20}{white}
\begin{longtable}{|>{\centering}p{5cm}|>{\centering\arraybackslash}p{10cm}|}
\hline
\rowcolor{white}
    \textbf{Aspetto} & \textbf{Spiegazione} \\
\hline
\endfirsthead
\rowcolor{white}
\caption{Lista di controllo per gli errori ortografici, di lingua italiana e di forma.}
\label{table:Lista di controllo per gli errori ortografici, di lingua italiana e di forma}
\endlastfoot
    Errori di sintassi & Gli errori di sintassi (battitura o distrazione) devono essere rimossi.\\
\hline
    Errori di coniugazione & Gli errori di coniugazione devono essere rimossi. \\
\hline
    Forma non concisa & Le espressioni troppo verbose, dove possibile, devono essere ridotte.\\
\hline
    Non formalità & Le espressioni non formali devono essere sostituite con le corrispondenti
    espressioni formali. \\
\hline
    Richiamo errato al documento & I richiami ai documenti, devono seguire la seguente forma:
    \textit{NomeDocumento vX.X.X} (e.g. \textit{Piano di Progetto v1.0.0}).\\
\hline
    Acronimi non in maiuscolo & Gli acronimi devono essere completamente in maiuscolo. \\
\hline
\end{longtable}
\subsubsection{Non conformità con le \textit{Norme di Progetto v1.0.0}}
\renewcommand{\arraystretch}{2.5}
\rowcolors{2}{gray!20}{white}
\begin{longtable}{|>{\centering}p{5cm}|>{\centering\arraybackslash}p{10cm}|}
\hline
\rowcolor{white}
    \textbf{Aspetto} & \textbf{Spiegazione} \\
\hline
\endfirsthead
\rowcolor{white}
    \caption{Lista di controllo per le non conformità con le \textit {Norme di Progetto v1.0.0}.}
    \label{table: Lista di controllo per le non conformità con le Norme di Progetto}
\endlastfoot
    Formato date errato & Il formato delle date deve essere aaaa-mm-dd all'interno dei documenti. \\
    \hline
    Punteggiatura scorretta negli elenchi &  Ogni elemento di un elenco, numerato o non, deve terminare con un ``;”, ad eccezione dell'ultima riga, la quale deve terminare con ``.”. \\ 
    \hline
    ``:” non in grassetto negli elenchi & Gli elenchi nella forma ``termine: testo”, devono
    includere ``:” nel grassetto. \\
    \hline
    Maiuscole nei titoli & La prima lettera di ogni titolo deve essere maiuscola. Il resto del
    titolo dovrebbe essere in minuscolo (tolte particolari eccezioni). \\
    \hline
    Ruoli in minuscolo & Tutti i ruoli del progetto devono avere la prima lettera in maiuscolo. \\
    \hline
    Termine non presente nel glossario & Ogni termine segnato con la formattazione da glossario deve essere presente nel glossario. \\
\hline
\end{longtable}
\newpage
\section{Cruscotto della \glossterm{qualità}}\label{sec:cruscotto della qualita}
\subsection{\glossterm{Qualità} di \glossterm{processo} - Fornitura}
\subsubsection{MPC-EAC Estimate at Completion}
\begin{figure}[h!]
    \centering
    \includegraphics[width=1\textwidth]{images_pdq/EAC.png}
    \caption{Proiezione grafica di EAC.}
    \label{fig:Proiezione grafica di EAC}
\end{figure}
\textbf{\glossterm{RTB}:} EAC indica, per ogni periodo, la stima del budget finale prefiguratasi rispetto al valore pianificato definito dal BAC, o Budget at Completion. Questo valore viene calcolato come il rapporto tra BAC e CPI dove CPI, dove CPI sta per Cost Performance Index.\\
Risulta evidente come la metrica in questione non rientrasse all'interno dei valori di accettazione già dal primo \glossterm{Sprint}. Ciò è dovuto alla scarsa esperienza del team nell'affrontare la pianificazione degli Sprint e all'altrettanta poca conoscenza riguardante le attività prioritarie al fine di una buona gestione del progetto. Negli Sprint successivi il team ha preso provvedimenti, adottando con maggiore accortezza una pianificazione efficace e adoperando un metodo lavorativo atto a far rientrare l'EAC all'interno dei valori di accettazione, mantenendo alta, al contempo, la produttività.\\
Le problematiche riscontrate nel sesto periodo hanno avuto un effetto evidente, seppur lieve, sull'andamento della metrica pur rimanendo all'interno dei valori di accettazione.\\
Il gruppo può trarre esperienza dalla proiezione rappresentata nel grafico in previsione della seconda revisione, mantenendo l'andamento già individuato dal quarto Sprint.\\
\textbf{\glossterm{PB}:} Il periodo in previsone della revisione PB è caratterizzato da un mantenimento dell'equilibrio precedentemente individuato dal team. Esso è stato facilitato in parte da Sprint di durata maggiore nella quale i vari obiettivi sono stati distribuiti. La maggiore consapevolezza in fase di pianificazione ha permesso di ipotizzare preventivi realistici rispetto ai periodi precedenti.
\clearpage
\subsubsection{MPC-EV Earned Value e MPC-PV Planned Value}
\begin{figure}[h!]
    \centering
    \includegraphics[width=1\textwidth]{images_pdq/EV_PV.png}
    \caption{Proiezione grafica di EV e PV.}
    \label{fig:Proiezione grafica di EV e PV}
\end{figure}
\textbf{\glossterm{RTB}:} EV rappresenta il valore generato dal prodotto sviluppato fino a un dato momento, mentre PV costituisce il valore del lavoro pianificato fino a un dato momento. Il loro confronto permette di comprendere lo scostamento che avviene tra i preventivi e i consuntivi di lavoro effettuati e quindi ottenere una chiara visione dell'andamento del lavoro svolto dal team.\\
L'andamento delle due metriche è pressoché coincidente, come dimostrato anche nel grafico successivo dalla metrica SV. Dal secondo periodo in poi il valore EV è generalmente inferiore al PV segnalando che il team ha effettuato preventivi ottimistici rispetto all'effettivo risultato prodotto. Ciononostante, questa differenza rimane minimale.\\
\textbf{\glossterm{PB}:} I due valori mantengono l'andamento di pari passo seppur con lieve e costante superamento del valore pianificato rispetto al valore prodotto segno che il gruppo non eccede quanto preventivato in fase di pianificazione.
\clearpage
\subsubsection{MPC-BV Budget Variance e MPC-SV Schedule Variance}
\begin{figure}[h!]
    \centering
    \includegraphics[width=1\textwidth]{images_pdq/BV_SV.png}
    \caption{Proiezione grafica di BV e SV.}
    \label{fig:Proiezione grafica di BV e SV}
\end{figure}
\textbf{\glossterm{RTB}:} BV consiste nella differenza tra il valore guadagnato (EV) e i costi effettivamente sostenuti (AC). Un valore negativo di BV indica che si sta spendendo più di quanto si stia guadagnando. SV, invece, rappresenta la differenza tra il valore guadagnato (EV) e il valore pianificato (PV). Un valore negativo indica un ritardo rispetto alla produzione e consegna del lavoro preventivato.\\
Il valore del BV, nonostante sia certamente all'interno dei valori di accettazione, si è mantenuto negativo per la maggior parte degli Sprint effettuati, diventando positiva solamente all'ultimo. Ciò dimostra come i costi effettivamente sostenuti abbiano ecceduto il lavoro prodotto. I motivi per cui dal quinto periodo la BV è diventata positiva si possono individuare principalmente in due situazioni ben distinte ovvero l'operazione di profonda verifica del materiale prodotto in previsione della revisione RTB e l'incombenza di impegni personali e di studio che hanno rallentato la produttività del gruppo.\\
La SV ha mantenuto valori oscillanti intorno allo 0, indice di una pianificazione abbastanza realistica rispetto al valore effettivamente prodotto. Ancora una volta, le problematiche riscontrate hanno determinato un lieve peggioramento del valore guadagnato rispetto al valore pianificato.\\
In generale, queste due metriche mantengono valori altamente accettabili.\\
\textbf{\glossterm{PB}:} Durante il settimo Sprint la percentuale di SV non è migliorata e ciò è dovuto principalmente al periodo di pausa che il team ha affrontato. Questo ha impedito di produrre più valore rispetto a quanto preventivato portando ad un lieve ritardo nella creazione del prodotto, successivamente recuperato nei periodi seguenti. La BV ha mantenuto l'andamento positivo rimarcando che il gruppo sta producendo più di quanto sta spendendo.
\clearpage
\subsubsection{MPC-AC Actual Cost e MPC-ETC Estimate To Complete}
\begin{figure}[h!]
    \centering
    \includegraphics[width=1\textwidth]{images_pdq/ETC_AC.png}
    \caption{Proiezione grafica di AC e ETC.}
    \label{fig:Proiezione grafica di AC e ETC}
\end{figure}
\textbf{\glossterm{RTB}:} AC rappresenta il costo sostenuto fino a un dato momento, mentre ETC consiste nella conseguente stima dei costi da sostenere per il completamento del progetto.\\
Totalmente in linea con quanto rilevato dalla proiezione delle metriche di EAC, BV e SV, l'AC e l'ETC dimostrano l'avanzamento stimato del progetto. I sei \glossterm{Sprint} rappresentati corrispondono all'arco di tempo definito da 12 settimane, corrispondenti a oltre metà del progetto. Come ci si aspettava, quindi, le due metriche si sono incontrate in corrispondenza del quinto Sprint, definito dal team come il periodo di revisione RTB, nonché decima settimana del progetto.\\
\textbf{\glossterm{PB}:} Queste metriche hanno mantenuto l'andamento corretto evidenziando l'andamento effettivamente riscontrato nei periodi di riferimento.
\clearpage
\subsubsection{MPC-CPI Cost Performance Index}
\begin{figure}[h!]
    \centering
    \includegraphics[width=1\textwidth]{images_pdq/CPI.png}
    \caption{Proiezione grafica di CPI.}
    \label{fig:Proiezione grafica di CPI}
\end{figure}
\textbf{\glossterm{RTB}:} CPI indica il rapporto tra il valore guadagnato (EV) e i costi sostenuti (AC). Un valore di CPI uguale a 1 indica che i costi sostenuti sono in linea con il bugdet pianificato (BAC), mentre un valore negativo comporta il discostamento della stima del budget al completamento (EAC) dal budget effettivo.\\
Coerentemente con quanto indicato dall'EAC, il valore di CPI è rimasto al di fuori dei valori di accettazione fino al quarto \glossterm{Sprint}. Specialmente nei primi periodi, il team ha sostenuto costi eccessivi non solo rispetto al valore pianificato, ma soprattutto rispetto al valore prodotto. Le accortezze adottate negli Sprint successivi hanno avuto gli effetti desiderati, portando il valore di CPI vicino ad 1 e, di conseguenza, ad una diminuzione dell'EAC. Il team si impegna nei prossimi periodi a mantenere l'indice all'interno delle soglie di accettazione. \\
\textbf{\glossterm{PB}:} Il CPI ha raggiunto valori ottimali come dimostrato anche dalla metrica EAC. Ciò è il risultato dell'attenta pianificazione che il team ha svolto nei periodi successivi al conseguimento della revisione RTB.
\clearpage
\subsubsection{MPC-VAC Variance At Completion}
\begin{figure}[h!]
    \centering
    \includegraphics[width=1\textwidth]{images_pdq/VAC.png}
    \caption{Proiezione grafica di VAC.}
    \label{fig:Proiezione grafica di VAC}
\end{figure}
\textbf{\glossterm{RTB}:} VAC indica la variazione relativa del budget pianificato, ovvero il BAC, rispetto al budget stimato dall'EAC. L'intervallo di accettazione di questa metrica è posto ad una differenza in percentuale del 10\% rispetto al BAC.\\
Coerentemente con quanto indicato dall'EAC, il valore di VAC è risultato maggiore del -10\%, indicando un EAC eccessivamente superiore rispetto al BAC pianificato. Le accortezze adottate negli \glossterm{Sprint} successivi hanno avuto gli effetti desiderati, portando il valore di VAC all'interno dell'intervallo di accettazione dal terzo Sprint.\\
\textbf{\glossterm{PB}:} Sempre in linea con le metriche EAC e CPI, anche il VAC raggiunge valori ottimali.
\subsection{\glossterm{Qualità} di \glossterm{processo} - Sviluppo}
\subsubsection{MPC-RSI Requirements Stability Index}
\renewcommand{\arraystretch}{2.5}
\rowcolors{2}{gray!20}{white}
\begin{longtable}{|>{\centering}p{3cm}|>{\centering\arraybackslash}p{3cm}|}
    \hline
    \rowcolor{white}
    \textbf{Requisiti iniziali} & \textbf{Requisiti Finali} \\
    \hline
    \endfirsthead 
    \rowcolor{white}
    \caption{Tabella del numero di requisiti del progetto.} 
    \label{table:Tabella del numero di requisiti del progetto}
    \endlastfoot 
    89 & 99 \\
    \hline
\end{longtable}
\textbf{\glossterm{PB}:} RSI rappresenta la facilità con cui i requisiti subiscono modifiche nell'avanzamento del progetto. Questa viene calcolata come rapporto tra i requisiti inizialmente definiti e i requisiti presenti in un determianto momento. Questo valore rientra nella soglia di accettazione ma nonarriva al valore ottimale. Questo poiché varie problematiche sorte e richieste della \glossterm{Proponente} hanno portato all'aumento dei requisiti richiesti da un valore inziale di 89 a un valore finale di 99.
\clearpage
\subsection{\glossterm{Qualità} di \glossterm{processo} - Documentazione}
\subsubsection{Indice Gulpease}
\begin{figure}[h!]
    \centering
    \includegraphics[width=1\textwidth]{images_pdq/IG.png}
    \caption{Proiezione grafica dell'indice gulpease.}
    \label{fig:Proiezione grafica dell'indice gulpease}
\end{figure}
\textbf{\glossterm{RTB}:} L'indice Gulpease è un'indice di leggibilità per testi in lingua italiana ed è indicato da una percetuale di leggibilità, con soglia di accettazione al 60\%. Tutti i documenti prodotti dal team hanno prodotto un indice di leggibilità alto e nella maggior parte anche ottimale. L'unica eccezione si individua nel \textit{Glossario}, che fino al terzo \glossterm{Sprint} ha avuto valori inferiori alla soglia di accettazione. La necessità del team di concentrarsi sulla revisione di altri documenti ha impedito il miglioramento dell'indice in tale documento. Sarà di fondamentale importanza per il team migliorarlo al più presto.\\
\textbf{\glossterm{PB}:} Dal settimo Sprint si riscontra un lieve miglioramento del \textit{Glossario}. Date le poche modifiche effettuate negli Sprint successivi alla RTB, molti non hanno variato particolarmente il valore dell'indice fino all'ottavo periodo. Sono stati introdotti due nuovi documenti con i rispettivi valori che nei primi periodi di PB risultano sufficienti ma non ottimali, data la loro mancata completezza.
\clearpage
\subsubsection{Correttezza Ortografica}
\begin{figure}[h!]
    \centering
    \includegraphics[width=1\textwidth]{images_pdq/CO.png}
    \caption{Proiezione grafica della correttezza ortografica.}
    \label{fig:Proiezione grafica della correttezza ortografica}
\end{figure}
\textbf{\glossterm{RTB}:} Il team ha costantemente effettuato un controllo approfondito della correttezza ortografica dei documenti prodotti. I redattori hanno il compito di assicurarsene prima ancora della verifica del documento da parte di membri terzi. Successivamente, in vista delle revisioni, questi vengono ricontrollati e approvati. In questo modo ogni documento viene sempre visionato e controllato da ogni membro del gruppo, ottenendo come risultato un esiguo numero di errori e, di conseguenza, l'ottima proiezione rappresentata.\\
\textbf{\glossterm{PB}:} Conseguentemente alle poche modifiche apportate ai documenti fino all'ottavo Sprint, non si riscontrano particolari errori nella correttezza ortografica dei documenti. I nuovi documenti sono particolarmente sottoposti ad un controllo accurato.
\clearpage
\subsection{\glossterm{Qualità} di \glossterm{processo} - Verifica}
\subsubsection{MPC-CC Code Coverage}
\begin{figure}[h!]
    \centering
    \includegraphics[width=1\textwidth]{images_pdq/CC.png}
    \caption{Proiezione grafica del code coverage.}
    \label{fig:Proiezione grafica del code coverage}
\end{figure}
\textbf{\glossterm{PB}:} Il Code Coverage consiste nella percentuale di codice eseguito durante l'esecuzione dei test. L'andamento del CC è sempre rimasto sopra la soglia di accettazione con un lieve calo nello Sprint 7, probabilmente dovuto all'opera di sistemazione di varie problematiche insorte. Nonostante questo la rifinitura del codice dei sensori ha portato ad aumentare questa percentuale.
\clearpage
\subsection{\glossterm{Qualità} di \glossterm{processo} - Gestione della qualità}
\subsubsection{MPC-QMS Quality Metrics Satisfied}
\begin{figure}[h!]
    \centering
    \includegraphics[width=1\textwidth]{images_pdq/QMS.png}
    \caption{Proiezione grafica di QMS.}
    \label{fig:Proiezione grafica di QMS}
\end{figure}
\textbf{\glossterm{RTB}:} In linea con quanto già specificato, le metriche di EAC, CPI, VAC e IG non sono state soddisfatte fin da subito. Il VAC e l'indice Gulpease, in particolare per il \textit{Glossario}, sono rientrati nei valori accettabili solamente dal terzo \glossterm{Sprint}, mentre l'EAC e il CPI dal quarto. Nonostante le problematiche riscontrate nel quarto e sesto Sprint tutte le metriche rientrano nei valori di accettazione. \\Con ottica di retrospettiva, il team nota come è stata necessaria l'acquisizione di esperienza e accortezza nella pianificazione e organizzazione delle attività del gruppo, cause primarie della mancata tollerabilità delle metriche citate. Questo non deve distogliere il team dal mantenere l'andamento individuato, affiché tutte le metriche vengano rispettate, il prodotto venga sviluppato senza intoppi e il progetto possa essere portato a termine senza imprevisti.\\
\textbf{\glossterm{PB}:}  Tutte le altre metriche mantengono un andamento ottimale.
\clearpage
\subsection{\glossterm{Qualità} di \glossterm{processo} - Gestione dei processi}
\subsubsection{MPC-NR Non-calculated Risk}
\begin{figure}[h!]
    \centering
    \includegraphics[width=1\textwidth]{images_pdq/NR.png}
    \caption{Proiezione grafica dei rischi non calcolati.}
    \label{fig:Proiezione grafica dei rischi non calcolati}
\end{figure}
\textbf{\glossterm{RTB}:} Come mostrato dalla proiezione, il quarto \glossterm{Sprint} è stato soggetto ad un rischio non previsto particolarmente limitante. Su suggerimento della \glossterm{Proponente}, il team ha affrontato un profondo lavoro di \glossterm{refactoring} del codice prodotto fino a quel momento, che non solo ha comportato un certo dispendio di risorse destinate ad altre attività, ma ha anche limitato e modificato l'organizzazione che il team aveva precedentemente pianificato per il suddetto periodo. Ciononostante, grazie al lavoro svolto negli Sprint antecedenti e soprattutto alle accortezze adottate nelle fasi di pianificazione, il refactoring non ha comportato problematiche rispetto al preventivo e alle metriche di fornitura.\\
Un ulteriore rischio non previsto è stato riscontrato nel sesto Sprint. Gli impegni personali e universitari si sono intensificati a tal punto da portare il team alla scelta di intraprendere un periodo di assestamento atto a mitigare il rischio occorso. Lo scopo è, quindi, quello di riprendere al più presto con il normale regime di avanzamento e produttività.\\
\textbf{\glossterm{PB}:} Conseguentemente alla revisione RTB il team ha affrontato un periodo di pausa dovuto alle esigenze di studio. Ciò ha chiaramente comportato un rischio. Tale rischio era stato individuato ma non era stato considerato l'impatto che questo avrebbe avuto nello svolgimento del progetto. Per questo motivo viene definito non previsto. Dall'ottavo periodo in poi il problema è stato mitigato grazie alla collaborazione di tutto il team.
\clearpage
\subsection{\glossterm{Qualità} di \glossterm{prodotto} - Funzionalità}
\subsubsection{MPD-ROS Requisiti Obbligatori Soddisfatti}
\begin{figure}[h!]
    \centering
    \includegraphics[width=1\textwidth]{images_pdq/ROS.png}
    \caption{Proiezione grafica dei requisiti obbligatori soddisfatti.}
    \label{fig:Proiezione grafica dei requisiti obbligatori soddisfatti}
\end{figure}
\textbf{\glossterm{PB}:} Durante lo svolgimento del progetto si è cercato di mantenere costante il grado di soddisfazione dei requisiti obbligatori. Durante la presentazione del \glossterm{PoC} avvenuta in fase di \glossterm{RTB} il team aveva già integrato più della metà dei requisiti; inoltre, sono stati completamente soddisfatti durante lo Sprint 9, in relazione all'incontro di collaudo con la Proponente.
\clearpage
\subsubsection{MPD-RDS Requisiti Desiderabili Soddisfatti}
\begin{figure}[h!]
    \centering
    \includegraphics[width=1\textwidth]{images_pdq/RDS.png}
    \caption{Proiezione grafica dei requisiti desiderabili soddisfatti.}
    \label{fig:Proiezione grafica dei desiderabili obbligatori soddisfatti}
\end{figure}
\textbf{\glossterm{PB}:} L'integrazione dei requisiti desiderabili è avvenuta in parallelo rispetto ai requisiti obbligatori. Nonostante il team, per ovvi motivi, si sia dedicato più tardi su di essi, li ha comunque soddisfatti nel giro degli ultimi Sprint.
\clearpage
\subsubsection{MPD-ROPS Requisiti Opzionali Soddisfatti}
\begin{figure}[h!]
    \centering
    \includegraphics[width=1\textwidth]{images_pdq/ROPS.png}
    \caption{Proiezione grafica dei requisiti opzionali soddisfatti.}
    \label{fig:Proiezione grafica dei requisiti opzionali soddisfatti}
\end{figure}
\textbf{\glossterm{PB}:} A differenza dei requisiti precedenti, il team non è riuscito a soddisfare i requisiti opzionali. Ciò è stato dovuto ovviamente alla priorità minore di integrazione e dal desiderio della stessa Proponente, la quale ha preferito l'integrazione e la concretizzazione di altri requisiti.
\clearpage
\subsection{\glossterm{Qualità} di \glossterm{prodotto} - Affidabilità}
\subsubsection{MPD-BC \glossterm{Branch} Coverage}
\begin{figure}[h!]
    \centering
    \includegraphics[width=1\textwidth]{images_pdq/BC.png}
    \caption{Proiezione grafica della percentuale di branch coverage.}
    \label{fig:Proiezione grafica della percentuale di branch coverage}
\end{figure}
\textbf{\glossterm{PB}:} Il BC rappresenta la percentuale di percorsi eseguiti nella totalità dei percorsi possibili nel codice. Indica quanto il codice esplori le ramificazioni. Nonostante i test per i primi sensori implementati fossero pronti già in fase di \glossterm{RTB} è stato valutato immaturo iniziare i test relativi ad un \glossterm{PoC}. In fase iniziale del \glossterm{PB} i test sono iniziati dando esiti leggermente al di sotto delle aspettative, mostrando possibili ramificazioni condizionali mancanti. \\ Il team, quindi, ha dedicato maggiore attenzione sulla copertura del codice sorgente da test manuali e automatici. \\ Nonostante non sia riuscito ad ottenere il risultato ottimale desiderato, il team è comunque riuscito a ottenere un risultato accettabile, mantenendosi poco sotto il 90\% di copertura alla conclusione del progetto.
\clearpage
\subsubsection{MPD-SC \glossterm{Statement} Coverage}
\begin{figure}[h!]
    \centering
    \includegraphics[width=1\textwidth]{images_pdq/SC.png}
    \caption{Proiezione grafica della percentuale di statement coverage.}
    \label{fig:Proiezione grafica della percentuale di statement coverage}
\end{figure}
\textbf{\glossterm{PB}:} SC indica la percentuale di istruzioni eseguite sulla totalità delle istruzioni presenti nel codice. Come precedentemente scritto, è stato valutato immaturo iniziare i test relativi ad un \glossterm{PoC}. In fase iniziale del \glossterm{PB} i test sono iniziati mostrando una copertura delle istruzioni del codice accettabile. \\ Il team, quindi, ha dedicato il resto degli Sprint cercando di migliorare questa percentuale, in relazione con l'ampliazione del codice scritto. \\ Nonostante non sia riuscito ad ottenere il risultato ottimale desiderato, il team è comunque riuscito a ottenere un risultato accettabile, mantenendosi sopra il 90\% di copertura alla conclusione del progetto.
\clearpage
\subsubsection{MPD-FD Failure Density}
\begin{figure}[h!]
    \centering
    \includegraphics[width=1\textwidth]{images_pdq/FD.png}
    \caption{Proiezione grafica della percentuale di failure density.}
    \label{fig:Proiezione grafica della percentuale di failure density}
\end{figure}
\textbf{\glossterm{PB}:} FD rappresenta il rapporto tra il numero di guasti in un determinato momento rispetto al numero di guasti presenti dopo aver aggiunto parti di codice. Deve quindi essere il meno possibile. Nella fase iniziale, il team aveva precedentemente sistemato i guasti di implementazione delle tecnologie e della scrittura del generatore di sensori. Tuttavia, durante la fase di implementazione dei nuovi sensori (livello dell'acqua, i guasti elettrici, eccetera) e del sistema di \glossterm{stream processing}, il team ha riscontrato seri \glossterm{bug} implementativi non previsti, i quali hanno comportato un aumento della metrica al di sopra della soglia massima accettabile.\\ In previsione della riunione di collaudo con la Proponente, il team si è impegnato alla risoluzione dei suddetti problemi, mantenendo poi una percentuale di densità dei guasti buona ma non ottimale come desiderato.
\clearpage
\subsubsection{MPD-PTCP Passed Test Cases Percentage}
\begin{figure}[h!]
    \centering
    \includegraphics[width=1\textwidth]{images_pdq/PTCP.png}
    \caption{Proiezione grafica della percentuale di test passati.}
    \label{fig:Proiezione grafica della percentuale di test passati}
\end{figure}
\textbf{\glossterm{PB}:} PTCP rappresenta la percentuale di test superati. Come precedentemente enunciato, i test integrati prematuramente sono stati utilizzati in queste fasi iniziali, con una percentuale massima di successo. Il problema di implementazione dei nuovi sensori ha comportato la visualizzazione di test falliti (2/30), risolti nello Sprint successivo.\\Le migliorie apportate sul prodotto si sono comunque poco riflettute nei test, e il team ha ripristinato indi la percentuale massima di test passati.
\clearpage
\subsection{\glossterm{Qualità} di \glossterm{prodotto} - Usabilità}
\subsubsection{MPD-FU Facilità di Utilizzo}
\begin{figure}[h!]
    \centering
    \includegraphics[width=1\textwidth]{images_pdq/FU.png}
    \caption{Proiezione grafica della facilità di utilizzo (numero medio di click).}
    \label{fig:Proiezione grafica della facilità di utilizzo (numero medio di click)}
\end{figure}
\textbf{\glossterm{PB}:} La facilità di utilizzo è definita dal numero di click medi per raggiungere ogni funzionalità. Durante lo svolgimento degli Sprint il numero medio di click per raggiungere uno specifico panel, sono sempre rimasti 4: 2 per la sezione di login (compresa una parte di cambio password), 1 di scelta della \glossterm{dashboard} e 1 di selezione del panello preciso
\clearpage
\subsubsection{MPD-TA Tempo di Apprendimento}
\begin{figure}[h!]
    \centering
    \includegraphics[width=1\textwidth]{images_pdq/TA.png}
    \caption{Proiezione grafica del tempo di apprendimento medio del prodotto.}
    \label{fig:Proiezione grafica del tempo di apprendimento medio del prodotto}
\end{figure}
\textbf{\glossterm{PB}:} Il TA si misura in minuti necessari all'apprendimento dell'utilizzo del prodotto. Così come il numero medio di click, anche il tempo di apprendimento è rimasto generalmente costante. Nonostante il team si aspettasse un tempo di apprendimento inferiore, l'interfaccia grafica integrata di Grafana è risultata leggermente più fuorviante del previsto.
\clearpage
\subsection{\glossterm{Qualità} di \glossterm{prodotto} - Efficienza}
\subsubsection{MPD-UR Utilizzo Risorse}
\renewcommand{\arraystretch}{2.5}
\rowcolors{2}{gray!20}{white}
\begin{longtable}{|>{\centering}p{3.5cm}|>{\centering\arraybackslash}p{3cm}|>{\centering\arraybackslash}p{3.5cm}|>{\centering\arraybackslash}p{3cm}|}
    \hline
    \rowcolor{white}
    \textbf{Metrica - Risorsa} & \textbf{Valore Medio} & \textbf{Valore Accettabile} & \textbf{Valore Ottimo} \\
    \hline
    \endfirsthead 
    \rowcolor{white}
    \caption{Tabella dei valori medi dell'utilizzo di risorse} 
    \label{table:Tabella dei valori medi dell'utilizzo di risorse}
    \endlastfoot 
    MPD-CPUU & $\sim$ 2,7\% & $\leq$ 20\% & $\leq$ 5\% \\
    \hline 
    MPD-RAMU & $\sim$ 196.7MB & $\leq$ 500MB & $\leq$ 200MB \\
    \hline 
    MPD-TDE & $\sim$ 4.5s & $\leq$ 6s & $\leq$ 3s \\
    \hline
\end{longtable}
\textbf{\glossterm{PB}:} UR definisce la percentuale di risorse occupate nel utilizzare il prodotto. Le metriche sono state calcolate utilizzando un elaboratore con 16GB RAM DDR4-3200 e processore AMD Ryzen 5 5600U 4,2GHz.\\ Grazie all'utilizzo di \glossterm{docker} per l'implementazione del sistema, si sono riusciti ad ottenere delle prestazioni ottime; il numero delle immagini ha tuttavia incrementato il tempo medio di avvio del sistema.
\subsection{\glossterm{Qualità} di \glossterm{prodotto} - Manutenibilità}
\subsubsection{MPD-CC Complessità Ciclomatica}
\renewcommand{\arraystretch}{2.5}
\rowcolors{2}{gray!20}{white}
\begin{longtable}{|>{\centering}p{3cm}|>{\centering\arraybackslash}p{3cm}|>{\centering\arraybackslash}p{3.5cm}|>{\centering\arraybackslash}p{3cm}|}
    \hline
    \rowcolor{white}
    \textbf{Metrica} & \textbf{Valore Medio} & \textbf{Valore Accettabile} & \textbf{Valore Ottimo} \\
    \hline
    \endfirsthead 
    \rowcolor{white}
    \caption{Tabella del valore medio di complessità ciclomatica.} 
    \label{table:Tabella del valore medio di complessità ciclomatica}
    \endlastfoot 
    MPD-CC & $\sim$ 3.4 & 11-20 & 1-10 \\
    \hline
\end{longtable}
\textbf{\glossterm{PB}:} La CC è costituita dal numero di cammini linearmente indipendenti percorribili nel codice e rappresenta quindi la complessità del codice. Deve di conseguenza essere minore possibile. Di 39 file esaminati, 36 sono stati valutati con un classe "A" (con livello di complessità 1-5), mentre 3 con classe "B" (con livello di complessità 6-10). Facendo dunque una media dei valori possibili delle due classi, si è valutato che l'intero progetto sia di classe "A".\\ Ciò vuol dire non solo che la complessità è ben al di sotto del valore accettabile, ma anche del valore ottimale, rendendo il codice di facile comprensione.
% \subsubsection{MPC-CS Code Smell}
% \renewcommand{\arraystretch}{2.5}
% \rowcolors{2}{gray!20}{white}
% \begin{longtable}{|>{\centering}p{3cm}|>{\centering\arraybackslash}p{3cm}|>{\centering\arraybackslash}p{3.5cm}|>{\centering\arraybackslash}p{3cm}|}
%     \hline
%     \rowcolor{white}
%     \textbf{Metrica} & \textbf{Valore Medio} & \textbf{Valore Accettabile} & \textbf{Valore Ottimo} \\
%     \hline
%     \endfirsthead 
%     \rowcolor{white}
%     \caption{Tabella del valore di punteggio di code smell.} 
%     \label{table:Tabella del valore di punteggio di code smell}
%     \endlastfoot 
%     MPD-CS & $\sim$ 93.7\% & 100\% & 100\% \\
%     \hline
% \end{longtable}
% \textbf{\glossterm{PB}:} Percentuale di codice non esposto a possibili debolezze. Esaminando il codice è risultato una percentuale di rating intorno al 93.7\%, con imperfezioni dovute prevalentemente a problemi di cattiva identazione o per stringhe di documentazioni mancanti. Ciò ha impedito di ottenere un risultato dal team considerato sia accettabile che ottimale.
% \clearpage
\end{document}
