% changelog: "0.1.0, 2024-07-25, Davide Donanzan, Continuazione stesura"

\documentclass[8pt]{article}
\usepackage[italian]{babel}
\usepackage[utf8]{inputenc}
\usepackage[letterpaper, left=1in, right=1in, bottom=0.75in, top=0.75in]{geometry}
\usepackage{amsmath}
\usepackage{subfiles}
\usepackage{lipsum}
\usepackage{csquotes}
\usepackage{amsfonts}
\usepackage[sfdefault]{plex-sans}
\usepackage{float}
\usepackage{pifont}
\usepackage{mathabx}
\usepackage[euler]{textgreek}
\usepackage{makecell}
\usepackage{tikz}
\usepackage{wrapfig}
\usepackage{siunitx}
\usepackage{amssymb} 
\usepackage{tabularx}
\usepackage{adjustbox}
\usepackage[document]{ragged2e}
\usepackage{floatflt}
\usepackage[hidelinks]{hyperref}
\usepackage{graphicx}
\usepackage{hyperref}
\setcounter{tocdepth}{4}
\usepackage{caption}
\usepackage{multicol}
\usepackage{tikz}
\setlength\parindent{0pt}
\captionsetup{font=footnotesize}
\usepackage{fancyhdr} 
\usepackage{graphicx}
\usepackage{capt-of}% 
\usepackage{booktabs}
\usepackage{varwidth}
\usepackage{datetime2}
\usepackage{xcolor}
\usepackage{longtable}
\usepackage{array}
\usepackage{ragged2e}
\usepackage{colortbl}
\usepackage{verbatim}
\usepackage{enumitem}

\newcommand{\customtitle}{SPECIFICA TECNICA}% o ESTERNO

% -- STILE COLONNA CENTRATA PER TABELLE -- %
\newcolumntype{M}[1]{>{\centering\arraybackslash}m{#1}}

% -- STILE INTESTAZIONE -- %
\fancypagestyle{mystyle}{
	\fancyhf{} 
	\fancyhead[R]{\includegraphics[height=1cm]{../../template/images/logos/NaN1fy_logo.png}} 
	\fancyhead[L]{\leftmark} 
	\renewcommand{\headrulewidth}{1pt} 
	\fancyhead[L]{\customtitle} 
	\renewcommand{\headsep}{1.3cm} 
	\fancyfoot[C]{\thepage} 
}

% -- PER LA FIRMA -- %
\newcommand{\signatureline}[1]{%
	 \par\vspace{0.5cm}
	\noindent\makebox[\linewidth][r]{\rule{0.2\textwidth}{0.5pt}\hspace{3cm}\makebox[0pt][r]{\vspace{3pt}\footnotesize #1}}%
}

% -- PER IL GLOSSARIO -- %
\newcommand{\glossterm}[1]{#1\textsuperscript{G}} % inserisci \glossterm{termine}

% -- per abilitare 4x sottosezioni es 2.1.1.1
\setcounter{secnumdepth}{4}
\newcommand{\subsubsubsection}[1]{\paragraph{#1}\mbox{}\\\\}

\begin{document}
\definecolor{myblue}{RGB}{23,103,162}
\begin{titlepage}
	\begin{tikzpicture}[remember picture, overlay]
		\node[anchor=south east, opacity=0.2, yshift = -4cm, xshift= 2em] at (current page.south east) {\includegraphics[width=0.7\textwidth, trim=0cm 0cm 5cm 0cm, clip]{../../template/images/logos/Universita_Padova_transparent.png}}; 
		\node[anchor=north west, opacity=1, yshift = 4.2cm, xshift= 1.4cm, scale=1.6] at (current page.south west) {\includegraphics[width=4cm]{../../template/images/logos/NaN1fy_logo.png}};
	\end{tikzpicture}
	
	\begin{minipage}[t]{0.47\textwidth}
		{\large{\textsc{Destinatari}}
			\vspace{3mm}
			\\ \large{\textsc{Prof. Tullio Vardanega}}
			\\ \large{\textsc{Prof. Riccardo Cardin}}
		}
	\end{minipage}
	\hfill
	\begin{minipage}[t]{0.47\textwidth}\raggedleft
		{\large{\textsc{Redattori}}
			\vspace{3mm}
			{\\\large{\textsc{Davide Donanzan}\\}} % massimo due 
			{\large{\textsc{XXXX XXXX}}}
			
			
		}
		\vspace{8mm}
		
		{\large{\textsc{Verificatori}}
			\vspace{3mm}
			{\\\large{\textsc{XXXX XXXX}\\}} 
			{\large{\textsc{XXXX XXXX}\\}} 
			{\large{\textsc{XXXX XXXX}\\}}
			
		}
		\vspace{2mm}\vspace{2mm}
	\end{minipage}
	\vspace{4cm}
	\begin{center}
		\begin{flushright}
			{\fontsize{30pt}{52pt}\selectfont \textbf{Specifica Tecnica}} % o ESTERNO
		\end{flushright}
		\vspace{3cm}
	\end{center}
	\vspace{10 cm}
	{\small \textsc{\href{mailto: nan1fyteam.unipd@gmail.com}{nan1fyteam.unipd@gmail.com}}}
\end{titlepage}
\pagestyle{mystyle}
\section*{Registro delle Modifiche}
\begin{table}[ht!]	
	\centering
	\begin{tabular}{p{1.2cm} p{2cm} p{5cm} p{3cm} p{3cm}}
		\toprule
		\textbf{Versione}& \textbf{Data} & \textbf{Descrizione} & \textbf{Redattore} & \textbf{Verificatore} \\
		\midrule
		    0.1.0 & 2024-07-25 & Stesura sezione \ref{sec:tec}. & Davide Donanzan & Guglielmo Barison \\\\
    		0.0.0 & 2024-07-01 & Stesura struttura e sezione \ref{sec:intro}. & Davide Donanzan & Guglielmo Barison \\
		\bottomrule
		% Ruolo Redattore o Verificatore
	\end{tabular}
	\caption*{Tabella: Registro delle modifiche.}
	\label{table:Registro delle modifiche}
\end{table}
\newpage
\tableofcontents
\newpage
\listoffigures
\newpage
\listoftables
\newpage
\justifying
\section{Introduzione}\label{sec:intro}
\subsection{Scopo del documento}
Il presente documento si propone come risorsa esaustiva per la comprensione degli aspetti
tecnici chiave del progetto SyncCity. La sua finalità primaria è fornire una descrizione
dettagliata e approfondita dell’\glossterm{architettura} implementativa del \glossterm{sistema}. 
In particolare, si prevede un’analisi approfondita che si estenda anche al livello di design più
basso, includendo definizione e spiegazione dettagliata dei \glossterm{design pattern} utilizzati.\\
L’obiettivo principale del presente documento è triplice:
motivare le scelte di sviluppo adottate; fungere da guida fondamentale per l’attività di codifica e di manutenzione; monitorare la copertura dei requisiti identificati nel
documento \textit{Analisi dei Requisiti}. \\
L’adeguatezza del documento e dell’architettura viene costantemente monitorata e modificata sulla base dei
requisiti e dei feedback ricevuti da parte della \glossterm{Proponente}.
\subsection{Scopo del prodotto}
L'obiettivo del progetto SyncCity è quello di creare una piattaforma atta al monitoraggio
di sensori sparsi geograficamente nel territorio di una città. I sensori in questione
permettono la misurazione e segnalazione di dati \glossterm{real-time} riguardanti le più disparate
caratteristiche e necessità del territorio quali temperatura ed umidità esterna, occupazione di
stalli di parcheggio, funzionamento o guasto elettrico di colonnine di ricarica, traffico stradale e via
dicendo. La Proponente richiede la simulazione di alcuni dei sensori nominati nonchè la
gestione dei dati, della loro persistenza e della loro rappresentazione grafica attraverso \glossterm{widget} e
grafici. 
\\\\SyncCity permetterà un miglioramento della \glossterm{qualità} dei servizi offerti dalla città attraverso il continuo monitoraggio della stessa, ottenendo, gestendo e successivamente condividendo i dati con gli utenti. 
\subsection{Glossario}
Per garantire chiarezza nel linguaggio utilizzato nei documenti, è stato redatto un Glossario contenente le definizioni dei termini con significato specifico da disambiguare. Tali termini sono contrassegnati con una G ad apice. L'inserimento di un termine nel Glossario è considerato completo solo dopo averne fornito la definizione.
\subsection{Riferimenti}
\subsubsection{Normativi}
\begin{itemize}
	\item \textit{Norme di Progetto v2.0.0};
	\item Presentazione e documentazione del \glossterm{capitolato} d’appalto C6 - SyncCity:
	\begin{itemize}
		\item \href{https://www.math.unipd.it/~tullio/IS-1/2023/Progetto/C6p.pdf}{\color{myblue}https://www.math.unipd.it\textasciitilde{}tullio/IS-1/2023/Progetto/C6p.pdf} (Ultimo accesso: \today)
		\item \href{https://www.math.unipd.it/~tullio/IS-1/2023/Progetto/C6.pdf}{\color{myblue}https://www.math.unipd.it/\textasciitilde{}tullio/IS-1/2023/Progetto/C6.pdf} (Ultimo accesso: \today)
	\end{itemize}
	\item Regolamento di progetto:
	\begin{itemize}
		\item \href{https://www.math.unipd.it/~tullio/IS-1/2023/Dispense/PD2.pdf}{\color{myblue}https://www.math.unipd.it/\textasciitilde{}tullio/IS-1/2023/Dispense/PD2.pdf} (Ultimo accesso: \today)
	\end{itemize}
\end{itemize}
\clearpage
\subsubsection{Informativi}
\begin{itemize}
    \item \textit{Analisi dei Requisiti v2.0.0};
    \item \textit{Glossario v2.0.0};
    \item Diagrammi delle classi (UML) - corso di Ingegneria del Software a.a. 2023/2024:
    \begin{itemize}
        \item \href{https://www.math.unipd.it/~rcardin/swea/2023/Diagrammi%20delle%20Classi.pdf}{\color{myblue}https://www.math.unipd.it/\textasciitilde{}rcardin/swea/2023/Diagrammi\%20delle\%20Classi.pdf} (Ultimo accesso: \today)
    \end{itemize}
    \item Diagrammi di Sequenza (UML) - corso di Ingegneria del Software a.a. 2023/2024:
    \begin{itemize}
        \item \href{https://www.math.unipd.it/~rcardin/swea/2022/Diagrammi%20di%20Sequenza.pdf}{\color{myblue}https://www.math.unipd.it/\textasciitilde{}rcardin/swea/2022/Diagrammi\%20di\%20Sequenza.pdf} (Ultimo accesso: \today)
    \end{itemize}
    \item Progettazione - corso di Ingegneria del Software a.a. 2023/2024:
    \begin{itemize}
        \item \href{https://www.math.unipd.it/~tullio/IS-1/2023/Dispense/T6.pdf}{\color{myblue}https://www.math.unipd.it/\textasciitilde{}tullio/IS-1/2023/Dispense/T6.pdf} (Ultimo accesso: \today)
    \end{itemize}
\end{itemize}
\subsubsection{Tecnici}
\begin{itemize}
    \item Documentazione \glossterm{Docker}:
    \begin{itemize}
		\item \href{https://docs.docker.com/}{\color{myblue}https://docs.docker.com/} (Ultimo accesso: \today)
	\end{itemize}
    \item Documentazione \glossterm{Docker Compose}:
    \begin{itemize}
		\item \href{https://docs.docker.com/compose/}{\color{myblue}https://docs.docker.com/compose/} (Ultimo accesso: \today)
	\end{itemize}
    \item Documentazione \glossterm{Python}:
    \begin{itemize}
		\item \href{https://docs.python.org/3/}{\color{myblue}https://docs.python.org/3/} (Ultimo accesso: \today)
	\end{itemize}
    \item Documentazione pytest - \glossterm{Python}:
    \begin{itemize}
		\item \href{https://docs.pytest.org/en/7.1.x/contents.html}{\color{myblue}https://docs.pytest.org/en/7.1.x/contents.html} (Ultimo accesso: \today)
	\end{itemize}
    \item Documentazione unittest - \glossterm{Python}:
    \begin{itemize}
		\item \href{https://docs.python.org/3/library/unittest.html}{\color{myblue}https://docs.python.org/3/library/unittest.html} (Ultimo accesso: \today)
	\end{itemize}
    \item Documentazione Apache \glossterm{Kafka}:
    \begin{itemize}
		\item \href{https://kafka.apache.org/20/documentation.html}{\color{myblue}https://kafka.apache.org/20/documentation.html} (Ultimo accesso: \today)
	\end{itemize}
    \item Documentazione Apache \glossterm{Flink}:
    \begin{itemize}
		\item \href{https://nightlies.apache.org/flink/flink-docs-stable/}{\color{myblue}https://nightlies.apache.org/flink/flink-docs-stable/} (Ultimo accesso: \today)
	\end{itemize}
    \item Documentazione \glossterm{ClickHouse}:
    \begin{itemize}
		\item \href{https://clickhouse.com/docs/en/intro}{\color{myblue}https://clickhouse.com/docs/en/intro} (Ultimo accesso: \today)
	\end{itemize}
    \item Documentazione \glossterm{Grafana}:
    \begin{itemize}
		\item \href{https://grafana.com/docs/grafana/latest}{\color{myblue}https://grafana.com/docs/grafana/latest} (Ultimo accesso: \today)
	\end{itemize}
    \item Documentazione variables - \glossterm{Grafana}:
    \begin{itemize}
		\item \href{https://grafana.com/docs/grafana/latest/variables/}{\color{myblue}https://grafana.com/docs/grafana/latest/variables/} (Ultimo accesso: \today)
	\end{itemize}
    \item Documentazione alerts - \glossterm{Grafana}:
    \begin{itemize}
		\item \href{https://grafana.com/docs/grafana/latest/alerting/}{\color{myblue}https://grafana.com/docs/grafana/latest/alerting/} (Ultimo accesso: \today)
	\end{itemize}
    \item Documentazione notification policies - \glossterm{Grafana}:
    \begin{itemize}
		\item \href{https://grafana.com/docs/grafana/latest/alerting-rules/create-notification-policy/}{\color{myblue}https://grafana.com/docs/grafana/latest/alerting-rules/create-notification-policy/} (Ultimo accesso: \today)
	\end{itemize}
\end{itemize}
\newpage
\clearpage
\section{Tecnologie}\label{sec:tec}
In questa sezione verranno illustrati gli strumenti e le tecnologie impiegati per lo sviluppo del software nonché infrastrutture, linguaggi di programmazione, \glossterm{librerie} e \glossterm{framework} utilizzati.
\subsection{Docker}
In fase di sviluppo, testing e produzione, sono stati utilizzati \glossterm{container} \glossterm{Docker} per creare ambienti di lavoro efficaci e consistenti.
\subsubsection{Ambienti}
Sono stati creati tre ambienti di lavoro con focus operativi differenti.
\begin{itemize}
    \item \textbf{Ambiente di produzione:}
    \begin{itemize}
        \item Ambiente dove i programmatori possono testare il codice nella produzione dei dati simulati;
        \item Il focus è quello della produzione dei dati simulati nell'ottica della fruizione del servizio fornito dall'applicativo.
        \item L'intera \glossterm{data pipeline} è in funzione.
    \end{itemize}
    \item \textbf{Ambiente di development:}
    \begin{itemize}
        \item Ambiente dove i programmatori possono controllare il corretto collegamento tra le tecnologie implementate;
        \item Non è attiva la generazione dei dati simulati;
        \item Il focus è l'interazione fra le tecnologie della \glossterm{data pipeline}.
    \end{itemize}
    \item \textbf{Ambiente di test:}
    \begin{itemize}
        \item Ambiente che simula l'ambiente di produzione;
        \item Utilizza test automatizzati per verificare la correttezza del codice nelle sue unità e nell'integrazione di queste tra loro.
    \end{itemize}
\end{itemize}
\subsubsection{Images}
Di seguito sono elencate le immagini \glossterm{Docker} utilizzate:
\begin{itemize}
    \item \textbf{\glossterm{Kafka}}
    \begin{itemize}
        \item \textbf{Image:} bitnami/kafka:3.7.0
        \item \textbf{Riferimento:}
        \item \textbf{Ambiente:}
        \begin{itemize}
            \item Produzione;
            \item Test.
        \end{itemize}
    \end{itemize}
    \item \textbf{\glossterm{Flink}}
    \begin{itemize}
        \item \textbf{Image:} 
        \item \textbf{Riferimento:}
        \item \textbf{Ambiente:}
    \end{itemize}
    \item \textbf{\glossterm{ClickHouse}}
    \begin{itemize}
        \item \textbf{Image:} clickhouse/clickhouse-server:latest
        \item \textbf{Riferimento:}
        \item \textbf{Ambiente:}
        \begin{itemize}
            \item Produzione;
            \item Development;
            \item Test.
        \end{itemize}
    \end{itemize}
    \item \textbf{\glossterm{Grafana}}
    \begin{itemize}
        \item \textbf{Image:} grafana/grafana-oss:latest
        \item \textbf{Riferimento:}
        \item \textbf{Ambiente:}
        \begin{itemize}
            \item Produzione;
            \item Development.
        \end{itemize}
    \end{itemize}
\end{itemize}
\subsection{Linguaggi e formato dati}
\subsubsection{\glossterm{Python}}
\begin{itemize}
    \item \textbf{Versione:} 
    \item \textbf{Utilizzo:}
    \item \textbf{Librerie e framework:}
    \begin{itemize}
        \item \textbf{pytest}
        \begin{itemize}
            \item \textbf{Versione:}
            \item \textbf{Utilizzo:}
        \end{itemize}
    \end{itemize}
\end{itemize}
\subsubsection{\glossterm{SQL}}
\begin{itemize}
    \item \textbf{Versione:} 
    \item \textbf{Utilizzo:}
\end{itemize}
\subsubsection{\glossterm{JSON}}
\begin{itemize}
    \item \textbf{Versione:} 
    \item \textbf{Utilizzo:}
\end{itemize}
\subsubsection{\glossterm{Java}}
\begin{itemize}
    \item \textbf{Versione:} 
    \item \textbf{Utilizzo:}
\end{itemize}
\subsubsection{\glossterm{YAML}}
\begin{itemize}
    \item \textbf{Versione:} 
    \item \textbf{Utilizzo:}
\end{itemize}
\subsection{Servizi della pipeline}
\section{Architettura di sistema}
\subsection{Modello architetturale}
\subsection{Data-flow}
\subsection{Architettura dei simulatori}
\subsection{Kafka}
\subsection{Flink}
\subsection{Database}
\subsection{Grafana}
\section{Tracciamento requisiti funzionali}
\end{document}