% changelog: "1.0.0, 2024-08-12, Linda Barbiero, Approvazione per PB."

\documentclass[8pt]{article}
\usepackage[italian]{babel}
\usepackage[utf8]{inputenc}
\usepackage[letterpaper, left=1in, right=1in, bottom=0.75in, top=0.75in]{geometry}
\usepackage{amsmath}
\usepackage{subfiles}
\usepackage{lipsum}
\usepackage{csquotes}
\usepackage{amsfonts}
\usepackage[sfdefault]{plex-sans}
\usepackage{float}
\usepackage{pifont}
\usepackage{mathabx}
\usepackage[euler]{textgreek}
\usepackage{makecell}
\usepackage{tikz}
\usepackage{wrapfig}
\usepackage{siunitx}
\usepackage{amssymb} 
\usepackage{tabularx}
\usepackage{threeparttable}
\usepackage{adjustbox}
\usepackage[document]{ragged2e}
\usepackage{floatflt}
\usepackage[hidelinks]{hyperref}
\usepackage{graphicx}
\usepackage{hyperref}
\setcounter{tocdepth}{4}
\usepackage{caption}
\usepackage{multicol}
\usepackage{tikz}
\captionsetup{font=footnotesize}
\usepackage{fancyhdr} 
\usepackage{graphicx}
\usepackage{capt-of}% 
\usepackage{booktabs}
\usepackage{varwidth}
\setlength\parindent{0pt}
% -- IMPORTANTE -- %
% utilizzare `` " per le virgolette (``esempio") altrimenit non si chiudono
% non utilizzare \glossterm nei verbali

% -- TITOLO INTESTAZIONE -- %
\newcommand{\customtitle}{VERBALE ESTERNO DEL 2024-07-16} % o ESTERNO

% -- STILE INTESTAZIONE -- %
\fancypagestyle{mystyle}{
	\fancyhf{} 
  	\fancyhead[R]{\includegraphics[height=1cm]{../../../template/images/logos/NaN1fy_logo.png}} 
	\fancyhead[L]{\leftmark} 
	\renewcommand{\headrulewidth}{1pt} 
	\fancyhead[L]{\customtitle} 
	\renewcommand{\headsep}{1.3cm} 
	\fancyfoot[C]{\thepage} 
}

% -- PER LA FIRMA -- %
\newcommand{\signatureline}[1]{%
	 \par\vspace{0.5cm}
	\noindent\makebox[\linewidth][r]{\rule{0.2\textwidth}{0.5pt}\hspace{3cm}\makebox[0pt][r]{\vspace{3pt}\footnotesize #1}}%
}

% -- PER IL GLOSSARIO -- %
\newcommand{\glossterm}[1]{#1\textsuperscript{G}} % inserisci \glossterm{termine}

% -- per abilitare 4x sottosezioni es 2.1.1.1
% quando lo utilizi metti un // esterno al comando subsubsubsection{esempio} altrimenti è attacatto al titoletto, non modificare il comando (se metti //// nel comando è troppo spaziato quando ci sono table e figure
\setcounter{secnumdepth}{4}
\newcommand{\subsubsubsection}[1]{\paragraph{#1}\mbox{}\\}

\begin{document}
\definecolor{myblue}{RGB}{23,103,162}
\begin{titlepage}
	\begin{tikzpicture}[remember picture, overlay]
		\node[anchor=south east, opacity=0.2, yshift = -4cm, xshift= 2em] at (current page.south east)
      {\includegraphics[width=0.7\textwidth, trim=0cm 0cm 5cm 0cm, clip]{../../../template/images/logos/Universita_Padova_transparent.png}}; 
		\node[anchor=north west, opacity=1, yshift = 4.2cm, xshift= 1.4cm, scale=1.6] at (current
      page.south west) {\includegraphics[width=4cm]{../../../template/images/logos/NaN1fy_logo.png}};
	\end{tikzpicture}
	
	\begin{minipage}[t]{0.47\textwidth}
		{\large{\textsc{Destinatari}}
			\vspace{3mm}
			\\ \large{\textsc{Prof. Tullio Vardanega}}
			\\ \large{\textsc{Prof. Riccardo Cardin}}
		}
	\end{minipage}
	\hfill
	\begin{minipage}[t]{0.47\textwidth}\raggedleft
		{\large{\textsc{Redattori}}
			\vspace{3mm}
			{\\\large{\textsc{Pietro Busato}\\}}
			
			
		}
		\vspace{8mm}
		
		{\large{\textsc{Verificatori}}
			\vspace{3mm}
			{\\\large{\textsc{Linda Barbiero}\\}} % massimo due 
			{\large{\textsc{Veronica Tecchiati}}}
			
		}
		\vspace{4mm}\vspace{4mm}
	\end{minipage}
	\vspace{4cm}
	\begin{center}
		\begin{flushright}
			{\fontsize{30pt}{52pt}\selectfont \textbf{Verbale Esterno Del\\2024-07-16\\}} % o ESTERNO
		\end{flushright}
		\vspace{3cm}
	\end{center}
	\vspace{8.5 cm}
	{\small \textsc{\href{mailto: nan1fyteam.unipd@gmail.com}{nan1fyteam.unipd@gmail.com}}}
\end{titlepage}
\pagestyle{mystyle}
\section*{Registro delle Modifiche}
\begin{table}[ht!]	
	\centering
	\begin{tabular}{p{1.2cm} p{2cm} p{5cm} p{3cm} p{3cm}}
		\toprule
		\textbf{Versione}& \textbf{Data} & \textbf{Descrizione} & \textbf{Redattore} & \textbf{Verificatore} \\
		\midrule
		1.0.0 & 2024-08-12 & \textbf{Approvazione per PB} & & Linda Barbiero \\
		0.0.1 & 2024-08-01 & Prima stesura. & Pietro Busato & Veronica Tecchiati \\
		\bottomrule
		% Ruolo Redattore o Verificatore
	\end{tabular}
	\caption{Registro delle modifiche.}
	\label{table:Registro delle modifiche}
\end{table}
\newpage
\tableofcontents
\clearpage
\newpage
\justifying
\section{Contenuti del Verbale}
\subsection{Informazioni sulla riunione}
\begin{itemize}
	\setlength\itemsep{0em}
	\item\textbf{Luogo:} Chiamata Google Meet;
	\item\textbf{Ora di inizio:} 15:00;
	\item\textbf{Ora di fine:}  16:00.
\end{itemize}
\begin{table}[ht!]
	\begin{minipage}[t]{0.5\linewidth}
		\centering
		\begin{tabular}{p{3cm} p{3cm}}
			\toprule
			\textbf{Partecipante} & \textbf{Durata presenza} \\
			\midrule
			Guglielmo Barison & 0.0 h \\
			Linda Barbiero &  1.0 h \\
			Pietro Busato & 1.0 h \\
			Oscar Konieczny & 1.0 h \\
			Davide Donanzan & 1.0 h \\
			Veronica Tecchiati & 1.0 h \\
			\bottomrule
		\end{tabular}
		\caption{Partecipanti NaN1fy.}
		\label{table:Partecipanti NaN1fy}
	\end{minipage} 
	\begin{minipage}[t]{0.5\linewidth} % -- COMMENTA/DECOMMENTA DA QUI
		\centering
		\begin{tabular}{p{3cm} p{3cm}}
			\toprule
			\textbf{Partecipante} & \textbf{Durata presenza} \\
			\midrule
			Andrea Dorigo & 1.0 h \\
			Fabio Pallaro & 1.0 h \\
			Daniele Zorzi & 1.0 h \\
			\bottomrule
		\end{tabular}
		\caption{Partecipanti SyncLab.}
		\label{table:Partecipanti SyncLab}
	\end{minipage} % -- A QUI PER TOGLIERE AGGIUNGERE
\end{table}
\subsection{Ordine del giorno}
\begin{itemize}
	\setlength\itemsep{0em}
	\item Revisione lavoro svolto in seguito al periodo di assestamento;	
	\item Discussione riguardante alcune criticità incontrate;
    \item Discussione riguardante gli obiettivi e le finestre di consegna del prodotto.
\end{itemize}
\subsection{Sintesi dell'incontro}
Come di consueto, la riunione ha visto un'iniziale discussione riguardante i progressi fatti durante l'ultimo Sprint, ovvero la quasi completa realizzazione delle dashboard richieste e l'implementazione finale del sistema di alerting per Grafana; successivamente, come anche già specificato tramite mail apposita, si è dettagliatamente discussa l'organizzazione delle fasi finali del progetto, comprendente individuazione e valutazione di obiettivi a breve, medio e lungo termine, oltre che alle varie difficoltà riscontrabili nella realizzazione degli  stessi e, infine, dell'esatta finestra di consegna del progetto. Le criticità principali individuate sono:
\begin{itemize}
\setlength\itemsep{0em}
    \item Scarsa conoscenza della tecnologia di stream processing da implementare, ovvero Apache Flink;
    \item Aggiunta della richiesta di accesso alla piattaforma, funzionalità rivelatasi necessaria, attuabile tramite funzionalità di Grafana.
\end{itemize}   
Si riporta di seguito il piano di fine progetto specificato tramite mail all'azienda Proponente.\\\\
\setlength\parindent{0pt}
\textbf{Risorse disponibili:}
\begin{itemize}
\setlength\itemsep{0em}
    \item Budget rimanente: 4252,5 \texteuro;
    \item Monte ore: 217 ore da distribuire nell'arco di 5-6 settimane (o 8 nel caso di consegna il 2024-09-06).
\end{itemize} 

\textbf{Data prevista di completamento progetto:} 2024-08-16 / 2024-08-23 (con data ultima di scadenza 2024-09-06); \\\\
\textbf{Obiettivi del prodotto finale:}
\begin{itemize}
\setlength\itemsep{0em}
    \item Conformazione del prodotto finale: versione ampliata, rifinita e più complessa del prodotto precedentemente presentato, comprendente più tipologie di simulazioni di sensori e più dashboard visualizzabili, un sistema di notifica, stream processing e infine interfaccia grafica utente personalizzata con la brand identity della Proponente. L'obiettivo finale è la creazione di un'interfaccia accessibile da ente pubblico (amministratore) tramite login, grazie alla quale visualizzare i dati relativi al territorio urbano e ai suoi servizi;
    \item Dashboard: suddivise semanticamente per ambito di interesse, svolgeranno funzioni di monitoraggio dati o di analisi degli stessi, anche tramite stream processing, offrendo diverse modalità di visualizzazione dati (time series, gauge, barchart ecc.). I dati saranno di pertinenza ambientale (temperatura, umidità ecc.) e urbanistica (occupazione parcheggi, malfunzionamenti elettrici ecc.);
    \item Stream Processing: realizzato tramite la tecnologia Apache Flink, verrà utilizzato per la semplificazione dell'analisi dei dati ambientali e delle loro interazioni, oltre che dei dati urbanistici per valutare l'efficienza e le modalità d'uso dei servizi cittadini;
    \item Stream dati: combinazione dati di occupazione parcheggio con colonnina di ricarica e dati relativi all'effettivo utilizzo e consumo di quest'ultima; combinazione dati occupazione parcheggio e pagamento parcheggio; varie correlazioni tra eventi atmosferici, come ad esempio il cambiamento del livello dell'acqua a causa della temperatura, umidità e/o delle precipitazioni;
    \item Sistema di notifica: realizzato tramite notifiche via mail/webhook, invia messaggi al superamento di determinate soglie, descritte nel documento \textit{Analisi dei Requisiti};
    \item Documentazione: dashboard e pannelli saranno ampiamente e dettagliatamente descritti nel documento \textit{Analisi dei Requisiti}, così come il sistema di notifica; l'esame delle tecnologie di stream processing sarà spiegato in un documento apposito, che conterrà inoltre le motivazioni a supporto della scelta di Apache Flink; verranno redatti inoltre il \textit{Manuale Utente} e il documento di \textit{Specifica Tecnica}; la qualità del prodotto invece verrà mantenuta e consolidata dal \textit{Piano di Qualifica}.
\end{itemize}
\textbf{Organizzazione e divisione obiettivi:}
\begin{itemize}
\setlength\itemsep{0em}
    \item Breve termine (previsione: entro lo Sprint attuale):
    \begin{itemize}
    \setlength\itemsep{0em}
        \item Ultimazione implementazione sensori;
        \item Ultimazione e rifinitura Dashboard;
        \item Ultimazione sistema di notifica o di buona sua parte.
    \end{itemize}
    \item Medio termine (previsione: entro la settimana 2024-07-29 / 2024-08-04):
    \begin{itemize}
    \setlength\itemsep{0em}
        \item Studio e implementazione stream processing tramite Apache Flink;
        \item Personalizzazione interfaccia grafica;
        \item Iniziale stesura della documentazione di supporto.
    \end{itemize}
    \item Lungo termine (previsione: entro la settimana 2024-08-12 / 2024-08-18):
    \begin{itemize}
    \setlength\itemsep{0em}
        \item Completamento stesura documentazione;
        \item Ultimazione MVP e sua rifinitura;
        \item Presentazione MVP e conclusione progetto.
    \end{itemize}
\end{itemize}
% -- togliere, testo per riempiere il template
\subsection{Decisioni prese}
\begin{itemize}
	\setlength\itemsep{0em}
	\item Viene presa la decisione di studiare nel dettaglio Apache Flink al fine di implementare tale tecnologia nel prodotto finale;
	\item Viene presa la decisione di modificare il design dell'applicativo Grafana tramite framework React, a immagine della Proponente;
	\item Viene presa la decisione di fare un'iniziale stesura dei documenti finali di supporto, quali Specifica Tecnica, Manuale Utente e a riguardo di Apache Flink.
	% consigliata la forma \textit{Viene adottato} quando viene adottato un certo modo di fare/strumento
	% per nomi di aziene e capitolati usare \texttt{}, e.g \texttt{Easy meal} \texttt{C6} 
\end{itemize}
\newpage
\section{Attività da svolgere}
Viene indetta una riunione interna per delineare la pianificazione dell'ottavo Sprint e le attività da svolgere in vista della revisione relativa alla milestone PB.
% commentare per togliere la firma
\signatureline{Padova, 2024-08-12}
\end{document}
