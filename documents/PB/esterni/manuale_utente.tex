% changelog: "0.0.0, 2024-07-01, Davide Donanzan, Stesura struttura e sezione intro"

\documentclass[8pt]{article}
\usepackage[italian]{babel}
\usepackage[utf8]{inputenc}
\usepackage[letterpaper, left=1in, right=1in, bottom=0.75in, top=0.75in]{geometry}
\usepackage{amsmath}
\usepackage{subfiles}
\usepackage{lipsum}
\usepackage{csquotes}
\usepackage{amsfonts}
\usepackage[sfdefault]{plex-sans}
\usepackage{float}
\usepackage{pifont}
\usepackage{mathabx}
\usepackage[euler]{textgreek}
\usepackage{makecell}
\usepackage{tikz}
\usepackage{wrapfig}
\usepackage{siunitx}
\usepackage{amssymb} 
\usepackage{tabularx}
\usepackage{adjustbox}
\usepackage[document]{ragged2e}
\usepackage{floatflt}
\usepackage[hidelinks]{hyperref}
\usepackage{graphicx}
\usepackage{hyperref}
\setcounter{tocdepth}{4}
\usepackage{caption}
\usepackage{multicol}
\usepackage{tikz}
\setlength\parindent{0pt}
\captionsetup{font=footnotesize}
\usepackage{fancyhdr} 
\usepackage{graphicx}
\usepackage{capt-of}% 
\usepackage{booktabs}
\usepackage{varwidth}
\usepackage{datetime2}
\usepackage{xcolor}
\usepackage{longtable}
\usepackage{array}
\usepackage{ragged2e}
\usepackage{colortbl}
\usepackage{verbatim}
\usepackage{enumitem}

\newcommand{\customtitle}{MANUALE UTENTE}% o ESTERNO

% -- STILE COLONNA CENTRATA PER TABELLE -- %
\newcolumntype{M}[1]{>{\centering\arraybackslash}m{#1}}

% -- STILE INTESTAZIONE -- %
\fancypagestyle{mystyle}{
	\fancyhf{} 
	\fancyhead[R]{\includegraphics[height=1cm]{../../template/images/logos/NaN1fy_logo.png}} 
	\fancyhead[L]{\leftmark} 
	\renewcommand{\headrulewidth}{1pt} 
	\fancyhead[L]{\customtitle} 
	\renewcommand{\headsep}{1.3cm} 
	\fancyfoot[C]{\thepage} 
}

% -- PER LA FIRMA -- %
\newcommand{\signatureline}[1]{%
	 \par\vspace{0.5cm}
	\noindent\makebox[\linewidth][r]{\rule{0.2\textwidth}{0.5pt}\hspace{3cm}\makebox[0pt][r]{\vspace{3pt}\footnotesize #1}}%
}

% -- PER IL GLOSSARIO -- %
\newcommand{\glossterm}[1]{#1\textsuperscript{G}} % inserisci \glossterm{termine}

% -- per abilitare 4x sottosezioni es 2.1.1.1
\setcounter{secnumdepth}{4}
\newcommand{\subsubsubsection}[1]{\paragraph{#1}\mbox{}\\\\}

\begin{document}
\definecolor{myblue}{RGB}{23,103,162}
\begin{titlepage}
	\begin{tikzpicture}[remember picture, overlay]
		\node[anchor=south east, opacity=0.2, yshift = -4cm, xshift= 2em] at (current page.south east) {\includegraphics[width=0.7\textwidth, trim=0cm 0cm 5cm 0cm, clip]{../../template/images/logos/Universita_Padova_transparent.png}}; 
		\node[anchor=north west, opacity=1, yshift = 4.2cm, xshift= 1.4cm, scale=1.6] at (current page.south west) {\includegraphics[width=4cm]{../../template/images/logos/NaN1fy_logo.png}};
	\end{tikzpicture}
	
	\begin{minipage}[t]{0.47\textwidth}
		{\large{\textsc{Destinatari}}
			\vspace{3mm}
			\\ \large{\textsc{Prof. Tullio Vardanega}}
			\\ \large{\textsc{Prof. Riccardo Cardin}}
		}
	\end{minipage}
	\hfill
	\begin{minipage}[t]{0.47\textwidth}\raggedleft
		{\large{\textsc{Redattori}}
			\vspace{3mm}
			{\\\large{\textsc{Davide Donanzan}\\}} % massimo due 
			{\large{\textsc{XXXX XXXX}}}
			
			
		}
		\vspace{8mm}
		
		{\large{\textsc{Verificatori}}
			\vspace{3mm}
			{\\\large{\textsc{XXXX XXXX}\\}} 
			{\large{\textsc{XXXX XXXX}\\}} 
			{\large{\textsc{XXXX XXXX}\\}}
			
		}
		\vspace{2mm}\vspace{2mm}
	\end{minipage}
	\vspace{4cm}
	\begin{center}
		\begin{flushright}
			{\fontsize{30pt}{52pt}\selectfont \textbf{Manuale Utente}} % o ESTERNO
		\end{flushright}
		\vspace{3cm}
	\end{center}
	\vspace{10 cm}
	{\small \textsc{\href{mailto: nan1fyteam.unipd@gmail.com}{nan1fyteam.unipd@gmail.com}}}
\end{titlepage}
\pagestyle{mystyle}
\section*{Registro delle Modifiche}
\begin{table}[ht!]	
	\centering
	\begin{tabular}{p{1.2cm} p{2cm} p{6cm} p{3cm} p{2cm}}
		\toprule
		\textbf{Versione}& \textbf{Data} & \textbf{Descrizione} & \textbf{Autore} & \textbf{Ruolo} \\
		\midrule
		    0.0.0 & 2024-07-01 & Stesura struttura e sezione \ref{sec:intro}. & Davide Donanzan & Redattore \\
		\bottomrule
		% Ruolo Redattore o Verificatore
	\end{tabular}
	\caption*{Tabella: Registro delle modifiche.}
	\label{table:Registro delle modifiche}
\end{table}
\newpage
\tableofcontents
\newpage
\listoffigures
\newpage
\listoftables
\newpage
\justifying
\section{Introduzione}\label{sec:intro}
\subsection{Scopo del documento}
Il presente manuale è concepito per fornire un supporto agli utenti nell’utilizzo efficace del software, 
consentendo loro di sfruttare appieno tutte le sue funzionalità al fine di garantire un’esperienza ottimale.
In tal modo, si vuole informare l’amministratore pubblico dei requisiti minimi da soddisfare per
poter usufruire del prodotto, degli step necessari per la sua installazione e di tutte le funzionalità a disposizione.
% Poiché l’installazione del software è gestita da personale tecnico specializzato, questo
% manuale non include istruzioni dettagliate per l’installazione, ma si concentra piuttosto sui
% passaggi necessari per utilizzare il softwareG una volta installato correttamente
\subsection{Scopo del prodotto}
L'obiettivo del progetto SyncCity è quello di creare una piattaforma atta al monitoraggio
di sensori sparsi geograficamente nel territorio di una città. I sensori in questione
permettono la misurazione e segnalazione di dati \glossterm{real-time} riguardanti le più disparate
caratteristiche e necessità del territorio quali temperatura ed umidità esterna, occupazione di
stalli di parcheggio, funzionamento o guasto elettrico di colonnine di ricarica, traffico stradale e via
dicendo. La \glossterm{Proponente} richiede la simulazione di alcuni dei sensori nominati nonchè la
gestione dei dati, della loro persistenza e della loro rappresentazione grafica attraverso \glossterm{widget} e
grafici. 
\\\\SyncCity permetterà un miglioramento della \glossterm{qualità} dei servizi offerti dalla città attraverso il continuo monitoraggio della stessa, ottenendo, gestendo e successivamente condividendo i dati con gli utenti. 
% \subsection{Accesso alla piattaforma}
\subsection{Glossario}
Per garantire chiarezza nel linguaggio utilizzato nei documenti, è stato redatto un Glossario contenente le definizioni dei termini con significato specifico da disambiguare. Tali termini sono contrassegnati con una G ad apice. L'inserimento di un termine nel Glossario è considerato completo solo dopo averne fornito la definizione.
\subsection{Riferimenti}
\subsubsection{Normativi}
\begin{itemize}
	\item \textit{Norme di Progetto v2.0.0};
	\item Presentazione e documentazione del \glossterm{capitolato} d’appalto C6 - SyncCity:
	\begin{itemize}
		\item \href{https://www.math.unipd.it/~tullio/IS-1/2023/Progetto/C6p.pdf}{\color{myblue}https://www.math.unipd.it\textasciitilde{}tullio/IS-1/2023/Progetto/C6p.pdf} (Ultimo accesso: \today)
		\item \href{https://www.math.unipd.it/~tullio/IS-1/2023/Progetto/C6.pdf}{\color{myblue}https://www.math.unipd.it/\textasciitilde{}tullio/IS-1/2023/Progetto/C6.pdf} (Ultimo accesso: \today)
	\end{itemize}
	\item Regolamento di progetto:
	\begin{itemize}
		\item \href{https://www.math.unipd.it/~tullio/IS-1/2023/Dispense/PD2.pdf}{\color{myblue}https://www.math.unipd.it/\textasciitilde{}tullio/IS-1/2023/Dispense/PD2.pdf} (Ultimo accesso: \today)
	\end{itemize}
\end{itemize}
% \clearpage
\subsubsection{Informativi}
\begin{itemize}
    \item \textit{Analisi dei Requisiti v2.0.0};
    \item \textit{Glossario v2.0.0};
    \item \textit{Specifica Tecnica v1.0.0};
\end{itemize}
\clearpage
\section{Requisiti}
\subsection{Requisiti hardware}
\subsection{Requisiti di sistema operativo}
\subsection{Requisiti software}
\subsection{Requisiti browser}
\section{Istruzioni all'uso}
\subsection{Avvio}
\subsection{Login}
\subsection{Informazioni generali}
\subsection{Informazioni dashboard e pannelli}
\subsection{Dashboard ``Sensori"}
\subsection{Dashboard ``Ambientale"}
\subsection{Dashboard ``Urbanistica"}
\subsection{Dashboard ``Superamento Soglie"}
\subsection{Allerte}
\subsection{Gestione e funzionalità dashboard}
\section{Supporto tecnico}
Per assistenza tecnica relativa all’utilizzo del prodotto software SyncCity, viene fornito il
seguente indirizzo email:
\begin{center}
    \parbox{\linewidth}{\centering
        \href{mailto: nan1fyteam.unipd@gmail.com}{nan1fyteam.unipd@gmail.com}
    }
\end{center}
Si è gentilmente pregati di includere nel corpo dell’email una descrizione quanto più completa del problema riscontrato, 
insieme ad eventuali istantanee o dettagli aggiuntivi che possano aiutare la comprensione del problema e la risoluzione dello stesso.
Si invita, inoltre, a descrivere eventuali passaggi già tentati, in modo che il team possa fornire un’assistenza più mirata.
\end{document}