\subsection{Modello Agile}
Un approccio metodologico agile fornisce un insieme di caratteristiche che lo rendono particolarmente idoneo per gestire le tempistiche limitate di rilascio, garantendo comunque un progetto di qualità buona.
Questa metodologia prevede la realizzazione di molteplici rilasci successivi, ciascuno dei quali porta con sé un incremento delle funzionalità. Adottando tale approccio, diventa essenziale identificare e classificare i requisiti in modo da definire un ordine di priorità nello sviluppo, consentendo di ottenere, dopo ogni incremento, un prodotto stabile e funzionante, seppur incompleto. Per garantire la stabilità del prodotto sin dalle prime fasi, è cruciale che i primi incrementi soddisfino i requisiti più significativi, mentre quelli di minore importanza saranno integrati in un secondo momento, permettendo loro di stabilizzarsi gradualmente nel contesto del prodotto.
\\\\
I vantaggi da evidenziare sono i seguenti:
\begin{itemize}
\item
\textbf{\emph{Iterazioni rapide}}: Sprint brevi e frequenti permettono di ottenere risultati tangibili in tempi rapidi e di adattare il piano di sviluppo in base ai feedback e alle esigenze emergenti.
\item
\textbf{\emph{Flessibilità e adattabilità}}: Si portanno apportare modifiche al prodotto anche in corso d'opera senza dover ripensare completamente il \textit{Piano di Progetto}.
\item
\textbf{\emph{Coinvolgimento del cliente / proponente}}: Permette il coinvolgimento col proponente continuo durante tutto il ciclo di sviluppo, così da avere feedback tempestivi e garantire che il prodotto finale soddisfi effettivamente le esigenze.
\item
\textbf{\emph{Focus sulla qualità}}: La pratica di sviluppo incrementale, al testing continuo e al coinvolgimento degli stakeholder consente comunque di mantere alta la qualità del software.
\item
\textbf{\emph{Comunicazione e collaborazione}}: I membri del team rimangono allineati sugli obiettivi e sulle priorità del progetto, riducendo i rischi di fraintendimenti e conflitti.
\end{itemize}