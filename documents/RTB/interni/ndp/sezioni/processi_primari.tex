
\subsection{Fornitura}
\subsubsection{Introduzione}
Il processo primario di fornitura, definito dallo standard ISO/IEE 12207:1995 (con ultima versione datata 2018), individua l'insieme di attivitá atte alla realizzazione di un prodotto software che soddisfi pienamente, a partire dalla formulazione della proposta al committente fino alla sua ultimazione e consegna, i requisiti e le necessitá a cui deve rispondere, garantendo un percorso consono, strutturato ed efficiente, oltre che efficace.

\subsubsection{Attivitá}
Tali sono le attivitá definite nel processo primario di fornitura:

\begin{itemize}
    \item Acquisizione e preparazione: si individuano le necessitá del cliente e vengono definiti eventuali requisiti, con associate analisi dei costi pecuniari e temporali (preventivo), e valutazione delle varie opzioni;
    \item Contrattazione: vengono negoziati tra fornitore e cliente i termini e le condizioni contrattuali, e vengono stipulati di comune accordo obiettivi, costi, tempistiche e responsabilitá di entrambe le parti;
    \item Pianificazione: vengono pianificate le attivitá e le stesure dei documenti utili alla realizzazione del progetto nel rispetto degli accordi fatti;
    \item Attuazione e controllo: vengono eseguite le attvitá pianificate, controllando regolarmente e consistentemente lo stato di avanzamento e il rispetto degli impegni prefissati conformemente al costo, alle tempistiche e ai requisiti accordati in precedenza;
    \item Revisione e valutazione: vengono effettuate revisioni periodiche e confronti con il cliente, per assicurare il corretto svolgimento del progetto secondo i termini prefissati e risolvere dubbi, incertezze e rischi occorsi;
    \item Completamento e consegna: una volta completato il progetto, viene consegnato al cliente il prodotto finale, secondo quanto stipulato nel contratto.
\end{itemize}

\subsubsection{Documentazione fornita}
A corredo delle attivitá volte alla realizzazione del progetto, vengono stesi e resi disponibili all'azienda propronente SyncLab e ai committenti \textit{Prof}. Vardanega e \textit{Prof}. Cardin i seguenti documenti:

\subsubsubsection{Analisi dei Requisiti}
Il documento Analisi dei Requisiti illustra e descrive in dettaglio i casi d'uso e i requisiti del progetto, nonché le funzionalitá che ci si aspetta che il prodotto finale abbia, sulla base degli obiettivi posti riguardo al progetto. Tale documento funge quindi da base preliminare per la progettazione del software, eliminando incertezze e dubbi preliminari.

\subsubsubsection{Dichiarazione Impegni}
Tale documento manifesta la volontá, da parte del team di sviluppo, di impegnarsi nella realizzazione del prodotto di un dato capitolato (in particolare nel nostro caso di InnovaCity, proposto dall'azienda SyncLab); tale documento fornisce inoltre:

\begin{itemize}
    \item una suddivisione del monte ore e dei singoli ruoli tra i vari membri del gruppo;
    \item una descrizione dei ruoli e del loro rapporto relativo alle specifiche del progetto;
    \item una analisi preliminare dei rischi e dei dubbi sorti durante il processo di vaglio dei capitolati;
    \item preventivo preliminare dei costi;
    \item scadenza prefissata previsa;
\end{itemize}

\subsubsubsection{Glossario}
Ai fini di disambiguare ogni possibile dubbio o incertezza, nonché per rispondere preventivamente e in maniera esaustiva a possibili domande di natura tecnico-linguistica/etimologica da parte dei membri del gruppo, o di chiunque abbia la possibilitá in futuro di leggere i documenti relativi a tale progetto, viene istituito un Glossario, contenente una definizione chiara e univoca di tutti quei termini rilevanti che portano con sé un significato specifico, onde evitare interpretazioni arbitrarie e fraintendimenti riguardo tali specifici concetti.

\subsubsubsection{Lettera di Presentatione}
La Lettera di Presentazione é un documento che accompagna la documentazione e il prodotto forniti all'azienda proponente durante le fasi di revisione del progetto, il cui scopo é quello di dare un veloce contesto intorno allo stato di avanzamento dei lavori (o del loro inizio, per quanto concerne la lettera di presentazione per i capitolati), e fornire una breve panoramica della documentazione redatta fino a quel momento. 

\subsubsubsection{Norme di Progetto}
La finalitá del documento noto come Norme di Progetto é quello di redigere e definire un insieme di standard e regole rigurado ai processi e la loro normazione, da seguire imperativamente, da parte del team di sviluppo, durante tutto il ciclo di vita del progetto, in modo da garantirne la qualitá e la conformitá agli obiettivi e requisiti prefissati con il cliente.

\subsubsubsection{Piano Di Progetto}
Il documento Piano di Progetto si occupa di delineare la pianificazione e la gestione delle attività volte alla realizzazione del progetto; tra queste si ha in particolare l'analisi dei rischi e metodi per la loro possibile mitigazione, il modello di sviluppo adottato, il preventivo dei costi ed il consuntivo, nonché un aggionramento costante sullo stato di avanzamento dei lavori, corredato di diagrammi di Gantt che descrivono lo stato di avanzamento di ogni singolo documento o attivitá relativa al progetto.

\subsubsubsection{Piano di Qualifica}
Il documento Piano di Qualifica offre una panoramica dettagliata delle strategie di verifica e validazione adottate per assicurare la \glossterm{qualità} del prodotto e dei processi del progetto in questione. Costantemente aggiornato per riflettere l'evoluzione del progetto, si pone come documento dinamico e incrementale che illustra le pratiche per il controllo di qualità degli artefatti e dei processi, con particolare enfasi sulle metriche di valutazione del prodotto. È un documento dunque progettato per guidare l'adozione di processi mirati al miglioramento continuo, fornendo misure quantitative per valutare il progresso del progetto, fornendo gli strumenti per risolvere tempestivamente eventuali criticità, aggiornandosi regolarmente per adattarsi alle esigenze mutevoli del progetto, garantendo infine la crescita e l'evoluzione sia del processo che del prodotto nel tempo.

%\subsubsubsection{Valutazione Capitolati}
%Il documento Valutazione dei Capitolati ha il puro scopo informativo di elencare le motivazioni per le quali si é scelto di proporsi per un determinato progetto test

\subsubsubsection{Verbali interni}
Viene riportata la documentazione concernente le riunioni interne del team di sviluppo, il cui scopo é quello di fissare per iscritto ció che é stato detto durante suddette riunioni. %da ampliare 

\subsubsubsection{Verbali esterni}
Viene riportata la documentazione concernente le riunioni tenutesi con i rappresentanti dell'azienda proponente. % da ampliare

\subsection{Sviluppo}