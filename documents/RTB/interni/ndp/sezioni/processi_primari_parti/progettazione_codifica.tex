\subsubsection{Progettazione} \label{sec:progettazione}
Lo scopo della progettazione è definire una soluzione possibile ai requisiti dichiarati nell'\textit{Analisi dei Requisiti}.
\subsubsubsection{Descrizione}\\
La parte di progettazione è soggetta a due parti:
\begin{itemize}
    \item{\textbf{Progettazione logica}}: tecnologie, \glossterm{framework} e \glossterm{librerie}, utilizzate per la realizzazione del prodotto, dimostrando la sua adeguatezza utilizzando il \glossterm{PoC}.
    \begin{itemize}
        \item Framework e tencologie utilizzate;
        \item \glossterm{Proof of Concept};
        \item Diagrammi \glossterm{UML}.
    \end{itemize}
    \item{\textbf{Progettazione di dettaglio}}: base architetturale del prodotto basandosi sulla Progettazione logica.
    \begin{itemize}
        \item Diagrammi delle classi;
        \item Tracciamento delle classi;
        \item Test di unità per componente.
    \end{itemize}
\end{itemize}
\subsubsection{Codifica} \label{sec:codifica}
Lo scopo della codifica è implementare le specifiche individuate concretizzandole su un prodotto utilizzabile.
\subsubsubsection{Commenti}\\
Nel caso in cui sia necessario documentare porzioni di codice tramite commenti mirati, è d'obbligo che essi siano concisi e chiari.
\subsubsubsection{Nomi dei file}\\
I nomi dei file devono essere univoci ma soprattutto esplicativi, dichiarando con parole chiave il loro contenuto.
