\subsection{Sviluppo}

\subsubsection{Introduzione}
Il processo di sviluppo comprende tutte le attività da svolgere, da parte dei membri del team di sviluppo, il cui fine è quello di ottenere un prodotto in linea con le esigenze e i requisiti fissati dalla Proponente; in particolare si compone di:
\begin{itemize}
    \item Analisi dei Requisiti;
    \item Progettazione;
    \item Codifica.
\end{itemize} 

\subsubsection{Analisi dei Requisiti} \label{sec:analisi-rischi}

\subsubsubsection{Descrizione}\\
Lo scopo dell'analisi dei requisiti è quello di definire esaustivamente i casi d'uso, i requisiti e i vincoli concordati insieme alla Proponente, per garantire una corretta codifica e implementazione del prodotto software da parte del team.\\
L'Analisi dei Requisiti, redatta dagli Analisti, contiene:
\begin{itemize}
    \item Una descrizione esplicita delle funzionalità attese dal prodotto;
    \item L'insieme degli attori, ovvero gli utilizzatori del prodotto finale;
    \item L'insieme dei casi d'uso, ovvero le interazioni tra gli attori e il sistema;
    \item L'insieme dei requisiti e delle caratteristiche da soddisfare.
\end{itemize}
\subsubsubsection{Obiettivo}\\
Tale processo si propone di comprendere in maniera ultima le esigenze degli utenti, le caratteristiche del prodotto e le condizioni in cui esso dovrà operare.\\
Questa attività di analisi comporta:
\begin{itemize}
    \item L'identificazione, in armonia con le esigenze e le proposte dell'azienda, degli obiettivi e delle finalità del prodotto che si intende sviluppare;
    \item La fornitura di una base chiara e comprensibile ai progettisti per definire più facilmente l'architettura e il design del sistema;
    \item La facilitazione della comunicazione tra fornitori e stakeholder.
\end{itemize}
\subsubsubsection{Diagrammi UML dei casi d'uso}\\
Un diagramma di caso d'uso è uno strumento di modellazione utilizzato nella documentazione e descrizione delle funzionalità di un sistema. La sua utilità si manifesta nella possibilità di rappresentare visivamente, e quindi in maniera veloce e comprensibile, l'interazione tra utente e sistema in uno specifico scenario. Gli scenari d'uso descrivono le azioni atte a consentire all'utente di eseguire con successo una specifica attività; essi sono interconnessi mediante linee. La rappresentazione fornita dai diagrammi dei casi d'uso non si occupa di spiegare eventuali dettagli implementativi, poichè esula dallo scopo principale del documento.\\
I diagrammi dei casi d'uso sono composti da:\\
\begin{itemize}
    \item \textbf{Attore}: agente estraneo che interagisce con il sistema per soddisfare una sua necessità;
    %Attore
        \begin{figure}[H]
            \centering
            \begin{tikzpicture}
                \umlactor{Attore}
            \end{tikzpicture}
            \caption{Rappresentazione UML di un attore}
            \label{fig:Rappresentazione UML di un attore}
        \end{figure}
    \item \textbf{Caso d'uso}: funzionalità messa a disposizione da parte del sistema, con la quale l'attore può interagire;
    %Caso d'uso
        \begin{figure}[H]
        \centering
        \begin{tikzpicture}
            \umlusecase[width=80]{UC-X Titolo caso d'uso}
        \end{tikzpicture}
        \caption{Rappresentazione UML di un caso d'uso}
        \label{fig:Rappresentazione UML di un caso d'uso}
        \end{figure}
    \item \textbf{Sottocasi}: i sottocasi espongono in maniera più dettagliata e specifica alcune istanze di casi d'uso più generali; 
    %Sottocasi
    \begin{figure}[H]
        \centering
        \begin{tikzpicture}
            \umlactor[x=0, y=-1]{Attore}
            \begin{umlsystem}[x=6, y=0]{UC-X Titolo caso d'uso}
                \umlusecase[x=0, y=0, width=80, name=UC-X1]{UC-X.1 \\ Titolo sottocaso d'uso}
                \umlusecase[x=0, y=-2, width=80, name=UC-X2]{UC-X.2\\ Titolo sottocaso d'uso}
            \end{umlsystem}
                \umlassoc{Attore}{UC-X1}
                \umlassoc{Attore}{UC-X2}
        \end{tikzpicture}
        \caption{Rappresentazione UML sottocasi d'uso}
        \label{fig:Rappresentazione UML sottocasi d'uso}
    \end{figure}
    \item \textbf{Sistema}: identifica il prodotto software in produzione, e contiene al suo interno i casi d'uso; al di fuori di esso saranno invece posizionati gli attori (esterni);
        %Sistema
        \begin{figure}[H]
            \centering
            \begin{tikzpicture}
                \begin{umlsystem}{Sistema}\end{umlsystem}
            \end{tikzpicture}
            \caption{Rappresentazione UML di un sistema}
            \label{fig:Rappresentazione UML di un sistema}
        \end{figure}
    \item \textbf{Relazione tra attori e casi d'uso}: tale relazione esplica l'interazione tra un attore, ovvero l'utilizzo di una funzionalità del sistema, e un caso d'uso specifico, ovvero la funzionalitá stessa;
    %Relazioni tra attori e casi d'uso
        \begin{figure}[H]
            \centering
            \begin{tikzpicture}
                \umlactor{Attore}
                \umlusecase[x=5, y=0, width=80, name=UC-X]{UC-X Titolo caso d'uso}
                \umlassoc{Attore}{UC-X}
            \end{tikzpicture}  
            \caption{Rappresentazione UML relazione di associazione}
            \label{fig:Rappresentazione UML relazione di associazione}
        \end{figure}
    \item \textbf{Generalizzazione tra attori}: tale relazione identifica un rapporto di tipo ereditario tra due o più attori, uno dei quali (attore specializzato) ottiene caratteristiche e comportamenti da parte dell'attore ``padre" (attore base), stabilendo una gerarchia tra gli attori facenti parte dell'interazione con il sistema;
    %Relazioni tra attori
        \begin{figure}[H]
            \centering
            \begin{tikzpicture}
                \umlactor[x=0, y=0]{Attore padre}
                \umlactor[x=0, y=-4]{Attore figlio}
                \umlinherit{Attore figlio}{Attore padre}
            \end{tikzpicture}  
            \caption{Rappresentazione UML generalizzazione tra attori}
            \label{fig:Rappresentazione UML generalizzazione tra attori}
        \end{figure}
    \item \textbf{Relazioni tra casi d'uso}: 
        \begin{itemize}
        \item \textbf{Inclusione}: un caso d'uso viene definito ``includente'' quando, nell'interazione tra un attore e tale caso d'uso, viene eseguito come parte integrante del processo un altro caso d'uso, detto ``incluso''. Attraverso questa relazione si evita il problema della duplicazione di casi d'uso superflui; 
        %Inclusione
            \begin{figure}[H]
                \centering
                    \begin{tikzpicture}
                        \umlactor[x=0, y=0]{Attore}
                        \begin{umlsystem}[x=6, y=0]{Sistema}
                            \umlusecase[x=0, y=0, width=80, name=UC-X]{UC-X \\ Titolo caso d'uso}
                            \umlusecase[x=0, y=-3, width=80, name=UC-Y]{UC-Y \\ Titolo caso d'uso}
                        \end{umlsystem}
                            \umlassoc{Attore}{UC-X}
                            \umlinclude{UC-X}{UC-Y}
                    \end{tikzpicture}
                \caption{Rappresentazione UML relazione di inclusione}
                \label{fig:Rappresentazione UML relazione di inclusione}
            \end{figure}
        \item \textbf{Estensione}: un caso d'uso viene definito ``estendente'' quando, nell'interazione tra un attore e tale caso d'uso, al verificarsi di determinate condizioni può essere eseguito un altro caso d'uso, detto ``esteso'', a complemento o arricchimento del processo in atto. Attraverso tale relazione si è in grado di gestire situazioni particolari senza satollare lo scenario principale di eccezioni e condizioni particolari;
        %Estensione
            \begin{figure}[H]
                \centering
                    \begin{tikzpicture}
                        \umlactor[x=0, y=0]{Attore}
                        \begin{umlsystem}[x=6, y=0]{Sistema}
                            \umlusecase[x=0, y=0, width=80, name=UC-X]{UC-X \\ Titolo caso d'uso}
                            \umlusecase[x=0, y=-3, width=80, name=UC-Y]{UC-Y \\ Titolo caso d'uso}
                        \end{umlsystem}
                            \umlassoc{Attore}{UC-X}
                            \umlextend{UC-Y}{UC-X}
                    \end{tikzpicture}
                \caption{Rappresentazione UML relazione di estensione}
                \label{fig:Rappresentazione UML relazione di estensione}
            \end{figure}
        \item \textbf{Generalizzazione}: nel contesto di una generalizzazione un caso d'uso specifico eredita il proprio comportamento da parte di un altro caso d'uso piú generico. 
        %Generalizzazione
            \begin{figure}[H]
                \centering
                    \begin{tikzpicture}
                        \umlactor[x=0, y=0]{Attore}
                        \begin{umlsystem}[x=6, y=0]{Sistema}
                            \umlusecase[x=0, y=0, width=80, name=UC-X]{UC-X \\ Titolo caso d'uso}
                            \umlusecase[x=0, y=-3, width=80, name=UC-Y]{UC-Y \\ Titolo caso d'uso}
                        \end{umlsystem}
                            \umlassoc{Attore}{UC-X}
                            \umlinherit{UC-Y}{UC-X}
                    \end{tikzpicture}
                \caption{Rappresentazione UML relazione di generalizzazione}
                \label{fig:Rappresentazione UML relazione di generalizzazione}
            \end{figure}
        \end{itemize}
\end{itemize}

\subsubsubsection{Codice dei casi d'uso}\\
Ad ogni caso d'uso è associato un codice univoco definito nel seguente formato:
\begin{center}
    \textbf{UC-[Numero].[Specializzazione]}
\end{center}
Dove \textbf{Numero} è un identificativo e \textbf{Specializzazione} si riferisce ad un caso specifico
dello stesso caso d'uso.

\subsubsubsection{Requisiti} \\
I requisiti di un prodotto software, concordati a inizio progetto insieme alla Proponente, identificano in maniera dettagliata le funzionalità, le prestazioni, i vincoli e tutte le specifiche che tale prodotto deve soddisfare. Fungono quindi da guida per lo sviluppo, il testing e la valutazione finale del prodotto, garantendo la sua conformità alle esigenze e agli obiettivi prefissati.\\
Di seguito vengono elencati i vari tipi di requisito:\\
\begin{itemize} 
    \item \textbf{Funzionali}: questa tipologia di requisiti definisce le funzionalità del sistema, ovvero i servizi che esso deve essere in grado di fornire al fine di soddisfare le esigenze dell'utente. Un requisito funzionale descrive quindi il comportamento del sistema;
    \item \textbf{Qualità}: questa tipologia di requisiti definisce lo standard di qualità che il prodotto deve rispettare;
    \item \textbf{Vincolo}: questa tipologia di requisiti indica i limiti imposti dal capitolato che il prodotto deve rispettare;
    \item \textbf{Prestazionali}: questa tipologia di requisiti specifica le prestazioni che il sistema deve avere. 
\end{itemize}

\subsubsubsection{Codice dei requisiti}\\
A ciascun requisito è abbinato un codice univoco definito nel seguente modo:
\begin{center}
    \textbf{R[Tipologia]-[Numero]}
\end{center}
Dove \textbf{Tipologia} rappresenta il tipo di requisito e \textbf{[Numero]} è un identificativo progressivo.
