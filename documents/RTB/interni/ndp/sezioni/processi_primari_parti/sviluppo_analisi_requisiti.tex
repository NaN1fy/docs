\subsection{Sviluppo}

\subsubsection{Analisi dei Requisiti} \label{sec:analisi-rischi}

\subsubsubsection{Diagrammi UML dei casi d'uso}
%Attore
\begin{figure}[ht!]
    \centering
    \begin{tikzpicture}
        \umlactor{Attore}
    \end{tikzpicture}
    \caption{Rappresentazione UML di un attore}
    \label{fig:Rappresentazione UML di un attore}
\end{figure}

%Caso d'uso
\begin{figure}[ht!]
    \centering
    \begin{tikzpicture}
        \umlusecase[width=80]{UC-X Titolo caso d'uso}
    \end{tikzpicture}
    \caption{Rappresentazione UML di un caso d'uso}
    \label{fig:Rappresentazione UML di un caso d'uso}
\end{figure}

%Sistema
\begin{figure}[ht!]
    \centering
    \begin{tikzpicture}
        \begin{umlsystem}{Sistema}\end{umlsystem}
    \end{tikzpicture}
    \caption{Rappresentazione UML di un sistema}
    \label{fig:Rappresentazione UML di un sistema}
\end{figure}

%Sottocasi
\begin{figure}[ht!]
    \centering
        \begin{tikzpicture}
            \umlactor[x=0, y=-1]{Attore}
            \begin{umlsystem}[x=6, y=0]{UC-X Titolo caso d'uso}
                \umlusecase[x=0, y=0, width=80, name=UC-X1]{UC-X.1 \\ Titolo sottocaso d'uso}
                \umlusecase[x=0, y=-2, width=80, name=UC-X2]{UC-X.2\\ Titolo sottocaso d'uso}
            \end{umlsystem}
                \umlassoc{Attore}{UC-X1}
                \umlassoc{Attore}{UC-X2}
        \end{tikzpicture}
    \caption{Rappresentazione UML sottocasi d'uso}
    \label{fig:Rappresentazione UML sottocasi d'uso}
\end{figure}

%Relazioni tra attori e casi d'uso
\begin{figure}[ht!]
    \centering
    \begin{tikzpicture}
        \umlactor{Attore}
        \umlusecase[x=5, y=0, width=80, name=UC-X]{UC-X Titolo caso d'uso}
        \umlassoc{Attore}{UC-X}
    \end{tikzpicture}  
    \caption{Rappresentazione UML relazione di associazione}
    \label{fig:Rappresentazione UML relazione di associazione}
\end{figure}

%Relazioni tra attori
\begin{figure}[ht!]
    \centering
    \begin{tikzpicture}
        \umlactor[x=0, y=0]{Attore padre}
        \umlactor[x=0, y=-4]{Attore figlio}
        \umlinherit{Attore figlio}{Attore padre}
    \end{tikzpicture}  
    \caption{Rappresentazione UML generalizzazione tra attori}
    \label{fig:Rappresentazione UML generalizzazione tra attori}
\end{figure}

%Relazioni tra casi d'uso

%Inclusione
\begin{figure}[ht!]
    \centering
        \begin{tikzpicture}
            \umlactor[x=0, y=0]{Attore}
            \begin{umlsystem}[x=6, y=0]{Sistema}
                \umlusecase[x=0, y=0, width=80, name=UC-X]{UC-X \\ Titolo caso d'uso}
                \umlusecase[x=0, y=-3, width=80, name=UC-Y]{UC-Y \\ Titolo caso d'uso}
            \end{umlsystem}
                \umlassoc{Attore}{UC-X}
                \umlinclude{UC-X}{UC-Y}
        \end{tikzpicture}
    \caption{Rappresentazione UML relazione di inclusione}
    \label{fig:Rappresentazione UML relazione di inclusione}
\end{figure}

%Estensione
\begin{figure}[ht!]
    \centering
        \begin{tikzpicture}
            \umlactor[x=0, y=0]{Attore}
            \begin{umlsystem}[x=6, y=0]{Sistema}
                \umlusecase[x=0, y=0, width=80, name=UC-X]{UC-X \\ Titolo caso d'uso}
                \umlusecase[x=0, y=-3, width=80, name=UC-Y]{UC-Y \\ Titolo caso d'uso}
            \end{umlsystem}
                \umlassoc{Attore}{UC-X}
                \umlextend{UC-Y}{UC-X}
        \end{tikzpicture}
    \caption{Rappresentazione UML relazione di estensione}
    \label{fig:Rappresentazione UML relazione di estensione}
\end{figure}

%Generalizzazione
\begin{figure}[ht!]
    \centering
        \begin{tikzpicture}
            \umlactor[x=0, y=0]{Attore}
            \begin{umlsystem}[x=6, y=0]{Sistema}
                \umlusecase[x=0, y=0, width=80, name=UC-X]{UC-X \\ Titolo caso d'uso}
                \umlusecase[x=0, y=-3, width=80, name=UC-Y]{UC-Y \\ Titolo caso d'uso}
            \end{umlsystem}
                \umlassoc{Attore}{UC-X}
                \umlinherit{UC-Y}{UC-X}
        \end{tikzpicture}
    \caption{Rappresentazione UML relazione di generalizzazione}
    \label{fig:Rappresentazione UML relazione di generalizzazione}
\end{figure}