
\subsection{Fornitura}

\subsubsection{Introduzione}
Il \glossterm{processo} primario di fornitura, definito dallo standard \glossterm{ISO}/\glossterm{IEC} 12207:1995 (con ultima versione datata 2018), individua l'insieme di attività atte alla realizzazione di un prodotto software che soddisfi pienamente, a partire dalla formulazione della proposta al committente fino alla sua ultimazione e consegna, i requisiti e le necessità a cui deve rispondere, garantendo un percorso consono, strutturato ed efficiente, oltre che efficace.

\subsubsection{Attività}
Tali sono le attività definite nel processo primario di fornitura:

\begin{itemize}
    \item Acquisizione e preparazione: si individuano le necessità del cliente e vengono definiti eventuali requisiti, con associate analisi dei costi pecuniari e temporali (preventivo), e valutazione delle varie opzioni;
    \item Contrattazione: vengono negoziati tra fornitore e cliente i termini e le condizioni contrattuali, e vengono stipulati di comune accordo obiettivi, costi, tempistiche e responsabilitá di entrambe le parti;
    \item Pianificazione: vengono pianificate le attività e le stesure dei documenti utili alla realizzazione del progetto nel rispetto degli accordi fatti;
    \item Attuazione e controllo: vengono eseguite le attività pianificate, controllando regolarmente e consistentemente lo stato di avanzamento e il rispetto degli impegni prefissati conformemente al costo, alle tempistiche e ai requisiti accordati in precedenza;
    \item Revisione e valutazione: vengono effettuate revisioni periodiche e confronti con il cliente, per assicurare il corretto svolgimento del progetto secondo i termini prefissati e risolvere dubbi, incertezze e rischi occorsi;
    \item Completamento e consegna: una volta completato il progetto, viene consegnato al cliente il prodotto finale, secondo quanto stipulato nel contratto.
\end{itemize}

\subsubsection{Rapporti con la \glossterm{Proponente}}
L'azienda SyncLab si è resa disponibile per realizzare un canale di comunicazione costante e tempestivo, attraverso il quale è possibile colloquiare con i responsabili del progetto affidato al team per dubbi, consigli, organizzazione; tale canale, istituito sulla piattaforma \glossterm{Discord}, è accompagnato anche dalla possibilitá di comunicare con la Proponente via Gmail. È stato concordato con i responsabili di organizzare periodicamente delle riunioni di Stato Avanzamento Lavori (\glossterm{SAL}), fissate bisettimanalmente il venerdì alle 15:00 (con variazione di orario/data in base a eventuali inconvenienze riscontrate): tali riunioni, che coincidono con la fine dello \glossterm{Sprint} corrente e l'inizio di quello nuovo, consistono nell'esposizione dell'operato svolto dal team durante lo Sprint medesimo, con feedback da parte della controparte e successiva discussione di eventuali dubbi o problemi riscontrati e pianificazione dello Sprint successivo.
La \glossterm{proponente} mette anche a disposizione la possibilità di effettuare incontri di formazione riguardo alle le tecnologie proposte, tenuta da membri esperti dell'azienda e con struttura ``deep dive'', nelle quali chiarire dubbi e delucidare alcuni aspetti delle tecnologie utilizzate. 
Possono essere infine richieste delle riunioni speciali per risolvere tempestivamente possibili problematiche relative ai requisiti di progetto e ai vincoli imposti dalla proponente, o per aggiornare gli stessi.

\subsubsection{Documentazione fornita}
A corredo delle attività volte alla realizzazione del progetto, vengono stesi e resi disponibili all'azienda propronente SyncLab e ai committenti Prof. Vardanega e Prof. Cardin i seguenti documenti:

\subsubsubsection{Analisi dei Requisiti}\\
Il documento Analisi dei Requisiti illustra e descrive in dettaglio i casi d'uso e i requisiti del progetto, nonchè le \glossterm{funzionalità} che ci si aspetta che il prodotto finale abbia, sulla base degli obiettivi posti riguardo al progetto. Tale documento funge quindi anche da base preliminare per la progettazione del software, contenente:
\begin{itemize}
    \item Insieme dei casi d'uso, ovvero di tutti gli scenari possibili di utilizzo del software da parte degli utenti; per ogni caso d'uso si individuano e si analizzano: 
    \begin{itemize}
        \item Scenario
        \item Attori
        \item Azioni
    \end{itemize} 
    \item Lista dei requisiti e dei vincoli definiti e concordati con l'azienda proponente, volti alla realizzazione del prodotto finale.
\end{itemize}

\subsubsubsection{Dichiarazione Impegni}\\
Tale documento manifesta la volontà, da parte del team di sviluppo, di impegnarsi nella realizzazione del prodotto di un dato \glossterm{capitolato} (in particolare nel nostro caso di SyncCity, proposto dall'azienda SyncLab); tale documento fornisce inoltre:
\begin{itemize}
    \item Una suddivisione del monte ore e dei singoli ruoli tra i vari membri del gruppo;
    \item Una descrizione dei ruoli e del loro rapporto relativo alle specifiche del progetto;
    \item Una analisi preliminare dei rischi e dei dubbi sorti durante il processo di vaglio dei capitolati;
    \item Preventivo preliminare dei costi;
    \item Scadenza prefissata previsa;
\end{itemize}

\subsubsubsection{Glossario}\\
Ai fini di disambiguare ogni possibile dubbio o incertezza, nonchè per rispondere preventivamente e in maniera esaustiva a possibili domande di natura tecnico-linguistica/etimologica da parte dei membri del gruppo, o di chiunque abbia la possibilità e la necessità in futuro di leggere i documenti relativi a tale progetto, viene istituito un Glossario, contenente una definizione chiara e univoca di tutti quei termini rilevanti che portano con sè un significato specifico, onde evitare interpretazioni arbitrarie e fraintendimenti riguardo tali specifici concetti.

\subsubsubsection{Lettera di Presentatione}\\
La Lettera di Presentazione è un documento che accompagna la documentazione e il prodotto forniti all'azienda proponente durante le fasi di revisione del progetto, il cui scopo è quello di dare un veloce contesto intorno allo stato di avanzamento dei lavori (o del loro inizio, per quanto concerne la lettera di presentazione per i capitolati), e fornire una breve panoramica della documentazione redatta fino a quel momento. 

\subsubsubsection{Norme di Progetto}\\
La finalità del documento noto come Norme di Progetto è quello di redigere e definire un insieme di standard e regole rigurado ai processi e la loro normazione (o \glossterm{Way of Working}), da seguire imperativamente da parte del team di sviluppo, durante tutto il ciclo di vita del progetto, in modo da garantirne la \glossterm{qualità} e la conformità agli obiettivi e requisiti prefissati con il cliente. I contenuti sono divisi in:
\begin{itemize}
    \item Processi primari;
    \item Processi di supporto;
    \item Processi organizzativi;
    \item Metriche di qualità.
\end{itemize}

\subsubsubsection{Piano Di Progetto}\\
Il documento Piano di Progetto, redatto dal responsabile del team, si occupa di delineare la pianificazione e la gestione delle attività volte alla realizzazione del progetto; tra queste si ha in particolare:
\begin{itemize}
    \item Analisi dei rischi e delle eventuali problematiche sorte durante lo sviluppo del software, con conseguenti metodi per la loro possibile risoluzione o mitigazione;
    \item Modello di sviluppo adottato dal team, ovvero descrizione della metodologia e dell'organizzazione delle attività e dei processi utilizzate durante tutto lo sviluppo del software; 
    \item Preventivo dei costi stimati per ciascun periodo e relativo \glossterm{consuntivo} con conseguente:
    \item Aggiornamento costante sullo stato di avanzamento dei lavori, corredato di diagrammi di Gantt che descrivono lo stato di ogni singolo documento o attività relativa al progetto successivo al Sprint appena percorso.
\end{itemize}

\subsubsubsection{Piano di Qualifica}\\
Il documento Piano di Qualifica offre una panoramica dettagliata delle strategie di verifica e validazione adottate per assicurare la \glossterm{qualità} del prodotto e dei processi del progetto in questione. Costantemente aggiornato per riflettere l'evoluzione del progetto, si pone come documento dinamico e incrementale che illustra le pratiche per il controllo di qualità degli artefatti e dei processi, con particolare enfasi sulle metriche di valutazione del prodotto. Progettato per guidare l'adozione di processi mirati al miglioramento continuo, si compone di:
\begin{itemize}
    \item Qualità di processo: di fondamentale importanza è assicurarsi, attraverso controlli rigorosi e regolari, che i processi che supportano il prodotto siano ottimali. Applicato a tutte le attività, pratiche e metodologie, svoltesi durante l'intero ciclo di vita del software, si mira a integrare la qualità nel prodotto stesso;
    \item Qualità di prodotto: caratteristiche del prodotto che individuano la sua capacità di soddisfare le esigenze e le richieste del progetto. Concentrandosi su aspetti quali affidabilità, funzionalità, manutenibilità e usabilità, si assicura di garantire che il software non solo soddisfi le richieste del cliente e funzioni correttamente, ma che lo faccia conformemente agli standard di \glossterm{qualità} adottati.
    \item \glossterm{Test}: piano di testing che verrà utilizzato per garantire la correttezza finale del prodotto, comprendente test di unità, di integrazione, di sistema e di accettazione.
\end{itemize} 

\subsubsubsection{Valutazione Capitolati}\\
Il documento Valutazione dei Capitolati ha il puro scopo informativo di elencare le motivazioni per le quali si è scelto di proporsi per un determinato progetto tra i vari selezionati, soppesando i requisiti della proponente e ponderando le criticitá del progetto per ognuno di questi.  

\subsubsubsection{Verbali interni}\\
Viene riportata, sottoforma di verbale, la documentazione concernente le riunioni interne del team di sviluppo, attuate tramite la piattaforma comunicativa Discord a cadenza settimanale, il cui scopo é quello di fissare per iscritto:
\begin{itemize}
    \item Riassunto dell'andamento dell'ultimo periodo;
    \item Discussioni, dubbi, proposte, eventuali problemi riscontrati;
    \item Organizzazione per il prossimo periodo.
\end{itemize}

\subsubsubsection{Verbali esterni}\\
Viene riportata, sottoforma di verbale, la documentazione concernente le riunioni tenutesi con i rappresentanti dell'azienda proponente (SAL), attuate tramite la piattaforma comunicativa Google Meet ogni due settimane, il cui scopo è di fissare per iscritto:
\begin{itemize}
    \item Resoconto del lavoro svolto durante l'ultimo Sprint con feedback della controparte;
    \item Discussioni, dubbi, proposte, eventuali problemi riscontrati;
    \item Organizzazione e obiettivi per il prossimo Sprint.
Al contrario dei verbali interni, i verbali esterni verranno, una volta stesi, inviati telematicamente alla Proponente per una loro verifica e convalida del documento.
\end{itemize} 

\subsubsection{Strumenti}
Di seguito viene riportata una lista di strumenti utilizzati per i processi di fornitura:
\begin{itemize}
    \item \glossterm{Discord}, come piattaforma per comunicare da remoto tra membri del gruppo e per raccogliere informazioni e dati utili, come link, appunti et al., oltre che per comunicare in maniera asincrona con l'azienda proponente;
    \item \glossterm{Github}, come piattaforma dove condividere il codice del prodotto e i documenti da stendere;
    \item Gmail, come mezzo aggiuntivo e alternativo per la comunicazione asincrona con l'azienda;
    \item Google Calendar, utilizzato dalla Proponente per segnare le date dei vari incontri fissati;
    \item Google Meet, come piattaforma dove vengono svolti gli incontri con l'azienda e i responsabili del progetto; 
    \item Microsoft Excel, dove vengono inseriti e organizzati i dati relativi a preventivi e consuntivi dei vari periodi; 
    \item \glossterm{Telegram}, come piattaforma per comunicare in maniera asincrona tra membri del gruppo. 
\end{itemize}
