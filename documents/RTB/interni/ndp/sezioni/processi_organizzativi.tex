\subsection{Gestione di Processo}
\subsubsection{Coordinamento}
\subsubsubsection{Comunicazioni interne}\\
Le comunicazioni interne avvengono principalmente tra membri del gruppo di pari livello. Si rimanda alla sezione \hyperref[sec:infrastruttura]{\ref{sec:infrastruttura} Infrastruttura} per approfondire gli strumenti utilizzati.
\begin{itemize}
  \item{\textbf{Discord}}
  \begin{itemize}
    \item \textbf{Informazioni}: comunicazioni semi-formali per la condivisione di risorse.
    \item \textbf{Canali Testuali}: comunicazioni semi-formali per la condivisione di appunti sugli incontri.
    \item \textbf{Canali Vocali}: comunicazioni semi-formali per le riunioni interne; informali per argomenti non inerenti al progetto.
  \end{itemize}
  \item{\textbf{Telegram}}
  \begin{itemize}
    \item{\textbf{Chat di gruppo:} comunicazioni semi-formali, brevi, inerenti al progetto}
    \item{\textbf{Chat individuali:} comunicazioni informali, inerenti al progetto}
  \end{itemize}
  \item{\textbf{Google Calendar}}
  \begin{itemize}
    \item{Comunicazioni informali, brevi, inerenti all'evento di calendario relativo.}
  \end{itemize}
\end{itemize}
\subsubsubsection{Comunicazioni esterne}\\
Le comunicazioni esterne con il committente e proponente vengono considerate ovviamente di importanza maggiore, dunque trattate col rispetto dovuto.\\
Il registro utilizzato è esclusivamente formale, cercando di adottare vocaboli consoni e concisi.
\begin{itemize}
  \item{\textbf{Discord}}
  \item{\textbf{Google Meet}}
  \item{\textbf{Google Mail}, indirizzo e-mail condiviso \texttt{nan1fyteam.unipd@gmail.com}}
\end{itemize}
\subsubsubsection{Riunioni interne}\\
Le riunioni interne avvengono tra i membri del gruppo, settimanalmente, con cadenza al giovedì ore 15 esclusi casi eccezionali, con durata media di 1 ora / 1 ora e mezza.\\
In caso di mancata partecipazione da parte di uno dei membri, oltre alla sezione di appunti presente nei canali di comunicazione, è comunque possibile accedere ai verbali degli incontri.\\
I verbali vengono redatti dai membri a rotazione, così come la verifica di esse.\\
Ogni riunione ha un tema principale, "l'ordine del giorno", su cui implicitamente si basano le discussioni principali.
Oltre a ciò, generalmente le riunioni avvengono nella seguente modalità:
\begin{itemize}
  \item{Discussione da parte di ognuno sulle proprie task assegnate con retrospettiva in merito (quel che si è fatto e quel che c'è da fare)}
  \item{Discussione su temi venuti fuori durante lo svolgimento delle task}
  \item{Controllo generale della task board e delle scadenze imminenti}
\end{itemize}
\subsubsubsection{Riunioni esterne}\\
Le riunioni esterne avvengono prevalentemente tra committente e proponente con cadenza bisettimanale/trisettimanale. Vengono utilizzate come incontri di \glossterm{SAL}, e come deadline degli sprint.\\
La durata media è intorno ai 40 minuti, ed è compito dei membri del gruppo esporre in modo conciso e preparato lo stato di avanzamento dei lavori ed eventuali dubbi sorti.\\
Vengono successivamente redatti i relativi verbali, consegnati all'azienda per poter essere validati, approvati e firmati.
\subsubsubsection{Reperibilità}\\
Ciascun componente del gruppo gode dell'autonomia di pianificare il proprio orario di lavoro individuale in modo asincrono, conformemente agli obblighi accademici, personali e alle disposizioni stabilite nel programma preventivamente concordato.
In un accordo mirato a bilanciare l'efficacia della comunicazione asincrona con la tutela del tempo personale, i partecipanti si impegnano a garantire la propria disponibilità per questioni inerenti al progetto didattico durante il seguente intervallo orario: dal lunedì al venerdì, dalle 9:00 alle 13:00 e dalle 15:00 alle 19:00. Qualsiasi modifica agli orari o ai giorni di disponibilità può essere concordata previamente tra i membri del gruppo. È importante sottolineare che questo intervallo di disponibilità non deve essere interpretato come tempo di lavoro attivo, ma un limite temporale entro il quale i membri si impegnano a essere raggiungibili per eventuali necessità connesse al progetto.
\subsubsection{Pianificazione}
\subsubsubsection{Ruoli del Progetto}\\
I ruoli sono stati distribuiti equamente tra tutti i componenti del gruppo e saranno:
\begin{itemize}
	\item \textbf{Responsabile del progetto:} Guida il team nel rispetto delle scadenze, nell'allocazione delle risorse e nella pianificazione generale, assicurando che il progetto proceda in modo efficiente e soddisfi gli obiettivi.
	\begin{itemize}
    \item{Pianifica lo sprint definendo le task relative, definendo il preventivo ore e costì}
    \item{Calcola il consuntivo delle ore e costi alla fine dello sprint}
    \item{Tiene traccia dello stato generale di progresso dell'intero progetto}
    \item{Fa da intermediario tra il gruppo e l'azienda proponente}
  \end{itemize}
	\item \textbf{Amministratore:} Gestisce l'infrastruttura e le risorse necessarie per il progetto, inclusi gli strumenti e le tecnologie che definiscono il modo di lavorare del team.
	\begin{itemize}
    \item{Gestisce l'infrastruttura del progetto e i suoi strumenti}
    \item{Automatizza i processi e ne individua punti di miglioramento}
    \item{Si occupa dell'effettiva redazione dei documenti che definiscono il \glossterm{way of working} del gruppo}
	\end{itemize}
  \item \textbf{Programmatore:} Responsabile della scrittura del codice seguendo le specifiche del progetto e traducendo i requisiti in un'applicazione funzionante
	\begin{itemize}
    \item{Si occupa della stesura del codice in conformità ai requisiti e alla sua manutenibilità}
    \item{Scrive i test per il codice prodotto}
    \item{Si occupa della documentazione relativa alla comprensione del codice, sia da parte dell'utente che da parte del programmatore}
  \end{itemize}
  \item \textbf{Progettista:} Definisce l'architettura del software, pianifica la sua struttura e l'organizzazione nel dettaglio.
	\begin{itemize}
    \item{Sviluppa l'architettura in conformità ai requisiti e alla sua manutenibilità, al minimo livello di dipendenze possibili}
    \item{Approfondisce le conoscenze e strumenti tecnici utili allo sviluppo}
  \end{itemize}
  \item \textbf{Verificatore:} Garantisce la qualità del software eseguendo test e controlli per assicurare il corretto funzionamento e il rispetto degli standard di qualità.
	\begin{itemize}
    \item{Verifica il livello atteso e il rispetto della qualità in ambito tecnico}
    \item{Verifica il livello atteso e il rispetto della qualità in ambito funzionale}
    \item{Verifica il livello atteso e il rispetto della qualità in ambito organizzativo}
  \end{itemize}
  \item \textbf{Analista:} Si concentra sull'analisi dei requisiti, aiuta a definire le funzionalità del software e si assicura di comprendere i bisogni del cliente.
  \begin{itemize}
    \item{Valuta il dominio applicativo delle richieste del proponente}
    \item{Scompone le richieste ed esigenze del proponente in sotto attività così da poter essere risolte individualmente e/o parallelamente}
  \end{itemize}
\end{itemize}
\subsubsubsection{Gestione delle task}\\
Durante la fase di Sprint Planning vengono definite tutte le attività, le quali saranno poi associate alle relative task.\\
Ognuna di esse viene assegnata ad almeno un membro, il quale si occuperà del suo ciclo di vita. Vengono assegnate in modo da poter essere svolte il più parallelamente e asincronamente possibile.\\
Per le task di processi primari (sviluppo) e di supporto (documentazione) viene utilizzato GitHub.
Il ciclo di vita di una task è come segue:
\begin{itemize}
  \item{Creazione: la task definita viene aperta come issue su GitHub.}
  \item{Assegnazione: la task viene assegnata ad uno o più membri del gruppo.}
  \item{Completamento: la task viene completata, prevalentemente su un branch distinto dal principale.}
  \item{Pull request: viene fatta una pull request del branch di sviluppo dell'attività completata, collegando la richiesta alla relativa issue.}
  \item{Verifica: almeno due verificatori effettuano il controllo qualità.}
  \item{Accettazione: quando la fase di verifica è conclusa senza intoppi, viene approvata la pull request, si chiude la issue relativa e viene cancellato il branch relativo all'attività.}
\end{itemize}
Per ulteriori dettagli si rimanda alla sezione \hyperref[sec:infrastruttura]{\ref{sec:infrastruttura}}.\\
La dimensione e importanza dell'attività dipendono dal processo (primario, di supporto o organizzativo) di cui fa parte. All'apertura della sua task relativa viene valutato il tempo ragionevole di svolgimento e 
La tracciabilità dei cambiamenti, eventuali commenti e il generale stato dell'attività sono presenti nella piattaforma, nelle pagine della issue collegata.
\subsubsubsection{Metodo di Lavoro}\\
Per lo svolgimento dell'attività il gruppo ha scelto di adottare la modalità Agile SCRUM.\\
Ciò permette la suddivisione del tempo di lavoro in intervalli di tempo (Sprint), in modo da definire attività da svolgere, quel che è stato fatto e si è fatto, e avere scadenze tangibili durante lo svolgimento del progetto.
Ogni sprint, della durata media di almeno due settimane, prevede le seguente fasi:
\begin{itemize}
  \item{\textbf{Sprint Planning:} Pianificazione dello sprint, in concomitanza con il suo inizio. Gli incontri di SAL vegnono utilizzati come punto di riferimento di conclusione dello Sprint e inizio del successivo.}
  \begin{itemize}
    \item{Discussione sui nuovi obiettivi post incontro di SAL da completare.}
    \item{Discussione di quel che rimane da fare dal precedente Sprint e con quale priorità.}
    \item{Definizione di obiettivi concreti e di issue relative, definendole su GitHub.}
    \item{Preventivo ore da parte di ogni membro, in base alle attività da svolgere e i ruoli da assegnare; preventivo dei costi in base alle ore dichiarate.}
  \end{itemize}
  \item{\textbf{Sprint Review:} Revisione dello Sprint nel suo ultimo giorno.}
  \begin{itemize}
    \item{Consuntivo ore secondo la produttività individuale, sia in difetto che in eccesso; consuntivo costi in base alle ore dichiarate.}
    \item{Discussione sugli obiettivi raggiunti e non di ogni membro. Gli obiettivi non raggiunti verranno presentati allo Sprint successivo.}
  \end{itemize}
  \item{\textbf{Sprint Retrospective:} Retrospettiva effettivamente conclusiva dello Sprint, valutandone l'andamento generale. Viene valutato ciò che è andato positivamente e cosa negativamente, così da poter avere delle linee guida sul come proseguire al meglio.}
  \begin{itemize}
    \item{Good: ciò che effettivamente è andato bene durante lo svolgimento dello Sprint.}
    \item{To Imrove: ciò che invece va migliorato per i Sprint successivi.}
  \end{itemize}
\end{itemize} 
\subsection{Infrastruttura}\label{sec:infrastruttura}
Fanno parte dell'infrastruttura organizzativa tutti gli strumenti che permettono al gruppo di attuare in modo efficace ed efficiente i processi organizzativi. In particolare tali strumenti permettono la \textbf{comunicazione}, il \textbf{coordinamento} e la \textbf{pianificazione}.
\subsubsection{Strumenti}
\subsubsubsection{GitHub}\\
È il principale servizio di hosting della repository di gruppo e di controllo della versione distribuita. 
\\\\
Generalmente il workflow adottato dal gruppo è il GitHub Flow, che sinteticamente segue il seguente schema:
\begin{itemize}
  \item riallineamento della repository locale con quella remota;
  \item creazione di un branch locale su cui effettuare le modifiche;
  \item push del branch locale verso repository remota;
  \item creazione di una pull request;
  \item verifica da parte di due membri del gruppo e successivo merge del branch con le modifiche;
  \item eliminazione del branch utilizzato dalla repository remota.
\end{itemize}
Per i dettagli consultare la documentazione ufficiale: 
\begin{itemize}
  \item \underline{https://docs.github.com/en/get-started/quickstart/github-flow}
\end{itemize}
\medskip
Viene utilizzato anche il sistema di \textbf{project management} fornito.\\\\
La \textbf{board principale} è divisa nelle seguenti liste:
\begin{itemize}
  \item \textbf{Todo}: contiene task da fare in base allo sprint corrente. Durante la fase di \texttt{Sprint Planning} sono state definite ed ognuna preassegnata ad almeno un membro, il quale si occuperà del suo ciclo di vita.
  \item \textbf{In Progress}: contiene task in corso. Ogni task che non sia relativa alla stesura dei verbali viene assegnata ad almeno due membri, così che possano lavorarci asincronamente
  \item \textbf{Verify}: contiene task completate che necessitano di verifica. Qualsiasi membro del gruppo può autonomamente diventare verificatore di una task aggiungendosi a essa, con la precondizione che non sia la stessa persona che l'ha svolta. Questa fase avviene in parallelo con la relativa \texttt{Pull Request}, dove ulteriori due membri devono verificare ed approvare lo svolgimento della task. Nel caso di modifiche lievi sarà discrezione del verificatore apportarle; nel caso di modifiche più importanti sarà premura del \texttt{Responsabile} valutare la situazione.\\In generale, in caso di esito negativo la \texttt{Pull Request} viene annullata e si ritorna in fase "In Progress". Nel caso invece sia conforme viene approvata e spostata in fase "Done".
  \item \textbf{Done}: contiene task completate, verificate e accettate. In prossimità del prossimo sprint, vengono archiviate le task completate relative a sprint precedenti così da non sovraffollare la board.
\end{itemize}
\medskip
Per quanto riguarda le \textbf{task}, ognuna è costituita da:
\begin{itemize}
  \item \textbf{titolo}: breve descrizione sintentica su cosa consiste la task.
  \item \textbf{descrizione}: opzionale e breve per dettagli importanti;
  \item \textbf{membro/i}: elementi del gruppo coinvolti nella task;
  \item \textbf{milestone}: a quale milestone fanno riferimento.
\end{itemize}
Si vuole far notare che, nonostante tutto, le task non vengono mai cancellate ma soltanto archiviate. \\Per navigare più facilmente nella bacheca è possibile impostare dei filtri, ad esempio per membro o milestone.
\subsubsubsection{Discord}\\
Principale strumento di \textbf{comunicazione interna sincrona} e \textbf{asincrona}. Vengono utilizzati 3 categorie di canali:
\begin{itemize}
  \item \textbf{Informazioni}: prevalentemente utilizzato per la condivisione di risorse.
  \item \textbf{Canali Testuali}: utilizzato per la condivisione di appunti su incontri interni ed esterni, e simili.
  \item \textbf{Canali Vocali}: comunicazioni vocali tra i membri del gruppo, con possibilità di condivisione schermo.
\end{itemize}
\medskip
Viene utilizzato questo strumento anche per la \textbf{comunicazione asincrona} con l'azienda proponente nel loro canale personale.
\subsubsubsection{Telegram}\\
Principale strumento di \textbf{comunicazione interna testuale asincrona}. Viene utilizzato in due modalità:
\begin{itemize}
  \item \textbf{Gruppo}: chat condivisa utilizzata, con parsimonia, per comunicazioni rivolte a tutti i membri;
  \item \textbf{Individuale}: ogni membro del gruppo può essere contattato singolarmente.
\end{itemize}
\medskip
\subsubsubsection{Google Calendar}\\
Calendario condiviso del gruppo utilizzato per tenere traccia di:
\begin{itemize}
  \item \textbf{Meeting Esterni}: con proponente o committente;
  \item qualsiasi altra attività o evento che può essere collocato in un tempo specifico.
\end{itemize}
\subsubsubsection{Google Drive}\\
Strumento utilizzato come:
\begin{itemize}
  \item \textbf{directory condivisa} dai membri del gruppo per documenti temporanei o non ufficiali;
  \item accesso alla \textbf{suite Google}: Docs, Sheets, Slides.
\end{itemize}
\subsubsubsection{Google Meet}\\
Strumento di videochiamata utilizzato principalmente per la comunicazione esterna con committente e proponente.
\subsubsubsection{Google Mail}\\
Utilizzo dell'indirizzo e-mail condiviso \texttt{nan1fyteam.unipd@gmail.com} per le \textbf{comunicazione esterna} come gruppo con i proponenti e il committente.

\subsection{Miglioramento}
Nel corso della redazione della documentazione e dello sviluppo del software, il gruppo si impegnerà a perseguire un miglioramento costante delle attività, con l'obiettivo di evitare di ripetere errori precedentemente commessi e di fornire soluzioni ottimali, con un riguardo particolari sui temi quali:
\begin{itemize}
  \item{Organizzazione}
  \item{Ruoli}
  \item{Strumenti di lavoro}
\end{itemize}
\subsection{Formazione}
Al fine di promuovere un ambiente di lavoro asincrono efficiente e equo, garantendo un progresso organizzato e uniforme senza lasciare alcun membro indietro, si richiede a ciascun componente del gruppo di assumersi autonomamente la responsabilità di colmare eventuali lacune relative agli strumenti e alle tecnologie impiegate per la documentazione e lo sviluppo del progetto. Questo può avvenire mediante lo studio individuale o, alternativamente, con la condivisione delle proprie conoscenze con gli altri membri al fine di accelerare il processo di apprendimento.
\\Di seguito sono elencati gli strumenti e le tecnologie utilizzati, insieme ai principali riferimenti adottati dal gruppo:
\begin{itemize}
  \item{\textbf{LaTeX:}} https://www.overleaf.com/learn.
  \item{\textbf{Git:}} https://docs.github.com/en/get-started/using-git/about-git.
  \item{\textbf{GitHub:}} https://docs.github.com.
  \item{\textbf{GitHub Flow:}} https://docs.github.com/en/get-started/quickstart/github-flow.
\end{itemize}
