\subsection{Documentazione}
\subsubsection{Introduzione}
Viene denominato ``documentazione" l'insieme di tutti i documenti a corredo del progetto software, che forniscono tutte le informazioni e i dettagli utili a sviluppatori, utenti e distributori riguardanti il prodotto. Utilizzata prevalentemente dal team di sviluppo per facilitare organizzazione e svolgimento delle attivitá durante il ciclo di vita del software, essa viene mantenuta coerente grazie al tracciamento di tutti i processi e attivitá che la coinvolgono, in maniera tale da migliorarne costantemente lo stato e semplificare la manutenzione.
\subsubsection{Lista dei documenti}
Di seguito viene elencato l'insieme dei documenti redatti:
\begin{itemize}
    \item Analisi dei Requisiti;
    \item Dichiarazione Impegni;
    \item Glossario;
    \item Lettera di Presentatione;
    \item Norme di Progetto;
    \item Piano di Progetto;
    \item Piano di Qualifica;
    \item Valutazione Capitolati;
    \item Verbali Esterni;
    \item Verbali Interni;
\end{itemize}

\subsubsection{Documentation as Code}
Riguardo la documentazione viene adottata la \glossterm{Way of Working} del ``Documentation as Code'', che prevede la stesura, la gestione e la distribuzione dei documenti utilizzando pratiche e strumenti solitamente utilizzati nello sviluppo di codice o, più in generale, di software. I punti principali di tale Way of Working sono:
\begin{itemize}
    \item Automazione;
    \item Collaborazione;
    \item Distribuzione;
    \item integrazione incrementale;
    \item \glossterm{Versionamento};
\end{itemize}

\subsubsection{Ciclo di vita dei documenti}
Ogni documento viene redatto e manutenuto secondo questo \glossterm{processo}:
\begin{enumerate}
    \item Viene creato un \glossterm{branch} per la stesura del documento nella \glossterm{repository} docs: se il file è un verbale, allora viene clonato il template relativo nel branch;
    \item Vengono assegnati i membri del gruppo che dovranno redigere il documento; 
    \item Ogni volta che una sezione o comunque parte del documento viene completata, viene aggiunto nel registro delle modifiche del file una riga in coda, nel formato <versione x.y.z>,<data>,<oggetto>,<membro del team>,<ruolo>;  
    \item Viene poi caricata nella repository tramite commit sul branch adeguato;
    \item Può venire richiesta una pull request per fondere le modifiche fatte fino a quel momento nel branch "main", in maniera tale da costituire una \glossterm{milestone} per i singoli documenti, senza ovviamente chiudere il branch;
    \item Una volta completato il documento, viene richiesta la verifica del contenuto e della forma da parte di almeno due membri del gruppo, prima di procedere con il merge finale e la chiusura della branch; 
    \item A richiesta dei verificatori, possono essere ripetute parti del processo, in vista di una ottimizzazione del documento;
    \item Una volta dichiarato finito il documento, la sua branch viene chiusa e la rispettiva \glossterm{issue} viene marcata come chiusa.
    \item A discrezione del team di sviluppo, la branch di un documento che si ritiene debba essere aggiornato ulteriormente, manutenuto o modificato, previa valida motivazione, può essere riaperto, e il processo ripreso. 
\end{enumerate}
Il ciclo di vita dei documenti e delle loro issue consiste nei seguenti quattro stati:
\begin{itemize} 
    \item Stato ``to-do": il documento deve essere iniziato;
    \item Stato ``in progress": il documento sta venendo scritto, completato o aggiornato;
    \item Stato ``verify'': il documento è in attesa di essere posto sotto scrutinio;
    \item Stato ``done'': il documento è completo e verificato;
\end{itemize}
Un documento che deve passare il vaglio per una milestone principale, quale \glossterm{RTB} o \glossterm{PB} (o eventi straordinari che assumono la stessa ``gravitas"), viene controllato e verificato un'ultima volta, in separata sede, da tutti i membri del team.
In corrispondenza di questo ultimo controllo, il documento subisce uno scatto di versione (da 0.x.y a 1.x.y, o 2.x.y, e cosí via) che determina la versione definitiva di quel periodo.

\subsubsection{Template Verbali}
Per semplificare, velocizzare e standardizzare la stesura del documento piú comune prodotto dal team, ovvero i verbali, indipenddentemente dal loro essere esterni o interni, `(e) stato sviluppato un template in \glossterm{\LaTeX}, con il quale essi vengono omologati e verificati molto facilmente.

\subsubsection{Nomenclatura}
Per riferirsi a un documento prodotto dal team, la formulazione del nome è la seguente: <nome\textunderscore del\textunderscore file> + <versione> aggiunta in automatico. Per i verbali viene utilizzato invece una nomenclatura del tipo: <VE> (se esterno)/<VI> (se interno)\textunderscore <data\textunderscore verbale> con data verbale nel formato ``yyyy\textunderscore mm\textunderscore dd", dove yyyy indica l'anno, mm il mese e dd il giorno. 

\subsubsection{Versionamento}
Il versionamento, necessario per il tracciamento delle modifiche dei documenti, è strutturato nel formato \textit{X.Y.Z}, come convenzione della versione dei documenti, dove:
\begin{itemize}
    \item \textbf{X:} cifra che viene incrementata quando avviene un rilascio, che nel caso del progetto corrisponde ai raggiungimenti di RTB, PB e CA;
    \item \textbf{Y:} rappresenta un'aggiunta o modifica sostanziale, come ad esempio l'aggiunta di una sezione;
    \item \textbf{Z:} indica una piccola modifica, come per esempio la correzione di errori o aggiunte di piccole dimensioni.
\end{itemize}
Un documento parte sempre dalla versione \textit{0.0.0}. \\
Ogni modifica di un numero di versione, comporta l'azzeramento di tutti i numeri alla sua destra.

\subsubsection{Struttura}
Ogni file segue una rigorosa struttura delle pagine, organizzata come segue.
\subsubsubsection{Prima Pagina}\\
Nella prima pagina di ogni documento sono presenti, partendo dall'alto e proseguendo verso destra:
\begin{itemize}
    \item Destinatari del documento;
    \item Redattori;
    \item Verificatori;
    \item Nome del file; 
    \item Logo del team;
    \item Logo dell'università di Padova;
    \item Mail ufficiale del team.
\end{itemize}
Al contenuto di ogni pagina, esclusa la prima, precede un'intestazione con nome del file a sinistra e logo del team a destra, mentre segue un piè di pagina contenente semplicemente il numero della pagina. 
\subsubsubsection{Registro delle modifiche}\\
Alla prima pagina segue la pagina contenente il changelog del documento; esso è strutturato sottoforma di tabella, ogni riga della quale contiene i seguenti dati:
\begin{itemize}
    \item Versione del documento;
    \item Data dell'evento;
    \item Breve descrizione dell'evento;
    \item Membri che hanno partecipato all'evento;
    \item Ruolo dei membri (redattore o verificatore).
\end{itemize}
Nel caso di una revisione per milestone prinicipale (RTB, PB et similia), essendo che tutti i membri del gruppo vi partecipano, non viene specificato né autore né ruolo. 
\subsubsubsection{Indice}\\
Nella pagina successiva al Registro Modifiche, viene reso disponibile, grazie alla scrittura in \LaTeX, un indice interattivo che segna ogni sezione e sottosezione del contenuto del documento, comprendente la possibilità di navigare nel file tramite click sulla sezione interessata.
\subsubsubsection{Contenuto del documento}\\
Le restanti pagine del documento sono riservate al contenuto del documento stesso, che non segue una precisa organizzazione trasversale tra i documenti, data la loro eterogeneità di contenuti e forma.
Unica eccezione sono i verbali, interni ed esterni, che presentano una struttura omologata e omogenea:
\begin{itemize}
    \item Informazioni generali, riguardanti l'incontro e i partecipanti all'incontro;
    \item Ordine del giorno, in cui si descrive brevemente lo scopo dell'incontro;
    \item Sintesi dell'incontro, in cui si descrive ciò che è stato detto durante l'incontro;
    \item Conclusioni, in cui si espone le scelte decise durante l'incontro;
    \item Attività da svolgere, in cui si dichiara gli impegni e le attività da svolgere decise sulla base delle conclusioni tratte dall'incontro.
\end{itemize}
Nell'ultima pagina dei verbali viene poi apposta, nel corrispettivo spazio dedicatogli, la firma da parte di un rappresenante della \glossterm{Proponente}, da porre dopo la visione e conferma di tale documento della controparte.

\subsubsection{Convenzioni stilistiche}
\subsubsubsection{Registro modifiche}\\
All'interno della sezione ``Descrizione '', nel registro delle modifiche di un documento, ogni entry deve terminare con un punto.
\subsubsubsection{Elenchi puntati}\\
Le voci di ogni elenco, salvo eccezioni, iniziano con lettera maiuscola e terminano con punto e virgola `;', eccetto l'ultima voce che termina con punto normale `.'.
\subsubsubsection{Descrizione immagini}\\
Ogni immagine o tabella presenta una descrizione associata, utile a fornire una breve descrizione o spiegazione del contenuto visivo.
\subsubsubsection{Formato delle date}\\
Viene adottato il formato “yyyy-mm-dd”, dove yyyy indica l'anno (4 cifre), mm il mese (2 cifre), e dd il giorno (2 cifre).
\subsubsubsection{Abbreviazioni}\\
Numerosi sono i casi in cui vengono utilizzate sigle relative a entità e artefatti relativi al progetto. Tra questi ci sono:
\begin{itemize}
    \item \textbf{Concetti chiave del progetto}:
        \begin{itemize}
            \item CA: Customer Acceptance;
            \item PoC: \glossterm{Proof of Concept};
            \item MVP: Minimum Viable Product;
            \item PB: Product \glossterm{Baseline};
            \item RTB: Requirements and Technology Baseline.
        \end{itemize}
    \item \textbf{Ruoli del progetto}:
        \begin{itemize}
            \item Am: Amministratore;
            \item An: Analista;
            \item Pr: Programmatore;
            \item Pt: Progettista;
            \item Re: Responsabile;
            \item Ve: Verificatore.
        \end{itemize}
\end{itemize}

\subsubsection{Strumenti}
Di seguito vengono elencati tutti gli strumenti unitariamente utilizzati dal team per la stesura, scrittura, manutenzione e verifica dei documenti:
\begin{itemize}
    \item \textbf{\glossterm{GitHub}}: piattaforma di hosting di codice sorgente, utilizzato per la condivisione dei file, della loro verifica e del loro versionamento, nonché per la automatizzazione di taluni processi, in conformità al concetto di ``Documentation as Code";  
    \item \textbf{\LaTeX}: linguaggio di markup, molto famoso in ambito accademico/scientifico, utilizzato per la stesura dei documenti in maniera pulita ed efficiente.
\end{itemize}





