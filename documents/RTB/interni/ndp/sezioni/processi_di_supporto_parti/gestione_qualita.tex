L'obiettivo è assicurarsi che i processi e il prodotto rispettino le esigenze del cliente e lo facciano con il massimo livello di qualità possibile, monitorando anche futuri progressi attraverso verifiche retrospettive.
\subsubsection{Piano di Qualifica}
Il documento \textit{Piano di Qualifica} è fondamentale per il completamento degli obiettivi di questo processo. 
\\Il documento comprende:
\begin{itemize}
    \item Fissare gli obiettivi di qualità;
    \item Definire le metriche di visione quantitativa;
    \item Definire test di qualità e funzionamento e relativa documentazione;
    \item Avere una visione dello stato attuale del prodotto e del progetto;
    \item Fornire margine di retrospettiva ai fini di miglioramento.
\end{itemize}
\subsubsection{Testing}
Il documento \textit{Piano di Qualifica} fornisce obiettivi di qualità sia del processo che del prodotto. Le metriche relative garantiscono la verifica sugli aspetti di accessibilità; i test invece garantiscono la qualità generale del software. 
\\Le categorie interessate sono le seguenti:
\begin{itemize}
	\item \textbf{Test di unità:} si verifica il corretto funzionamento delle unità componenti il \glossterm{sistema}. Un’unità rappresenta un elemento indivisibile e indipendente del \glossterm{sistema}; 
	\item \textbf{Test di integrazione:} si verifica il corretto funzionamento di più unità che cooperano per svolgere uno specifico compito (tali unità devono certamente aver superato i loro test di unità precedentemente);
	\item \textbf{\glossterm{Test di sistema}:} si verifica il corretto funzionamento del \glossterm{sistema} nella sua interezza. I requisiti funzionali obbligatori, di vincolo, di \glossterm{qualità} e di prestazione, precedentemente concordati con la Proponente mediante stipulazione del contratto, devono essere soddisfatti per intero;
	\item \textbf{Test di accettazione:} si verifica il soddisfacimento della Proponente rispetto al prodotto software. Il loro superamento permette di procedere con il rilascio del prodotto.
\end{itemize}
Per ogni test vengono indicati i risultati del test di qualità del processo e della qualità del prodotto software.
\subsubsection{Metriche}
Il documento \textit{Piano di Qualifica} fornisce le metriche da applicare all'esecuzione dei test di qualità.
\\Ogni metrica è codificata come segue:
\begin{center}
    \textbf{M[Tipologia metrica]-[Sigla identificativa Metrica]}
\end{center}
in particolar modo:
\begin{itemize}
	\item{\textbf{Tipologia metrica}: l'effettiva categoria di appartenenza della \glossterm{metrica}}
	\begin{itemize}
    \item \textbf{PC}: Processo;
		\item \textbf{PD}: Prodotto.
	\end{itemize}
	\item{\textbf{Sigla identificativa metrica}: sigla di indentificazione della specifica metrica}
\end{itemize}
\clearpage
\subsubsubsection{Gestione della qualità} \label{sec:gestione_qualita}
\begin{table}[H]	
	\centering
	\begin{tabular}{p{2cm} p{3cm} p{6cm} p{4cm}}
		\toprule
		\textbf{Metrica}& \textbf{Sigla} & \textbf{Descrizione} & \textbf{Formula} \\
		\midrule
		MPC-QMS & Quality Metrics Satisfied (QMS) & Numero di metriche di qualità soddisfatte. & - \\
		\bottomrule
	\end{tabular}
	\caption{Valori accettabili e ottimi per ogni metrica riguardante il processo di gestione della qualità.}
	\label{table:Valori accettabili e ottimi per ogni metrica riguardante il processo di gestione della qualità.}
\end{table}

\subsubsection{Aspettative}
A seguito di questo processo, le aspettative attese sono le seguenti:
\begin{itemize}
    \item{Ottima qualità del prodotto realizzato}
    \item{Ottima qualità dei processi di sviluppo e di comunicazione}
    \item{Buona visione quantitativa dello stato di avanzamento dei lavori}
    \item{Test con alta frequenza e predicibili}
    \item{Miglioramento generale costante}
    \item{Soddisfazione delle aspettative del \glossterm{proponente}}
\end{itemize}
