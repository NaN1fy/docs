\subsection{Gestione della configurazione}

\subsubsection{Introduzione}

Attuato durante tutto il ciclo di vita di un progetto software, il processo di gestione della configurazione norma il tracciamento e il controllo delle modifiche a documenti e codice prodotti, in maniera da rendere organizzata la procedura di modifica di tali artefatti e la loro evoluzione.
% Expand the introduction a bit

\subsubsection{Tecnologie Utilizzate}

\begin{itemize}
    \item \textbf{Git:} software utilizzato per il controllo di versione dei Configuration Item;
    \item \textbf{GitHub:} piattaforma web per il controllo di versione utilizzata per l'hosting e coordinamento delle operazioni. Offre anche un Issue Tracking System.
\end{itemize}

\subsubsection{Versionamento}

Il versionamento è necessario per il tracciamento delle modifiche che avvengono ai documenti. Grazie al versionamento è possibile visualizzare le modifiche che un file ha subito e, nel caso fosse necessario, far regredire il documento ad una versione precedente. \\
Il gruppo utilizza il formato \textit{X.Y.Z} come convenzione della versione dei documenti, dove:
\begin{itemize}
    \item \textbf{X:} cifra che viene incrementata quando avviene un rilascio, che nel caso del progetto corrisponde ai raggiungimenti di RTB, PB e CA;
    \item \textbf{Y:} rappresenta un'aggiunta o modifica sostanziale, come ad esempio l'aggiunta di una sezione;
    \item \textbf{Z:} indica una piccola modifica, come per esempio la correzione di errori o aggiunte di piccole dimensioni.
\end{itemize}
Un documento parte sempre dalla versione \textit{0.0.0}. \\
Ogni modifica di un numero di versione, comporta l'azzeramento di tutti i numeri alla sua destra.

\subsubsection{Repository}

Il gruppo \textit{NaN1fy} utilizza le seguenti repository facenti parte della organizzazione \textit{NaN1fy-unipd} in \glossterm{Github}.

\subsubsubsection{Lista Repository}

\begin{itemize}
    \item \textbf{docs:} repository dedicata alla documentazione del progetto;
    \item \textbf{SyncCity:} repository dedicata alla scrittura e implementazione del progetto software;
    \item \textbf{NaN1fy.github.io:} repository del sito vetrina.
\end{itemize}

\subsubsubsection{Struttura Repository \texttt{docs}} \\
Questo repository è suddiviso in due branch principali:
\begin{itemize}
    \item \textit{main:} branch contenente tutti la documentazione prodotta in formato \textit{.pdf};
    \item \textit{sources:} branch contenente i file \textit{.tex} dei rispettivi documenti della repository precedente. Quando vengono aggiunti o modificati dei file in questa branch, vengono automaticamente compilati, caricando il risultato nella branch \textit{main}, cosicché possano essere visualizzati da chiunque.
\end{itemize}
Di seguito viene riportata la struttura della repository, i termini in \textbf{grassetto} indicano il nome di una directory:
 \begin{itemize}
    \item \textbf{RTB}:
    \begin{itemize}
        \item \textbf{Esterni}:
        \begin{itemize}
            \item \textbf{Verbali};
            \item Analisi dei requisiti;
            \item Piano di progetto;
            \item Piano di qualifica;
        \end{itemize}
        \item \textbf{Interni}:
        \begin{itemize}
            \item \textbf{Verbali};
            \item Glossario;
            \item Norme di progetto;
        \end{itemize}
    \end{itemize}
    \item \textbf{Candidatura}:
    \begin{itemize}
        \item \textbf{Verbali}:
        \begin{itemize}
            \item \textbf{Esterni};
            \item \textbf{Interni};
        \end{itemize}
        \item Preventivo costi e assunzione impegni;
        \item Lettera di presentazione;
        \item Valutazione capitolati.
    \end{itemize}
 \end{itemize}

\subsubsubsection{Struttura Repository \textit{SyncCity}} \\
Di seguito viene riportata la struttura della repository, i termini in \textbf{grassetto} indicano il nome di una directory:
\begin{itemize}
    \item \textbf{\glossterm{ClickHouse}}: directory contenente i config e codice necessario per l'utilizzo di ClickHouse;
    \item \textbf{\glossterm{Grafana}}: directory contenente i config di Grafana;
    \item \textbf{PyMockSensors}: directory contenente il codice sorgente di \textit{PyMockSensors}, ovvero il generatore simulato di dati provenienti da sensori;
    \item docker-compose.yaml: file di configurazione per l'utilizzo di \glossterm{Docker Compose}.
\end{itemize}

\subsubsubsection{Sito Vetrina} \\
La repository \textit{NaN1fy.github.io} contiene il codice del sito del progetto, il cui scopo è quello di fornire una veloce e intuitiva interfaccia, nella quale poter visualizzare ergonomicamente e in maniera organizzata i documenti relativi al progetto stesso. \\
Il sito vetrina propone anche la visualizzazione diretta del glossario del progetto, contenente una definizione chiara e univoca di tutti i termini rilevanti, onde evitare interpretazioni arbitrarie e fraintendimenti riguardo tali specifici concetti. \\
Il codice del sito comprende una funzionalità di auto aggiornamento dei documenti contenuti nella vetrina e nel glossario.

\subsubsection{Sincronizzazione}

Attraverso la piattaforma \glossterm{GitHub}, ogni attività prevista viene affiancata da una issue a cui corrisponde una branch, separata e parallela alle altre; tale suddivisione permette lo svolgimento in maniera autonoma e sicura di ogni singola attività, garantendo un'avanzamento simultaneo dei lavori.

\subsubsubsection{Branching} \\
La suddivisione in branch, già precedentemente esposta, presuppone l'utilizzo della metodologia single-purpose, che prevede l'utilizzo di una branch solo ed esclusivamente per lo svolgimento di una singola attività. In tale modo, il lavoro viene parcelizzato, garantendo un flusso stabile di lavoro. \\
Una volta che una data attività viene portata a termine, la corrispondente branch viene unita alla branch principale e successivamente cancellata.

\subsubsubsection{Pull Request} \\
Al termine di un'attività (o al completamento di una sua parte), il membro a cui è stata assegnata questa attività (o uno di essi), si assume la responsabilità di aprire una Pull Request indicando i verificatori. Questi ultimi, dopo aver confermato la correttezza delle modifiche, hanno il compito di fare il merge della pull request, chiudendo poi la issue.

\subsection{Verifica}

\subsubsection{Introduzione}

La verifica è un processo svolto, caso per caso, da una parte dei membri del team, con lo scopo di garantire l'efficienza e la correttezza di ogni attività sottoposta a scrutinio. \\
È un processo presente per l'intera durata del ciclo di vita del software e abbraccia, nella sua esecuzione, sia la documentazione che il prodotto in sè: la sua attuazione, difatti, non è organizzata ricorrentemente, ma viene effetuata in occasione del completamento di un'attività, documento o parte significativa del prodotto. \\
Questo processo trae le sue basi dai vincoli di qualità e dalle linee guida individuate all'interno del documento \textit{Piano\textunderscore di\textunderscore Qualifica\textunderscore v1.0.0}, basi che il verificatore è tenuto a rispettare per garantire uniformità, coerenza e ripetibilità al processo di verifica.

\subsubsection{Analisi Statica}

L'analisi statica è una modalità di analisi che prevede una verifica del prodotto e/o dei documenti senza la necessità di esecuzione o di mediazione di terze parti. \\
Le due metodologie principali per condurre l'analisi statica, sono: l'\textit{inspection} e il \textit{walkthrough}.

\subsubsubsection{Inspection} \\
Tale metodologia utilizza un'approccio metodologico e ben strutturato per rilevare possibili difetti nel prodotto e nella documentazione; questi difetti vengono definiti e specificati a priori in delle liste, cosidette di controllo, le quali verranno poi utilizzate in maniera programmatica nella valutazione del documento e codice del prodotto.

\subsubsubsection{Walkthrough} \\
Contrariamente al metodo \textit{inspection}, il \textit{walkthrough} si preoccupa della verifica del documento o del codice tramite un maggiore dialogo tra il verificatore e l'autore, eseguendo una verifica più sostanziale e profonda, non alla ricerca di problemi specifici, bensì di una correttezza più generica e adattiva. In tale maniera, pur risultando in un maggiore impiego di risorse, garantisce una correttezza maggiore. \\
Per questo progetto il team \textit{NaN1fy} si propone di utilizzare per la documentazione un'approccio di questo tipo, principalmente considerando che l'approccio di tipo \textit{inspection} è fin troppo rigido per poter garantire una corretta valutazione di un'insieme così eterogeneo e variegato di dati, quale è la documentazione del progetto.

\subsubsection{Analisi Dinamica}

La metodologia \textit{analisi dinamica} pone l'attenzione sul comportamento del codice e suoi eventuali difetti durante l'esecuzione dello stesso. \\
Tale metodologia è specifica per ogni progetto in sè, in quanto l'insieme di attività di verifica, composte da test, dipendono esclusivamente dal contenuto del codice e dai requisiti del progetto. \\
Tali test garantiscono, grazie alla loro ripetibilità, una valutazione oggettiva del codice e delle funzionalità del prodotto, in quanto in grado, dato un'insieme di caratteristiche, di generare lo stesso risultato più volte, indipendentemente da fattori esterni o casuali.

\subsubsubsection{Test di unità} \\
Test che hanno come obiettivo la verifica delle singole unità del sistema, che possono essere funzioni o metodi esulando dal resto del sistema. \\
I test di unità si dividono in due categorie:
\begin{itemize}
    \item \textbf{Test Funzionali:} verificano che l'output corrisposta al valore atteso;
    \item \textbf{Test Strutturali:} verificano la struttura interna dell'unità e il flusso dei dati.
\end{itemize}

\subsubsubsection{Test di integrazione} \\
Test che hanno come obiettivo la valutazione del comportamento delle unità quando vengono combinate tra di loro identificando eventuali problemi dell'interazione tra le componenti integrate e verificando l'efficacia e il soddisfacimento dei requisiti posti dal progetto.

\subsubsubsection{Test di sistema} \\
Test che hanno come obiettivo di valutare il sistema come singola e unica entità verificando la corretta esecuzione e completezza del prodotto, in linea con i requisiti fissati nell'analisi dei requisiti.

\subsubsubsection{Test di accettazione} \\
Test il cui obiettivo è di mostrare la conformità del software rispetto le richieste e aspettative della Proponente, garantendo il soddisfacimento di quest'ultima riguardo il prodotto finale.

\subsubsubsection{Identificazione dei test} \\
Ogni singolo test possiede un codice univoco identificativo con il seguente formato:
\begin{center}
	\textbf{T[Tipologia]-[Numero]}
\end{center}
Dove \textbf{Tipologia} indica la tipologia del test: 
\begin{itemize}
    \item \textbf{U:} di unità;
	\item \textbf{I:} di integrazione;
	\item \textbf{S:} di sistema;
	\item \textbf{A:} di accettazione.
\end{itemize}

\subsubsubsection{Stato dei test} \\
I test possiedono anche uno stato:
\begin{itemize}
	\item \textbf{V:} Verificato. Il test ha esito positivo;
	\item \textbf{NV:} Non Verificato. Il test ha esito negativo; 
	\item \textbf{NI:} Non Implementato.
\end{itemize}