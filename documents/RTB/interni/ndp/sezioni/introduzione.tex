\subsection{Scopo del documento}
Il seguente documento ha come scopo quello di elencare le norme che ogni membro del gruppo \texttt{NaN1fy} è tenuto a rispettare durante lo svolgimento del progetto \texttt{SyncCity} presentato dall'azienda \glossterm{proponente} \texttt{SyncLab}.\\
Inoltre, si delineano le convenzioni relative all'utilizzo dei diversi strumenti selezionati per l'implementazione del prodotto, esponendo dettagliatamente i procedimenti adottati.
\subsection{Obiettivi del prodotto}
L'obiettivo del progetto \texttt{SyncCity} è quello di creare una \textit{piattaforma} atta al monitoraggio di \textit{sensori} sparsi geograficamente nel territorio di una città. I sensori in questione permettono la misurazione e segnalazione di dati \textit{real-time} riguardanti le più disparate caratteristiche e necessità del territorio quali temperatura ed umidità esterna, occupazione di stalli di parcheggio, funzionamento o guasto elettrico di colonnine HPC, traffico stradale e via dicendo. La \glossterm{proponente} richiede la simulazione di alcuni dei sensori nominati nonché la gestione dei dati, della loro persistenza e della loro rappresentazione grafica attraverso \textit{widgets} e \textit{grafici}. \\\\\texttt{SyncCity} permetterà un miglioramento della qualità dei servizi offerti dalla città attraverso il continuo monitoraggio della stessa, ottenendo, gestendo e successivamente condividendo i dati con gli utenti. 
\\\\
Il prodotto si struttura nelle seguenti funzionalità principali:
\begin{itemize}
	\setlength\itemsep{0em}
	\item Raccolta dati;
	\item Persistenza e strutturazione dati;
	\item Rappresentazione grafica dati.
\end{itemize}
\subsection{Glossario}
Al fine di ovviare a possibili ambiguità dovute al linguaggio e ai termini utilizzati nel seguente documento, viene fornito un \textit{Glossario v0.0.0} contenente le definizioni dei termini utilizzati aventi un significato specifico. Tali termini saranno evidenziati con l'uso del corsivo e dalla presenza di una G a pedice.
\subsection{Riferimenti}
\subsubsection{Riferimenti Normativi}
\begin{itemize}
	\setlength\itemsep{0em}
	\item \textit{Norme di Progetto v0.0.0}
	\item \glossterm{Capitolato} \texttt{C6 - SyncCity: Smart city monitoring platform:} \\ \url{https://www.math.unipd.it/~tullio/IS-1/2023/Progetto/C6.pdf} (Ultimo accesso: \today) \\ \url{https://www.math.unipd.it/~tullio/IS-1/2023/Progetto/C6p.pdf} (Ultimo accesso: \today)
	\item \textit{Verbale Esterno 2024/04/19}
    \item \textit{Verbale Esterno 2024/03/12}
	\item \textit{Verbale Esterno 2024/04/03}
	\item Regolamento progetto didattico: \\ \url{https://www.math.unipd.it/~tullio/IS-1/2023/Dispense/PD2.pdf} (Ultimo accesso: \today)
\end{itemize}
\subsubsection{Riferimenti Informativi}
\begin{itemize}
	\setlength\itemsep{0em}
	\item Norme di Progetto - corso di Ingegneria del Software a.a. 2023/2024: \\ \url{https://www.math.unipd.it/~tullio/IS-1/2023/Dispense/T5.pdf} (Ultimo accesso: \today)
\end{itemize}
\newpage
