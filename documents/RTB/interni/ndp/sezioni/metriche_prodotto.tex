\subsection{Introduzione}
Al fine di garantire una valutazione oggettiva e misurabile del prodotto e dei processi che lo compongono rispetto agli standard fissati, oltre che per fornire una guida atta al miglioramento del prodotto in sè e dei suoi processi, verrano utilizzate le seguenti metriche di product management. 

\subsection{Metriche di \glossterm{processo}}
\begin{itemize}
    \item \textbf{Fornitura}
        \begin{itemize}
            \item \textbf{MPC-EV}: Earned Value - Rappresenta il valore del lavoro prodotto fino ad un dato momento;\\
                EV = BAC x \%lavoro\_prodotto\\  
            \item \textbf{MPC-PV}: Planned Value - Rappresenta il valore del lavoro pianificato fino ad un dato momento;\\
                PV = BAC x \%lavoro\_pianificato\\
            \item \textbf{MPC-AC}: Actual Cost - Rappresenta il costo effettivamente sostenuto fino ad un dato momento. Il suo valore è reperibile nel Piano di Progetto;\\

            \item \textbf{MPC-CPI}: Cost Performance Index - Rappresenta l'indice di produzione rispetto al costo sostenuto;\\
                CPI = EV / AC\\
            \item \textbf{MPC-EAC}: Estimate At Completion - Rappresenta il valore stimato per il completamento del progetto in un dato momento;\\
                EAC = BAC / CPI\\ 
            \item \textbf{MPC-ETC}: Estimate To Completion - Rappresenta il valore stimato per completare il progetto;\\
                ETC = EAC - AC\\
            \item \textbf{MPC-VAC}: Variance At Completion - Rappresenta la variazione relativa del budget stimato rispetto al budget pianificato;\\
                VAC = (BAC - EAC) / BAC\\
            \item \textbf{MPC-SV}: Schedule Variance - Rappresenta la variazione relativa del valore prodotto rispetto a quello pianificato;\\
                SV = (EV - PV) / BAC\\
            \item \textbf{MPC-BV}: Budget Variance - Rappresenta la variazione relativa tra il valore pianificato e i costi sostenuti.\\
                BV = (PV - AC) / BAC\\
        \end{itemize}
    \item \textbf{Documentazione}
        \begin{itemize}
            \item \textbf{MPC-IG}: Indice Gulpease - Indice di leggibilità di un testo tarato sulla lingua italiana, considera la lunghezza delle parole e della frase rispetto al numero di lettere nella frase stessa.\\
                GULPEASE = 89 + ((300 x numero\_frasi - 10 x numero\_lettere) / numero\_parole)\\\\
                Il risultato è un numero compreso tra 0 e 100, dove ``100'' indica la leggibilità più alta e ``0'' quella più bassa. Vi sono alcune soglie che sono ufficialmente considerate rilevanti, in particolare:
                \begin{itemize}
                    \item testi con risultato inferiore all'80 sono considerati di difficile lettura per chi possiede la licenza elementare;
                    \item testi con risultato inferiore all'60 sono considerati di difficile lettura per chi possiede la licenza media;
                    \item testi con risultato inferiore all'40 sono considerati di difficile lettura per chi possiede la licenza superiore.
                \end{itemize}\\
            \item \textbf{MPC-CO}: Correttezza Ortografica - Numero di errori ortografici e grammaticali presenti nel documento.\\
        \end{itemize}
    \item \textbf{Gestione della qualità}
        \begin{itemize}
            \item \textbf{MPC-QMS}: Quality Metrics Satisfied - Rappresenta il numero di metriche che sono state soddisfatte in un determinato momento.\\
        \end{itemize}
    \item \textbf{Gestione dei processi}
        \begin{itemize}
            \item \textbf{MPC-NR}: Non-calculated Risks - Rappresenta il numero di rischi non calcolati incontrati fino ad un dato momento.\\
        \end{itemize}
\end{itemize}

% \subsection{Metriche di prodotto}
% \begin{itemize}
%     \item
%     \item
%     \item
%     \item
%     \item
%     \item
%     \item
%     \item
%     \item
%     \item
%     \item
%     \item
%     \item
%     \item
%     \item
%     \item
%     \item
% \end{itemize}