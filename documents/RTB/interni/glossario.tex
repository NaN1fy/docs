% changelog: "2.0.0, 2024-08-15, Approvazione per PB"
\documentclass[8pt]{article}
\usepackage[italian]{babel}
\usepackage[utf8]{inputenc}
\usepackage[letterpaper, left=1in, right=1in, bottom=0.75in, top=0.75in]{geometry}
\usepackage{amsmath}
\usepackage{subfiles}
\usepackage{lipsum}
\usepackage{csquotes}
\usepackage{amsfonts}
\usepackage[sfdefault]{plex-sans}
\usepackage{float}
\usepackage{pifont}
\usepackage{mathabx}
\usepackage[euler]{textgreek}
\usepackage{makecell}
\usepackage{tikz}
\usepackage{wrapfig}
\usepackage{siunitx}
\usepackage{amssymb} 
\usepackage{tabularx}
\usepackage{adjustbox}
\usepackage[document]{ragged2e}
\usepackage{floatflt}
\usepackage[hidelinks]{hyperref}
\usepackage{graphicx}
\usepackage{hyperref}
\setcounter{tocdepth}{4}
\usepackage{caption}
\usepackage{multicol}
\usepackage{tikz}
\setlength\parindent{0pt}
\captionsetup{font=footnotesize}
\usepackage{fancyhdr} 
\usepackage{graphicx}
\usepackage{capt-of} 
\usepackage{booktabs}
\usepackage{varwidth}
\usepackage{csvsimple}
\usepackage{xstring}
\usepackage{datatool}

% -- TITOLO -- %
\newcommand{\customtitle}{GLOSSARIO} % o ESTERNO

% -- PER LA FIRMA -- %
\newcommand{\signatureline}[1]{%
	\par\vspace{0.5cm}
	\noindent\makebox[\linewidth][r]{\rule{0.2\textwidth}{0.5pt}\hspace{3cm}\makebox[0pt][r]{\vspace{3pt}\footnotesize #1}}%
}

% -- PER IL GLOSSARIO -- %
\newcommand{\glossterm}[1]{#1\textsuperscript{G}}

% -- INTESTAZIONE -- %
\fancypagestyle{mystyle}{
	\fancyhf{} 
	\fancyhead[R]{\includegraphics[height=1cm]{../../template/images/logos/NaN1fy_logo.png}} 
	\fancyhead[L]{\leftmark} 
	\renewcommand{\headrulewidth}{1pt} 
	\fancyhead[L]{\customtitle} 
	\renewcommand{\headsep}{1.3cm} 
	\fancyfoot[C]{\thepage} 
}

% -- Rimozione numeri sezioni -- %
\renewcommand{\thesection}{}
\renewcommand{\thesubsection}{\arabic{section}.\arabic{subsection}}
\makeatletter
\def\@seccntformat#1{\csname #1ignore\expandafter\endcsname\csname the#1\endcsname\quad}
\let\sectionignore\@gobbletwo
\let\latex@numberline\numberline
\def\numberline#1{\if\relax#1\relax\else\latex@numberline{#1}\fi}
\makeatother

\setcounter{tocdepth}{1}

\newcommand\myprevious{}%
\newcommand\mycurrent{}%

\newcommand{\creaGlossario}[2]{
	\DTLsetseparator{#1}
	\DTLloaddb{glossary}{#2}
	\DTLsort{termine}{glossary}
	\gdef\mycurrent{}
	\gdef\myprevious{}
	\DTLforeach{glossary}{\myterm=termine,\mydescription=descrizione}{
		\StrLeft{\myterm}{1}[\mycurrent]
		\IfStrEq{\mycurrent}{\myprevious}{}{
			\section*{\mycurrent}
			\addcontentsline{toc}{section}{\mycurrent}
			\global\let\myprevious=\mycurrent
		}
		\subsection*{\myterm}
		\mydescription
		%\par
	}
}

\begin{document}
	\definecolor{myblue}{RGB}{23,103,162}
	\begin{titlepage}
		\begin{tikzpicture}[remember picture, overlay]
			\node[anchor=south east, opacity=0.2, yshift = -4cm, xshift= 2em] at (current page.south east) {\includegraphics[width=0.7\textwidth, trim=0cm 0cm 5cm 0cm, clip]{../../template/images/logos/Universita_Padova_transparent.png}}; 
			\node[anchor=north west, opacity=1, yshift = 4.2cm, xshift= 1.4cm, scale=1.6] at (current page.south west) {\includegraphics[width=4cm]{../../template/images/logos/NaN1fy_logo.png}};
		\end{tikzpicture}
		
		\begin{minipage}[t]{0.47\textwidth}
    {\large{\textsc{Destinatari}}
			\vspace{3mm}
			\\ \large{\textsc{Prof. Tullio Vardanega}}
			\\ \large{\textsc{Prof. Riccardo Cardin}}
		}
		\end{minipage}
		\hfill
    \begin{minipage}[t]{0.47\textwidth}\raggedleft
		{\large{\textsc{Redattori}}
			\vspace{3mm}
			{\\\large{\textsc{Guglielmo Barison}\\}} % massimo due
			{\large{\textsc{Oscar Konieczny}}}


		}
		\vspace{8mm}

		{\large{\textsc{Verificatori}}
			\vspace{3mm}
			{\\\large{\textsc{Veronica Tecchiati}\\}}
			{\large{\textsc{Linda Barbiero}}}

		}
		\vspace{4mm}\vspace{4mm}
	\end{minipage}
		\vspace{4cm}
		\begin{center}
			\begin{flushright}
				{\fontsize{30pt}{52pt}\selectfont \textbf{Glossario}} 
			\end{flushright}
			\vspace{3cm}
		\end{center}
		\vspace{10 cm}
		{\small \textsc{\href{mailto: nan1fyteam.unipd@gmail.com}{nan1fyteam.unipd@gmail.com}}}
	\end{titlepage}
	\pagestyle{mystyle}
	\section*{Registro delle Modifiche}
	\begin{table}[ht!]	
		\centering
		\begin{tabular}{p{1.2cm} p{2cm} p{5cm} p{3cm} p{3cm}}
			\toprule
			\textbf{Versione}& \textbf{Data} & \textbf{Descrizione} & \textbf{Redattore} & \textbf{Verificatore} \\
			\midrule
				2.0.0 & 2024-08-15 & \textbf{Approvazione per PB} & & \\\\
				1.1.0 & 2024-08-15 & Aggiunta nuovi termini. & Oscar Konieczny & Veronica Tecchiati \\
			\bottomrule
		\end{tabular}
		\caption{Registro delle modifiche.}
		\label{table:Registro delle modifiche}
	\end{table}
	\clearpage
	\begin{table}[ht!]	
		\centering
		\begin{tabular}{p{1.2cm} p{2cm} p{6cm} p{3cm} p{2cm}}
			\toprule
			\textbf{Versione}& \textbf{Data} & \textbf{Descrizione} & \textbf{Autore} & \textbf{Ruolo} \\
			\midrule
        		1.0.0 & 2024-06-03 & \textbf{Approvazione per RTB} & & \\\\
      			0.0.3 & 2024-06-03 & Verifica completa, correzioni minori & Veronica Tecchiati, Linda Barbiero & Verificatore \\\\
				0.0.2 & 2024-06-03 & Aggiunta numerosi termini in vista dell'RTB. & Oscar Konieczny & Redattore \\\\
				0.0.1 & 2024-05-03 & Inserimento ``Topic", ``Smart City", ``Kafka", ``JSON", ``Budget At Completion", ``Consuntivo" e ``Data pipeline".  & Guglielmo Barison & Redattore \\\\% spazio tra le righe
    			0.0.0 & 2024-04-06 & Struttura di base e aggiunta di alcuni termini.  & Guglielmo Barison & Redattore \\% spazio tra le righe
			\bottomrule
			% Ruolo Redattore o Verificatore
		\end{tabular}
		\caption{Registro delle modifiche.}
		\label{table:Registro delle modifiche}
	\end{table}
	\newpage
	\tableofcontents
	\clearpage
	\newpage
	\justifying
	\section{Introduzione}
	Il presente documento è stato creato con l'obiettivo di prevenire ambiguità o incomprensioni riguardanti la terminologia utilizzata nella varia documentazione del progetto. A tale scopo, vengono fornite le definizioni dei termini specifici del dominio d'uso, seguendo una struttura alfabetica per consentire una facile navigazione all'interno del documento. L'indicazione della presenza di un termine nel \texttt{Glossario} nei restanti documenti avviene mediante l'applicazione di \glossterm{questo stile}.
	\newpage
	
	% separatore ; nel .csv
	% nel .csv i file non devono essere per forza in ordine alfabetico, li ordina la macro in latex
	\creaGlossario{;}{glossario.csv}
		
\end{document}
