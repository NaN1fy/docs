% changelog: "0.0.0, 2024-04-06, Guglielmo Barison, Struttura di base e aggiunta di alcuni termini"
\documentclass[8pt]{article}
\usepackage[italian]{babel}
\usepackage[utf8]{inputenc}
\usepackage[letterpaper, left=1in, right=1in, bottom=0.75in, top=0.75in]{geometry}
\usepackage{amsmath}
\usepackage{subfiles}
\usepackage{lipsum}
\usepackage{csquotes}
\usepackage{amsfonts}
\usepackage[sfdefault]{plex-sans}
\usepackage{float}
\usepackage{pifont}
\usepackage{mathabx}
\usepackage[euler]{textgreek}
\usepackage{makecell}
\usepackage{tikz}
\usepackage{wrapfig}
\usepackage{siunitx}
\usepackage{amssymb} 
\usepackage{tabularx}
\usepackage{adjustbox}
\usepackage[document]{ragged2e}
\usepackage{floatflt}
\usepackage[hidelinks]{hyperref}
\usepackage{graphicx}
\usepackage{hyperref}
\setcounter{tocdepth}{4}
\usepackage{caption}
\usepackage{multicol}
\usepackage{tikz}
\setlength\parindent{0pt}
\captionsetup{font=footnotesize}
\usepackage{fancyhdr} 
\usepackage{graphicx}
\usepackage{capt-of}% 
\usepackage{booktabs}
\usepackage{varwidth}

% -- TITOLO -- %
\newcommand{\customtitle}{GLOSSARIO} % o ESTERNO

% -- PER LA FIRMA -- %
\newcommand{\signatureline}[1]{%
	\par\vspace{0.5cm}
	\noindent\makebox[\linewidth][r]{\rule{0.2\textwidth}{0.5pt}\hspace{3cm}\makebox[0pt][r]{\vspace{3pt}\footnotesize #1}}%
}

% -- PER IL GLOSSARIO -- %
\newcommand{\glossterm}[1]{#1\textsuperscript{G}}


% -- INTESTAZIONE -- %
\fancypagestyle{mystyle}{
	\fancyhf{} 
	\fancyhead[R]{\includegraphics[height=1cm]{../../template/images/logos/NaN1fy_logo.png}} 
	\fancyhead[L]{\leftmark} 
	\renewcommand{\headrulewidth}{1pt} 
	\fancyhead[L]{\customtitle} 
	\renewcommand{\headsep}{1.3cm} 
	\fancyfoot[C]{\thepage} 
}

\setcounter{tocdepth}{1}

\begin{document}
	\definecolor{myblue}{RGB}{23,103,162}
	\begin{titlepage}
		\begin{tikzpicture}[remember picture, overlay]
			\node[anchor=south east, opacity=0.2, yshift = -4cm, xshift= 2em] at (current page.south east) {\includegraphics[width=0.7\textwidth, trim=0cm 0cm 5cm 0cm, clip]{../../template/images/logos/Universita_Padova_transparent.png}}; 
			\node[anchor=north west, opacity=1, yshift = 4.2cm, xshift= 1.4cm, scale=1.6] at (current page.south west) {\includegraphics[width=4cm]{../../template/images/logos/NaN1fy_logo.png}};
		\end{tikzpicture}
		
		\begin{minipage}[t]{0.47\textwidth}
			{\large{\textsc{Destinatari}}
				\vspace{3mm}
				\\ \large{\textsc{Prof. Tullio Vardanega}}
				\\ \large{\textsc{Prof. Riccardo Cardin}}
			}
		\end{minipage}
		\hfill
		\begin{minipage}[t]{0.47\textwidth}\raggedleft
			{\large{\textsc{Redattori}}
				\vspace{3mm}
				{\\\large{\textsc{Guglielmo Barison}\\}} % massimo due 
				{\large{\textsc{Oscar Konieczny}}}
				
				
			}
			\vspace{8mm}
			
			{\large{\textsc{Verificatori}}
				\vspace{3mm}
				{\\\large{\textsc{XXXX XXXX}\\}} % massimo due 
				{\large{\textsc{XXXX XXXX}}}
				
			}
			\vspace{4mm}\vspace{4mm}
		\end{minipage}
		\vspace{4cm}
		\begin{center}
			\begin{flushright}
				{\fontsize{30pt}{52pt}\selectfont \textbf{Glossario}} 
			\end{flushright}
			\vspace{3cm}
		\end{center}
		\vspace{10 cm}
		{\small \textsc{\href{mailto: nan1fyteam.unipd@gmail.com}{nan1fyteam.unipd@gmail.com}}}
	\end{titlepage}
	\pagestyle{mystyle}
	\section*{Registro delle Modifiche}
	\begin{table}[ht!]	
		\centering
		\begin{tabular}{p{1.2cm} p{2cm} p{6cm} p{3cm} p{2cm}}
			\toprule
			\textbf{Versione}& \textbf{Data} & \textbf{Descrizione} & \textbf{Autore} & \textbf{Ruolo} \\
			\midrule
			0.0.0 & 2024-04-06 & Struttura di base e aggiunta di alcuni termini.  & Guglielmo Barison & Redattore \\% spazio tra le righe
			\bottomrule
			% Ruolo Redattore o Verificatore
		\end{tabular}
		\caption{Registro delle modifiche.}
		\label{table:Registro delle modifiche}
	\end{table}
	\newpage
	\tableofcontents
	\clearpage
	\newpage
	\justifying
	\section{Introduzione}
	Il presente documento è stato creato con l'obiettivo di prevenire ambiguità o incomprensioni riguardanti la terminologia utilizzata nella varia documentazione del progetto. A tale scopo, vengono fornite le definizioni dei termini specifici del dominio d'uso, seguendo una struttura alfabetica per consentire una facile navigazione all'interno del documento. L'indicazione della presenza di un termine nel \texttt{Glossario} nei restanti documenti avviene mediante l'applicazione di \glossterm{questo stile}.
	\newpage
	
	\section*{A}
	\addcontentsline{toc}{section}{A}
	\subsection*{Action}
	Su GitHub, un ``action" (azione) si riferisce a GitHub Actions, che è un servizio di automazione integrato direttamente nella piattaforma GitHub. GitHub Actions consente l'automatizzazione di diversi processi nel ciclo di vita dello sviluppo del software, quali la compilazione del codice, i test automatici, la distribuzione e altro ancora.
	\newpage
	\section*{B}
	\addcontentsline{toc}{section}{B}
	\subsection*{Baseline}
	Una ``baseline" indica una configurazione stabile e controllata di un sistema o di un prodotto in un particolare momento del suo ciclo di vita. Questa configurazione può comprendere specifiche tecniche, documentazione, codice sorgente o altri elementi critici che definiscono le caratteristiche e le funzionalità del sistema. Le baseline costituiscono un punto di riferimento essenziale per valutare eventuali modifiche, monitorare il progresso e garantire la coerenza nel tempo.
	\subsection*{Best practice}
	Il modo migliore per condurre la produzione. Si basa su procedure ripetibili che nel tempo hanno dimostrato di essere le migliori sia per l'efficienza (meno sforzo richiesto) sia per l'efficacia (migliori risultati). Seguire le buone pratiche assicura il raggiungimento degli obiettivi massimizzando economia e qualità.
	\subsection*{Board}
	Su GitHub, un project board, o più semplicemente board, rappresenta uno strumento fondamentale per organizzare e monitorare le attività di un progetto. È composto da colonne che riflettono lo stato delle attività. Gli issue e le pull request possono essere assegnati a tali colonne per indicarne lo stato attuale.
	\subsection*{Branch}
	Nel campo dello sviluppo software, un branch costituisce una copia indipendente del codice sorgente che consente agli sviluppatori di lavorare su modifiche separate senza impattare direttamente il flusso principale. I branch agevolano lo sviluppo simultaneo e la gestione delle versioni.
	\newpage
	\section*{C}
	\addcontentsline{toc}{section}{C}
	\subsection*{Capitolato}
	Un capitolato è un documento che definisce le specifiche, i requisiti e le principali condizioni di un progetto o di un appalto. Serve a delineare in modo dettagliato ciò che deve essere realizzato, le prestazioni attese e le regole da seguire. Fornisce una base solida per la pianificazione e l'esecuzione del progetto, garantendo una chiara comprensione delle aspettative da parte di tutte le parti coinvolte e identificando un problema o un bisogno da risolvere.
	\newpage
	\section*{D}
	\addcontentsline{toc}{section}{D}
	\subsection*{Database}
	Un sistema organizzato per la raccolta, la memorizzazione e la gestione di dati strutturati in modo che possano essere facilmente recuperati, aggiornati ed elaborati.
	\subsection*{Discord}
	Discord è una piattaforma di comunicazione online che unisce chat testuale, vocale e video. Gli utenti possono creare server, organizzare discussioni in canali e personalizzare l'esperienza di comunicazione. Discord offre anche un sistema di ruoli e autorizzazioni per controllare l'accesso degli utenti e definire le loro capacità all'interno del server. Supporta inoltre bot e integrazioni di terze parti per aggiungere funzionalità personalizzate ai server.
	\subsection*{Docker}
	Una piattaforma open-source che automatizza il processo di distribuzione delle applicazioni all'interno di contenitori leggeri e portabili.
	\subsection*{Docker Compose}
	Uno strumento per definire ed eseguire applicazioni Docker multi-container. Permette di configurare un'applicazione complessa con più servizi e relative dipendenze in un singolo file.
	\newpage
	\section*{E}
	\addcontentsline{toc}{section}{E}
	\section*{F}
	\addcontentsline{toc}{section}{F}
	\subsection*{Funzionalit\`a}
	Attributo di un particolare prodotto o componente software utilizzato per scopi operativi.
	\newpage
	\section*{G}
	\addcontentsline{toc}{section}{G}
	\subsection*{Git}
	Git è un sistema di controllo delle versioni distribuito (DVCS), progettato per tracciare le modifiche nel codice sorgente durante lo sviluppo del software. È uno strumento essenziale per la gestione del controllo delle versioni, che consente agli sviluppatori di lavorare in modo collaborativo e di tenere traccia delle modifiche apportate al codice nel tempo, permettendo inoltre il ripristino a versioni precedenti del software in modo controllato.
	\subsection*{GitHub}
	GitHub è una piattaforma di sviluppo collaborativo basata su Git che consente agli sviluppatori di lavorare insieme, gestire le versioni del codice sorgente e facilitare la collaborazione in progetti software. Le sue principali caratteristiche includono repository per il controllo delle versioni, strumenti di tracciamento dei problemi (issues) e funzionalità di gestione dei progetti (project boards).
	\newpage
	\section*{H}
	\addcontentsline{toc}{section}{H}
	
	\section*{I}
	\addcontentsline{toc}{section}{I}
	\subsection*{Issue}
	In ambito di GitHub, un issue (problema) rappresenta un meccanismo attraverso il quale gli utenti possono segnalare, discutere e tenere traccia di attività specifiche all'interno di un progetto. Ogni issue ha un thread di commenti associato, consentendo la comunicazione tra i membri del team o i collaboratori esterni. Questo strumento facilita la collaborazione e la gestione delle attività nello sviluppo del software.
	\newpage
	\section*{J}
	\addcontentsline{toc}{section}{J}
	
	\section*{K}
	\addcontentsline{toc}{section}{K}
	
	\section*{L}
	\addcontentsline{toc}{section}{L}
	\subsection*{LaTeX}
	Un sistema di preparazione dei documenti basato su un linguaggio di markup. È spesso utilizzato per la creazione di documenti tecnici, accademici e scientifici.
	\newpage
	\section*{M}
	\addcontentsline{toc}{section}{M}
	\subsection*{Milestone}
	Una milestone è una data di calendario che indica un punto di avanzamento atteso. Nel contesto di GitHub o di altre piattaforme di gestione dei progetti, le milestone sono spesso utilizzate per organizzare e tenere traccia di gruppi di issues o di attività correlate. Associare un gruppo di issues a una milestone può aiutare a monitorare il progresso e a stabilire obiettivi intermedi, contribuendo a gestire in modo più efficace lo sviluppo del progetto.
	\newpage
	\section*{N}
	\addcontentsline{toc}{section}{N}
	\section*{O}
	\addcontentsline{toc}{section}{O}
	\section*{P}
	\addcontentsline{toc}{section}{P}
	\subsection*{Proponente}
	Il proponente è colui che presenta ufficialmente l'iniziativa e promuove la sua realizzazione. È il cliente rispetto alle esigenze del prodotto e il mentore rispetto alle scelte di sviluppo.
	\subsection*{Proof of Concept}
	Artificio usa-e-getta, sotto forma di demo eseguibile, realizzato all'inizio del progetto per valutare la fattibilità tecnologica del prodotto atteso rispetto a specifiche funzionalità individuate con il proponente. Non si pone quindi l'obiettivo di essere una baseline architetturale bensì tecnologica, per dimostrare la padronanza delle tecnologie necessarie da parte del team.
	\newpage
	\subsection*{Python}
	Linguaggio di programmazione ad alto livello, interpretato e general-purpose, noto per la sua sintassi chiara, facilità d'uso e ampia disponibilità di librerie e framework.
	\newpage
	\section*{Q}
	\addcontentsline{toc}{section}{Q}
	\section*{R}
	\addcontentsline{toc}{section}{R}
	\subsection*{Repository}
	Un repository, o più semplicemente repo, è uno spazio di archiviazione digitale utilizzato per conservare e gestire i file di un progetto, in particolare nel contesto del controllo delle versioni del software. Un repository può includere file di codice, documentazione, risorse multimediali o qualsiasi altro elemento necessario per il progetto.
	\subsection*{RTB}
	Nel contesto dell'ingegneria del software, è una fase iniziale e fondamentale del processo di sviluppo del software. In questa fase, l'obiettivo principale è stabilire e comprendere i requisiti del sistema e definire la base tecnologica sulla quale il progetto si svilupperà. Le tre principali attività di questa fase sono: l'analisi dei requisiti, la definizione della baseline tecnologica e la definizione della baseline di progetto.
	\newpage
	\section*{S}
	\addcontentsline{toc}{section}{S}
	\subsection*{Sprint}
	Periodo di tempo prefissato entro il quale lavorare producendo dei risultati documentati: sono il nucleo delle metodologie Agile, atte a produrre risultati discreti in dimensione ma in maniera costante.
	\newpage
	\section*{T}
	\addcontentsline{toc}{section}{T}
	\subsection*{Telegram}
	Telegram è un'applicazione di messaggistica istantanea e una piattaforma di comunicazione che consente agli utenti di scambiare messaggi di testo, foto, video, documenti e altri tipi di file.
	\newpage	
	\section*{U}
	\addcontentsline{toc}{section}{U}
	\subsection*{UML}
	Acronimo di Unified Modeling Language, utilizzato comunemente nel contesto dello sviluppo software per descrivere ed analizzare in modo immediato un progetto, illustrando le interazioni con il sistema da parte degli attori coinvolti.
	\newpage
	\section*{V}
	\addcontentsline{toc}{section}{V}
	\subsection*{Versionamento}
	Processo che realizza il cosiddetto ``version control", stabilendo la cronologia delle azioni compiute per una determinata attività, tracciando i cambiamenti e fornendo la possibilità di tornare a uno stato precedente se necessario.
	\newpage
	\section*{W}
	\addcontentsline{toc}{section}{W}
	\subsection*{Way of Working}
	Il Way of Working, noto anche come ``modo di lavorare", stabilisce l'organizzazione ottimale delle attività di progetto per garantire un'operatività professionale del team. Questo comprende processi operativi, procedure, norme comportamentali e l'utilizzo di strumenti o tecnologie specifiche. L'adozione di un Way of Working efficace può favorire il successo e migliorare la produttività nell'esecuzione delle attività necessarie.
	\newpage
	\section*{X}
	\addcontentsline{toc}{section}{X}
	
	\section*{Y}
	\addcontentsline{toc}{section}{Y}
	
	\section*{Z}
	\addcontentsline{toc}{section}{Z}
	
	
		
\end{document}