% changelog: "0.0.1, 2024-05-13, Linda Barbiero, Verifica completa e aggiunta informazione sulla discussione stato refactor del repository"

\documentclass[8pt]{article}
\usepackage[italian]{babel}
\usepackage[utf8]{inputenc}
\usepackage[letterpaper, left=1in, right=1in, bottom=0.75in, top=0.75in]{geometry}
\usepackage{amsmath}
\usepackage{subfiles}
\usepackage{lipsum}
\usepackage{csquotes}
\usepackage{amsfonts}
\usepackage[sfdefault]{plex-sans}
\usepackage{float}
\usepackage{pifont}
\usepackage{mathabx}
\usepackage[euler]{textgreek}
\usepackage{makecell}
\usepackage{tikz}
\usepackage{wrapfig}
\usepackage{siunitx}
\usepackage{amssymb} 
\usepackage{tabularx}
\usepackage{threeparttable}
\usepackage{adjustbox}
\usepackage[document]{ragged2e}
\usepackage{floatflt}
\usepackage[hidelinks]{hyperref}
\usepackage{graphicx}
\usepackage{hyperref}
\setcounter{tocdepth}{4}
\usepackage{caption}
\usepackage{multicol}
\usepackage{tikz}
\setlength\parindent{0pt}
\captionsetup{font=footnotesize}
\usepackage{fancyhdr} 
\usepackage{graphicx}
\usepackage{capt-of}% 
\usepackage{booktabs}
\usepackage{varwidth}

% -- IMPORTANTE -- %
% utilizzare `` " per le virgolette (``esempio") altrimenit non si chiudono
% non utilizzare \glossterm nei verbali

% -- TITOLO INTESTAZIONE -- %
\newcommand{\customtitle}{VERBALE INTERNO DEL 2024-05-09} % o ESTERNO

% -- STILE INTESTAZIONE -- %
\fancypagestyle{mystyle}{
	\fancyhf{} 
  \fancyhead[R]{\includegraphics[height=1cm]{../../../template/images/logos/NaN1fy_logo.png}} 
	\fancyhead[L]{\leftmark} 
	\renewcommand{\headrulewidth}{1pt} 
	\fancyhead[L]{\customtitle} 
	\renewcommand{\headsep}{1.3cm} 
	\fancyfoot[C]{\thepage} 
}

% -- PER LA FIRMA -- %
\newcommand{\signatureline}[1]{%
	 \par\vspace{0.5cm}
	\noindent\makebox[\linewidth][r]{\rule{0.2\textwidth}{0.5pt}\hspace{3cm}\makebox[0pt][r]{\vspace{3pt}\footnotesize #1}}%
}

% -- PER IL GLOSSARIO -- %
\newcommand{\glossterm}[1]{#1\textsuperscript{G}} % inserisci \glossterm{termine}

% -- per abilitare 4x sottosezioni es 2.1.1.1
% quando lo utilizi metti un // esterno al comando subsubsubsection{esempio} altrimenti è attacatto al titoletto, non modificare il comando (se metti //// nel comando è troppo spaziato quando ci sono table e figure
\setcounter{secnumdepth}{4}
\newcommand{\subsubsubsection}[1]{\paragraph{#1}\mbox{}\\}

\begin{document}
\definecolor{myblue}{RGB}{23,103,162}
\begin{titlepage}
	\begin{tikzpicture}[remember picture, overlay]
		\node[anchor=south east, opacity=0.2, yshift = -4cm, xshift= 2em] at (current page.south east)
      {\includegraphics[width=0.7\textwidth, trim=0cm 0cm 5cm 0cm, clip]{../../../template/images/logos/Universita_Padova_transparent.png}}; 
		\node[anchor=north west, opacity=1, yshift = 4.2cm, xshift= 1.4cm, scale=1.6] at (current
      page.south west) {\includegraphics[width=4cm]{../../../template/images/logos/NaN1fy_logo.png}};
	\end{tikzpicture}
	
	\begin{minipage}[t]{0.47\textwidth}
		{\large{\textsc{Destinatari}}
			\vspace{3mm}
			\\ \large{\textsc{Prof. Tullio Vardanega}}
			\\ \large{\textsc{Prof. Riccardo Cardin}}
		}
	\end{minipage}
	\hfill
	\begin{minipage}[t]{0.47\textwidth}\raggedleft
		{\large{\textsc{Redattori}}
			\vspace{3mm}
			{\\\large{\textsc{Guglielmo Barison}\\}} % massimo due 
		  {\large{\textsc{}}}
    }
		\vspace{8mm}
		
		{\large{\textsc{Verificatori}}
			\vspace{3mm}
			{\\\large{\textsc{Linda Barbiero}\\}} % massimo due 
			{\large{\textsc{Pietro Busato}}}
			
		}
		\vspace{4mm}\vspace{4mm}
	\end{minipage}
	\vspace{4cm}
	\begin{center}
		\begin{flushright}
			{\fontsize{30pt}{52pt}\selectfont \textbf{Verbale Interno Del\\2024-05-09\\}} % o ESTERNO
		\end{flushright}
		\vspace{3cm}
	\end{center}
	\vspace{8cm}
	{\small \textsc{\href{mailto: nan1fyteam.unipd@gmail.com}{nan1fyteam.unipd@gmail.com}}}
\end{titlepage}
\pagestyle{mystyle}
\section*{Registro delle Modifiche}
\begin{table}[ht!]	
	\centering
	\begin{tabular}{p{1.2cm} p{2cm} p{6cm} p{3cm} p{2cm}}
		\toprule
		\textbf{Versione}& \textbf{Data} & \textbf{Descrizione} & \textbf{Autore} & \textbf{Ruolo} \\
		\midrule
		0.0.1 & 2024-05-13 & Verifica completa e aggiunta informazione sulla discussione stato refactor
      del repository. & Linda Barbiero & Verificatore \\\\
		0.0.0 & 2024-05-09 & Prima stesura documento. & Guglielmo Barison & Redattore \\
		\bottomrule
		% Ruolo Redattore o Verificatore
	\end{tabular}
	\caption{Registro delle modifiche.}
	\label{table:Registro delle modifiche}
\end{table}
\newpage
\tableofcontents
\clearpage
\newpage
\justifying
\section{Contenuti del Verbale}
\subsection{Informazioni sulla riunione}
\begin{itemize}
	\setlength\itemsep{0em}
	\item\textbf{Luogo:} Chiamata Discord;
	\item\textbf{Ora di inizio:} 15:00;
	\item\textbf{Ora di fine:}  16:00.
\end{itemize}
\begin{table}[ht!]
	\begin{minipage}[t]{0.5\linewidth}
		\centering
		\begin{tabular}{p{3cm} p{3cm}}
			\toprule
			\textbf{Partecipante} & \textbf{Durata presenza} \\
			\midrule
			Guglielmo Barison & 1.0 h \\
			Linda Barbiero &  1.0 h \\
			Pietro Busato & 1.0 h \\
			Oscar Konieczny & 1.0 h \\
			Davide Donanzan & 1.0 h \\
			Veronica Tecchiati & 1.0 h \\
			\bottomrule
		\end{tabular}
		\caption{Partecipanti NaN1fy.}
		\label{table:Partecipanti NaN1fy}
	\end{minipage} 
	% \begin{minipage}[t]{0.5\linewidth} % -- COMMENTA/DECOMMENTA DA QUI
	% 	\centering
	% 	\begin{tabular}{p{3cm} p{3cm}}
	% 		\toprule
	% 		\textbf{Partecipante} & \textbf{Durata presenza} \\
	% 		\midrule
	% 		XXXX XXXX & X.X h \\
	% 		XXXX XXXX &  X.X h \\
	% 		\bottomrule
	% 	\end{tabular}
	% 	\caption{Partecipanti XXXX}
	% 	\label{table:Partecipanti XXXX}
	% \end{minipage} % -- A QUI PER TOGLIERE AGGIUNGERE
\end{table}
\subsection{Ordine del giorno}
\begin{itemize}
	\setlength\itemsep{0em}
  \item Discussione pre ``Diario di Bordo" del 2024-05-10;	
  \item Punto della situazione del prodotto software;
  \item Punto della situazione su alcuni documenti.
\end{itemize}
\subsection{Sintesi dell'incontro}
La riunione ha preso avvio con un aggiornamento fornito dai vari membri del team sullo stato di
avanzamento delle attivit\`{a}. Si è ritenuto essenziale individuare la corretta indicazione per i
termini del \textit{Glossario} all'interno della documentazione. In particolare, ci si è chiesti come
comportarsi nel caso di più istanze dello stesso termine e in casi particolari, come le intestazioni
delle tabelle. Il dubbio è rimasto irrisolto e il gruppo ha scelto all'unanimità di porre questa
questione nel prossimo ``Diario di Bordo". Prosegue la codifica del software rispettando cos\`{i} la data
di scadenza del progetto, prestando particolare attenzione alla stesura del documento \textit{Norme di
Progetto}, che era stato trascurato nei periodi precedenti. La stesura dei casi d'uso nell'\textit{Analisi
dei Requisiti} viene rimandata poiché la maggior parte di essi riguarda la componente di
visualizzazione, che ad oggi non è ancora sufficientemente matura.
\\
Riguardante il tema dell refactor del repository di GitHub per poter assegnare delle Epiche alle attività,
il gruppo ha concluso come le priorità attualmente siano altre, scegliendo di procrastinare questo genere di miglioria al prossimo Sprint.
\subsection{Decisioni prese}
\begin{itemize}
	\setlength\itemsep{0em}
  \item Completare la stesura delle \textit{Norme di Progetto} in tempo ragionevole.
	% consigliata la forma \textit{Viene adottato} quando viene adottato un certo modo di fare/strumento
	% per nomi di aziene e capitolati usare \texttt{}, e.g \texttt{Easy meal} \texttt{C6} 
\end{itemize}
\newpage
\section{Attività da svolgere}
\begin{table}[ht!]
	\centering
	\begin{tabular}{p{7cm}cp{7cm}}
		\toprule
		\textbf{Titolo} & \textbf{\# Issue} & \textbf{Redattori} \\
		\midrule
		\href{https://github.com/NaN1fy/docs/issues/14}{\underline{Continuazione ``Glossario"}} & 14 & Guglielmo Barison, Oscar Konieczny\\\\
		\href{https://github.com/NaN1fy/docs/issues/15}{\underline{Continuazione ``Analisi dei requisiti"}} & 15 & Davide Donanzan, Veronica Tecchiati\\\\
		\href{https://github.com/NaN1fy/docs/issues/16}{\underline{Continuazione ``Piano di progetto"}}  & 16 & Guglielmo Barison\\\\
		\href{https://github.com/NaN1fy/docs/issues/17}{\underline{Continuazione ``Piano di qualifica"}} & 17 & Davide Donanzan, Guglielmo Barison\\\\
		\href{https://github.com/NaN1fy/docs/issues/20}{\underline{Continuazione ``Norme di progetto"}} & 20 & Linda Barbiero, Pietro Busato, Oscar Konieczny\\\\
        \href{https://github.com/NaN1fy/SyncCity/issues/9}{\underline{Aggiunta nuove tipologie di sensori}} & 9*\tnote{*} & Davide Donanzan, Veronica Tecchiati, Guglielmo Barison\\\\
		\href{https://github.com/NaN1fy/SyncCity/issues/10}{\underline{Migliorie dashboard}} & 10*\tnote{*} & Davide Donanzan, Linda Barbiero\\\\
		\bottomrule
	\end{tabular}
	\begin{tablenotes}
		\vspace{1em}
		\item * indica che la issue fa parte della repository SyncCity
	\end{tablenotes}
	\caption{Attività da svolgere.}
	\label{table:Attivita da svolgere}
\end{table}
\end{document}
