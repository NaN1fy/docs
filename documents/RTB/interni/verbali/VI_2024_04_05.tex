% changelog: "1.0.0, 2024-04-16, Approvazione per RTB"
\documentclass[8pt]{article}
\usepackage[italian]{babel}
\usepackage[utf8]{inputenc}
\usepackage[letterpaper, left=1in, right=1in, bottom=0.75in, top=0.75in]{geometry}
\usepackage{amsmath}
\usepackage{subfiles}
\usepackage{lipsum}
\usepackage{csquotes}
\usepackage{amsfonts}
\usepackage[sfdefault]{plex-sans}
\usepackage{float}
\usepackage{pifont}
\usepackage{mathabx}
\usepackage[euler]{textgreek}
\usepackage{makecell}
\usepackage{tikz}
\usepackage{wrapfig}
\usepackage{siunitx}
\usepackage{amssymb} 
\usepackage{tabularx}
\usepackage{threeparttable}
\usepackage[document]{ragged2e}
\usepackage{floatflt}
\usepackage[hidelinks]{hyperref}
\usepackage{graphicx}
\usepackage{hyperref}
\setcounter{tocdepth}{4}
\usepackage{caption}
\usepackage{multicol}
\usepackage{tikz}
\setlength\parindent{0pt}
\captionsetup{font=footnotesize}
\usepackage{fancyhdr} 
\usepackage{graphicx}
\usepackage{capt-of}% 
\usepackage{booktabs}
\usepackage{varwidth}

% -- TITOLO INTESTAZIONE -- %
\newcommand{\customtitle}{VERBALE INTERNO DEL 2024-04-05} % o ESTERNO

% -- STILE INTESTAZIONE -- %
\fancypagestyle{mystyle}{
	\fancyhf{} 
	\fancyhead[R]{\includegraphics[height=1cm]{../../../template/images/logos/NaN1fy_logo.png}} 
	\fancyhead[L]{\leftmark} 
	\renewcommand{\headrulewidth}{1pt} 
	\fancyhead[L]{\customtitle} 
	\renewcommand{\headsep}{1.3cm} 
	\fancyfoot[C]{\thepage} 
}

% -- PER LA FIRMA -- %
\newcommand{\signatureline}[1]{%
	 \par\vspace{0.5cm}
	\noindent\makebox[\linewidth][r]{\rule{0.2\textwidth}{0.5pt}\hspace{3cm}\makebox[0pt][r]{\vspace{3pt}\footnotesize #1}}%
}

% -- PER IL GLOSSARIO -- %
\newcommand{\glossterm}[1]{#1\textsuperscript{G}} % inserisci \glossterm{termine}

\begin{document}
\definecolor{myblue}{RGB}{23,103,162}
\begin{titlepage}
	\begin{tikzpicture}[remember picture, overlay]
		\node[anchor=south east, opacity=0.2, yshift = -4cm, xshift= 2em] at (current page.south east) {\includegraphics[width=0.7\textwidth, trim=0cm 0cm 5cm 0cm, clip]{../../../template/images/logos/Universita_Padova_transparent.png}}; 
		\node[anchor=north west, opacity=1, yshift = 4.2cm, xshift= 1.4cm, scale=1.6] at (current page.south west) {\includegraphics[width=4cm]{../../../template/images/logos/NaN1fy_logo.png}};
	\end{tikzpicture}
	
	\begin{minipage}[t]{0.47\textwidth}
		{\large{\textsc{Destinatari}}
			\vspace{3mm}
			\\ \large{\textsc{Prof. Tullio Vardanega}}
			\\ \large{\textsc{Prof. Riccardo Cardin}}
		}
	\end{minipage}
	\hfill
	\begin{minipage}[t]{0.47\textwidth}\raggedleft
		{\large{\textsc{Redattori}}
			\vspace{3mm}
			{\\\large{\textsc{Oscar Konieczny}\\}} % massimo due 
			{\large{\textsc{}}}
			
			
		}
		\vspace{8mm}
		
		{\large{\textsc{Verificatori}}
			\vspace{3mm}
			{\\\large{\textsc{Guglielmo Barison}\\}} % massimo due 
			{\large{\textsc{Pietro Busato}}}
			
		}
		\vspace{4mm}\vspace{4mm}
	\end{minipage}
	\vspace{4cm}
	\begin{center}
		\begin{flushright}
			{\fontsize{30pt}{52pt}\selectfont \textbf{Verbale Interno Del\\2024-04-05\\}} % o ESTERNO
		\end{flushright}
		\vspace{3cm}
	\end{center}
	\vspace{8 cm}
	{\small \textsc{\href{mailto: nan1fyteam.unipd@gmail.com}{nan1fyteam.unipd@gmail.com}}}
\end{titlepage}
\pagestyle{mystyle}
\section*{Registro delle Modifiche}
\begin{table}[ht!]	
	\centering
	\begin{tabular}{p{1.2cm} p{2cm} p{6cm} p{3cm} p{2cm}}
		\toprule
		\textbf{Versione} & \textbf{Data} & \textbf{Descrizione} & \textbf{Autore} & \textbf{Ruolo} \\
		\midrule
  		1.0.0 & 2024-04-22 & \textbf{Approvazione per RTB} 
    		0.0.3 & 2024-04-22 & Correzione finale & Pietro Busato & Verificatore \\\\
		0.0.2 & 2024-04-16 & Ampliamento attività da svolgere, correzioni punteggiatura & Guglielmo Barison & Verificatore \\\\ 
		0.0.1 & 2024-04-16 & Correzioni varie & Oscar Konieczny & Redattore \\\\ % spazio tra le righe
		0.0.0 & 2024-04-05 & Prima stesura del verbale & Oscar Konieczny & Redattore \\
		\bottomrule
		% Ruolo Redattore o Verificatore
	\end{tabular}
	\caption{Registro delle modifiche.}
	\label{table:Registro delle modifiche}
\end{table}
\newpage
\tableofcontents
\clearpage
\newpage
\justifying
\section{Contenuti del Verbale}
\subsection{Informazioni sulla riunione}
\begin{itemize}
	\setlength\itemsep{0em}
	\item\textbf{Luogo:} Chiamata Discord;
	\item\textbf{Ora di inizio:} 15:00;
	\item\textbf{Ora di fine:}  16:30.
\end{itemize}
\begin{table}[ht!]
	\begin{minipage}[t]{0.5\linewidth}
		\centering
		\begin{tabular}{p{3cm} p{3cm}}
			\toprule
			\textbf{Partecipante} & \textbf{Durata presenza} \\
			\midrule
			Guglielmo Barison & 1.5 h \\
			Linda Barbiero &  0.0 h \\
			Pietro Busato & 1.5 h \\
			Oscar Konieczny & 1.5 h \\
			Davide Donanzan & 1.5 h \\
			Veronica Tecchiati & 1.5 h \\
			\bottomrule
		\end{tabular}
		\caption{Partecipanti NaN1fy}
		\label{table:Partecipanti NaN1fy}
	\end{minipage}
\end{table}
\subsection{Ordine del giorno}
\begin{itemize}
	\setlength\itemsep{0em}
	\item Discussione a riguardo degli incontri avvenuti il giorno precedente;
	\item Organizzazione del lavoro in vista del RTB.
\end{itemize}
\subsection{Sintesi dell'incontro}
La riunione è iniziata con la discussione degli incontri avvenuti il giorno precedente. In particolare si è discusso su come procedere in vista del prossimo SAL fissato con \texttt{SyncLab}. Successivamente sono stati assegnati i ruoli per le seguenti due settimane, corrispondenti al primo sprint. Si sono infine stabiliti i vari redattori per i vari documenti che necessitano la stesura.
\subsection{Decisioni prese}
\begin{itemize}
	\setlength\itemsep{0em}
	\item Suddivisione ruoli per lo sprint, la cui fine cade il 2024-04-19;
    \item La possibilità di eseguire modifiche direttamente sul ramo principale delle varie repository è stata bloccata. Da adesso in poi, ogni aggiornamento dovrà essere richiesto tramite ``pull-request" e approvato da un terzo, il verificatore;
    \item \textit{Viene adottato} il procedimento di iniziare ad associare le decisioni prese in seguito alle riunioni che scatenano delle azioni con le loro corrispondenti issue tramite collegamento ipertestuale. A differenza del colore utilizzato negli altri testi, nella tabella Attività da svolgere (\ref{table:Attivita da svolgere}) i collegamenti ipertestuali verranno indicati con sottolineatura;
	\item Assegnamento delle nuove attività da svolgere.
	% consigliata la forma \textit{Viene adottato} quando viene adottato un certo modo di fare/strumento
	% per nomi di aziene e capitolati usare \texttt{}, e.g \texttt{Easy meal} \texttt{C6} 
\end{itemize}
\newpage
\section{Attività da svolgere}
\begin{table}[ht!]
	\centering
	\begin{tabular}{lcl}
		\toprule
		\textbf{Titolo} & \textbf{\# Issue} & \textbf{Redattori} \\
		\midrule
		\href{https://github.com/NaN1fy/docs/issues/14}{\underline{Stesura ``Glossario"}} & 14 & Guglielmo Barison, Oscar Konieczny\\\\
		\href{https://github.com/NaN1fy/docs/issues/15}{\underline{Stesura ``Analisi dei requisiti"}} & 15 & Davide Donanzan, Veronica Tecchiati\\\\
		\href{https://github.com/NaN1fy/docs/issues/16}{\underline{Stesura ``Piano di progetto"}}  & 16 & Linda Barbiero, Veronica Tecchiati\\\\
		\href{https://github.com/NaN1fy/docs/issues/17}{\underline{Stesura ``Piano di qualifica"}} & 17 & Davide Donanzan, Guglielmo Barison\\\\
		\href{https://github.com/NaN1fy/docs/issues/20}{\underline{Stesura ``Norme di progetto"}} & 20 & Linda Barbiero, Pietro Busato\\\\
		\href{https://github.com/NaN1fy/SyncCity/issues/1}{\underline{Stesura codice generatore dati Python}} & 1*\tnote{*} & Pietro Busato, Guglielmo Barison\\
		\bottomrule
	\end{tabular}
	\begin{tablenotes}
		\vspace{1em}
		\item * indica che la issue fa parte della repository \texttt{SyncCity}
	\end{tablenotes}
	\caption{Attività da svolgere.}
	\label{table:Attivita da svolgere}
\end{table}
\end{document}
