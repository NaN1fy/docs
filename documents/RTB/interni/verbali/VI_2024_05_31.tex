% changelog: "1.0.0, 2024-06-03, Approvazione per RTB"

\documentclass[8pt]{article}
\usepackage[italian]{babel}
\usepackage[utf8]{inputenc}
\usepackage[letterpaper, left=1in, right=1in, bottom=0.75in, top=0.75in]{geometry}
\usepackage{amsmath}
\usepackage{subfiles}
\usepackage{lipsum}
\usepackage{csquotes}
\usepackage{amsfonts}
\usepackage[sfdefault]{plex-sans}
\usepackage{float}
\usepackage{pifont}
\usepackage{mathabx}
\usepackage[euler]{textgreek}
\usepackage{makecell}
\usepackage{tikz}
\usepackage{wrapfig}
\usepackage{siunitx}
\usepackage{amssymb} 
\usepackage{tabularx}
\usepackage{threeparttable}
\usepackage{adjustbox}
\usepackage[document]{ragged2e}
\usepackage{floatflt}
\usepackage[hidelinks]{hyperref}
\usepackage{graphicx}
\usepackage{hyperref}
\setcounter{tocdepth}{4}
\usepackage{caption}
\usepackage{multicol}
\usepackage{tikz}
\setlength\parindent{0pt}
\captionsetup{font=footnotesize}
\usepackage{fancyhdr} 
\usepackage{graphicx}
\usepackage{capt-of}% 
\usepackage{booktabs}
\usepackage{varwidth}

% -- IMPORTANTE -- %
% utilizzare `` " per le virgolette (``esempio") altrimenit non si chiudono
% non utilizzare \glossterm nei verbali

% -- TITOLO INTESTAZIONE -- %
\newcommand{\customtitle}{VERBALE INTERNO DEL 2024-05-31} % o ESTERNO

% -- STILE INTESTAZIONE -- %
\fancypagestyle{mystyle}{
	\fancyhf{} 
  \fancyhead[R]{\includegraphics[height=1cm]{../../../template/images/logos/NaN1fy_logo.png}} 
	\fancyhead[L]{\leftmark} 
	\renewcommand{\headrulewidth}{1pt} 
	\fancyhead[L]{\customtitle} 
	\renewcommand{\headsep}{1.3cm} 
	\fancyfoot[C]{\thepage} 
}

% -- PER LA FIRMA -- %
\newcommand{\signatureline}[1]{%
	 \par\vspace{0.5cm}
	\noindent\makebox[\linewidth][r]{\rule{0.2\textwidth}{0.5pt}\hspace{3cm}\makebox[0pt][r]{\vspace{3pt}\footnotesize #1}}%
}

% -- PER IL GLOSSARIO -- %
\newcommand{\glossterm}[1]{#1\textsuperscript{G}} % inserisci \glossterm{termine}

% -- per abilitare 4x sottosezioni es 2.1.1.1
% quando lo utilizi metti un // esterno al comando subsubsubsection{esempio} altrimenti è attacatto al titoletto, non modificare il comando (se metti //// nel comando è troppo spaziato quando ci sono table e figure
\setcounter{secnumdepth}{4}
\newcommand{\subsubsubsection}[1]{\paragraph{#1}\mbox{}\\}

\begin{document}
\definecolor{myblue}{RGB}{23,103,162}
\begin{titlepage}
	\begin{tikzpicture}[remember picture, overlay]
		\node[anchor=south east, opacity=0.2, yshift = -4cm, xshift= 2em] at (current page.south east)
      {\includegraphics[width=0.7\textwidth, trim=0cm 0cm 5cm 0cm, clip]{../../../template/images/logos/Universita_Padova_transparent.png}}; 
		\node[anchor=north west, opacity=1, yshift = 4.2cm, xshift= 1.4cm, scale=1.6] at (current
      page.south west) {\includegraphics[width=4cm]{../../../template/images/logos/NaN1fy_logo.png}};
	\end{tikzpicture}
	
	\begin{minipage}[t]{0.47\textwidth}
		{\large{\textsc{Destinatari}}
			\vspace{3mm}
			\\ \large{\textsc{Prof. Tullio Vardanega}}
			\\ \large{\textsc{Prof. Riccardo Cardin}}
		}
	\end{minipage}
	\hfill
	\begin{minipage}[t]{0.47\textwidth}\raggedleft
		{\large{\textsc{Redattori}}
			\vspace{3mm}
			{\\\large{\textsc{Linda Barbiero}\\}} % massimo due 
		  {\large{\textsc{}}}
    }
		\vspace{8mm}
		
		{\large{\textsc{Verificatori}}
			\vspace{3mm}
			{\\\large{\textsc{Oscar Konieczny}\\}} % massimo due 
			{\large{\textsc{Davide Donanzan}}}
			
		}
		\vspace{4mm}\vspace{4mm}
	\end{minipage}
	\vspace{4cm}
	\begin{center}
		\begin{flushright}
			{\fontsize{30pt}{52pt}\selectfont \textbf{Verbale Interno Del\\2024-05-31\\}} % o ESTERNO
		\end{flushright}
		\vspace{3cm}
	\end{center}
	\vspace{8.2cm}
	{\small \textsc{\href{mailto: nan1fyteam.unipd@gmail.com}{nan1fyteam.unipd@gmail.com}}}
\end{titlepage}
\pagestyle{mystyle}
\section*{Registro delle Modifiche}
\begin{table}[ht!]	
	\centering
	\begin{tabular}{p{1.2cm} p{2cm} p{6cm} p{3cm} p{2cm}}
		\toprule
		\textbf{Versione}& \textbf{Data} & \textbf{Descrizione} & \textbf{Autore} & \textbf{Ruolo} \\
		\midrule
		1.0.0 & 2024-06-03 & \textbf{Approvazione per RTB} & & \\\\
		0.0.2 & 2024-06-03 & Verifica e modifiche minori. & Davide Donanzan & Verificatore \\\\
		0.0.1 & 2024-06-03 & Piccole modifiche. & Oscar Konieczny & Verificatore \\\\
		0.0.0 & 2024-05-31 & Stesura documento. & Linda Barbiero & Redattore \\
		\bottomrule
		% Ruolo Redattore o Verificatore
	\end{tabular}
	\caption{Registro delle modifiche.}
	\label{table:Registro delle modifiche}
\end{table}
\newpage
\tableofcontents
\clearpage
\newpage
\justifying
\section{Contenuti del Verbale}
\subsection{Informazioni sulla riunione}
\begin{itemize}
	\setlength\itemsep{0em}
	\item\textbf{Luogo:} Chiamata Discord;
	\item\textbf{Ora di inizio:} 16:40;
  \item\textbf{Ora di fine:}  18:00;
\end{itemize}
\begin{table}[ht!]
	\begin{minipage}[t]{0.5\linewidth}
		\centering
		\begin{tabular}{p{3cm} p{3cm}}
			\toprule
			\textbf{Partecipante} & \textbf{Durata presenza} \\
			\midrule
			Guglielmo Barison & 1.3 h \\
			Linda Barbiero &  1.3 h \\
			Pietro Busato & 1.3 h \\
			Oscar Konieczny & 1.3 h \\
			Davide Donanzan & 1.3 h \\
			Veronica Tecchiati & 1.3 h \\
			\bottomrule
		\end{tabular}
		\caption{Partecipanti NaN1fy.}
		\label{table:Partecipanti NaN1fy}
	\end{minipage} 
	% \begin{minipage}[t]{0.5\linewidth} % -- COMMENTA/DECOMMENTA DA QUI
	% 	\centering
	% 	\begin{tabular}{p{3cm} p{3cm}}
	% 		\toprule
	% 		\textbf{Partecipante} & \textbf{Durata presenza} \\
	% 		\midrule
	% 		XXXX XXXX & X.X h \\
	% 		XXXX XXXX &  X.X h \\
	% 		\bottomrule
	% 	\end{tabular}
	% 	\caption{Partecipanti XXXX}
	% 	\label{table:Partecipanti XXXX}
	% \end{minipage} % -- A QUI PER TOGLIERE AGGIUNGERE
\end{table}
\subsection{Ordine del giorno}
\begin{itemize}
	\setlength\itemsep{0em}
    \item Discussione post SAL avvenuto in data 2024-05-31 con l'azienda SyncLab;
    \item Retrospettiva del quarto Sprint concluso e consuntivo ore;
	\item Pianificazione del quinto ed ultimo Sprint prima dell'RTB, suddivisione dei ruoli e calcolo preventivo ore.
\end{itemize}
\subsection{Sintesi dell'incontro}
Alla luce delle riflessioni esposte durante l'ultimo ``Diario di Bordo", la riunione si è svolta
successivamente al SAL tenutosi poco prima nel pomeriggio, con la consapevolezza del prossimo incontro di SAL in data 2024-06-07.
\\
L'incontro è iniziato valutando le considerazioni della proponente, in revisione dell'RTB,
con scadenza ultima prevista per il giorno 2024-06-07. Il gruppo dunque conclude ufficialmente lo Sprint precedente valutando il fatto e le prossime rifiniture da apportare,
oltre al calcolo del consuntivo relativo.
Data la scadenza precedentemente esposta, lo Sprint imminente sarà dunque di una settimana, a differenza delle precedenti, e per tal motivo il gruppo ha valutato di mantenere lo stesso responsabile.\\
La riunione si è infine conclusa effettuando lo sprint planning e il preventivo ore del prossimo Sprint.

\subsection{Decisioni prese}
\begin{itemize}
	\setlength\itemsep{0em}
	\item Mantenimento dello stesso responsabile del Sprint precedente: Pietro Busato;
	\item Suddivisione dei restanti ruoli e calcolo preventivo ore per lo Sprint con fine il
      2024-06-07;
	\item Suddivisione della revisione dei documenti e delle ultime rifiniture tra i membri.
	\item Preparazione della presentazione dell'RTB.
\end{itemize}
\newpage
\section{Attività da svolgere}
\begin{table}[ht!]
	\centering
	\begin{tabular}{p{7cm}cp{7cm}}
		\toprule
		\textbf{Titolo} & \textbf{\# Issue} & \textbf{Redattori} \\
		\midrule
		\href{https://github.com/NaN1fy/docs/issues/19}{\underline{Stesura lettera di presentazione RTB}} & 19 & Guglielmo Barison \\\\
		\href{https://github.com/NaN1fy/SyncCity/issues/25}{\underline{Improvement Dashboard Urbanistica}} & 25*\tnote{*} & Guglielmo Barison \\\\
		\href{https://github.com/NaN1fy/SyncCity/issues/27}{\underline{Improvement Dashboard Ambientale}} & 27*\tnote{*} & Linda Barbiero, Guglielmo Barison \\\\
		\bottomrule
	\end{tabular}
	\begin{tablenotes}
		\vspace{1em}
		\item * indica che la issue fa parte della repository SyncCity
	\end{tablenotes}
	\caption{Attività da svolgere.}
	\label{table:Attivita da svolgere}
\end{table}
\end{document}
