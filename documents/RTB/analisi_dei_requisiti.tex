% changelog: "0.0.0, 2024-04-09, Davide Donanzan, Stesura del file"
\documentclass[8pt]{article}
\usepackage[italian]{babel}
\usepackage[utf8]{inputenc}
\usepackage[letterpaper, left=1in, right=1in, bottom=0.75in, top=0.75in]{geometry}
\usepackage{amsmath}
\usepackage{subfiles}
\usepackage{lipsum}
\usepackage{csquotes}
\usepackage{amsfonts}
\usepackage[sfdefault]{plex-sans}
\usepackage{float}
\usepackage{pifont}
\usepackage{mathabx}
\usepackage[euler]{textgreek}
\usepackage{makecell}
\usepackage{tikz}
\usepackage{wrapfig}
\usepackage{siunitx}
\usepackage{amssymb} 
\usepackage{tabularx}
\usepackage{adjustbox}
\usepackage[document]{ragged2e}
\usepackage{floatflt}
\usepackage[hidelinks]{hyperref}
\usepackage{graphicx}
\usepackage{hyperref}
\setcounter{tocdepth}{4}
\usepackage{caption}
\usepackage{multicol}
\usepackage{tikz}
\setlength\parindent{0pt}
\captionsetup{font=footnotesize}
\usepackage{fancyhdr} 
\usepackage{graphicx}
\usepackage{capt-of}% 
\usepackage{booktabs}
\usepackage{varwidth}

% -- TITOLO INTESTAZIONE -- %
\newcommand{\customtitle}{Analisi dei Requisiti} % o ESTERNO

% -- STILE INTESTAZIONE -- %
\fancypagestyle{mystyle}{
	\fancyhf{} 
	\fancyhead[R]{\includegraphics[height=1cm]{../template/images/logos/NaN1fy_logo.png}} 
	\fancyhead[L]{\leftmark} 
	\renewcommand{\headrulewidth}{1pt} 
	\fancyhead[L]{\customtitle} 
	\renewcommand{\headsep}{1.3cm} 
	\fancyfoot[C]{\thepage} 
}

% -- PER LA FIRMA -- %
\newcommand{\signatureline}[1]{%
	 \par\vspace{0.5cm}
	\noindent\makebox[\linewidth][r]{\rule{0.2\textwidth}{0.5pt}\hspace{3cm}\makebox[0pt][r]{\vspace{3pt}\footnotesize #1}}%
}

% -- PER IL GLOSSARIO -- %
\newcommand{\glossterm}[1]{#1\textsuperscript{G}} % inserisci \glossterm{termine}

\begin{document}
\definecolor{myblue}{RGB}{23,103,162}
\begin{titlepage}
	\begin{tikzpicture}[remember picture, overlay]
		\node[anchor=south east, opacity=0.2, yshift = -4cm, xshift= 2em] at (current page.south east) {\includegraphics[width=0.7\textwidth, trim=0cm 0cm 5cm 0cm, clip]{../template/images/logos/Universita_Padova_transparent.png}}; 
		\node[anchor=north west, opacity=1, yshift = 4.2cm, xshift= 1.4cm, scale=1.6] at (current page.south west) {\includegraphics[width=4cm]{../template/images/logos/NaN1fy_logo.png}};
	\end{tikzpicture}
	
	\begin{minipage}[t]{0.47\textwidth}
		{\large{\textsc{Destinatari}}
			\vspace{3mm}
			\\ \large{\textsc{Prof. Tullio Vardanega}}
			\\ \large{\textsc{Prof. Riccardo Cardin}}
		}
	\end{minipage}
	\hfill
	\begin{minipage}[t]{0.47\textwidth}\raggedleft
		{\large{\textsc{Redattori}}
			\vspace{3mm}
			{\\\large{\textsc{Davide Donanzan}\\}} % massimo due 
			% {\large{\textsc{XXXX XXXX}}}
			
			
		}
		\vspace{8mm}
		
		{\large{\textsc{Verificatori}}
			\vspace{3mm}
			{\\\large{\textsc{XXXX XXXX}\\}} % massimo due 
			{\large{\textsc{XXXX XXXX}}}
			
		}
		\vspace{4mm}\vspace{4mm}
	\end{minipage}
	\vspace{4cm}
	\begin{center}
		\begin{flushright}
			{\fontsize{30pt}{52pt}\selectfont \textbf{Analisi dei Requisiti\\}} % o ESTERNO
		\end{flushright}
		\vspace{3cm}
	\end{center}
	\vspace{8 cm}
	{\small \textsc{\href{mailto: nan1fyteam.unipd@gmail.com}{nan1fyteam.unipd@gmail.com}}}
\end{titlepage}
\pagestyle{mystyle}
\section*{Registro delle Modifiche}
\begin{table}[ht!]	
	\centering
	\begin{tabular}{p{1.2cm} p{2cm} p{6cm} p{3cm} p{2cm}}
		\toprule
		\textbf{Versione}& \textbf{Data} & \textbf{Descrizione} & \textbf{Autore} & \textbf{Ruolo} \\
		\midrule
		0.0.0 & 2024-04-09 & Stesura del file. & Davide Donanzan &
		Analista \\\\ % spazio tra le righe
		% X.X.X & YYYY-MM-DD & Lorem ipsum dolor sit amet, consectetur adipiscing elit.  & XXXX XXXX & --- \\
		\bottomrule
		% Ruolo Redattore o Verificatore
	\end{tabular}
	\caption{Registro delle modifiche.}
	\label{table:Registro delle modifiche}
\end{table}
\newpage
\tableofcontents
\clearpage
\newpage
\justifying
\section{Introduzione}
\subsection{Scopo del documento}
Il seguente documento ha come scopo quello di elencare in modo esaustivo i casi d'uso e i requisiti del progetto \texttt{InnovaCity} a seguito di un'attenta analisi del capitolato \texttt{C6} presentato dall'azienda proponente \texttt{SyncLab} e di una successiva discussione con essa attraverso gli incontri svolti.
\subsection{Glossario}
Al fine di ovviare a possibili ambiguità dovute al linguaggio e ai termini utilizzati nel seguente documento, viene fornito un \textit{Glossario v0.0.0} contenente le definizioni dei termini utilizzati aventi un significato specifico. Tali termini saranno evidenziati con l'uso del corsivo e dalla presenza di una G a pedice.
\subsection{Riferimenti}
\subsubsection{Riferimenti Normativi}
\begin{itemize}
	\setlength\itemsep{0em}
	\item \textit{Norme di Progetto v0.0.0}
	\item Capitolato \texttt{C6 - InnovaCity: Smart city monitoring platform:} \\ \url{https://www.math.unipd.it/~tullio/IS-1/2023/Progetto/C6.pdf} \\ \url{https://www.math.unipd.it/~tullio/IS-1/2023/Progetto/C6p.pdf}
	\item \textit{Verbale Esterno 2024/03/12}
	\item \textit{Verbale Esterno 2024/04/03}
	\item Regolamento progetto didattico: \\ \url{https://www.math.unipd.it/~tullio/IS-1/2023/Dispense/PD2.pdf}
\end{itemize}
\subsubsection{Riferimenti Informativi}
\begin{itemize}
	\setlength\itemsep{0em}
	\item Analisi dei requisiti - corso di Ingegneria del Software a.a. 2023/2024: \\ \url{https://www.math.unipd.it/~tullio/IS-1/2023/Dispense/T5.pdf}
    \item Analisi e descrizione delle funzionalità: Use Case e relativi diagammi (UML) - corso di Ingegneria del Software a.a. 2023/2024: \\ \url{https://www.math.unipd.it/~rcardin/swea/2022/Diagrammi%20di%20Attivit%C3%A0.pdf}
\end{itemize}
\newpage
\section{Descrizione}
\subsection{Obiettivi del prodotto}
L'obiettivo del progetto \texttt{InnovaCity} è quello di creare una \textit{piattaforma} atta al monitoraggio di \textit{sensori} sparsi grograficamente nel territorio di una città. I sensori in questione permettono la misurazione e segnalazione di dati \textit{real-time} riguardanti le più disparate caratteristiche e necessità del territorio quali temperatura ed umidità esterna, occupazione di stalli di parcheggio, funzionamento o guasto elettrico di colonnine HPC, traffico stradale e via dicendo. Il \textit{proponente} richiede la simulazione di alcuni dei sensori nominati nonché la gestione dei dati, della loro persistenza e della loro rappresentazione grafica attraverso \textit{widgets} e \textit{grafici}. \\\\\texttt{InnovaCity} permetterà un miglioramento della qualità dei servizi offerti dalla città attraverso il continuo monitoraggio della stessa, ottenendo, gestendo e successivamente condividendo i dati con gli utenti. 
\subsection{Funzionalità del prodotto}
Il prodotto si struttura nelle seguenti funzionalità principali:
\begin{itemize}
	\setlength\itemsep{0em}
	\item Raccolta dati;
	\item Persistenza e strutturazione dati;
	\item Rappresentazione grafica dati.
\end{itemize}
\subsection{Caratteristiche utente}
\subsection{Tecnologie utilizzate}
Il dominio tecnologico utillizzato per il funzionamento del prodotto raccoglie principalmente:
\begin{itemize}
	\setlength\itemsep{0em}
	\item Python: attraverso \textit{faker} si simula l'informazione fornita dai sensori
	\item Apache Kafka: \textit{broker} per distinguere i dati rilevati
	\item ClickHouse: \textit{database OLAP} per la persistenza dei dati ottenuti
	\item Grafana: \textit{piattaforma} per monitorare e analizzare i dati in tempo reale attraverso la loro rappresentazione grafica
\end{itemize}
% togli il commento per la firma
% \signatureline{Padova, YYYY-MM-DD}
%\signatureline{Padova, YYYY-MM-DD}
\end{document}
