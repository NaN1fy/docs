% changelog: "0.1.0, 2024-05-20, Veronica Tecchiati, Stesura sezione 2.3 e inizio sezione 3"
\documentclass[8pt]{article}
\usepackage[italian]{babel}
\usepackage[utf8]{inputenc}
\usepackage[letterpaper, left=1in, right=1in, bottom=0.75in, top=0.75in]{geometry}
\usepackage{amsmath}
\usepackage{subfiles}
\usepackage{lipsum}
\usepackage{csquotes}
\usepackage{amsfonts}
\usepackage[sfdefault]{plex-sans}
\usepackage{float}
\usepackage{pifont}
\usepackage{mathabx}
\usepackage[euler]{textgreek}
\usepackage{makecell}
\usepackage{tikz}
\usepackage{wrapfig}
\usepackage{siunitx}
\usepackage{amssymb} 
\usepackage{tabularx}
\usepackage{adjustbox}
\usepackage[document]{ragged2e}
\usepackage{floatflt}
\usepackage[colorlinks=true,linkcolor=nan1fyblue,urlcolor=nan1fyblue]{hyperref}
\usepackage{graphicx}
\setcounter{tocdepth}{4}
\usepackage{caption}
\usepackage{multicol}
\usepackage{tikz}
\setlength\parindent{0pt}
\captionsetup{font=footnotesize}
\usepackage{fancyhdr} 
\usepackage{graphicx}
\usepackage{capt-of}% 
\usepackage{booktabs}
\usepackage{varwidth}

\definecolor{nan1fyblue}{RGB}{23,103,162}

% -- TITOLO INTESTAZIONE -- %
\newcommand{\customtitle}{ANALISI DEI REQUISITI} % o ESTERNO

% -- STILE INTESTAZIONE -- %
\fancypagestyle{mystyle}{
	\fancyhf{} 
	\fancyhead[R]{\includegraphics[height=1cm]{../../template/images/logos/NaN1fy_logo.png}} 
	\fancyhead[L]{\leftmark} 
	\renewcommand{\headrulewidth}{1pt} 
	\fancyhead[L]{\customtitle} 
	\renewcommand{\headsep}{1.3cm} 
	\fancyfoot[C]{\thepage} 
}

% -- PER LA FIRMA -- %
\newcommand{\signatureline}[1]{%
	 \par\vspace{0.5cm}
	\noindent\makebox[\linewidth][r]{\rule{0.2\textwidth}{0.5pt}\hspace{3cm}\makebox[0pt][r]{\vspace{3pt}\footnotesize #1}}%
}

% -- PER IL GLOSSARIO -- %
\newcommand{\glossterm}[1]{#1\textsuperscript{G}} % inserisci \glossterm{termine}

\begin{document}
\definecolor{myblue}{RGB}{23,103,162}
\begin{titlepage}
	\begin{tikzpicture}[remember picture, overlay]
		\node[anchor=south east, opacity=0.2, yshift = -4cm, xshift= 2em] at (current page.south east)
      {\includegraphics[width=0.7\textwidth, trim=0cm 0cm 5cm 0cm, clip]{../../template/images/logos/Universita_Padova_transparent.png}}; 
		\node[anchor=north west, opacity=1, yshift = 4.2cm, xshift= 1.4cm, scale=1.6] at (current
      page.south west) {\includegraphics[width=4cm]{../../template/images/logos/NaN1fy_logo.png}};
	\end{tikzpicture}
	
	\begin{minipage}[t]{0.47\textwidth}
		{\large{\textsc{Destinatari}}
			\vspace{3mm}
			\\ \large{\textsc{Prof. Tullio Vardanega}}
			\\ \large{\textsc{Prof. Riccardo Cardin}}
		}
	\end{minipage}
	\hfill
	\begin{minipage}[t]{0.47\textwidth}\raggedleft
		{\large{\textsc{Redattori}}
			\vspace{3mm}
			{\\\large{\textsc{Davide Donanzan}\\}}
			{\large{\textsc{Veronica Tecchiati\\}}}	
			{\large{\textsc{Guglielmo Barison}}}
			
		}
		\vspace{8mm}
		
		{\large{\textsc{Verificatori}}
			\vspace{3mm}
			{\\\large{\textsc{XXXX XXXX}\\}} 
			{\large{\textsc{XXXX XXXX}}}
			
		}
		\vspace{4mm}\vspace{4mm}
	\end{minipage}
	\vspace{4cm}
	\begin{center}
		\begin{flushright}
			{\fontsize{30pt}{52pt}\selectfont \textbf{Analisi dei Requisiti\\}} % o ESTERNO
		\end{flushright}
		\vspace{3cm}
	\end{center}
	\vspace{9.5cm}
	{\small \textsc{\href{mailto: nan1fyteam.unipd@gmail.com}{\color{black}nan1fyteam.unipd@gmail.com}}}
\end{titlepage}
\pagestyle{mystyle}
\section*{Registro delle Modifiche}
\begin{table}[ht!]
\hypersetup{hidelinks}
	\centering
	\begin{tabular}{p{1.2cm} p{2cm} p{6cm} p{3cm} p{2cm}}
		\toprule
		\textbf{Versione}& \textbf{Data} & \textbf{Descrizione} & \textbf{Autore} & \textbf{Ruolo} \\
		\midrule
            0.1.0 & 2024-05-20 & Stesura sezione \ref{sec:users} e inizio sezione \ref{sec:use-case} & Veronica Tecchiati & Redattore \\\\
		0.0.0 & 2024-04-09 & Stesura del file. & Davide Donanzan & Redattore \\
		\bottomrule
	\end{tabular}
	\caption{Registro delle modifiche.}
	\label{table:Registro delle modifiche}
\end{table}
\newpage
{\hypersetup{hidelinks} \tableofcontents}
\clearpage
\newpage
{\hypersetup{hidelinks} \listoffigures}
\newpage
{\hypersetup{hidelinks} \listoftables}
\newpage
\justifying
\section{Introduzione}
\subsection{Scopo del documento}
Il seguente documento ha come scopo quello di elencare in modo esaustivo i casi d'uso e i requisiti
del progetto SyncCity a seguito di un'attenta analisi del capitolato C6 presentato dall'azienda proponente SyncLab e di una successiva discussione con essa attraverso gli incontri svolti.

\subsection{Glossario}
Al fine di ovviare a possibili ambiguità dovute al linguaggio e ai termini utilizzati nel seguente
documento, viene fornito un \textit{Glossario vX.X.X} contenente le definizioni dei termini utilizzati aventi un significato specifico. Tali termini saranno evidenziati con l'uso del corsivo e dalla presenza di una G a pedice.
\subsection{Riferimenti}
\subsubsection{Riferimenti Normativi}
\begin{itemize}
	\setlength\itemsep{0em}
	\item \textit{Norme di Progetto vX.X.X};	
	\item \textit{Verbale Esterno 2024-03-12};
	\item \textit{Verbale Esterno 2024-04-03;}
  \item Presentazione e documentazione del capitolato d’appalto C6 - SyncCity:
	\begin{itemize}
		\item \href{https://www.math.unipd.it/~tullio/IS-1/2023/Progetto/C6p.pdf}{https://www.math.unipd.it\textasciitilde{}tullio/IS-1/2023/Progetto/C6p.pdf} (Ultimo accesso: \today)
		\item \href{https://www.math.unipd.it/~tullio/IS-1/2023/Progetto/C6.pdf}{https://www.math.unipd.it/\textasciitilde{}tullio/IS-1/2023/Progetto/C6.pdf} (Ultimo accesso: \today)
\end{itemize}
	\item Regolamento progetto didattico: 
      \begin{itemize}
          \item \href{https://www.math.unipd.it/~tullio/IS-1/2023/Dispense/PD2.pdf}{https://www.math.unipd.it/\textasciitilde{}tullio/IS-1/2023/Dispense/PD2.pdf} (Ultimo accesso: \today)
    \end{itemize}
\end{itemize}
\subsubsection{Riferimenti Informativi}
\begin{itemize}
	\setlength\itemsep{0em}
	\item Dispense T5 - Analisi dei requisiti:
    \begin{itemize}
        \item \href{https://www.math.unipd.it/~tullio/IS-1/2023/Dispense/T5.pdf}{https://www.math.unipd.it/\textasciitilde{}tullio/IS-1/2023/Dispense/T5.pdf} (Ultimo accesso: \today)
    \end{itemize}
    \item Dispense P3 - Analisi e descrizione delle funzionalità: Diagrammi delle attività (UML):
        \begin{itemize}
            \item
                \href{https://www.math.unipd.it/~rcardin/swea/2022/Diagrammi\%20di\%20Attivita.pdf}{https://www.math.unipd.it/\textasciitilde{}rcardin/swea/2022/Diagrammi\%20di\%20Attivit\`{a}.pdf}
                \\ (Ultimo accesso: \today)
        \end{itemize}
\end{itemize}
\newpage

\section{Descrizione}
\subsection{Obiettivi del prodotto}
L'obiettivo del progetto SyncCity è quello di creare una piattaforma atta al monitoraggio di sensori
sparsi geograficamente nel territorio di una città. I sensori in questione permettono la misurazione
e segnalazione di dati real-time riguardanti le più disparate caratteristiche e necessità del
territorio quali temperatura ed umidità esterna, occupazione di stalli di parcheggio, funzionamento
o guasto elettrico di colonnine HPC, traffico stradale e via dicendo. Il proponente richiede la
simulazione di alcuni dei sensori nominati nonché la gestione dei dati, della loro persistenza e
della loro rappresentazione grafica attraverso widgets e grafici. \\\\SyncCity permetterà un miglioramento della qualità dei servizi offerti dalla città attraverso il continuo monitoraggio della stessa, ottenendo, gestendo e successivamente condividendo i dati con gli utenti. 
\subsection{Funzionalità del prodotto}
Il prodotto si struttura nelle seguenti funzionalità:
\begin{itemize}
	\setlength\itemsep{0em}
  \item Data pipeline in grado di:
  \begin{itemize}  
  \item Raccogliere dati; 
	\item Consentirne la persistenza e processare dati provenineti da più sorgenti in real-time.
  \end{itemize}  
	\item Una dashboard che permette di visualizzare i dati raccolti.
\end{itemize}
La piattaforma è progettata principalemente per un solo tipo di utente: l'amministratore pubblico. Quest'ultimo potrà avere accesso a diverse metriche e indicatori sullo stato della città attraverso l'utilizzo delle dashbaord.
\subsection{Caratteristiche degli utenti} \label{sec:users}
L'applicativo è destinato principalmente agli amministratori pubblici. La visualizzazione in tempo reale dei dati, sintetizzati in una \glossterm{dashboard}, consente all'amministrazione di vigilare costantemente sullo stato di salute della città e prendere decisioni tempestive volte al miglioramento della qualità dei servizi urbani. \\ Si assume che i fruitori della piattaforma siano in grado di comprendere ed interpretare le informazioni visualizzate. Gli utenti potranno accedere alla piattaforma utilizzando un qualsiasi dispositivo desktop o mobile connesso alla rete.

\subsection{Tecnologie utilizzate}
Il dominio tecnologico dell'applicativo comprende:
\begin{itemize}
	\setlength\itemsep{0em}
  \item \textbf{Python: }attraverso il quale si simula l'informazione fornita dai sensori;
  \item \textbf{Apache Kafka:} per gestire il gathering dei dati da più fonti;
  \item \textbf{ClickHouse:} database OLAP per la persistenza dei dati ottenuti;
  \item \textbf{Grafana}: piattaforma per monitorare e analizzare i dati in tempo reale attraverso la loro rappresentazione grafica
\end{itemize}


\tikzset{every picture/.style={line width=0.75pt}} %set default line width to 0.75pt        

\begin{figure}
\begin{tikzpicture}[x=0.75pt,y=0.75pt,yscale=-1,xscale=1]
%uncomment if require: \path (0,427); %set diagram left start at 0, and has height of 427

%Image [id:dp8158510236522856] 
    \draw (225.23,191.41) node  {\includegraphics[width=78.34pt,height=36.62pt]{image_adr/kafka-icon-2048x935-cvu4503l.png}};
%Image [id:dp09901368341649897] 
\draw (564.23,189.48) node  {\includegraphics[width=99.34pt,height=51.78pt]{image_adr/grafana_logo_icon_171049.png}};
%Image [id:dp04743607370190417] 
    \draw (389.28,194.72) node  {\includegraphics[width=36.43pt,height=36.43pt]{image_adr/clickhouse.png}};
%Shape: Square [id:dp14758393708670448] 
\draw  [line width=1.5]  (24.67,159.98) -- (84,159.98) -- (84,219.32) -- (24.67,219.32) -- cycle ;
%Straight Lines [id:da729481581554552] 
\draw [line width=1.5]    (94.91,191) -- (162.33,191) ;
\draw [shift={(165.33,191)}, rotate = 180] [color={rgb, 255:red, 0; green, 0; blue, 0 }  ][line width=1.5]    (14.21,-4.28) .. controls (9.04,-1.82) and (4.3,-0.39) .. (0,0) .. controls (4.3,0.39) and (9.04,1.82) .. (14.21,4.28)   ;
%Straight Lines [id:da639851528451087] 
\draw [line width=1.5]    (285.91,191) -- (353.33,191) ;
\draw [shift={(356.33,191)}, rotate = 180] [color={rgb, 255:red, 0; green, 0; blue, 0 }  ][line width=1.5]    (14.21,-4.28) .. controls (9.04,-1.82) and (4.3,-0.39) .. (0,0) .. controls (4.3,0.39) and (9.04,1.82) .. (14.21,4.28)   ;
%Straight Lines [id:da2205341167852083] 
\draw [line width=1.5]    (425.91,190) -- (493.33,190) ;
\draw [shift={(496.33,190)}, rotate = 180] [color={rgb, 255:red, 0; green, 0; blue, 0 }  ][line width=1.5]    (14.21,-4.28) .. controls (9.04,-1.82) and (4.3,-0.39) .. (0,0) .. controls (4.3,0.39) and (9.04,1.82) .. (14.21,4.28)   ;

% Text Node
\draw (345,116) node [anchor=north west][inner sep=0.75pt]   [align=left] {\begin{minipage}[lt]{52.74pt}\setlength\topsep{0pt}
\begin{center}
{\scriptsize Persistenza }\\{\scriptsize e Aggregazione}
\end{center}

\end{minipage}};
% Text Node
\draw (559,115) node [anchor=north west][inner sep=0.75pt]   [align=left] {\begin{minipage}[lt]{30.49pt}\setlength\topsep{0pt}
\begin{center}
{\scriptsize Analitica}\\{\scriptsize real-time}
\end{center}

\end{minipage}};
% Text Node
\draw (207,115) node [anchor=north west][inner sep=0.75pt]   [align=left] {\begin{minipage}[lt]{36.06pt}\setlength\topsep{0pt}
\begin{center}
{\scriptsize Gathering }\\{\scriptsize dei dati}
\end{center}

\end{minipage}};
% Text Node
\draw (116,166) node [anchor=north west][inner sep=0.75pt]   [align=left] {{\tiny dati}};
% Text Node
\draw (306,166) node [anchor=north west][inner sep=0.75pt]   [align=left] {{\tiny dati}};
% Text Node
\draw (448,166) node [anchor=north west][inner sep=0.75pt]   [align=left] {{\tiny dati}};
% Text Node
\draw (36,185) node [anchor=north west][inner sep=0.75pt]   [align=left] {{\scriptsize Sensori}};
\end{tikzpicture}
\caption{Stack tecnologico}
\end{figure}
\newpage

\section{Casi d'uso} \label{sec:use-case}
\subsection{Introduzione}
Di seguito sono elencati i casi d'uso individuati in seguito all'analisi del capitolato e al
confronto con la Proponente. Ciascuno di essi è corredato di un codice identificativo, la cui
struttura è descritta alla sezione vattelappesca del documento \textit{Norme di Progetto vX.X.X}. %INSERIRE RIMANDO SEZIONE CORRETTA
\subsection{Attori}
Gli attori che interagiscono con il sistema sono i seguenti:
\begin{itemize}
    \item \textbf{Amministratore pubblico}: utente in grado di accedere al sistema ed usufruire di tutte le sue funzionalità. In particolare, può visualizzare la \glossterm{dashboard} contenente i dati provenienti dai sensori. L'applicativo non richiede autenticazione; %NOT SO SURE ABOUT DIS
    \item \textbf{Sensore}: dispositivo in grado di rilevare dati dall'ambiente esterno e inviare le misurazioni effettuate al sistema, in modo da consentirne l'archiviazione e la successiva visualizzazione.
\end{itemize}
\clearpage
\subsection{Codice dei casi d'uso}
Ogni caso d'uso è associato ad un codice univoco definito nel seguente formato:
\begin{center}
    \textbf{UC-[Numero].[Specializzazione]}
\end{center}
Dove \textbf{Numero} è un identificativo e \textbf{Specializzazione} si riferisce ad un caso specifico
dello stesso caso d'uso.

\subsection{Elenco dei casi d'uso}
\subsubsection*{UC-0: Visualizzazione menu dashboard}
\addcontentsline{toc}{subsubsection}{\protect\numberline{}UC-0: Visualizzazione menu dashboard}
\begin{itemize}
    \item \textbf{Attore principale:} amministratore pubblico;
    \item \textbf{Precondizioni:} il sistema è operativo e accessibile;
    \item \textbf{Postcondizioni:} l'amministratore pubblico visualizza un menu di selezione da cui
        può scegliere che dashboard visualizzare: Sensori, Ambientale e Urbanistica.
    \item \textbf{Scenario principale:} 
        \begin{enumerate}
        \item l'amministratore pubblico accede alla piattaforma di visualizzazione.
        \end{enumerate}
        % immagine
    \item \textbf{User Story associata: }in qualità di amministratore pubblico, desidero accedere alla dashboard per monitorare in tempo reale i dati raccolti dai vari sensori dislocati nella città. Questo mi permetterà di valutare rapidamente lo stato complessivo della città e di prendere decisioni informate e tempestive riguardo alla gestione delle risorse e all’implementazione dei servizi.
\end{itemize}

\subsubsection*{UC-1: Visualizzazione dashboard sensori}
\addcontentsline{toc}{subsubsection}{\protect\numberline{}UC-1: Visualizzazione dashboard sensori}
\begin{itemize}
    \item \textbf{Attore principale:} amministratore pubblico;
    \item \textbf{Precondizioni:} nessuna;
    \item \textbf{Postcondizioni:} l'amministratore pubblico visualizza una lista di pannelli
        contenenti dati relativi allo stato dei sensori;
    \item \textbf{Scenario principale:}
    \begin{enumerate}
    \item l’amministratore pubblico accede alla piattaforma di visualizzazione;
    \item l’amministratore pubblico seleziona la visualizzazione della dashboard relativa ai sensori.
    \end{enumerate}
    \item \textbf{User Story associata:} in qualità di amministratore pubblico, desidero accedere
        alla dashboard per monitorare lo stato dei sensori dislocati nella città;
    \item \textbf{Specializzazioni:} [UC-1.1];
    \item \textbf{Estensioni:} [UC-4];
\end{itemize}

\subsubsection*{UC-1.1: Visualizzazione posizione sensori su mappa}
\addcontentsline{toc}{subsubsection}{\protect\numberline{}UC-1.1: Visualizzazione posizione sensori su mappa}
\begin{itemize}
    \item \textbf{Attore principale:} amministratore pubblico;
    \item \textbf{Precondizioni:} l’amministratore pubblico ha selezionato la visualizzazione
        relativa al dominio dei sensori;
    \item \textbf{Postcondizioni:} l'amministratore pubblico visualizza un panello contenente un
        mappa che mostra la posizione dei sensori;
    \item \textbf{Scenario principale:}
    \begin{enumerate}
      \item L’amministratore pubblico accede alla piattaforma di visualizzazione;
      \item L’amministratore pubblico seleziona la visualizzazione della dashboard relativa ai sensori.
      \item L’amministratore pubblico seleziona la visualizzazione del pannello geomap per la
          posizione dei sensori.
    \end{enumerate}
    \item \textbf{User Story associata:} in qualità di amministratore pubblico, desidero accedere
        alla dashboard per conoscere la posizione dei sensori dislocati nella città. La mappa deve indicare chiaramente la posizione di ciascun sensore e deve essere etichettata in modo da consentire un riconoscimento immediato della tipologia di ogni sensore.
\end{itemize}

\subsubsection*{UC-2: Visualizzazione dashboard ambientale}
\addcontentsline{toc}{subsubsection}{\protect\numberline{}UC-2: Visualizzazione dashboard ambientale}
\begin{itemize}
    \item \textbf{Attore principale:} amministratore pubblico;
    \item \textbf{Precondizioni: }nessuna;
    \item \textbf{Postcondizioni:} l'amministratore pubblico visualizza una lista di pannelli
        contenenti dati relativi al dominio ambientale;
    \item \textbf{Scenario principale:} 
    \begin{enumerate}
    \item L’amministratore pubblico accede alla piattaforma di visualizzazione;
    \item L’amministratore pubblico seleziona la visualizzazione della dashboard relativa al dominio
        ambientale.
    \end{enumerate}
    \item \textbf{User Story associata:} in qualità di amministratore pubblico, desidero accedere alla dashboard per monitorare il dominio ambientale;
    \item \textbf {Specializzazioni:} [UC-2.1], [UC-2.2], [UC-2.3], [UC-2.4];  
    \item \textbf{Estensioni:} [UC-4];
\end{itemize}

\subsubsection*{UC-2.1: Visualizzazione pannello time series per temperatura}
\addcontentsline{toc}{subsubsection}{\protect\numberline{}UC-2.1: Visualizzazione pannello time series per temperatura}
\begin{itemize}
    \item \textbf{Attore principale:} amministratore pubblico;
    \item \textbf{Precondizioni: }l’amministratore pubblico ha selezionato la visualizzazione
        relativa al dominio ambientale;
    \item \textbf{Postcondizioni:} L’amministratore pubblico visualizza un pannello con un grafico che mostra la temperatura, espressa in gradi Celsius (°C), in formato time series.
    \item \textbf{Scenario principale}:
    \begin{enumerate}
    \item L’amministratore pubblico accede alla piattaforma di visualizzazione;
    \item L’amministratore pubblico seleziona la visualizzazione della dashboard relativa al dominio
        ambientale;
    \item L’amministratore pubblico seleziona la visualizzazione del pannello time series per la
        temperatura.
    \end{enumerate}
\item \textbf{User Story associata:} in qualità di amministratore pubblico, desidero accedere al
        pannello per monitorare l'andamento della temperatura. Questo consente di semplificare la comprensione e la comparazione delle misurazioni.
\end{itemize}
\subsubsection*{UC-2.2: Visualizzazione pannello time series per umidità}
\addcontentsline{toc}{subsubsection}{\protect\numberline{}UC-2.2: Visualizzazione pannello time
series per umidità}
\begin{itemize}
    \item \textbf{Attore principale:} amministratore pubblico;
    \item \textbf{Precondizioni: }l’amministratore pubblico ha selezionato la visualizzazione
        relativa al dominio ambientale;
    \item \textbf{Postcondizioni:} L’amministratore pubblico visualizza un pannello con un grafico
        che mostra l'umidità relativa, espressa in percentuale, in formato time series.
    \item \textbf{Scenario principale}:
    \begin{enumerate}
    \item L’amministratore pubblico accede alla piattaforma di visualizzazione;
    \item L’amministratore pubblico seleziona la visualizzazione della dashboard relativa al dominio
        ambientale; 
    \item L’amministratore pubblico seleziona la visualizzazione del pannello time series per
        l'umidità.
    \end{enumerate}
\item \textbf{User Story associata:} in qualità di amministratore pubblico, desidero accedere al
        pannello per monitorare l'andamento dell'umidità relativa. Questo consente di semplificare la comprensione e la comparazione delle misurazioni.
\end{itemize}
\subsubsection*{UC-2.3: Visualizzazione pannello time series per temperatura percepita}
\addcontentsline{toc}{subsubsection}{\protect\numberline{}UC-2.3: Visualizzazione pannello time series per temperatura percepita}
\begin{itemize}
    \item \textbf{Attore principale:} amministratore pubblico;
    \item \textbf{Precondizioni: }l’amministratore pubblico ha selezionato la visualizzazione
        relativa al dominio ambientale;
    \item \textbf{Postcondizioni:} L’amministratore pubblico visualizza un pannello con un grafico
        che mostra la temperatura, in gradi Celsius (°C), in formato time series in relazione
        all'umidità.
    \item \textbf{Scenario principale}:
    \begin{enumerate}
    \item L’amministratore pubblico accede alla piattaforma di visualizzazione;
    \item L’amministratore pubblico seleziona la visualizzazione della dashboard relativa al dominio
        ambientale; 
    \item L’amministratore pubblico seleziona la visualizzazione del pannello time series per la temperatura percepita.
    \end{enumerate}
\item \textbf{User Story associata:} in qualità di amministratore pubblico, desidero accedere al
        pannello per monitorare l'andamento della temperatura percepita. Questo consente di semplificare la comprensione e la comparazione delle misurazioni.
\end{itemize}
\subsubsection*{UC-2.4: Visualizzazione pannello Gauge per temperatura media}
\addcontentsline{toc}{subsubsection}{\protect\numberline{}UC-2.4: Visualizzazione pannello Gauge per temperatura media}
\begin{itemize}
    \item \textbf{Attore principale:} amministratore pubblico;
    \item \textbf{Precondizioni: }l’amministratore pubblico ha selezionato la visualizzazione
        relativa al dominio ambientale;
    \item \textbf{Postcondizioni:} L’amministratore pubblico visualizza un pannello con un grafico
        che mostra la media della temperatura, in gradi Celsius (°C), in formato di diagramma di
        Gauge considerando tutti i sensori attivi nell'intervallo di tempo all'interno della dashbaord.
    \item \textbf{Scenario principale}:
    \begin{enumerate}
    \item L’amministratore pubblico accede alla piattaforma di visualizzazione;
    \item L’amministratore pubblico seleziona la visualizzazione della dashboard relativa al dominio
        ambientale; 
    \item L’amministratore pubblico seleziona la visualizzazione del pannello per la temperatura
        media.
    \end{enumerate}
\item \textbf{User Story associata:} in qualità di amministratore pubblico, desidero accedere al
        pannello per monitorare la temperatura media. Questo consente di semplificare la comprensione e la comparazione delle misurazioni.
\end{itemize}
\subsubsection*{UC-3: Visualizzazione dashboard urbanistica}
\addcontentsline{toc}{subsubsection}{\protect\numberline{}UC-3: Visualizzazione dashboard urbanistica}
\begin{itemize}
    \item \textbf{Attore principale:} amministratore pubblico;
    \item \textbf{Precondizioni: }nessuna;
    \item \textbf{Postcondizioni:} l'amministratore pubblico visualizza una lista di pannelli
        contenenti dati relativi al dominio urbanistico;
    \item \textbf{Scenario principale:} 
    \begin{enumerate}
    \item L’amministratore pubblico accede alla piattaforma di visualizzazione;
    \item L’amministratore pubblico seleziona la visualizzazione della dashboard relativa al dominio
        urbanistico.
    \end{enumerate}
    \item \textbf{User Story associata:} in qualità di amministratore pubblico, desidero accedere
        alla dashboard per monitorare il dominio urbanistico;
    \item \textbf {Specializzazioni:} [UC-3.1], [UC-3.2], [UC-3.3].
    \item \textbf{Estensioni:} [UC-4];
\end{itemize}

\subsubsection*{UC-3.1: Visualizzazione pannello alert list per notifiche di pagamento}
\addcontentsline{toc}{subsubsection}{\protect\numberline{}UC-3.1: Visualizzazione pannello alert list per notifiche di pagamento}
\begin{itemize}
    \item \textbf{Attore principale:} amministratore pubblico;
    \item \textbf{Precondizioni:} l’amministratore pubblico ha selezionato la visualizzazione
        relativa al dominio urbanistico.
    \item \textbf{Postcondizioni:} l'amministratore pubblico visualizza una lista di pannelli
        contenenti dati relativi al dominio urbanistico;
    \item \textbf{Scenario principale:} 
    \begin{enumerate}
    \item L’amministratore pubblico accede alla piattaforma di visualizzazione;
    \item L’amministratore pubblico seleziona la visualizzazione della dashboard relativa al dominio
        urbanistico;
    \item L’amministratore pubblico seleziona la visualizzazione del pannello notifiche di
        pagamento.
    \end{enumerate}
    \item \textbf{User Story associata:} in qualità di amministratore pubblico, desidero accedere
        alla dashboard per monitorare lo stato dei pagamenti. Questo consente di semplificare la
        comprensione e la comparazione delle misurazioni.
\end{itemize}
\subsubsection*{UC-3.2: Visualizzazione pannello status history per parcheggi}
\addcontentsline{toc}{subsubsection}{\protect\numberline{}UC-3.2: Visualizzazione pannello status history per parcheggi}
\begin{itemize}
    \item \textbf{Attore principale:} amministratore pubblico;
    \item \textbf{Precondizioni:} l’amministratore pubblico ha selezionato la visualizzazione
        relativa al dominio urbanistico.
    \item \textbf{Postcondizioni:} l'amministratore pubblico visualizza una lista di pannelli
        contenenti dati relativi al dominio urbanistico;
    \item \textbf{Scenario principale:} 
    \begin{enumerate}
    \item L’amministratore pubblico accede alla piattaforma di visualizzazione;
    \item L’amministratore pubblico seleziona la visualizzazione della dashboard relativa al dominio
        urbanistico;
    \item L’amministratore pubblico seleziona la visualizzazione del pannello notifiche di
        pagamento.
    \end{enumerate}
    \item \textbf{User Story associata:} in qualità di amministratore pubblico, desidero accedere
        alla dashboard per monitorare lo stato dei singoli parcheggi. Questo consente di semplificare la
        comprensione e la comparazione delle misurazioni.
\end{itemize}
\subsubsection*{UC-3.3: Visualizzazione pannello ...}
\subsubsection*{UC-4: Visualizzazione errore nessun dato}
\addcontentsline{toc}{subsubsection}{\protect\numberline{}UC-4: Visualizzazione errore nessun dato}
\begin{itemize}
    \item \textbf{Attore principale:} amministratore pubblico;
    \item \textbf{Precondizioni:} il sistema di visualizzazione non ottiene alcun dato da mostrare all’interno di un pannello;
    \item \textbf{Postcondizioni:} l’amministratore pubblico visualizza un messaggio di errore segnalante l’assenza di dati da mostrare;
    \item \textbf{Scenario principale:}
        \begin{enumerate}
        \item L’amministratore pubblico vuole visualizzare qualche pannello;
        \item Il sistema non ha dati con cui popolare tale pannello.
        \end{enumerate}
    \item \textbf{User Story associata:} in qualità di amministratore pubblico, desidero che qualora
        i dati non siano disponibili o ci sia un errore nel funzionamento del software questo mi
        venga notificato.
\end{itemize}
\subsubsection*{UC-5: Applicazione filtri}
\addcontentsline{toc}{subsubsection}{\protect\numberline{}UC-5: Applicazione filtri}
\begin{itemize}
    \item \textbf{Attore principale:} amministratore pubblico;
    \item \textbf{Precondizioni:} l’amministratore pubblico sta visualizzando uno o più pannelli con i dati;
    \item \textbf{Postcondizioni:} l’amministratore pubblico visualizza solamente i dati relativi al filtro
applicato;
    \item \textbf{Scenario principale:}
        \begin{enumerate}
        \item L’amministratore pubblico seleziona l’icona o il pulsante relativo al filtro dei dati;
        \item L’amministratore pubblico seleziona secondo quali valori filtrare il pannello.
        \end{enumerate}
    \item \textbf{User Story associata:} in qualità di amministratore pubblico, desidero poter
        visualizzare i dati secondo criteri specifici, in modo tale da semplificare la comprensione
        delle informazioni.
    \item \textbf{Specializzazioni:} [UC-5.1], [UC-5.2].
\end{itemize}
\subsubsection*{UC-5.1: Filtro sotto-insieme di sensori su grafici time series}
\addcontentsline{toc}{subsubsection}{\protect\numberline{}UC-5.1: Filtro sotto-insieme di sensori su grafici time series}
\begin{itemize}
    \item \textbf{Attore principale:} amministratore pubblico;
    \item \textbf{Precondizioni:}
    \begin{itemize}
        \item L’amministratore pubblico sta visualizzando uno o più pannelli time series con i dati;
        \item Il pannello offre la funzionalità di filtro dei dati tramite selezione di uno o più sensori.
    \end{itemize}
    \item \textbf{Postcondizioni:} l’amministratore pubblico visualizza solamente i dati relativi ai sensori
selezionati, all’interno di tale.
    \item \textbf{Scenario principale: }l’amministratore pubblico seleziona il sensore da
        visualizzare tramite la legenda.
    \item \textbf{User Story associata:} in qualità di amministratore pubblico, desidero poter
        visualizzare i dati secondo criteri specifici, in modo tale da semplificare la comprensione
        delle informazioni. In particolare la possibilità di poter evidenziare/escludere i dati
        provenienti da determinati sensori.
\end{itemize}
\subsubsection*{UC-5.2: Filtro per intervallo temporale}
\addcontentsline{toc}{subsubsection}{\protect\numberline{}UC-5.2: Filtro per intervallo temporale}
\begin{itemize}
    \item \textbf{Attore principale:} amministratore pubblico;
    \item \textbf{Precondizioni:}
    \begin{itemize}
        \item L’amministratore pubblico sta visualizzando uno o più pannelli time series con i dati;
        \item Il pannello offre la funzionalità di filtro dei dati tramite selezione di uno o più sensori.
    \end{itemize}
    \item \textbf{Postcondizioni:} l’amministratore pubblico visualizza solamente i dati relativi all’intervallo
temporale selezionato, in tutti i pannelli della dashboard dove è stato applicato il filtro.
    \item \textbf{Scenario principale: }l’amministratore pubblico seleziona l’intervallo temporale desiderato.
    \item \textbf{User Story associata:} in qualità di amministratore pubblico, desidero poter
        visualizzare i dati secondo criteri specifici, in modo tale da semplificare la comprensione
        delle informazioni. In particolare la possibilità di poter evidenziare/escludere determinate
        scaglie temporali.
\end{itemize}
\subsubsection*{UC-6: Modifica layout pannelli}
\addcontentsline{toc}{subsubsection}{\protect\numberline{}UC-6: Modifica layout pannelli}
\begin{itemize}
    \item \textbf{Attore principale:} amministratore pubblico;
    \item \textbf{Precondizioni:} l'amministratore pubblico sta visualizzando almeno un pannello;
    \item \textbf{Postcondizioni:} l’amministratore pubblico visualizza il nuovo layout;
    \item \textbf{Scenario principale: }l’amministratore pubblico sposta o ridimensiona i pannelli a
        suo piacimento;
    \item \textbf{User Story associata:} in qualità di amministratore pubblico, desidero poter
        visualizzare i pannelli nella forma e dimensione a me più congeniali.
\end{itemize}
\subsubsection*{UC-7: Inserimento dati sensore}
\addcontentsline{toc}{subsubsection}{\protect\numberline{}UC-7: Inserimento dati sensore}
\begin{itemize}
    \item \textbf{Attore principale:} sensore;
    \item \textbf{Precondizioni:} il sensore è colleagato al sistema;
    \item \textbf{Postcondizioni:} il sistema ha archiviato correttamente i dati inviati dal sensore;
    \item \textbf{Scenario principale:}
        \begin{enumerate}
        \item Il sensore effettua una rilevazione;
        \item Il sensore formatta il messaggio da inviare al sistema;
        \item Il sensore invia il messaggio al sistema.
        \end{enumerate}
    \item \textbf{User Story associata:} in qualità di sensore desidero poter inviare al sistema i
        dati rilevati.
    \item \textbf{Specializzazioni:} [UC-7.1], [UC-7.2], [UC-7.3], [UC-7.4]
\end{itemize}
\subsubsection*{UC-7.1: Inserimento dati temperatura}
\addcontentsline{toc}{subsubsection}{\protect\numberline{}UC-7.1: Inserimento dati temperatura}
\begin{itemize}
    \item \textbf{Attore principale:} sensore;
    \item \textbf{Precondizioni:} il sensore è collegato al sistema;
    \item \textbf{Postcondizioni:} il sistema ha archiviato correttamente i dati inviati dal sensore;
    \item \textbf{Scenario principale:}
        \begin{enumerate}
        \item Il sensore di temperatura effettua una rilevazione;
        \item Il sensore formatta il messaggio da inviare al sistema, in modo da mandare la
temperatura, espressa in gradi Celsius (°C), il timestamp di rilevazione e le proprie
coordinate geografiche;
        \item Il sensore invia il messaggio al sistema.
        \end{enumerate}
    \item \textbf{User Story associata:} in qualità di sensore desidero poter inviare al sistema i
        dati rilevati.
\end{itemize}
\subsubsection*{UC-7.2: Inserimento dati umidità}
\addcontentsline{toc}{subsubsection}{\protect\numberline{}UC-7.2: Inserimento dati umidità}
\begin{itemize}
    \item \textbf{Attore principale:} sensore;
    \item \textbf{Precondizioni:} il sensore è collegato al sistema;
    \item \textbf{Postcondizioni:} il sistema ha archiviato correttamente i dati inviati dal sensore;
    \item \textbf{Scenario principale:}
        \begin{enumerate}
        \item Il sensore di umidità effettua una rilevazione;
        \item Il sensore formatta il messaggio da inviare al sistema, in modo da mandare l'umidità
            relativa, espressa in percentuale, il timestamp di rilevazione e le proprie
coordinate geografiche;
        \item Il sensore invia il messaggio al sistema.
        \end{enumerate}
    \item \textbf{User Story associata:} in qualità di sensore desidero poter inviare al sistema i
        dati rilevati.
\end{itemize}
\subsubsection*{UC-7.3: Inserimento dati parcheggio}
\addcontentsline{toc}{subsubsection}{\protect\numberline{}UC-7.3: Inserimento dati parcheggio}
\begin{itemize}
    \item \textbf{Attore principale:} sensore;
    \item \textbf{Precondizioni:} il sensore è collegato al sistema;
    \item \textbf{Postcondizioni:} il sistema ha archiviato correttamente i dati inviati dal sensore;
    \item \textbf{Scenario principale:}
        \begin{enumerate}
        \item Il sensore di parcheggio effettua una rilevazione;
        \item Il sensore formatta il messaggio da inviare al sistema, in modo da mandare lo stato
            di occupazione del parcheggio, il timestamp di rilevazione e le proprie
coordinate geografiche;
        \item Il sensore invia il messaggio al sistema.
        \end{enumerate}
    \item \textbf{User Story associata:} in qualità di sensore desidero poter inviare al sistema i
        dati rilevati.
\end{itemize}
\subsubsection*{UC-7.4: Inserimento dati pagamenti di parcheggio}
\addcontentsline{toc}{subsubsection}{\protect\numberline{}UC-7.4: Inserimento dati pagamenti di parcheggio}
\begin{itemize}
    \item \textbf{Attore principale:} sensore;
    \item \textbf{Precondizioni:} il sensore è collegato al sistema;
    \item \textbf{Postcondizioni:} il sistema ha archiviato correttamente i dati inviati dal sensore;
    \item \textbf{Scenario principale:}
        \begin{enumerate}
        \item Il sensore di pagamenti effettua una rilevazione;
        \item Il sensore formatta il messaggio da inviare al sistema, in modo da mandare lo stato
            del pagamento relativo a un parcheggio, il timestamp di rilevazione e le proprie
coordinate geografiche;
        \item Il sensore invia il messaggio al sistema.
        \end{enumerate}
    \item \textbf{User Story associata:} in qualità di sensore desidero poter inviare al sistema i
        dati rilevati.
\end{itemize}
\end{document}
