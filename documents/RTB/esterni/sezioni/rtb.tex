\textbf{Inizio:} 2024-04-01\\
\textbf{Fine prevista:} 2024-04-07
\subsection{Primo periodo} \label{sec:1_rtb}
\textbf{Inizio:} 2024-04-01\\
\textbf{Fine prevista:} 2024-04-19\\
\textbf{Giorni di ritardo:} (0)
\subsubsection{Pianificazione}
Nella fase iniziale, assume prioritaria importanza la discussione di tutte le regole precedentemente adottate e implementate nel corso del progetto, ma che non sono ancora state formalizzate mediante documentazione. Questo processo mira a produrre un documento scritto accessibile a tutti i membri del team, al fine di dissipare eventuali ambiguità riguardo l'esecuzione delle attività e l'utilizzo delle risorse.
\\
Inoltre, durante questa fase, è cruciale identificare tutti i potenziali rischi che potrebbero ostacolare il progresso del progetto, al fine di prevenirli e non essere colti impreparati. Un'altra priorità consiste nell'avviare la pianificazione delle prime attività e delle relative milestone, al fine di organizzare le risorse disponibili e fornire una stima preventiva dei tempi e dei costi.
\subsubsection{Rischi attesi}
I rischi che ci aspettiamo di correre in questo periodo sono i seguenti: 
\begin{itemize}
\item \textbf{RT-1:} Inesperienza;
\item \textbf{RP-1:} Stima errata delle tempistiche di progetto;
\item \textbf{RP-2:} Stima errata dei costi di progetto;
\item \textbf{RC-1:} Scarsa comunicazione interna;
\item \textbf{RC-3:} Conflitti interni.
\end{itemize}
Questo perchè, essendo all’inizio del progetto, siamo ancora incerti su molti aspetti di
quest’ultimo, ci stiamo attualmente organizzando e dobbiamo apprendere ancora molto, dunque la
probabilità di incorrere in qualche problema tra quelli riportati è abbastanza elevata.
\newpage
\subsubsection{Preventivo}
\subsubsubsection{Preventivo orario}
\begin{table}[ht!]
	\centering
	\begin{tabular}{p{4cm} p{1cm} p{1cm} p{1cm} p{1cm} p{1cm} p{1cm} p{3cm}}
		\toprule
        \textbf{Membro} & \multicolumn{6}{c}{\textbf{Ruoli}} & \textbf{Totale (persona)}\\
		& \textbf{AM} & \textbf{RE} & \textbf{PT} & \textbf{AN} & \textbf{PR} & \textbf{VE}\\
		\midrule
        Linda Barbiero          & -     & 3     & -     & 2     & -     & 1     & 6 \\
        Guglielmo Barison       & 3     & -     & 1     & 1     & 2,5   & 1     & 8,5\\
        Pietro Busato           & 3     & -     & 1     & 1     & 2,5   & 1     & 8,5 \\
        Davide Donanzan         & 3     & -     & -     & 3     & -     & 1,5   & 7,5 \\
        Oscar Konieczny         & 3     & -     & -     & 1     & -     & 1     & 5 \\
        Veronica Tecchiati      & 3     & -     & -     & 4     & -     & 1     & 8 \\
        \bottomrule
        \textbf{Totale (ruolo)} & 15     & 3     & 2     & 12   & 5     & 6,5   & 43,5 \\
	\end{tabular}
	\caption{Distribuzione delle ore del primo Sprint secondo ruolo e membro.}
	\label{table:Distribuzione delle ore del primo Sprint secondo ruolo e membro}
\end{table}
\begin{figure}[ht!]
    \centering
    \includegraphics[width=15cm]{./asset/istogramma_periodo_1.png}
    \caption{Visualizzazione dell’impegno temporale di ciascun membro nei rispettivi ruoli assegnati
    nel primo Sprint.}
    \label{figure:Visualizzazione dell’impegno temporale di ciascun membro nei rispettivi ruoli
    assegnati nel primo Sprint}
\end{figure}
\subsubsubsection{Preventivo economico}
\begin{table}[ht!]
	\centering
	\begin{tabular}{p{4cm} p{1cm} p{2cm}}
        \toprule
        \textbf{Ruolo} & \textbf{Ore} & \textbf{Costo (€)} \\
        \midrule
        Amministratore & 15 & 450 \\
        Responsabile & 3 & 60 \\
        Progettista & 2 & 50 \\
        Analista & 12 & 300 \\
        Programmatore & 5 & 75 \\
        Verificatore & 6,5 & 97,5 \\
        \bottomrule
        \textbf{Totale} & 43,5 & 1032,5
    \end{tabular}
    \caption{Preventivo dei costi del primo Sprint secondo ruolo.}
	\label{table:Preventivo dei costi del primo Sprint secondo ruolo}
\end{table}
\subsubsection{Consuntivo}
\subsubsubsection{Consuntivo orario}
\begin{table}[ht!]
	\centering
	\begin{tabular}{p{3cm} p{1.5cm} p{1.5cm} p{1.5cm} p{1.5cm} p{1.5cm} p{1.5cm} p{1cm}}
		\toprule
        \textbf{Membro} & \multicolumn{5}{c}{\textbf{Ruoli}} & \multicolumn{2}{r}{\textbf{Totale
        (pers.)}}\\
		& \textbf{AM} & \textbf{RE} & \textbf{PT} & \textbf{AN} & \textbf{PR} & \textbf{VE}\\
		\midrule
        Linda Barbiero          & -     & 3     & -     & 2     & 0     & 0.5 \textcolor{teal}{(-0,5)}   & 5,5 \\
        Guglielmo Barison       & 3     & -     & 1     & 1     & 2,5   & 1     & 8,5\\
        Pietro Busato           & 3     & -     & 1     & 1     & 2,5   & 1     & 8,5 \\
        Davide Donanzan         & 3     & -     & -     & 3     & -     & 2 \textcolor{teal}{(-0,5)}     & 8 \\
        Oscar Konieczny         & 3,5 \textcolor{teal}{(-0,5)}  & -     & -     & 1     & -     & 1     & 5,5 \\
        Veronica Tecchiati      & 3,5 \textcolor{teal}{(-0,5)}  & -     & -     & 5 \textcolor{teal}{(-1)}    & -     & 1     & 9,5 \\
        \bottomrule
        \textbf{Totale (ruolo)} & 16    & 3     & 2     & 13   & 5     & 6,5   & 45,5 \\
	\end{tabular}
	\caption{Distribuzione delle ore del primo Sprint secondo ruolo e membro.}
	\label{table:Distribuzione delle ore consuntive del primo Sprint secondo ruolo e membro}
\end{table}
\begin{table}[ht!]
	\centering
	\begin{tabular}{p{4cm} p{1cm} p{1cm} p{1cm} p{1cm} p{1cm} p{1cm} p{3cm}}
		\toprule
        \textbf{Membro} & \multicolumn{6}{c}{\textbf{Ruoli}} & \textbf{Totale (persona)}\\
		& \textbf{AM} & \textbf{RE} & \textbf{PT} & \textbf{AN} & \textbf{PR} & \textbf{VE}\\
		\midrule
        Linda Barbiero          & 7     & 5     & 20     & 11   & 24     & 22,5   & 89,5 \\
        Guglielmo Barison       & 4     & 9     & 19     & 11   & 21,5   & 22     & 86,5\\
        Pietro Busato           & 6     & 8     & 19     & 11   & 20,5   & 22     & 86,5 \\
        Davide Donanzan         & 6     & 8     & 20     & 9    & 23     & 21     & 87 \\
        Oscar Konieczny         & 4,5   & 7     & 20     & 11   & 24     & 23     & 89,5 \\
        Veronica Tecchiati      & 4,5   & 8     & 20     & 6    & 25     & 22     & 85,5 \\
        \bottomrule
        \textbf{Totale (ruolo)} & 32    & 45    & 118    & 59   & 138     & 132,5 & 524,5 \\
	\end{tabular}
	\caption{Ore rimaste dopo il primo Sprint secondo ruolo e membro.}
	\label{table:Ore rimaste dopo il primo Sprint secondo ruolo e membro}
\end{table}
\subsubsubsection{Consuntivo economico}
\begin{table}[ht!]
	\centering
	\begin{tabular}{p{4cm} p{2cm} p{2cm}}
        \toprule
        \textbf{Ruolo} & \textbf{Ore} & \textbf{Costo (€)} \\
        \midrule
        Amministratore & 16 \textcolor{red}{(+1)} & 480 \textcolor{red}{(+30)} \\
        Responsabile & 3 & 60 \\
        Progettista & 2 & 50 \\
        Analista & 13 \textcolor{teal}{(-1)} & 325 \textcolor{teal}{(-25)} \\
        Programmatore & 5 & 75 \\
        Verificatore & 6,5 & 97,5 \\
        \bottomrule
        \textbf{Totale} & 45,5 & 1087,5
    \end{tabular}
    \caption{Preventivo a finire dei costi del primo Sprint secondo ruolo.}
	\label{table:Preventivo a finire dei costi del primo Sprint secondo ruolo}
\end{table}
\subsubsubsection{Rischi occorsi, impatto e loro mitigazione}
\subsubsection{Retrospettiva}
Dal consuntivo è risultata una leggera discrepanza con il preventivo del periodo. Sono servite più ore da Amministratore e da Analista del preventivato.
Ciò ha comportato una spesa maggiore di 55\;\texteuro, per un complessivo di 1.\,087,5\;\texteuro\
rispetto ai 1\,.032,5\;\texteuro\ previsti.
In conclusione, il budget rimanente è di 10\,.342,5\;\texteuro. 
\begin{figure}[h!]
    \centering
    \includegraphics[width=15cm]{./asset/gantt1.jpeg}
    \caption{Diagramma di Gantt rappresentativo del primo periodo.}
    \label{figure:Diagramma di Gantt rappresentativo del primo periodo}
\end{figure}
\clearpage
\newpage
\subsection{Secondo periodo} \label{sec:2_rtb}
\textbf{Inizio:} 2024-04-22\\
\textbf{Fine prevista:} 2024-04-03\\
\textbf{Giorni di ritardo:} (0)
\subsubsection{Pianificazione}
In questa fase del processo, assume un'importanza fondamentale condurre un'analisi dettagliata del capitolato al fine di identificare accuratamente i casi d'uso necessari. Inoltre, al fine di evitare ambiguità e decisioni errate, si raccomanda di organizzare uno o più incontri con il proponente per condividere le idee e risolvere i dubbi emersi durante l'analisi, che sarà notevolmente più approfondita rispetto a quella svolta durante la selezione del capitolato. Da questo processo di analisi, si darà inizio alla stesura dell'\textit{Analisi dei Requisiti}, un documento di vitale importanza per il progetto poiché conterrà tutti i casi d'uso individuati, nonché i requisiti obbligatori, desiderabili e opzionali.
È altresì consigliabile redigere il \textit{Piano di Qualifica} in questa fase, il quale sarà fondamentale per definire i metodi volti a garantire la qualità dei processi e dei prodotti nel corso dello sviluppo.
\subsubsection{Rischi attesi}
I rischi che ci aspettiamo di correre in questo periodo sono i seguenti: 
\begin{itemize}
\item \textbf{RT-1:} Inesperienza;
\item \textbf{RT-2:} Produzione di codice incomprensibile;
\item \textbf{RP-1:} Stima errata delle tempistiche di progetto;
\item \textbf{RP-2:} Stima errata dei costi di progetto;
\item \textbf{RC-1:} Scarsa comunicazione interna;
\item \textbf{RC-3:} Conflitti interni.
\end{itemize}
I rischi identificati per il periodo attuale non differiscono significativamente da quelli presentati nella pianificazione del periodo precedente. Ciò è attribuibile al fatto che l’esperienza del gruppo è ancora limitata, come evidenziato nei verbali, e persistono lievi difficoltà nelle attività, come la suddivisione dei compiti. La novità consiste nell’aggiunta di \textbf{RT-2: Produzione di codice incomprensibile}, ritenuto necessario poichè durante questo periodo inizieremo con lo sviluppo del codice per il Proof of Concept (PoC).
\clearpage
\subsubsection{Preventivo}
\subsubsubsection{Preventivo orario}
\begin{table}[ht!]
	\centering
	\begin{tabular}{p{4cm} p{1cm} p{1cm} p{1cm} p{1cm} p{1cm} p{1cm} p{3cm}}
        \toprule
        \textbf{Membro} & \multicolumn{6}{c}{\textbf{Ruoli}} & \textbf{Totale (persona)}\\
		& \textbf{AM} & \textbf{RE} & \textbf{PT} & \textbf{AN} & \textbf{PR} & \textbf{VE}\\
		\midrule
        Linda Barbiero          & 3     & -     & 2     & -     & 4     & 3     & 12 \\
        Guglielmo Barison       & 1     & -     & 2,5     & -     & 2   & 1     & 6,5\\
        Pietro Busato           & 1     & -     & 2,5     & -     & 2   & 1     & 6,5 \\
        Davide Donanzan         & 1     & -     & 2     & 2     & 3     & 1   & 9 \\
        Oscar Konieczny         & -     & -     & 2     & -     & 6     & 1     & 9 \\
        Veronica Tecchiati      & 2     & 3     & -     & 2     & 0     & 1     & 8 \\
        \bottomrule
        \textbf{Totale (ruolo)} & 8     & 3     & 11     & 4   & 17     & 8   & 51 \\
	\end{tabular}
	\caption{Distribuzione delle ore del secondo Sprint secondo ruolo e membro.}
	\label{table:Distribuzione delle ore del secondo Sprint secondo ruolo e membro}
\end{table}
\begin{figure}[ht!]
    \centering
    \includegraphics[width=15cm]{./asset/istogramma_periodo_2.png}
    \caption{Visualizzazione dell’impegno temporale di ciascun membro nei rispettivi ruoli assegnati
    nel secondo Sprint.}
    \label{figure:Visualizzazione dell’impegno temporale di ciascun membro nei rispettivi ruoli
    assegnati nel secondo Sprint}
\end{figure}
\subsubsubsection{Preventivo economico}
\begin{table}[ht!]
	\centering
	\begin{tabular}{p{4cm} p{1cm} p{2cm}}
        \toprule
        \textbf{Ruolo} & \textbf{Ore} & \textbf{Costo (€)} \\
        \midrule
        Amministratore & 8 & 240 \\
        Responsabile & 3 & 60 \\
        Progettista & 11 & 275 \\
        Analista & 4 & 100 \\
        Programmatore & 17 & 255 \\
        Verificatore & 8 & 120 \\
        \bottomrule
        \textbf{Totale} & 51 & 1050
    \end{tabular}
    \caption{Preventivo dei costi del secondo Sprint secondo ruolo.}
	\label{table:Preventivo dei costi del secondo Sprint secondo ruolo}
\end{table}
\subsubsection{Consuntivo}
\subsubsubsection{Consuntivo orario}
\begin{table}[ht!]
	\centering
	\begin{tabular}{p{3cm} p{1.4cm} p{1.6cm} p{1.5cm} p{1.5cm} p{1.5cm} p{1.5cm} p{1cm}}
		\toprule
        \textbf{Membro} & \multicolumn{5}{c}{\textbf{Ruoli}} & \multicolumn{2}{r}{\textbf{Totale (pers.)}}\\
		& \textbf{AM} & \textbf{RE} & \textbf{PT} & \textbf{AN} & \textbf{PR} & \textbf{VE}\\
		\midrule
        Linda Barbiero          & 4 \textcolor{red}{(+1)} & - & 1 \textcolor{teal}{(-1)}  & - & 2
        \textcolor{teal}{(-2)} & 3 & 10 \\
        Guglielmo Barison       & 1 & - & 2,5 & -     & 3 \textcolor{red}{(+1)}  & 1     & 7,5\\
        Pietro Busato           & 1     & -     & 2,5     & -     & 3 \textcolor{red}{(+1)}  & 1     & 7,5 \\
        Davide Donanzan         & - \textcolor{teal}{(-1)}    & -     & 2     & 2     & 2
        \textcolor{teal}{(-1)}     & 1     & 7 \\
        Oscar Konieczny         & -   & -     & 2     & -     & 6     & 1     & 9 \\
        Veronica Tecchiati      & 2   & 3,5 \textcolor{red}{(+0,5)}     & -     & 2     & -     & 1     & 8,5 \\
        \bottomrule
        \textbf{Totale (ruolo)} & 7    & 3,5     & 10     & 4   & 16     & 8   & 48,5 \\
	\end{tabular}
	\caption{Distribuzione delle ore del secondo Sprint secondo ruolo e membro.}
	\label{table:Distribuzione delle ore consuntive del secondo Sprint secondo ruolo e membro}
\end{table}
\begin{table}[ht!]
	\centering
	\begin{tabular}{p{4cm} p{1cm} p{1cm} p{1cm} p{1cm} p{1cm} p{1cm} p{3cm}}
		\toprule
        \textbf{Membro} & \multicolumn{6}{c}{\textbf{Ruoli}} & \textbf{Totale (persona)}\\
		& \textbf{AM} & \textbf{RE} & \textbf{PT} & \textbf{AN} & \textbf{PR} & \textbf{VE}\\
		\midrule
        Linda Barbiero          & 3     & 5     & 19     & 11   & 22     & 19,5   & 79,5 \\
        Guglielmo Barison       & 4     & 9     & 16,5     & 11   & 18,5   & 21     & 80\\
        Pietro Busato           & 5     & 8     & 16,5     & 11   & 17,5   & 21     & 79\\
        Davide Donanzan         & 6     & 8     & 18     & 7    & 21     & 20     & 80\\
        Oscar Konieczny         & 4,5   & 7     & 18     & 11   & 18     & 22     & 80,5\\
        Veronica Tecchiati      & 2,5   & 4,5     & 20     & 4    & 25     & 21     & 77\\
        \bottomrule
        \textbf{Totale (ruolo)} & 25    & 41,5    & 108    & 55   & 122     & 124,5 & 476 \\
	\end{tabular}
	\caption{Ore rimaste dopo il secondo Sprint secondo ruolo e membro.}
	\label{table:Ore rimaste dopo il secondo Sprint secondo ruolo e membro}
\end{table}
\subsubsubsection{Consuntivo economico}
\begin{table}[ht!]
	\centering
	\begin{tabular}{p{4cm} p{2cm} p{2cm}}
        \toprule
        \textbf{Ruolo} & \textbf{Ore} & \textbf{Costo (€)} \\
        \midrule
        Amministratore & 7 \textcolor{red}{(+1)} & 210 \textcolor{red}{(+30)}\\
        Responsabile & 3,5 \textcolor{teal}{(-0,5)} & 70 \textcolor{teal}{(-10)} \\
        Progettista & 10 \textcolor{red}{(+1)} & 250 \textcolor{red}{(+25)} \\
        Analista & 4 & 100 \\
        Programmatore & 16 \textcolor{red}{(+1)} & 240 \textcolor{red}{(+15)} \\
        Verificatore & 8 & 120 \\
        \bottomrule
        \textbf{Totale} & 48,5 & 990
    \end{tabular}
    \caption{Preventivo a finire dei costi del secondo Sprint secondo ruolo.}
	\label{table:Preventivo a finire dei costi del secondo Sprint secondo ruolo}
\end{table}
\subsubsubsection{Rischi occorsi, impatto e loro mitigazione}
\subsubsection{Retrospettiva}
Al termine del secondo periodo è emerso un lieve divario tra i costi previsti e quelli effettivi.
Infatti, il consuntivo riporta una spesa totale di 990\;\texteuro\ anzichè i 1\,.050\;\texteuro\ calcolati
nel preventivo. La differenza di 60\;\texteuro\ è dovuta allo svolgimento di un minor quantitativo di
ore da Amministratore, Progettista e Programmatore rispetto alle stime, nonostante il leggero
surplus per il ruolo da Responsabile. Il budget rimanente ammonta dunque a 9\,.352,5\;\texteuro. 
\begin{figure}[h]
    \centering
    \includegraphics[width=13cm]{./asset/gantt2.png}
    \caption{Diagramma di Gantt rappresentativo del secondo periodo.}
    \label{figure:Diagramma di Gantt rappresentativo del secondo periodo}
\end{figure}
\newpage
\clearpage
\subsection{Terzo periodo} \label{sec:3_rtb}
\textbf{Inizio:} 2024-05-29\\
\textbf{Fine prevista:} 2024-05-17\\
\textbf{Giorni di ritardo:} (0)
\subsubsection{Pianificazione}
Con il consolidamento dell'\textit{Analisi dei Requisiti}, diventa fondamentale approfondire gli
strumenti necessari per l'implementazione finale del prodotto. Il \textit{Proof of Concept} (PoC) ha
raggiunto un buon grado di avanzamento, anche grazie all'inclusione della componente di
visualizzazione nello \textit{Sprint Backlog}, originariamente pianificata per questo periodo.
Prevediamo di completarlo entro il periodo attuale, consentendo così ai programmatori di terminare
il lavoro e successivamente a tutti i membri di dedicarsi al completamento della documentazione, in
vista della prima revisione del progetto. Questo permetterà di realizzare una versione semplificata
del prodotto finale, volta a fornire indicazioni sulla validità della direzione intrapresa e a
dimostrare al committente la correttezza dell'approccio di sviluppo.\\
Considerando il progresso fatto nelle \textit{Norme di Progetto} durante il primo e il secondo
periodo, ora l'attenzione sarà principalmente rivolta a quest'ultime, la cui struttura di base è
stata precedentemente definita. L'obiettivo sarà quello di consolidare ulteriormente il documento,
riflettendo sui dettagli emersi e aggiungendo nuove informazioni.\\
Le attivit\`{a} previste durante questo periodo sono quindi le seguenti:
\begin{itemize}
    \item Arricchimento del documento \textit{Norme di progetto} precedentemente avviato;
    \item Integrazione della terminologia mancante nel documento \textit{Glossario};
    \item Consolidamento del documento \textit{Analisi dei Requisiti};
    \item Ampliamento del \textit{Piano di Qualifica};
    \item Consolidamento del PoC (Proof of Concept).
\end{itemize}
\subsubsection{Rischi attesi}
I rischi che ci aspettiamo di correre in questo periodo sono i seguenti: 
\begin{itemize}
\item \textbf{RT-1:} Inesperienza;
\item \textbf{RP-1:} Stima errata delle tempistiche di progetto;
\item \textbf{RP-2:} Stima errata dei costi di progetto;
\item \textbf{RP-5: }Impegni personali ed accademici;
\item \textbf{RC-1:} Scarsa comunicazione interna;
\item \textbf{RC-3:} Conflitti interni.
\end{itemize}
Il consistente carico di lavoro di questo periodo richiederà inevitabilmente più tempo ad ogni membro del gruppo. I rischi identificati sono organizzativi e comuncativi: questi potrebbero causare rallentamenti generali e ritardi.
\clearpage
\subsubsection{Preventivo}
\subsubsubsection{Preventivo orario}
\begin{table}[ht!]
    \centering
    \begin{tabular}{p{4cm} p{1cm} p{1cm} p{1cm} p{1cm} p{1cm} p{1cm} p{3cm}}
        \toprule
        \textbf{Membro} & \multicolumn{6}{c}{\textbf{Ruoli}} & \textbf{Totale (persona)}\\
		& \textbf{AM} & \textbf{RE} & \textbf{PT} & \textbf{AN} & \textbf{PR} & \textbf{VE}\\
		\midrule
        Linda Barbiero       & -   & -   & 2   & -   & 5   & 4   & 11 \\
        Guglielmo Barison    & 2   & 4   & -   & -   & -   & 2,5 & 8,5 \\
        Pietro Busato        & 4   & -   & -   & 4   & -   & 2   & 10 \\
        Davide Donanzan      & -   & -   & 2   & -   & 5   & 4   & 11 \\
        Oscar Konieczny      & 4   & -   & -   & 4   & -   & 3   & 11 \\
        Veronica Tecchiati   & -   & -   & 2   & -   & 5   & 5   & 12 \\
        \bottomrule
        \textbf{Totali (ruolo)} & 10 & 4 & 6 & 8 & 15 & 20,5 & 63,5 \\
    \end{tabular}
    \caption{Distribuzione delle ore del terzo Sprint secondo ruolo e membro.}
    \label{table:Distribuzione delle ore del terzo Sprint secondo ruolo e membro}
\end{table}
\begin{figure}[ht!]
    \centering
    \includegraphics[width=15cm]{./asset/istogramma_periodo_3.png}
    \caption{Visualizzazione dell’impegno temporale di ciascun membro nei rispettivi ruoli assegnati
    nel terzo Sprint.}
    \label{figure:Visualizzazione dell’impegno temporale di ciascun membro nei rispettivi ruoli
    assegnati nel terzo Sprint}
\end{figure}
\subsubsubsection{Preventivo economico}
\begin{table}[ht!]
	\centering
	\begin{tabular}{p{4cm} p{1cm} p{2cm}}
        \toprule
        \textbf{Ruolo} & \textbf{Ore} & \textbf{Costo (€)} \\
        \midrule
        Amministratore & 10 & 300 \\
        Responsabile & 4 & 80 \\
        Progettista & 6 & 150 \\
        Analista & 8 & 200 \\
        Programmatore & 15 & 225 \\
        Verificatore & 20 & 307,5 \\
        \bottomrule
        \textbf{Totale} & 63,5 & 1262,5
    \end{tabular}
    \caption{Preventivo dei costi del terzo Sprint secondo ruolo.}
	\label{table:Preventivo dei costi del terzo Sprint secondo ruolo}
\end{table}
\subsubsection{Consuntivo}
\subsubsubsection{Consuntivo orario}
\subsubsubsection{Consuntivo economico}
\subsubsubsection{Rischi occorsi, impatto e loro mitigazione}
\\Nel corso di questo periodo il gruppo ha scelto di svolgere numerose attivit\`{a} in un tempo limitato.
In seguito ad un controllo generale della documentazione, sono emerse diverse imprecisioni in
particolari sezioni, sottolineando come alcune attivit\`{a} non fossero state svolte in maniera
ottimale. Gli analisti si sono occupati di sistemare queste parti. I rischi preventivati hanno avuto un impatto medio sul gruppo, causando principalmente stress, tuttavia, alla fine, sono stati mitigati nel miglior modo possibile.
\subsubsection{Retrospettiva}
Il terzo periodo si è rivelato, come da aspettative, il pi\`{u} intenso dei tre. Tutto sommato, il
carico di lavoro poteva essere distribuito diversamente, magari su pi\`{u} periodi di lavoro.
All'interno del team, è stata notevolmente valutata l'impegno di alcuni membri nel dedicare
ulteriori ore per prevenire un rallentamento eccessivo nell'avanzamento globale del progetto.
Tuttavia, sarebbe auspicabile evitare tali circostanze in futuro, sebbene abbiano messo in luce la
determinazione della maggioranza del gruppo a procedere con decisione.
\newpage
\clearpage
\subsection{Quarto periodo} \label{sec:4_rtb}
\textbf{Inizio:} 2024-05-20\\
\textbf{Fine prevista:} 2024-06-07\\
\textbf{Giorni di ritardo:} (0)
\subsubsection{Pianificazione}
%In questa fase conclusiva del processo, assume primaria importanza sia la progettazione che l'effettiva implementazione del \textit{PoC}. Parallelamente, diventa imprescindibile il perfezionamento e la verifica finale dei documenti in vista della revisione pianificata per l'inizio del mese di giugno.
\subsubsection{Rischi attesi}
\subsubsection{Preventivo}
\subsubsubsection{Preventivo orario}
\subsubsubsection{Preventivo economico}
\subsubsection{Consuntivo}
\subsubsubsection{Consuntivo orario}
\subsubsubsection{Consuntivo economico}
\subsubsubsection{Rischi occorsi, impatto e loro mitigazione}
\subsubsection{Retrospettiva}
