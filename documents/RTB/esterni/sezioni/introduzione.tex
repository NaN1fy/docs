\subsection{Scopo del documento}
Questo documento ha lo scopo di delineare la pianificazione e la gestione delle attività volte alla realizzazione del progetto. In particolare, saranno approfonditi temi fondamentali quali l'analisi dei rischi e la loro mitigazione, il modello di sviluppo adottato, il preventivo dei costi ed il consuntivo. Questo documento ha lo scopo di stabilire con chiarezza le modalità con la quale verranno eseguite
le attività dai membri del gruppo NaN1fy. In particolare verranno trattati i seguenti temi:
\begin{itemize}
\setlength\itemsep{0em}
    \item Analisi dei rischi;
    \item Organizzazione temporale delle attivit\`{a};
    \item Ripartizione dei compiti tra i componenti del gruppo;
    \item Stima dei costi e delle risorse delle varie iterazioni.
\end{itemize}
\subsection{Scopo del prodotto}
Il fine è creare una data pipeline che, partendo dalla generazione di dati da alcuni sensori simulati, possa gestire, archiviare, analizzare e visualizzare il flusso dati in tempo reale. I sensori simuleranno il campionamento di diverse caratteristiche della città in esame, come temperatura, vento, traffico e altri elementi analoghi.
\subsection{Glossario}
Per garantire chiarezza nel linguaggio utilizzato nei documenti, è stato redatto un Glossario contenente le definizioni dei termini con significato specifico da disambiguare. Tali termini sono contrassegnati con una G ad apice. L'inserimento di un termine nel Glossario è considerato completo solo dopo averne fornito la definizione.


\subsection{Preventivo iniziale}
Come riportato nel preventivo presentato in fase di candidatura, viene stimato il costo totale del
progetto per un ammontare di 11 .430,00\;\texteuro. Il gruppo prevede di consegnare il
prodotto finito entro la data \textbf{2024-09-06}.

\subsection{Riferimenti}
\subsubsection{Normativi}
\begin{itemize}
\setlength\itemsep{0em}
	\item \textit{Norme di progetto v.1.0.0};
	\item Presentazione e documentazione del capitolato d’appalto C6 - SyncCity:
	\begin{itemize}
            \setlength\itemsep{0em}
		\item \href{https://www.math.unipd.it/~tullio/IS-1/2023/Progetto/C6p.pdf}{https://www.math.unipd.it\textasciitilde{}tullio/IS-1/2023/Progetto/C6p.pdf} (Ultimo accesso: \today)
		\item \href{https://www.math.unipd.it/~tullio/IS-1/2023/Progetto/C6.pdf}{https://www.math.unipd.it/\textasciitilde{}tullio/IS-1/2023/Progetto/C6.pdf} (Ultimo accesso: \today)
	\end{itemize}
\end{itemize}

\subsubsection{Riferimenti informativi}
\begin{itemize}
\setlength\itemsep{0em}
    \item \textit{Glossario v1.0.0};
	\item Dispense T2 - Processi di ciclo di vita:
	\begin{itemize}
		\item
        \href{https://www.math.unipd.it/~tullio/IS-1/2023/Dispense/T2.pdf}{https://www.math.unipd.it/\textasciitilde{}tullio/IS-1/2023/Dispense/T2.pdf} (Ultimo accesso: \today)
	\end{itemize}
	\item Dispense T3 - Il ciclo di vita del software:
	\begin{itemize}
		\item
        \href{https://www.math.unipd.it/~tullio/IS-1/2023/Dispense/T3.pdf}{https://www.math.unipd.it/\textasciitilde{}tullio/IS-1/2023/Dispense/T3.pdf} (Ultimo accesso: \today)
	\end{itemize}
	\item Dispense T4 - Gestione di progetto:
	\begin{itemize}
		\item
        \href{https://www.math.unipd.it/~tullio/IS-1/2023/Dispense/T4.pdf}{https://www.math.unipd.it/\textasciitilde{}tullio/IS-1/2023/Dispense/T4.pdf} (Ultimo accesso: \today)
	\end{itemize}
\item Dispense T5 - Analisi dei requisiti:
	\begin{itemize}
		\item
        \href{https://www.math.unipd.it/~tullio/IS-1/2023/Dispense/T5.pdf}{https://www.math.unipd.it/\textasciitilde{}tullio/IS-1/2023/Dispense/T5.pdf} (Ultimo accesso: \today)
	\end{itemize}
    \end{itemize}
