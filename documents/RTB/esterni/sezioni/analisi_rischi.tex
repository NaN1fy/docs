Sono molteplici gli imprevisti che si possono manifestare durante la realizzazione di un progetto. Se non affrontate correttamente, le problematiche riscontrate possono impattare negativamente sullo svolgimento delle attività, causando un aumento dei costi, ritardi nel processo di sviluppo e una diminuzione della qualità del prodotto. Pertanto, si rivela essenziale condurre un'analisi approfondita dei rischi che possono insorgere, con l'obiettivo di prevenirli o quantomeno attenuarne gli effetti. \\
Il processo di mitigazione si compone quindi delle seguenti fasi: 
\begin{itemize}
\setlength\itemsep{0em}
    \item \textbf{Identificazione}: individuazione della minaccia, riconoscendone origine e contesto d'insorgenza;
    \item \textbf{Valutazione}: stima della probabilità di occorrenza e assegnazione di un grado di pericolosità indicativo del potenziale impatto sul progetto;
    \item \textbf{Mitigazione}: adozione di strategie di prevenzione e contrasto delle possibili conseguenze negative derivanti da una situazione avversa.
\end{itemize}
\textbf{Aggiornamento}: monitoraggio periodico delle attività in modo da riconoscere eventuali nuovi rischi o revisionare un processo di mitigazione precedentemente definito.

Di seguito sono elencati i rischi individuati, suddivisi per contesto.

\subsection{Rischi tecnologici}
\subsubsection{RT-1 Conoscenza limitata delle tecnologie}

\begin{itemize}
\setlength\itemsep{0em}
    \item \textbf{Identificazione}: ;
    \item \textbf{Valutazione}: ;
    \item \textbf{Mitigazione}: .
\end{itemize}

\subsubsection{RT-2 Produzione di codice incomprensibile}

\begin{itemize}
\setlength\itemsep{0em}
    \item \textbf{Identificazione}: ;
    \item \textbf{Valutazione}: ;
    \item \textbf{Mitigazione}: .
\end{itemize}

\subsubsection{RT-3 Dispersione dei file}

\begin{itemize}
\setlength\itemsep{0em}
    \item \textbf{Identificazione}: ;
    \item \textbf{Valutazione}: ;
    \item \textbf{Mitigazione}: .
\end{itemize}

\subsection{Rischi di comunicazione}
\subsubsection{RC-1 Scarsa comunicazione interna}

\begin{itemize}
\setlength\itemsep{0em}
    \item \textbf{Identificazione}: ;
    \item \textbf{Valutazione}: ;
    \item \textbf{Mitigazione}: .
\end{itemize}

\subsubsection{RC-2 Scarsa comunicazione con la proponente}

\begin{itemize}
\setlength\itemsep{0em}
    \item \textbf{Identificazione}: ;
    \item \textbf{Valutazione}: ;
    \item \textbf{Mitigazione}: .
\end{itemize}

\subsubsection{RC-3 Conflitti interni dovuti ad idee differenti}

\begin{itemize}
\setlength\itemsep{0em}
    \item \textbf{Identificazione}: ;
    \item \textbf{Valutazione}: ;
    \item \textbf{Mitigazione}: .
\end{itemize}

\subsubsection{RC-4 Confusione nella rotazione dei ruoli}

\begin{itemize}
\setlength\itemsep{0em}
    \item \textbf{Identificazione}: ;
    \item \textbf{Valutazione}: ;
    \item \textbf{Mitigazione}: .
\end{itemize}

\subsubsection{RC-5 Mancanza di fiducia}

\begin{itemize}
\setlength\itemsep{0em}
    \item \textbf{Identificazione}: ;
    \item \textbf{Valutazione}: ;
    \item \textbf{Mitigazione}: .
\end{itemize}

\subsection{Rischi di pianificazione}

\subsubsection{RP-1 Stima errata delle tempistiche di progetto}
%suddivisione temporale in sprint plasmata sulla cadenza dei SAL => poco margine in caso di problemi
\begin{itemize}
\setlength\itemsep{0em}
    \item \textbf{Identificazione}: ;
    \item \textbf{Valutazione}: ;
    \item \textbf{Mitigazione}: .
\end{itemize}

\subsubsection{RP-2 Stima errata dei costi di progetto}

\begin{itemize}
\setlength\itemsep{0em}
    \item \textbf{Identificazione}: ;
    \item \textbf{Valutazione}: ;
    \item \textbf{Mitigazione}: .
\end{itemize}

\subsubsection{RP-3 Stima errata del tempo di completamento di un'attività}

\begin{itemize}
\setlength\itemsep{0em}
    \item \textbf{Identificazione}: ;
    \item \textbf{Valutazione}: ;
    \item \textbf{Mitigazione}: .
\end{itemize}

\subsubsection{RP-4 Impegni personali ed accademici}

\begin{itemize}
\setlength\itemsep{0em}
    \item \textbf{Identificazione}: ;
    \item \textbf{Valutazione}: ;
    \item \textbf{Mitigazione}: .
\end{itemize}
