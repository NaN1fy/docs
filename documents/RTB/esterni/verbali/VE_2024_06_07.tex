% changelog: "1.0.0, 2024-06-10, Approvazione per RTB"
\documentclass[8pt]{article}
\usepackage[italian]{babel}
\usepackage[utf8]{inputenc}
\usepackage[letterpaper, left=1in, right=1in, bottom=0.75in, top=0.75in]{geometry}
\usepackage{amsmath}
\usepackage{subfiles}
\usepackage{lipsum}
\usepackage{csquotes}
\usepackage{amsfonts}
\usepackage[sfdefault]{plex-sans}
\usepackage{float}
\usepackage{pifont}
\usepackage{mathabx}
\usepackage[euler]{textgreek}
\usepackage{makecell}
\usepackage{tikz}
\usepackage{wrapfig}
\usepackage{siunitx}
\usepackage{amssymb} 
\usepackage{tabularx}
\usepackage{adjustbox}
\usepackage[document]{ragged2e}
\usepackage{floatflt}
\usepackage[hidelinks]{hyperref}
\usepackage{graphicx}
\usepackage{hyperref}
\setcounter{tocdepth}{4}
\usepackage{caption}
\usepackage{multicol}
\usepackage{tikz}
\setlength\parindent{0pt}
\captionsetup{font=footnotesize}
\usepackage{fancyhdr} 
\usepackage{graphicx}
\usepackage{capt-of}% 
\usepackage{booktabs}
\usepackage{varwidth}

% -- TITOLO INTESTAZIONE -- %
\newcommand{\customtitle}{VERBALE ESTERNO DEL 2024-06-07} % o ESTERNO

% -- STILE INTESTAZIONE -- %
\fancypagestyle{mystyle}{
	\fancyhf{} 
	\fancyhead[R]{\includegraphics[height=1cm]{../../../template/images/logos/NaN1fy_logo.png}} 
	\fancyhead[L]{\leftmark} 
	\renewcommand{\headrulewidth}{1pt} 
	\fancyhead[L]{\customtitle} 
	\renewcommand{\headsep}{1.3cm} 
	\fancyfoot[C]{\thepage} 
}

% -- PER LA FIRMA -- %
\newcommand{\signatureline}[1]{%
	 \par\vspace{0.5cm}
	\noindent\makebox[\linewidth][r]{\rule{0.2\textwidth}{0.5pt}\hspace{3cm}\makebox[0pt][r]{\vspace{3pt}\footnotesize #1}}%
}

% -- PER IL GLOSSARIO -- %
\newcommand{\glossterm}[1]{#1\textsuperscript{G}} % inserisci \glossterm{termine}

\begin{document}
\definecolor{myblue}{RGB}{23,103,162}
\begin{titlepage}
	\begin{tikzpicture}[remember picture, overlay]
		\node[anchor=south east, opacity=0.2, yshift = -4cm, xshift= 2em] at (current page.south east) {\includegraphics[width=0.7\textwidth, trim=0cm 0cm 5cm 0cm, clip]{../../../template/images/logos/Universita_Padova_transparent.png}}; 
		\node[anchor=north west, opacity=1, yshift = 4.2cm, xshift= 1.4cm, scale=1.6] at (current page.south west) {\includegraphics[width=4cm]{../../../template/images/logos/NaN1fy_logo.png}};
	\end{tikzpicture}
	
	\begin{minipage}[t]{0.47\textwidth}
		{\large{\textsc{Destinatari}}
			\vspace{3mm}
			\\ \large{\textsc{Prof. Tullio Vardanega}}
			\\ \large{\textsc{Prof. Riccardo Cardin}}
		}
	\end{minipage}
	\hfill
	\begin{minipage}[t]{0.47\textwidth}\raggedleft
		{\large{\textsc{Redattori}}
			\vspace{3mm}
			{\\\large{\textsc{Davide Donanzan}\\}}
		}
		\vspace{8mm}
		
		{\large{\textsc{Verificatori}}
			\vspace{3mm}
			{\\\large{\textsc{Guglielmo Barison}\\}} % massimo due 
			{\large{\textsc{Pietro Busato}}}
			
		}
		\vspace{4mm}\vspace{4mm}
	\end{minipage}
	\vspace{4cm}
	\begin{center}
		\begin{flushright}
			{\fontsize{30pt}{52pt}\selectfont \textbf{Verbale Esterno Del\\2024-06-07\\}} % o ESTERNO
		\end{flushright}
		\vspace{3cm}
	\end{center}
	\vspace{8.5 cm}
	{\small \textsc{\href{mailto: nan1fyteam.unipd@gmail.com}{nan1fyteam.unipd@gmail.com}}}
\end{titlepage}
\pagestyle{mystyle}

\section*{Registro delle Modifiche}
\begin{table}[ht!]	
	\centering
	\begin{tabular}{p{1.2cm} p{2cm} p{6cm} p{3cm} p{2cm}}
		\toprule
		\textbf{Versione}& \textbf{Data} & \textbf{Descrizione} & \textbf{Autore} & \textbf{Ruolo} \\
		\midrule
      1.0.0 & 2024-06-10 & \textbf{Approvazione per RTB} \\\\
  		0.0.1 & 2024-06-10 & Verifica e correzioni minori. & Pietro Busato, Guglielmo Barison & Verificatore \\\\
		0.0.0 & 2024-06-08 & Stesura verbale. & Davide Donanzan & Redattore \\ % spazio tra le righe
		%X.X.X & YYYY-MM-DD & Lorem ipsum dolor sit amet, consectetur adipiscing elit.  & XXXX XXXX & --- \\
		\bottomrule
		% Ruolo Redattore o Verificatore
	\end{tabular}
	\caption{Registro delle modifiche.}
	\label{table:Registro delle modifiche}
\end{table}
\newpage
\tableofcontents
\clearpage
\newpage
\justifying
\section{Contenuti del Verbale}
\subsection{Informazioni sulla riunione}
\begin{itemize}
	\setlength\itemsep{0em}
	\item\textbf{Luogo:} Chiamata Google Meet;
	\item\textbf{Ora di inizio:} 14:00;
	\item\textbf{Ora di fine:}  14:45.
\end{itemize}
\begin{table}[ht!]
	\begin{minipage}[t]{0.5\linewidth}
		\centering
		\begin{tabular}{p{3cm} p{3cm}}
			\toprule
			\textbf{Partecipante} & \textbf{Durata presenza} \\
			\midrule
			Guglielmo Barison & 0.75 h \\
			Linda Barbiero & 0.75 h \\
			Pietro Busato & 0.75 h \\
			Oscar Konieczny & 0.75 h \\
			Davide Donanzan & 0.75 h \\
			Veronica Tecchiati & 0.75 h \\
			\bottomrule
		\end{tabular}
		\caption{Partecipanti NaN1fy.}
		\label{table:Partecipanti NaN1fy}
	\end{minipage} 
	\begin{minipage}[t]{0.5\linewidth} % -- COMMENTA/DECOMMENTA DA QUI
		\centering
		\begin{tabular}{p{3cm} p{3cm}}
			\toprule
			\textbf{Partecipante} & \textbf{Durata presenza} \\
			\midrule
			Andrea Dorigo & 0.75 h \\
			Fabio Pallaro & 0.75 h \\
			Daniele Zorzi & 0.75 h \\
			\bottomrule
		\end{tabular}
		\caption{Partecipanti SyncLab.}
		\label{table:Partecipanti SyncLab}
	\end{minipage} % -- A QUI PER TOGLIERE AGGIUNGERE
\end{table}
\subsection{Ordine del giorno}
\begin{itemize}
	\item Discussione sull'avvenuta presentazione alla revisione RTB;	
	\item Revisione del lavoro svolto nel quinto Sprint;
	\item Discussione riguardante la documentazione prodotta dal team;
	\item Stesura degli obiettivi per il sesto Sprint.
\end{itemize}
\subsection{Sintesi dell'incontro}
L'incontro ha avuto inizio con la comunicazione da parte del team dell'avvenuta presentazione alla revisone RTB. 
Abbiamo così proseguito nel mostrare i progressi fatti nella visualizzazione delle dashboard e nelle rifiniture richieste dalla Proponente nello scorso SAL. \\
I membri della Proponente si sono detti soddisfatti dell'operato sia riguardante il prodotto in sè, sia riguardante la documentazione prodotta. \\
In attesa della risposta dei Committenti, il team insieme alla proponente ha valutato i prossimi obiettivi da perseguire con l'obiettivo di avvicinarsi alla produzione di un MVP. 
I prossimi passi saranno incentrati sull'arricchimento del generatore di dati con le nuove tipologie di sensori definite nell'\textit{Analisi dei Requisiti}.
Inoltre la Proponente ha richiesto una valutazione di possibili casistiche di processamento dei dati nella quale potesse essere necessaria l'implementazione di piattaforme per l'elaborazione di flussi di dati come ad esempio Apache Flink.\\
Il prossimo Sprint sarà di conseguenza dedicato allo studio di questa nuova tecnologia e dei suoi possibili utilizzi.
\subsection{Decisioni prese}
\begin{itemize}
	\setlength\itemsep{0em}
	\item Viene fissato il prossimo SAL il giorno 2024-06-21 alle ore 15:00.
\end{itemize}
\newpage
\section{Attività da svolgere}
Viene indetta una riunione interna per discutere l'organizzazione dei compiti da portare a termine nel sesto Sprint. 
% togli il commento per la firma
\signatureline{Padova, 2024-06-08}
%\signatureline{Padova, YYYY-MM-DD}
\end{document}
