% changelog: "0.0.1, 2024-05-23, Veronica Tecchiati, Verifica completa, modifica data firma"

\documentclass[8pt]{article}
\usepackage[italian]{babel}
\usepackage[utf8]{inputenc}
\usepackage[letterpaper, left=1in, right=1in, bottom=0.75in, top=0.75in]{geometry}
\usepackage{amsmath}
\usepackage{subfiles}
\usepackage{lipsum}
\usepackage{csquotes}
\usepackage{amsfonts}
\usepackage[sfdefault]{plex-sans}
\usepackage{float}
\usepackage{pifont}
\usepackage{mathabx}
\usepackage[euler]{textgreek}
\usepackage{makecell}
\usepackage{tikz}
\usepackage{wrapfig}
\usepackage{siunitx}
\usepackage{amssymb} 
\usepackage{tabularx}
\usepackage{adjustbox}
\usepackage[document]{ragged2e}
\usepackage{floatflt}
\usepackage[hidelinks]{hyperref}
\usepackage{graphicx}
\usepackage{hyperref}
\setcounter{tocdepth}{4}
\usepackage{caption}
\usepackage{multicol}
\usepackage{tikz}
\setlength\parindent{0pt}
\captionsetup{font=footnotesize}
\usepackage{fancyhdr} 
\usepackage{graphicx}
\usepackage{capt-of}% 
\usepackage{booktabs}
\usepackage{varwidth}


\newcommand{\customtitle}{VERBALE ESTERNO DEL 2024-05-17}

\fancypagestyle{mystyle}{
	\fancyhf{} 
	\fancyhead[R]{\includegraphics[height=1cm]{../../../template/images/logos/NaN1fy_logo.png}} 
	\fancyhead[L]{\leftmark} 
	\renewcommand{\headrulewidth}{1pt} 
	\fancyhead[L]{\customtitle} 
	\renewcommand{\headsep}{1.3cm} 
	\fancyfoot[C]{\thepage} 
}

\newcommand{\signatureline}[1]{%
	 \par\vspace{0.5cm}
	\noindent\makebox[\linewidth][r]{\rule{0.2\textwidth}{0.5pt}\hspace{3cm}\makebox[0pt][r]{\vspace{3pt}\footnotesize #1}}%
}

\newcommand{\glossterm}[1]{#1\textsuperscript{G}} % inserisci \glossterm{termine}

\begin{document}
\definecolor{myblue}{RGB}{23,103,162}
\begin{titlepage}
	\begin{tikzpicture}[remember picture, overlay]
      \node[anchor=south east, opacity=0.2, yshift = -4cm, xshift= 2em] at (current page.south east) {\includegraphics[width=0.7\textwidth, trim=0cm 0cm 5cm 0cm, clip]{../../../template/images/logos/Universita_Padova_transparent.png}};
		\node[anchor=north west, opacity=1, yshift = 4.2cm, xshift= 1.4cm, scale=1.6] at (current page.south west) {\includegraphics[width=4cm]{../../../template/images/logos/NaN1fy_logo.png}};
	\end{tikzpicture}
	
	\begin{minipage}[t]{0.47\textwidth}
		{\large{\textsc{Destinatari}}
			\vspace{3mm}
			\\ \large{\textsc{Prof. Tullio Vardanega}}
			\\ \large{\textsc{Prof. Riccardo Cardin}}
		}
	\end{minipage}
	\hfill
	\begin{minipage}[t]{0.47\textwidth}\raggedleft
		{\large{\textsc{Redattori}}
			\vspace{3mm}
			{\\\large{\textsc{Davide Donanzan}\\}} % massimo due 
		}
		\vspace{8mm}
		
		{\large{\textsc{Verificatori}}
			\vspace{3mm}
			{\\\large{\textsc{XXXX XXXX}\\}} % massimo due 
			{\large{\textsc{Veronica Tecchiati}}}
			
		}
		\vspace{4mm}\vspace{4mm}
	\end{minipage}
	\vspace{4cm}
	\begin{center}
		\begin{flushright}
			{\fontsize{30pt}{52pt}\selectfont \textbf{Verbale Esterno Del\\2024-05-17\\}} % o ESTERNO
		\end{flushright}
		\vspace{3cm}
	\end{center}
	\vspace{8.5 cm}
	{\small \textsc{\href{mailto: nan1fyteam.unipd@gmail.com}{nan1fyteam.unipd@gmail.com}}}
\end{titlepage}

\pagestyle{mystyle}
\section*{Registro delle Modifiche}
\begin{table}[ht!]	
	\centering
	\begin{tabular}{p{1.2cm} p{2cm} p{6cm} p{3cm} p{2cm}}
		\toprule
		\textbf{Versione}& \textbf{Data} & \textbf{Descrizione} & \textbf{Autore} & \textbf{Ruolo} \\
		\midrule
        0.0.1 & 2024-05-23 & Verifica completa, modifica data firma. & Veronica Tecchiati & Verificatore \\
        0.0.0 & 2024-05-17 & Stesura documento. & Davide Donanzan & Redattore \\
		\bottomrule
		% Ruolo Redattore o Verificatore
	\end{tabular}
	\caption{Registro delle modifiche.}
	\label{table:Registro delle modifiche}
\end{table}
\newpage
\tableofcontents
\clearpage
\newpage
\justifying
\section{Contenuti del Verbale}
\subsection{Informazioni sulla riunione}
\begin{itemize}
	\setlength\itemsep{0em}
	\item\textbf{Luogo:} Chiamata Google Meet;
	\item\textbf{Ora di inizio:} 16:00;
	\item\textbf{Ora di fine:}  17:00.
\end{itemize}
\begin{table}[ht!]
	\begin{minipage}[t]{0.5\linewidth}
		\centering
		\begin{tabular}{p{3cm} p{3cm}}
			\toprule
			\textbf{Partecipante} & \textbf{Durata presenza} \\
			\midrule
			Guglielmo Barison & 1.0 h \\
			Linda Barbiero &  1.0 h \\
			Pietro Busato & 1.0 h \\
			Oscar Konieczny & 0.0 h \\
			Davide Donanzan & 1.0 h \\
			Veronica Tecchiati & 1.0 h \\
			\bottomrule
		\end{tabular}
		\caption{Partecipanti NaN1fy.}
		\label{table:Partecipanti NaN1fy}
	\end{minipage} 
	\begin{minipage}[t]{0.5\linewidth} % -- COMMENTA/DECOMMENTA DA QUI
		\centering
		\begin{tabular}{p{3cm} p{3cm}}
			\toprule
			\textbf{Partecipante} & \textbf{Durata presenza} \\
			\midrule
			Andrea Dorigo & 1.0 h \\
			Fabio Pallaro &  1.0 h \\
			Daniele Zorzi & 0.5 h \\
			\bottomrule
		\end{tabular}
		\caption{Partecipanti SyncLab.}
		\label{table:Partecipanti SyncLab}
	\end{minipage} % -- A QUI PER TOGLIERE AGGIUNGERE
\end{table}

\subsection{Ordine del giorno}
\begin{itemize}
	\setlength\itemsep{0em}
	\item Revisione dello stato di avanzamento degli obbiettivi del terzo Sprint;
	\item Discussione sui problemi riscontrati nell'implementazione del sensore di tipo \textit{parking} e nella persistenza dei dati da esso prodotti attraverso ClickHouse;
    \item Definizione degli obiettivi per il prossimo Sprint e in previsione dell'RTB.
\end{itemize}
\subsection{Sintesi dell'incontro}
L'incontro con la Proponente ha visto come principale argomento la discussione sulle possibili soluzioni dei problemi riscontrati durante il terzo Sprint. Il team ha dedicato una consistente parte della riunione all'individuazioni delle migliorie da apportare al generatore di dati \textit{PyMockSensor} e alla rappresentazione dei dati da esso generati. Queste soluzioni delineano gli obiettivi proposti per il prossimo Sprint.

\subsubsection{Avanzamenti nel codice}
Gli obiettivi concordati erano l'\textbf{implementazione di due nuovi sensori} e la \textbf{rappresentazione dei dati} su Grafana con \textbf{dashboard significative}. Il team ha realizzato l'implementazione dei sensori, compresa di \textit{test di unità} per il controllo dell'efficacia del codice scritto. Come citato precedentemente sono state individuate difficoltà nella persistenza dei dati dei sensori di tipo \textit{parking}.

\subsubsection{Obiettivi dello Sprint}
La Proponente ha richiesto la riorganizzazione del codice per la riduzione dei dati generati e la produzione di dashboard con query maggiormente elaborate e analitiche. Al team viene proposto, inoltre, di rivalutare il metodo di segnalazione dei sensori: si vuole favorire un sistema \textit{event-driven} a discapito dell'attuale, basato su \textit{istantanee} dei sensori. La fine del quarto Sprint, e di conseguenza la data del prossimo SAL, è stata fissata per venerdì 31 maggio.\\
\texttt{SyncLab} ci informa la disponibilità di anticipare l'incontro qualora il team avesse completato gli obiettivi con anticipo rispetto alla data prefissata.

\subsection{Conclusioni}
\begin{itemize}
	\setlength\itemsep{0em}
	\item \textit{Viene fissato} per il 2024-05-31 alle ore 16:00 il prossimo SAL.
	% consigliata la forma \textit{Viene adottato} quando viene adottato un certo modo di fare/strumento
	% per nomi di aziene e capitolati usare \texttt{}, e.g \texttt{Easy meal} \texttt{C6} 
\end{itemize}

\section{Attività da svolgere}
È necessario stabilire subito la suddivisione dei compiti per il prossimo Sprint, continuando al contempo il lavoro di stesura dei vari documenti per l'RTB. Viene indetta, quindi, una riunione interna per definire i dettagli organizzativi del quarto Sprint. L'incontro è fissato per il giorno stesso in successione al SAL qui verbalizzato.\\
% commentare per togliere la firma
\signatureline{Padova, 2024-05-24}
\end{document}
