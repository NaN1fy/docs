% changelog: "0.0.0, 2024-05-31, Pietro Busato, Stesura documento"
\documentclass[8pt]{article}
\usepackage[italian]{babel}
\usepackage[utf8]{inputenc}
\usepackage[letterpaper, left=1in, right=1in, bottom=0.75in, top=0.75in]{geometry}
\usepackage{amsmath}
\usepackage{subfiles}
\usepackage{lipsum}
\usepackage{csquotes}
\usepackage{amsfonts}
\usepackage[sfdefault]{plex-sans}
\usepackage{float}
\usepackage{pifont}
\usepackage{mathabx}
\usepackage[euler]{textgreek}
\usepackage{makecell}
\usepackage{tikz}
\usepackage{wrapfig}
\usepackage{siunitx}
\usepackage{amssymb} 
\usepackage{tabularx}
\usepackage{adjustbox}
\usepackage[document]{ragged2e}
\usepackage{floatflt}
\usepackage[hidelinks]{hyperref}
\usepackage{graphicx}
\usepackage{hyperref}
\setcounter{tocdepth}{4}
\usepackage{caption}
\usepackage{multicol}
\usepackage{tikz}
\setlength\parindent{0pt}
\captionsetup{font=footnotesize}
\usepackage{fancyhdr} 
\usepackage{graphicx}
\usepackage{capt-of}% 
\usepackage{booktabs}
\usepackage{varwidth}

% -- TITOLO INTESTAZIONE -- %
\newcommand{\customtitle}{VERBALE ESTERNO DEL 2024-05-31} % o ESTERNO

% -- STILE INTESTAZIONE -- %
\fancypagestyle{mystyle}{
	\fancyhf{} 
	\fancyhead[R]{\includegraphics[height=1cm]{images/logos/NaN1fy_logo.png}} 
	\fancyhead[L]{\leftmark} 
	\renewcommand{\headrulewidth}{1pt} 
	\fancyhead[L]{\customtitle} 
	\renewcommand{\headsep}{1.3cm} 
	\fancyfoot[C]{\thepage} 
}

% -- PER LA FIRMA -- %
\newcommand{\signatureline}[1]{%
	 \par\vspace{0.5cm}
	\noindent\makebox[\linewidth][r]{\rule{0.2\textwidth}{0.5pt}\hspace{3cm}\makebox[0pt][r]{\vspace{3pt}\footnotesize #1}}%
}

% -- PER IL GLOSSARIO -- %
\newcommand{\glossterm}[1]{#1\textsuperscript{G}} % inserisci \glossterm{termine}

\begin{document}
\definecolor{myblue}{RGB}{23,103,162}
\begin{titlepage}
	\begin{tikzpicture}[remember picture, overlay]
		\node[anchor=south east, opacity=0.2, yshift = -4cm, xshift= 2em] at (current page.south east) {\includegraphics[width=0.7\textwidth, trim=0cm 0cm 5cm 0cm, clip]{images/logos/Universita_Padova_transparent.png}}; 
		\node[anchor=north west, opacity=1, yshift = 4.2cm, xshift= 1.4cm, scale=1.6] at (current page.south west) {\includegraphics[width=4cm]{images/logos/NaN1fy_logo.png}};
	\end{tikzpicture}
	
	\begin{minipage}[t]{0.47\textwidth}
		{\large{\textsc{Destinatari}}
			\vspace{3mm}
			\\ \large{\textsc{Prof. Tullio Vardanega}}
			\\ \large{\textsc{Prof. Riccardo Cardin}}
		}
	\end{minipage}
	\hfill
	\begin{minipage}[t]{0.47\textwidth}\raggedleft
		{\large{\textsc{Redattori}}
			\vspace{3mm}
			{\\\large{\textsc{Pietro Busato}\\}} 	
		}
		\vspace{8mm}
		
		{\large{\textsc{Verificatori}}
			\vspace{3mm}
			{\\\large{\textsc{XXXX XXXX}\\}} % massimo due 
			{\large{\textsc{XXXX XXXX}}}
			
		}
		\vspace{4mm}\vspace{4mm}
	\end{minipage}
	\vspace{4cm}
	\begin{center}
		\begin{flushright}
			{\fontsize{30pt}{52pt}\selectfont \textbf{Verbale Esterno Del\\2024-05-31\\}} % o ESTERNO
		\end{flushright}
		\vspace{3cm}
	\end{center}
	\vspace{8 cm}
	{\small \textsc{\href{mailto: nan1fyteam.unipd@gmail.com}{nan1fyteam.unipd@gmail.com}}}
\end{titlepage}
\pagestyle{mystyle}

\section*{Registro delle Modifiche}
\begin{table}[ht!]	
	\centering
	\begin{tabular}{p{1.2cm} p{2cm} p{6cm} p{3cm} p{2cm}}
		\toprule
		\textbf{Versione}& \textbf{Data} & \textbf{Descrizione} & \textbf{Autore} & \textbf{Ruolo} \\
		\midrule
		0.0.0 & 204-05-31 & Stesura verbale.  & Pietro Busato & Redattore
		--- \\\\ % spazio tra le righe
		%X.X.X & YYYY-MM-DD & Lorem ipsum dolor sit amet, consectetur adipiscing elit.  & XXXX XXXX & --- \\
		\bottomrule
		% Ruolo Redattore o Verificatore
	\end{tabular}
	\caption{Registro delle modifiche.}
	\label{table:Registro delle modifiche}
\end{table}
\newpage
\tableofcontents
\clearpage
\newpage
\justifying
\section{Contenuti del Verbale}
\subsection{Informazioni sulla riunione}
\begin{itemize}
	\setlength\itemsep{0em}
	\item\textbf{Luogo:} Chiamata Google Meet;
	\item\textbf{Ora di inizio:} 16:00;
	\item\textbf{Ora di fine:}  16:45.
\end{itemize}
\begin{table}[ht!]
	\begin{minipage}[t]{0.5\linewidth}
		\centering
		\begin{tabular}{p{3cm} p{3cm}}
			\toprule
			\textbf{Partecipante} & \textbf{Durata presenza} \\
			\midrule
			Guglielmo Barison & 0.75 h \\
			Linda Barbiero &  0.75 h \\
			Pietro Busato & 0.75 h \\
			Oscar Konieczny & 0.75 h \\
			Davide Donanzan & 0.75 h \\
			Veronica Tecchiati & 0.75 h \\
			\bottomrule
		\end{tabular}
		\caption{Partecipanti NaN1fy}
		\label{table:Partecipanti NaN1fy}
	\end{minipage} 
	\begin{minipage}[t]{0.5\linewidth} % -- COMMENTA/DECOMMENTA DA QUI
		\centering
		\begin{tabular}{p{3cm} p{3cm}}
			\toprule
			\textbf{Partecipante} & \textbf{Durata presenza} \\
			\midrule
			Andrea Dorigo & 0.33 h \\
			Fabio Pallaro &  0.75 h \\
			Daniele Zorzi &  0.75 h \\
			\bottomrule
		\end{tabular}
		\caption{Partecipanti SyncLab}
		\label{table:Partecipanti SyncLab}
	\end{minipage} % -- A QUI PER TOGLIERE AGGIUNGERE
\end{table}
\subsection{Ordine del giorno}
\begin{itemize}
	\setlexngth\itemsep{0em}
	\item Revisione dello stato di avanzamento degli obiettivi del quarto sprint;	
	\item Discussione riguardante lo Sprint successivo e l'avvento dell'RTB;
	\item Discussione riguardante l'eventuale avanzamento del progetto a fase CA.
\end{itemize}
\subsection{Sintesi dell'incontro}
Come consuetudine, l'incontro è iniziato con la presentazione, da parte del gruppo, dei progressi fatti durante lo Sprint. I membri della
Proponente si sono detti pienamente soddisfatti dell'operato riguardante il refactoring dei sensori e del loro funzionamento in relazione
alla pipeline Kafka-Clickhouse-Grafana; hanno inoltre reagito positivamente anche all'aggiunta e miglioramento dei grafici e delle viste
sui sensori sulla dashboard di Grafana, pur consigliando alcune semplici migliorie per alcune di esse. In particolare, l'aggiunta di una
mappa, a corredo delle tabelle di parcheggio, che mostri a vista d'occhio la disponibilità dei vari parcheggi registrati, oltre che un
grafico che mostri la percentuale di posti occupati per un determinato parcheggio. Con la prematura uscita dalla chiamata del Sig. Dorigo,
principale referente in materia di tecnologie e codice, la conversazione si è quindi spostata sull'organizzazione del prossimo Sprint, della
durata di una singola settimana, considerato ``di passaggio'' in quanto nella stessa settimana avverrà la presentazione dell'RTB, con conseguente
raggiungimento della prima milestone. Data la peculiare natura di tale Sprint, i membri dell'azienda hanno deciso di non porre nuovi obiettivi
al di là di una semplice ulteriore rifinitura dei documenti e del prodotto. Si è infine discusso della possibilità e della volontà, da parte del 
team, di procedere, una volta ultimato il PB, con l'avanzamento alla fase CA. Il team si è riservato di decidere in futuro se proporre o meno 
l'avanzamento a CA, dato l'alta probabilità di incorrere, nel prossimo periodo, in numerosi contrattempi e difficoltà, dovuti sia a questioni 
private che di università. 

\subsection{Decisioni prese}
\begin{itemize}
	\setlength\itemsep{0em}
	\item Si è fissato il prossimo SAL il giorno 2024-06-07 alle ore 15:00;
	\item Si è deciso di considerare questo Sprint come ``di passaggio'', senza dunque fissare obiettivi particolari.
\end{itemize}
\newpage
\section{Attività da svolgere}
Viene indetta una riunione interna per discutere della fase finale dell'RTB e dell'organizzazione del quinto Sprint, il cui focus
sarà l'ultimazione del perfezionamento dei documenti e del prodotto, in vista della presentazione. 
% togli il commento per la firma
\signatureline{Padova, 2024-06-04}
%\signatureline{Padova, YYYY-MM-DD}
\end{document}
