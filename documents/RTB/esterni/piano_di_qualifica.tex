% changelog: "0.1.0, 2024-04-09, Guglielmo Barison, Stesura Obiettivi di qualità"

\documentclass[8pt]{article}
\usepackage[italian]{babel}
\usepackage[utf8]{inputenc}
\usepackage[letterpaper, left=1in, right=1in, bottom=0.75in, top=0.75in]{geometry}
\usepackage{amsmath}
\usepackage{subfiles}
\usepackage{lipsum}
\usepackage{csquotes}
\usepackage{amsfonts}
\usepackage[sfdefault]{plex-sans}
\usepackage{float}
\usepackage{pifont}
\usepackage{mathabx}
\usepackage[euler]{textgreek}
\usepackage{makecell}
\usepackage{tikz}
\usepackage{wrapfig}
\usepackage{siunitx}
\usepackage{amssymb} 
\usepackage{tabularx}
\usepackage{adjustbox}
\usepackage[document]{ragged2e}
\usepackage{floatflt}
\usepackage[hidelinks]{hyperref}
\usepackage{graphicx}
\usepackage{hyperref}
\setcounter{tocdepth}{4}
\usepackage{caption}
\usepackage{multicol}
\usepackage{tikz}
\setlength\parindent{0pt}
\captionsetup{font=footnotesize}
\usepackage{fancyhdr} 
\usepackage{graphicx}
\usepackage{capt-of}% 
\usepackage{booktabs}
\usepackage{varwidth}
\usepackage{datetime2}
\newcommand{\customtitle}{PIANO DI QUALIFICA}% o ESTERNO

% -- STILE INTESTAZIONE -- %
\fancypagestyle{mystyle}{
	\fancyhf{} 
	\fancyhead[R]{\includegraphics[height=1cm]{../../template/images/logos/NaN1fy_logo.png}} 
	\fancyhead[L]{\leftmark} 
	\renewcommand{\headrulewidth}{1pt} 
	\fancyhead[L]{\customtitle} 
	\renewcommand{\headsep}{1.3cm} 
	\fancyfoot[C]{\thepage} 
}

% -- PER LA FIRMA -- %
\newcommand{\signatureline}[1]{%
	 \par\vspace{0.5cm}
	\noindent\makebox[\linewidth][r]{\rule{0.2\textwidth}{0.5pt}\hspace{3cm}\makebox[0pt][r]{\vspace{3pt}\footnotesize #1}}%
}

% -- PER IL GLOSSARIO -- %
\newcommand{\glossterm}[1]{#1\textsuperscript{G}} % inserisci \glossterm{termine}

% -- per abilitare 4x sottosezioni es 2.1.1.1
\setcounter{secnumdepth}{4}
\newcommand{\subsubsubsection}[1]{\paragraph{#1}\mbox{}\\}




\begin{document}
\definecolor{myblue}{RGB}{23,103,162}
\begin{titlepage}
	\begin{tikzpicture}[remember picture, overlay]
		\node[anchor=south east, opacity=0.2, yshift = -4cm, xshift= 2em] at (current page.south east) {\includegraphics[width=0.7\textwidth, trim=0cm 0cm 5cm 0cm, clip]{../../template/images/logos/Universita_Padova_transparent.png}}; 
		\node[anchor=north west, opacity=1, yshift = 4.2cm, xshift= 1.4cm, scale=1.6] at (current page.south west) {\includegraphics[width=4cm]{../../template/images/logos/NaN1fy_logo.png}};
	\end{tikzpicture}
	
	\begin{minipage}[t]{0.47\textwidth}
		{\large{\textsc{Destinatari}}
			\vspace{3mm}
			\\ \large{\textsc{Prof. Tullio Vardanega}}
			\\ \large{\textsc{Prof. Riccardo Cardin}}
		}
	\end{minipage}
	\hfill
	\begin{minipage}[t]{0.47\textwidth}\raggedleft
		{\large{\textsc{Redattori}}
			\vspace{3mm}
			{\\\large{\textsc{Guglielmo Barison}\\}} % massimo due 
			{\large{\textsc{Davide Donanzan}}}
			
			
		}
		\vspace{8mm}
		
		{\large{\textsc{Verificatori}}
			\vspace{3mm}
			{\\\large{\textsc{XXXX XXXX}\\}} 
			{\large{\textsc{XXXX XXXX}}}
			
		}
		\vspace{4mm}\vspace{4mm}
	\end{minipage}
	\vspace{4cm}
	\begin{center}
		\begin{flushright}
			{\fontsize{30pt}{52pt}\selectfont \textbf{Piano Di Qualifica}} % o ESTERNO
		\end{flushright}
		\vspace{3cm}
	\end{center}
	\vspace{10 cm}
	{\small \textsc{\href{mailto: nan1fyteam.unipd@gmail.com}{nan1fyteam.unipd@gmail.com}}}
\end{titlepage}
\pagestyle{mystyle}
\section*{Registro delle Modifiche}
\begin{table}[ht!]	
	\centering
	\begin{tabular}{p{1.2cm} p{2cm} p{6cm} p{3cm} p{2cm}}
		\toprule
		\textbf{Versione}& \textbf{Data} & \textbf{Descrizione} & \textbf{Autore} & \textbf{Ruolo} \\
		\midrule
		0.1.0 & 2024-04-18 & Stesura ``Obiettivi di qualità". & Guglielmo Barison & Redattore \\\\
		0.0.0 & 2024-04-09 & Struttura di base ed introduzione.  & Guglielmo Barison & Redattore \\
		\bottomrule
		% Ruolo Redattore o Verificatore
	\end{tabular}
	\caption*{Tabella: Registro delle modifiche.}
	\label{table:Registro delle modifiche}
\end{table}
\newpage
\tableofcontents
\newpage
\listoffigures
\newpage
\listoftables
\newpage
\justifying
\section{Introduzione}
\subsection{Scopo del documento}
Questo documento offre una panoramica dettagliata delle strategie di verifica e validazione adottate per assicurare la \glossterm{qualità} del prodotto e dei processi nel contesto del progetto in questione. Sarà costantemente aggiornato per riflettere l'evoluzione del progetto e concentrerà l'attenzione sui risultati delle verifiche per risolvere tempestivamente eventuali criticità.
\\\\
Il Piano di Qualifica, dinamico e incrementale, illustra le pratiche per il controllo di qualità degli artefatti e dei processi, con particolare enfasi sulle metriche di valutazione del prodotto. È progettato per guidare l'adozione di processi mirati al miglioramento continuo, fornendo misure quantitative per valutare il progresso del progetto. Questo impegno costante per la qualità si riflette nel regolare aggiornamento del documento per adattarsi alle esigenze mutevoli del progetto, garantendo così la crescita e l'evoluzione sia del processo che del prodotto nel tempo.
\subsection{Scopo del capitolato}
Lo scopo del \glossterm{capitolato} C6 è proporre una soluzione per la creazione di una piattaforma di monitoraggio per una smart city. Tale piattaforma deve raccogliere e analizzare in tempo reale una vasta gamma di dati provenienti da sensori distribuiti nella città, riguardanti aspetti quali traffico, qualità dell'aria, consumi energetici e altro ancora. L'obiettivo è fornire alle autorità locali informazioni dettagliate per prendere decisioni informate sulla gestione delle risorse e l'implementazione dei servizi, coinvolgendo anche i cittadini attraverso la condivisione di dati e la partecipazione attiva. La soluzione proposta prevede l'utilizzo di tecnologie per il data streaming processing e la simulazione dei dati dei sensori, con l'obiettivo di fornire dashboard intuitive per la visualizzazione dei dati raccolti e l'analisi delle condizioni della città.\\
L’applicativo sviluppato richiede una copertura dei \glossterm{test} di almeno l’80\%.

\subsection{Glossario}
Per garantire chiarezza nel linguaggio utilizzato nei documenti, è stato redatto un Glossario contenente le definizioni dei termini con significato specifico da disambiguare. Tali termini sono contrassegnati con una G ad apice. L'inserimento di un termine nel Glossario è considerato completo solo dopo averne fornito la definizione.
\subsection{Riferimenti}
\subsubsection{Normativi}
\begin{itemize}
	\item \textit{Norme di progetto v.X.X.X};
	\item Presentazione e documentazione del \glossterm{capitolato} d’appalto C6 - SyncCity:
	\begin{itemize}
		\item \href{https://www.math.unipd.it/~tullio/IS-1/2023/Progetto/C6p.pdf}{\color{myblue}https://www.math.unipd.it\textasciitilde{}tullio/IS-1/2023/Progetto/C6p.pdf} (Ultimo accesso: \today)
		\item \href{https://www.math.unipd.it/~tullio/IS-1/2023/Progetto/C6.pdf}{\color{myblue}https://www.math.unipd.it/\textasciitilde{}tullio/IS-1/2023/Progetto/C6.pdf} (Ultimo accesso: \today)
	\end{itemize}
	\item Regolamento di progetto:
	\begin{itemize}
		\item \href{https://www.math.unipd.it/~tullio/IS-1/2023/Dispense/PD2.pdf}{\color{myblue}https://www.math.unipd.it/\textasciitilde{}tullio/IS-1/2023/Dispense/PD2.pdf} (Ultimo accesso: \today)
	\end{itemize}
\end{itemize}
\clearpage
\subsubsection{Informativi}
\begin{itemize}
	\item Dispense T7 - Qualità del software:
	\begin{itemize}
		\item \href{https://www.math.unipd.it/~tullio/IS-1/2023/Dispense/T7.pdf}{\color{myblue}https://www.math.unipd.it/\textasciitilde{}tullio/IS-1/2023/Dispense/T7.pdf} (Ultimo accesso: \today)
	\end{itemize}
	\item Dispense T8 - Qualità di processo:
	\begin{itemize}
		\item \href{https://www.math.unipd.it/~tullio/IS-1/2023/Dispense/T8.pdf}{\color{myblue}https://www.math.unipd.it/\textasciitilde{}tullio/IS-1/2023/Dispense/T8.pdf} (Ultimo accesso: \today)
	\end{itemize}
	\item Dispense T9 - Verifica e validazione:
	\begin{itemize}
		\item \href{https://www.math.unipd.it/~tullio/IS-1/2023/Dispense/T9.pdf}{\color{myblue}https://www.math.unipd.it/\textasciitilde{}tullio/IS-1/2023/Dispense/T9.pdf} (Ultimo accesso: \today)
	\end{itemize}
	\item \glossterm{ISO} / \glossterm{IEC} 9126:
	\begin{itemize}
		\item \href{https://it.wikipedia.org/wiki/ISO/IEC_9126}{\color{myblue}https://it.wikipedia.org/wiki/ISO/IEC\_9126} (Ultimo accesso: \today)
	\end{itemize}
	\item \glossterm{ISO} / \glossterm{IEC} 12207-1995:
	\begin{itemize}
		\item \href{https://www.math.unipd.it/~tullio/IS-1/2009/Approfondimenti/ISO_12207-1995.pdf}{\color{myblue}https://www.math.unipd.it/\textasciitilde{}tullio/IS-1/2009/Approfondimenti/ISO\_12207-1995.pdf} \\ (Ultimo accesso: \today)
	\end{itemize}
\end{itemize}
\clearpage
\section{Obiettivi di qualità}
Ogni \glossterm{processo} viene valutato tramite l'utilizzo di metriche specifiche, le cui definizioni sono dettagliate nelle sezioni Metriche di qualità del processo e Metriche di qualità del prodotto del documento \textit{Norme di Progetto vX.X.X}. Queste sezioni delineano i criteri che le metriche devono rispettare per essere valutate come accettabili o eccellenti. La sigla MPC sta ad indicare le metriche di processo.
% ricordarsi di verficare effettivamente il nome della sezione in NdP, nome in corsivo per riferimenti nel testo a documentazione del gruppo?
\subsection{Qualità di processo}
La base della \glossterm{qualità} del processo risiede nell'idea che, per ottenere un prodotto conforme a determinati standard di qualità, sia fondamentale sottoporre i processi che lo supportano a controlli regolari, al fine di ottimizzarli. Il concetto di qualità del processo viene quindi applicato a tutte le attività, pratiche e metodologie utilizzate lungo l'intero ciclo di vita del software. In breve, la qualità del processo mira a integrare la qualità nel prodotto stesso, garantendo che sia intrinseca al processo e non solo un obiettivo secondario.
\subsubsection{Processi primari}
\subsubsubsection{Fornitura} 
\begin{table}[h]	
	\centering
	\begin{tabular}{lccc}
		\toprule
		\textbf{Metrica}& \textbf{Descrizione} & \textbf{Valore accettabile} & \textbf{Valore ottimo} \\
		\midrule
		MCP-EV & Earned value (EV) & $\geq$ 0 & $\leq$ EAC \\\\
		MPC-PV & Planned Value (PV) & $\geq$ 0 & $\leq$ BAC\\\\
		MPC-AC & Actual costo (AC) & $\geq$ 0 & $\leq$ EAC\\\\
		MPC-CPI & Cost Performance Index (CPI) & tra 0.95 e 1.05 & $\leq$ 1\\\\
		MPC-EAC & Estimate At Completion (EAC) & deviazione del $\pm$ 5\% dal BAC & BAC\\\\
		MPC-ETC & Estimate To Completion (ETC) & $\geq $ 0 & $\leq$ EAC\\\\
		MPC-VAC & Variance At Completion (VAC) & deviazione del $\pm$ 10\% dal BAC & 0\%\\\\
		MPC-SV & Schedule Variance (SV) & deviazione del $\pm$ 10\% dal BAC & 0\%\\\\
		MPC-BV & Budget Variance (BV) & deviazione del $\pm$ 10\% dal BAC  & 0\%\\
		\bottomrule
		% Ruolo Redattore o Verificatore
	\end{tabular}
	\caption{Metriche per il processo di fornitura.}
	\label{table:Tabella metriche per il processo di fornitura.}
\end{table}
\clearpage
\subsubsubsection{Sviluppo}
\begin{table}[h]	
	\centering
	\begin{tabular}{lccc}
		\toprule
		\textbf{Metrica}& \textbf{Descrizione} & \textbf{Valore accettabile} & \textbf{Valore ottimo} \\
		\midrule
		MCP-RSI & Requirements stability index (RSI) & $\geq $ 75\%  & $\leq$ 100\% \\\\
		MPC-SFIN & Structural Fan-In (SFIN) & - & Va massimizzato\\\\
		MPC-SFOUT & Structural Fan-Out (SFOUT) & - & Va minimizzato\\
		\bottomrule
		% Ruolo Redattore o Verificatore
	\end{tabular}
	\caption{Valori accettabili e ottimi per ogni metrica riguardante il processo di sviluppo.}
	\label{table:Valori accettabili e ottimi per ogni metrica riguardante il processo di sviluppo.}
\end{table}
\subsubsection{Processi di supporto}
\subsubsubsection{Documentazione}
\begin{table}[h]	
	\centering
	\begin{tabular}{lccc}
		\toprule
		\textbf{Metrica}& \textbf{Descrizione} & \textbf{Valore accettabile} & \textbf{Valore ottimo} \\
		\midrule
		MPC-IG & Indice Gulpease (IGI) & $\geq$ 60\% & 100 \\\\
		MPC-CO & Correttezza Ortografica (CO) & 0 & 0 \\
		\bottomrule
		% Ruolo Redattore o Verificatore
	\end{tabular}
	\caption{Metriche per il processo di documentazione.}
	\label{table:Tabella delle metriche per il processo di documentazione}
\end{table}
\subsubsubsection{Verifica}
\begin{table}[h]	
	\centering
	\begin{tabular}{lccc}
		\toprule
		\textbf{Metrica}& \textbf{Descrizione} & \textbf{Valore accettabile} & \textbf{Valore ottimo} \\
		\midrule
		MPC-CC & Code Coverage (CC) & $\geq$ 80\% & 100\% \\\\
		MPC-PT & Passed Test cases percentage (PT) & 100\% & 100\% \\
		\bottomrule
	\end{tabular}
	\caption{Metriche per il processo di documentazione.}
	\label{table:Tabella delle metriche per il processo di documentazione}
\end{table}
\subsubsubsection{Gestione della qualità}
\begin{table}[H]	
	\centering
	\begin{tabular}{lccc}
		\toprule
		\textbf{Metrica}& \textbf{Descrizione} & \textbf{Valore accettabile} & \textbf{Valore ottimo} \\
		\midrule
		MPC-QMS & Quality Metrics Satisfied (QMS) & $\geq$ 85\%& 100\%\\
		\bottomrule
	\end{tabular}
	\caption{Valori accettabili e ottimi per ogni metrica riguardante il processo di gestione della qualità.}
	\label{table:Valori accettabili e ottimi per ogni metrica riguardante il processo di gestione della qualità.}
\end{table}
\clearpage
\subsubsection{Processi organizzativi}
\subsubsubsection{Gestione dei processi}
\begin{table}[H]	
	\centering
	\begin{tabular}{lccc}
		\toprule
		\textbf{Metrica}& \textbf{Descrizione} & \textbf{Valore accettabile} & \textbf{Valore ottimo} \\
		\midrule
		MPC-NR & Non-calculated Risk (NR) & $\leq$ 3 & 0\\\\
		MPC-ET & Efficienza temporale (ET) & $\leq$ 3 & $\leq$ 1 \\
		\bottomrule
	\end{tabular}
	\caption{Valori accettabili e ottimi per ogni metrica riguardante il processo di gestione dei processi.}
	\label{table:Valori accettabili e ottimi per ogni metrica riguardante il processo di gestione dei processi.}
\end{table}
\subsection{Qualità di prodotto}
Si riferisce alle caratteristiche di un'entità risultante dallo sviluppo software, che influenzano la sua capacità di soddisfare sia le esigenze esplicite che implicite. In altre parole, è quanto il prodotto si adatta alle aspettative del cliente o agli standard predefiniti. Questo implica una valutazione completa del software realizzato, concentrandosi su attributi come usabilità, \glossterm{funzionalità}, affidabilità e manutenibilità, oltre alle prestazioni generali. L'obiettivo è garantire che il software non solo soddisfi le richieste del cliente e funzioni correttamente, ma che lo faccia conformemente ai rigidi standard di \glossterm{qualità} stabiliti. Per raggiungere questo obiettivo, il team si impegna a seguire le metriche di prodotto, indicate con la sigla MPD, come specificato nel documento \textit{Norme di Progetto vX.X.X}.
% ricordarsi di verficare effettivamente il nome della sezione in NdP, nome in corsivo per riferimenti nel testo a documentazione del gruppo?
\subsubsection{Funzionalità}
\begin{table}[H]	
	\centering
	\begin{tabular}{lccc}
		\toprule
		\textbf{Metrica}& \textbf{Descrizione} & \textbf{Valore accettabile} & \textbf{Valore ottimo} \\
		\midrule
		MPD-ROS& Requisiti Obbligatori Soddisfatti (ROS) & 100\% & 100\%\\\\
		MPD-RDS & Requisiti Desiderabili Soddisfatti (RDS) & $\geq$ 0\% & $\geq$ 75\% \\\\
		MPD-ROPS & Requisiti Opzionali Soddisfatti (ROPS) & $\geq$ 0\% & $\geq$ 75\% \\
		\bottomrule
	\end{tabular}
	\caption{Valori accettabili e ottimi per ogni metrica riguardante la funzionalità del prodotto.}
	\label{table:Valori accettabili e ottimi per ogni metrica riguardante la funzionalità del prodotto.}
\end{table}
\subsubsection{Affidabilità}
\begin{table}[H]	
	\centering
	\begin{tabular}{lccc}
		\toprule
		\textbf{Metrica}& \textbf{Descrizione} & \textbf{Valore accettabile} & \textbf{Valore ottimo} \\
		\midrule
		MPD-BC & Branch Coverage (BC) & $\geq$ 80\% & 100\%\\\\
		MPD-SC & Statement Coverage (SC) & $\geq$ 80\% & 100\% \\\\
		MPD-FD & Failure Density (FD) & $\geq$ 80\% & 100\% \\\\
		MPD-PTCP & Passed Test Cases Percentage (PTCP) & 100\%  & 100\% \\
		\bottomrule
	\end{tabular}
	\caption{Valori accettabili e ottimi per ogni metrica riguardante l’affidabilità del prodotto.}
	\label{table:Valori accettabili e ottimi per ogni metrica riguardante l’affidabilità del prodotto.}
\end{table}
\subsubsection{Usabilità}
\begin{table}[H]	
	\centering
	\begin{tabular}{lccc}
		\toprule
		\textbf{Metrica}& \textbf{Descrizione} & \textbf{Valore accettabile} & \textbf{Valore ottimo} \\
		\midrule
		MPD-FU & Facilità di Utilizzo (FU) & $\geq$ 9 click & $\geq$ 5 click \\\\
		MPD-TA & Tempo di Apprendimento (TA) & $\leq$ 15 minuti & $\leq$ 5 minuti \\
		\bottomrule
	\end{tabular}
	\caption{Valori accettabili e ottimi per ogni metrica riguardante l’usabilità del prodotto.}
	\label{table:Valori accettabili e ottimi per ogni metrica riguardante l’usabilità del prodotto.}
\end{table}
\subsubsection{Efficienza}
\begin{table}[H]	
	\centering
	\begin{tabular}{lccc}
		\toprule
		\textbf{Metrica}& \textbf{Descrizione} & \textbf{Valore accettabile} & \textbf{Valore ottimo} \\
		\midrule
		MPD-UR & Utilizzo risorse (UR) & $\geq$ 75\% & 100\% \\
		\bottomrule
	\end{tabular}
	\caption{Valori accettabili e ottimi per ogni metrica riguardante l’efficienza del prodotto.}
	\label{table:Valori accettabili e ottimi per ogni metrica riguardante l’efficienza del prodotto.}
\end{table}
\subsubsection{Manutenibilità}
\begin{table}[H]	
	\centering
	\begin{tabular}{lccc}
		\toprule
		\textbf{Metrica}& \textbf{Descrizione} & \textbf{Valore accettabile} & \textbf{Valore ottimo} \\
		\midrule
		MPD-CC & Complessità Ciclomatica (CC) & 11-20 & 1-10 \\\\
		MPD-CS & Code Smell (CS) & 0 & 0 \\
		\bottomrule
	\end{tabular}
	\caption{Valori accettabili e ottimi per ogni metrica riguardante la manutenibilità del prodotto.}
	\label{table:Valori accettabili e ottimi per ogni metrica riguardante la manutenibilità del prodotto.}
\end{table}
\clearpage
\end{document}
