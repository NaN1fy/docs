\section{Capitolato scelto CX - ...}
\subsection{Descrizione}
\begin{itemize}
    \setlength\itemsep{0em}
    \item Proponente: 
    \item Committenti: prof. Vardanega e prof. Cardin
    \item Obiettivo:
\end{itemize}

\subsection{Dominio applicativo}

\subsection{Dominio tecnologico}

\subsection{Aspetti positivi}
\begin{itemize}
    \setlength\itemsep{0em}
    \item 
\end{itemize}

\subsection{Fattori critici}
\begin{itemize}
    \setlength\itemsep{0em}
    \item 
\end{itemize}

\subsection{Conclusioni}


\section{Capitolato CY - ...}
\subsection{Descrizione}
\begin{itemize}
    \setlength\itemsep{0em}
    \item Proponente: 
    \item Committenti: prof. Vardanega e prof. Cardin
    \item Obiettivo:
\end{itemize}

\subsection{Dominio applicativo}

\subsection{Dominio tecnologico}

\subsection{Aspetti positivi}
\begin{itemize}
    \setlength\itemsep{0em}
    \item 
\end{itemize}

\subsection{Fattori critici}
\begin{itemize}
    \setlength\itemsep{0em}
    \item 
\end{itemize}

\subsection{Conclusioni}


\section{Capitolato C9 - ChatSQL: creare frasi SQL da linguaggio naturale}
\subsection{Descrizione}
\begin{itemize}
    \setlength\itemsep{0em}
    \item Proponente: Zucchetti;
    \item Committenti: prof. Vardanega e prof. Cardin;
    \item Obiettivo: sviluppare un sistema di \textit{prompt} in grado di tradurre nel linguaggio SQL una frase scritta in linguaggio naturale.
\end{itemize}

\subsection{Dominio applicativo}
Il progetto consiste nello sviluppo di un'applicazione in grado di generare un \textit{prompt}, basandosi sulla struttura di una base di dati e l'interrogazione della stessa formulata in linguaggio naturale. Il risultato ottenuto potrà poi essere fornito ad un modello di LLM (\textit{Large Language Model}) per ricavare il codice SQL corrispondente all'interrogazione specificata. \\ Il proposito di questo applicativo è la guida dell'intelligenza artificiale nella produzione dell'output desiderato: per fare ciò è necessario formulare le richieste in maniera ottimale. 

\subsection{Dominio tecnologico}
La proponente non ha espresso preferenze vincolanti per quanto riguarda le modalità di archiviazione della struttura della base di dati, che può essere gestita mediante un'architettura su \textit{database} o documenti di testo strutturati (XML, JSON). \\ Viene lasciata ampia libertà di scelta delle tecnologie anche per lo sviluppo della maschera di richiesta dell'interrogazione in linguaggio naturale. \\ Infine, per la verifica del \textit{prompt} generato dall'applicazione è stato suggerito l'uso di ChatGPT o altri modelli reperibili nella raccolta \href{https://huggingface.co}{Hugging Face}.

\subsection{Aspetti positivi}
\begin{itemize}
    \setlength\itemsep{0em}
    \item L'applicativo facilita la scrittura di codice in linguaggio SQL e la gestione di basi di dati anche per utenti non esperti;
    \item L'obiettivo del progetto consente di ampliare le capacità di LLM ad operazioni avanzate come l'esecuzione di calcoli e confronti tra dati strutturati;
    \item La possibilità di lavorare con tecnologie innovative ed attuali come le intelligenze artificiali è apparsa molto stimolante.
\end{itemize}

\subsection{Fattori critici}
\begin{itemize}
    \setlength\itemsep{0em}
    \item La presentazione del capitolato è risultata poco accattivante ed incompleta per quanto concerne il supporto tecnologico offerto dall'azienda;
    \item L'utilizzo di tecnologie estremamente recenti ed in rapida evoluzione (le intelligenze artificiali) aumenta le possibilità che il software prodotto sia soggetto ad una repentina obsolescenza e goda di minore supporto;
    \item La qualità del progetto è notevolmente influenzata dal modello di intelligenza artificiale impiegato, pertanto i risultati possono talvolta rivelarsi inaccurati;
    \item Il capitolato ha suscitato un minor interesse rispetto alle altre proposte.
\end{itemize}

\subsection{Conclusioni}
Il capitolato offre l'opportunità di lavorare con tecnologie estremamente innovative ampliando il proprio bagaglio di conoscenze nell'ambito delle intelligenze artificiali. Tuttavia, nessun componente del gruppo ha dimostrato un particolare entusiasmo per la proposta e le criticità riscontrate sono risultate complessivamente prevalenti sulle note positive. La scelta è quindi ricaduta su capitolati considerati maggiormente stimolanti da parte del gruppo.