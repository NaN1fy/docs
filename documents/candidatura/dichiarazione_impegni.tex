% changelog: "1.0.0, 2024-03-21, Approvazione per candidatura"
\documentclass[8pt]{article}
\usepackage[italian]{babel}
\usepackage[utf8]{inputenc}
\usepackage[letterpaper, left=1in, right=1in, bottom=0.75in, top=0.75in]{geometry}
\usepackage{amsmath}
\usepackage{subfiles}
\usepackage{lipsum}
\usepackage{csquotes}
\usepackage{amsfonts}
\usepackage[sfdefault]{plex-sans}
\usepackage{float}
\usepackage{pifont}
\usepackage{mathabx}
\usepackage[euler]{textgreek}
\usepackage{makecell}
\usepackage{tikz}
\usepackage{wrapfig}
\usepackage{siunitx}
\usepackage{amssymb} 
\usepackage{tabularx}
\usepackage{adjustbox}
\usepackage[document]{ragged2e}
\usepackage{floatflt}
\usepackage[hidelinks]{hyperref}
\usepackage{graphicx}
\usepackage{hyperref}
\setcounter{tocdepth}{4}
\usepackage{caption}
\usepackage{multicol}
\usepackage{tikz}
\setlength\parindent{0pt}
\captionsetup{font=footnotesize}
\usepackage{fancyhdr} 
\usepackage{graphicx}
\usepackage{capt-of}% 
\usepackage{booktabs}
\usepackage{varwidth}
\usepackage{colortbl}

% -- TITOLO -- %
\newcommand{\customtitle}{PREVENTIVO COSTI E ASSUNZIONE IMPEGNI} % o ESTERNO

% -- PER LA FIRMA -- %
\newcommand{\signatureline}[1]{%
	 \par\vspace{0.5cm}
	\noindent\makebox[\linewidth][r]{\rule{0.2\textwidth}{0.5pt}\hspace{3cm}\makebox[0pt][r]{\vspace{3pt}\footnotesize #1}}%
}

% -- INTESTAZIONE -- %
\fancypagestyle{mystyle}{
	\fancyhf{} 
	\fancyhead[R]{\includegraphics[height=1cm]{../template/images/logos/NaN1fy_logo.png}} 
    \fancyhead[L]{\leftmark} 
   	\renewcommand{\headrulewidth}{1pt} 
  	\fancyhead[L]{\customtitle} 
	\renewcommand{\headsep}{1.3cm} 
	\fancyfoot[C]{\thepage} 
}

\begin{document}
\definecolor{myblue}{RGB}{23,103,162}
\begin{titlepage}
	\begin{tikzpicture}[remember picture, overlay]
		\node[anchor=south east, opacity=0.2, yshift = -4cm, xshift= 2em] at (current page.south east) {\includegraphics[width=0.7\textwidth, trim=0cm 0cm 5cm 0cm, clip]{../template/images/logos/Universita_Padova_transparent.png}}; 
		\node[anchor=north west, opacity=1, yshift = 4.2cm, xshift= 1.4cm, scale=1.6] at (current page.south west) {\includegraphics[width=4cm]{../template/images/logos/NaN1fy_logo.png}};
	\end{tikzpicture}
	
	\begin{minipage}[t]{0.47\textwidth}
		{\large{\textsc{Destinatari}}
			\vspace{3mm}
			\\ \large{\textsc{Prof. Tullio Vardanega}}
			\\ \large{\textsc{Prof. Riccardo Cardin}}
		}
	\end{minipage}
	\hfill
	\begin{minipage}[t]{0.47\textwidth}\raggedleft
		{\large{\textsc{Redattori}}
			\vspace{3mm}
			{\\\large{\textsc{Guglielmo Barison}\\}} % massimo due 
			
		}
		\vspace{8mm}
		
		{\large{\textsc{Verificatori}}
			\vspace{3mm}
			{\\\large{\textsc{Pietro Busato}\\}} % massimo due 
			{\large{\textsc{Davide Donanzan}}}
			
		}
		\vspace{4mm}\vspace{4mm}
	\end{minipage}
	\vspace{4cm}
	\begin{center}
		\begin{flushright}
			{\fontsize{30pt}{52pt}\selectfont \textbf{Preventivo Costi \\e Assunzione Impegni\\}} % o ESTERNO
		\end{flushright}
		\vspace{3cm}
	\end{center}
	\vspace{8.5 cm}
	{\small \textsc{\href{mailto: nan1fyteam.unipd@gmail.com}{nan1fyteam.unipd@gmail.com}}}
\end{titlepage}
\pagestyle{mystyle}
\section*{Registro delle Modifiche}
\begin{table}[ht!]	
	\centering
	\begin{tabular}{p{1.2cm} p{2cm} p{6cm} p{3cm} p{2cm}}
		\toprule
		\textbf{Versione}& \textbf{Data} & \textbf{Descrizione} & \textbf{Autore} & \textbf{Ruolo} \\
		\midrule
		1.0.0 & 2024-03-21 & \textbf{Approvazione per candidatura} & &  \\\\
		0.0.1 & 2024-03-17 & Verifica completa, correzioni marginali  & Pietro Busato, Davide Donanzan & Verificatore \\\\
		0.0.0 & 2024-03-17 & Stesura del documento.  & Guglielmo Barison & Redattore \\
		\bottomrule
		% Ruolo Redattore o Verificatore
	\end{tabular}
	\caption{Registro delle modifiche.}
	\label{table:Registro delle modifiche}
\end{table}
\newpage
\tableofcontents
\clearpage
\newpage
\justifying
\section{Impegni orari}
Con la stesura del seguente documento, ciascun componente del gruppo NaN1fy si impegna a lavorare sul progetto \textbf{SyncCity: Smart city monitoring platform} proposto dall'azienda \textbf{Sync Lab}, contribuendo con un totale di 95 ore produttive e ricoprendo ogni ruolo previsto per un numero significativo di ore, al fine di garantire una distribuzione equa del lavoro all'interno del gruppo.
\vspace{1em}
\\Si riporta quindi il costo orario per ruolo, deciso dal gruppo:
% Definire un nuovo tipo di colonna che centra il testo
\newcolumntype{C}[1]{>{\centering\arraybackslash}p{#1}}
\begin{table}[ht!]   
	\centering
	\begin{tabular}{C{2.7cm} C{2.7cm} C{2.7cm} C{3.2cm}}
		\toprule
		\textbf{Ruoli} & 
		\textbf{Costo orario} \textbf{(\texteuro}\,\textbf{h\textsuperscript{-1})} & 
		\textbf{Ore per ruolo} & 
		\textbf{Ore per membro}\\
		\midrule
		Amministratore & 30 & 48 & 8 \\\\
		Responsabile & 20 & 48 & 8 \\\\
		Progettista & 25 & 120 & 20 \\\\
		Analista & 25 & 72 & 12 \\\\
		Programmatore & 15 & 144 & 24 \\\\
		Verificatore & 15 & 138 & 23 \\
		\bottomrule
		\\& \cellcolor{myblue!0} Totale costo (\texteuro) \textbf{11430} & Totale ore (h) \hspace{1em} 570 & Tot. ore membro (h) 95 
	\end{tabular}
	\caption{Suddivisione ruoli e costi.}
	\label{table:suddivisioneRuoli}
\end{table}

\begin{flushleft}
	Viene di seguito presentata la distribuzione del monte ore e dei ruoli nel dettaglio per membro:
\end{flushleft}

\begin{table}[ht!]   
	\centering
	\begin{tabular}{C{3cm} C{1cm} C{1cm} C{1cm} C{1cm} C{1cm} C{1cm}}
		\toprule
		\textbf{Membro} & 
		\textbf{Amm.} & 
		\textbf{Res.} & 
		\textbf{Prj.} &
		\textbf{Ana.} &
		\textbf{Prg.} &
		\textbf{Ver.}\\
		\midrule
		Linda Barbiero & 7 & 8  & 20  & 13 & 24  & 23 \\\\ 
		Guglielmo Barison & 7 & 9 & 20 & 12 & 24 & 23 \\\\ 
		Pietro Busato & 9 & 8 & 20 & 12 & 23 & 23 \\\\ 
		Davide Donanzan & 9 & 8 & 20 & 12 & 24 & 22 \\\\ 
		Oscar Konieczny & 8 & 7 & 20 & 12 & 24 & 24 \\\\ 
		Veronica Tecchiati & 8 & 8 & 20 & 11 & 25 & 23 \\\\ 
		\bottomrule
	\end{tabular}
	\caption{Suddivisione ore per membro.}
	\label{table:suddivisioneRuoliMembri}
\end{table}
\newpage

\section{Suddivisione ruoli}
I ruoli sono stati distribuiti equamente tra tutti i componenti del gruppo e saranno:
\begin{itemize}
	\item \textbf{Programmatore:} Responsabile della scrittura del codice seguendo le specifiche del progetto e traducendo i requisiti in un'applicazione funzionante. Valutato che il capitolato non richiede uno specifico ed estensivo utilizzo di codice
	ma altresì esige una buona progettazione, competenza nell'ambito delle tecnologie usate e nella loro integrazione: la decisione finale è stata quella di ritenere meno impattante il ruolo del programmatore e in quanto tale ne abbiamo diminuito le ore totali;
	\item \textbf{Progettista:} Definisce l'architettura del software, pianifica la sua struttura e l'organizzazione nel dettaglio. Dopo doverose considerazioni sul capitolato scelto e le sue specifiche, riteniamo che questo ruolo svolga una funzione cruciale per la buona riuscita del progetto;
	\item \textbf{Verificatore:} Garantisce la qualità del software eseguendo test e controlli per assicurare il corretto funzionamento e il rispetto degli standard di qualità. Tenendo a mente le considerazioni sopracitate e valutando la trasversalità del ruolo, ma altresì rammentando le motivazioni dietro la scelta espressa riguardo al ruolo del Programmatore riteniamo corretta la scelta di assegnarli un monte ore discreto e non eccessivo;
	\item \textbf{Analista:} Si concentra sull'analisi dei requisiti, aiuta a definire le funzionalità del software e si assicura di comprendere i bisogni del cliente. Ritenendo il ruolo del Progettista indispensabile per il buon esito del progetto, dato che una solida analisi dei requisiti funge da base per una solida progettazione, riteniamo sia altrettanto necessario assegnare al ruolo dell'Analista un ampio quantitativo di ore, al fine di assicurare una corretta analisi del requisiti;
	\item \textbf{Responsabile del progetto:} Guida il team nel rispetto delle scadenze, nell'allocazione delle risorse e nella pianificazione generale, assicurando che il progetto proceda in modo efficiente e soddisfi gli obiettivi. Riteniamo che le attività svolte dal responsabile siano soddisfacibili in un quantitativo di ore certamente inferiore rispetto a quelle degli altri ruoli: siffatta ragione ci porta alla conseguente diminuzione delle ore attribuite a tale ruolo rispetto all'ammontare orario medio;
	\item \textbf{Amministratore:} Gestisce l'infrastruttura e le risorse necessarie per il progetto, inclusi gli strumenti e le tecnologie che definiscono il modo di lavorare del team. Il ragionamento di cui sopra si può applicare anche al ruolo dell'Amministratore: giocoforza l'esito rimane il medesimo.	
\end{itemize}
\newpage
\section{Preventivo costi}
Il costo totale stimato per il progetto è di 11 .430,00\;\texteuro. Questa stima è stata calcolata in linea con la pianificazione fornita. Tale previsione considera ogni fase del progetto e riflette l'impegno per garantire un'implementazione del software accurata e di qualità, rispettando al contempo i vincoli finanziari stabiliti. 
\section{Scadenza consegna}
Il gruppo prevede di consegnare il prodotto finito relativo al capitolato C6, \textbf{SyncCity: Smart city monitoring platform} proposto dall'azienda \textbf{Sync Lab}, entro il 2024-09-06.
\end{document}
