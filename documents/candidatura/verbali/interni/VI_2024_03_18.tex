% changelog: "1.0.0, 2024-03-21, Approvazione per candidatura"
\documentclass[8pt]{article}
\usepackage[italian]{babel}
\usepackage[utf8]{inputenc}
\usepackage[letterpaper, left=1in, right=1in, bottom=0.75in, top=0.75in]{geometry}
\usepackage{amsmath}
\usepackage{subfiles}
\usepackage{lipsum}
\usepackage{csquotes}
\usepackage{amsfonts}
\usepackage[sfdefault]{plex-sans}
\usepackage{float}
\usepackage{pifont}
\usepackage{mathabx}
\usepackage[euler]{textgreek}
\usepackage{makecell}
\usepackage{tikz}
\usepackage{wrapfig}
\usepackage{siunitx}
\usepackage{amssymb} 
\usepackage{tabularx}
\usepackage{adjustbox}
\usepackage[document]{ragged2e}
\usepackage{floatflt}
\usepackage[hidelinks]{hyperref}
\usepackage{graphicx}
\usepackage{hyperref}
\setcounter{tocdepth}{4}
\usepackage{caption}
\usepackage{multicol}
\usepackage{tikz}
\setlength\parindent{0pt}
\captionsetup{font=footnotesize}
\usepackage{fancyhdr} 
\usepackage{graphicx}
\usepackage{capt-of}% 
\usepackage{booktabs}
\usepackage{varwidth}

% -- TITOLO -- %
\newcommand{\customtitle}{VERBALE INTERNO DEL 2024-03-18} % o ESTERNO

% -- PER LA FIRMA -- %
\newcommand{\signatureline}[1]{%
	 \par\vspace{0.5cm}
	\noindent\makebox[\linewidth][r]{\rule{0.2\textwidth}{0.5pt}\hspace{3cm}\makebox[0pt][r]{\vspace{3pt}\footnotesize #1}}%
}

% -- INTESTAZIONE -- %
\fancypagestyle{mystyle}{
	\fancyhf{} 
	\fancyhead[R]{\includegraphics[height=1cm]{../../../template/images/logos/NaN1fy_logo.png}} 
    \fancyhead[L]{\leftmark} 
   	\renewcommand{\headrulewidth}{1pt} 
  	\fancyhead[L]{\customtitle} 
	\renewcommand{\headsep}{1.3cm} 
	\fancyfoot[C]{\thepage} 
}

\begin{document}
\definecolor{myblue}{RGB}{23,103,162}
\begin{titlepage}
	\begin{tikzpicture}[remember picture, overlay]
		\node[anchor=south east, opacity=0.2, yshift = -4cm, xshift= 2em] at (current page.south east) {\includegraphics[width=0.7\textwidth, trim=0cm 0cm 5cm 0cm, clip]{../../../template/images/logos/Universita_Padova_transparent.png}}; 
		\node[anchor=north west, opacity=1, yshift = 4.2cm, xshift= 1.4cm, scale=1.6] at (current page.south west) {\includegraphics[width=4cm]{../../../template/images/logos/NaN1fy_logo.png}};
	\end{tikzpicture}
	
	\begin{minipage}[t]{0.47\textwidth}
		{\large{\textsc{Destinatari}}
			\vspace{3mm}
			\\ \large{\textsc{Prof. Tullio Vardanega}}
			\\ \large{\textsc{Prof. Riccardo Cardin}}
		}
	\end{minipage}
	\hfill
	\begin{minipage}[t]{0.47\textwidth}\raggedleft
		{\large{\textsc{Redattori}}
			\vspace{3mm}
			{\\\large{\textsc{Oscar Konieczny}\\}} % massimo due 
			
			
		}
		\vspace{8mm}
		
		{\large{\textsc{Verificatori}}
			\vspace{3mm}
			{\\\large{\textsc{Linda Barbiero}\\}} % massimo due 
			{\large{\textsc{Veronica Tecchiati}}}
			
		}
		\vspace{4mm}\vspace{4mm}
	\end{minipage}
	\vspace{4cm}
	\begin{center}
		\begin{flushright}
			{\fontsize{30pt}{52pt}\selectfont \textbf{Verbale Interno Del\\2024-03-18\\}} % o ESTERNO
		\end{flushright}
		\vspace{3cm}
	\end{center}
	\vspace{8.5 cm}
	{\small \textsc{\href{mailto: nan1fyteam.unipd@gmail.com}{nan1fyteam.unipd@gmail.com}}}
\end{titlepage}
\pagestyle{mystyle}
\section*{Registro delle Modifiche}
\begin{table}[ht!]	
	\centering
	\begin{tabular}{p{1.2cm} p{2cm} p{6cm} p{3cm} p{2cm}}
		\toprule
		\textbf{Versione}& \textbf{Data} & \textbf{Descrizione} & \textbf{Autore} & \textbf{Ruolo} \\
		\midrule
		% \\\\ % spazio tra le righe
		1.0.0 & 2024-03-21 & \textbf{Approvazione per candidatura} & & \\\\
		0.0.1 & 2024-03-20 & Verifica completa & Veronica Tecchiati & Verificatore \\\\
		0.0.1 & 2024-03-19 & Verifica completa, correzioni marginali & Linda Barbiero & Verificatore \\\\
		0.0.0 & 2024-03-19 & Stesura del verbale & Oscar Konieczny & Redattore \\
		\bottomrule
		% Ruolo Redattore o Verificatore
	\end{tabular}
	\caption{Registro delle modifiche.}
	\label{table:Registro delle modifiche}
\end{table}
\newpage
\tableofcontents
\clearpage
\newpage
\justifying
\section{Contenuti del Verbale}
\subsection{Informazioni sulla riunione}
\begin{itemize}
	\setlength\itemsep{0em}
	\item\textbf{Luogo:} Chiamata Discord;
	\item\textbf{Ora di inizio:} 15:00;
	\item\textbf{Ora di fine:}  17:30.
\end{itemize}
\begin{table}[ht!]
	\begin{minipage}[t]{0.5\linewidth}
		\centering
		\begin{tabular}{p{3cm} p{3cm}}
			\toprule
			\textbf{Partecipante} & \textbf{Durata presenza} \\
			\midrule
			Guglielmo Barison & 2.5 h \\
			Linda Barbiero &  2.5 h \\
			Pietro Busato & 2.3 h \\
			Oscar Konieczny & 2.5 h \\
			Davide Donanzan & 2.5 h \\
			Veronica Tecchiati & 2.5 h \\
			\bottomrule
		\end{tabular}
		\caption{Partecipanti NaN1fy.}
		\label{table:Partecipanti NaN1fy}
	\end{minipage}
\end{table}
\subsection{Revisione verbale precedente}
Si evidenzia che alcune delle attività delineate nel verbale precedente non sono ancora state portate a termine.
Si intende continuare ad aderire alle decisioni precedentemente concordate e ci si impegna a completare le attività rimanenti nel minor tempo possibile, entro e non oltre la consegna della candidatura.
\subsection{Ordine del giorno}
\begin{itemize}
	\setlength\itemsep{0em}
	\item Scelta del capitolato finale per cui presentarsi;
	\item Spartizione di tutti i compiti da svolgere entro la candidatura;
	\item Allineamento sulla stesura dei documenti.
\end{itemize}
\subsection{Sintesi dell'incontro}
Si è iniziata la riunione con la decisione del capitolato da portare in candidatura, ogni membro ha posto le sue preferenze e motivazioni, arrivando così ad una conclusione chiara. In seguito si è passati a verificare gli obiettivi da svolgere per la candidatura, andando ad assegnare i verificatori per tutti i documenti. Successivamente, si è dedicato del tempo per la risoluzione di dubbi riguardanti \LaTeX\: da parte di alcuni membri, cogliendo anche l'occasione per prendere decisioni sul versionamento dei documenti. Si sono poi discussi i contenuti da scrivere nei documenti per la candidatura, in particolare la valutazione dei capitolati, oltre a dedicare del tempo per calcolare il preventivo dei costi e la scadenza della consegna. Infine, ci si è organizzati per la revisione finale dei documenti in vista della candidatura.
\subsection{Decisioni prese}
\begin{itemize}
	\setlength\itemsep{0em}
	\item Viene scelto il capitolato \texttt{SyncCity} (\texttt{C6}) per la candidatura;
	\item Assegnati verificatori per ogni documento creato;
	\item Definito il versionamento dei documenti, con \texttt{0.0.0} come versione di partenza;
	\item Calcolato il preventivo dei costi e la data di consegna prevista;
	\item Organizzata la revisione finale dei documenti in prossimità alla candidatura.
	\\\\
	% consigliata la forma \textit{Viene adottato} quando viene adottato un certo modo di fare/strumento
	% per nomi di aziene e capitolati usare \texttt{}, e.g \texttt{Easy meal} \texttt{C6} 
\end{itemize}
\section{Attività da svolgere}
Non sono state stabilite ulteriori attività da realizzare.
%\begin{table}[ht!]
%	\centering
%	\begin{tabular}{lcl}
	%		\toprule
	%		\textbf{Titolo} & \textbf{\# Issue} & \textbf{Verificatori} \\
	%		\midrule
	%		Stesura lettera di presentazione & 8 & Pietro Busato, Linda Barbiero\\\\
	%		Stesura della dichiarazione impegni  &  6 & Davide Donanzan, Pietro Busato \\\\
	%		Stesura della valutazione dei capitolati & 7 & Davide Donanzan, Oscar Konieczny \\
	%		\bottomrule
	%	\end{tabular}
%	\caption{Attività da svolgere}
%	\label{table:Attivita da svolgere}
%\end{table}

% togli il commento per la firma
%\signatureline{Padova, YYYY-MM-DD}
\end{document}
