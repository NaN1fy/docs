\section{Contenuti del Verbale}
\subsection{Informazioni sulla riunione}
\begin{itemize}
	\setlength\itemsep{0em}
	\item\textbf{Luogo:} Chiamata Discord;
	\item\textbf{Ora di inizio:} 16:00;
	\item\textbf{Ora di fine:}  18:30;
\end{itemize}
\begin{table}[ht!]
	\begin{tabular}{p{3cm} p{3cm}}
		\toprule
		\textbf{Partecipante} & \textbf{Durata presenza} \\
		\midrule
		Guglielmo Barison & 2.5 h \\
		Linda Barbiero &  2.5 h \\
		Pietro Busato & 2.5 h \\
		Oscar Konieczny & 2.5 h \\
		Davide Donanzan & 2.5 h \\
		Veronica Tecchiati & 2.5 h \\
		\bottomrule
	\end{tabular}
	\caption{Informazioni sulla partecipazione}
	\label{table:Informazioni sulla partecipazione}
\end{table}
\subsection{Ordine del giorno}
\begin{itemize}
	\setlength\itemsep{0em}
	\item Scelta del nome e del logo del gruppo;	
	\item Confronto sui vari capitolati;
	\item Organizzazione riunioni interne future.
\end{itemize}
\subsection{Sintesi dell'incontro}
Durante la riunione, si è discusso dei nomi e dei loghi proposti da alcuni membri del gruppo. 
Successivamente, abbiamo proceduto ad esaminare i tre capitolati rimasti disponibili, esaminandoli uno per uno e discutendo i loro rispettivi aspetti positivi e negativi, nonché le loro caratteristiche. Ogni membro del gruppo ha espresso le proprie preferenze riguardo ai capitolati. \\
Inoltre, durante la riunione sono state discusse le questioni da approfondire, le quali saranno utilizzate per orientare i successivi colloqui esplorativi. Valutando le domande a cui non sono state fornite risposte nelle presentazioni dei capitolati, al fine di garantire una comprensione più completa dei progetti proposti.
Infine, il gruppo si è organizzato per stabilire quando e come affrontare altri argomenti tramite prossime riunioni interne, iniziando quindi a definire il proprio \textit{way of working} da utilizzare.
\subsection{Decisioni prese}
\begin{itemize}
	\setlength\itemsep{0em}
	\item \textit{Viene adottato} come nome del gruppo \texttt{NaN1fy} e il relativo logo;
	\item Abbiamo concluso, a votazione, che i capitolati di maggiore interesse del gruppo fossero \texttt{C6} e \texttt{C3};
	\item \textit{Viene adottato} \LaTeX\: per la stesura di documenti ufficiali;
	\item \textit{Viene adottato} Discord come strumento per lo svolgimento di riunioni interne;
	\item \textit{Viene adottato} Telegram come canale di messaggistica istantanea;
	\item Dato il fatto che il capitolato \texttt{SyncCity} dell’azienda \texttt{Sync Lab} è risultato essere il più interessante si è deciso di richiedere un incontro conoscitivo per porre domande sulla natura del progetto in data 2024-03-12;
	\item Dato il fatto che il capitolato \texttt{Easy meal} dell’azienda \texttt{Imola Informatica} ha suscitato interesse in alcuni membri del gruppo, si è deciso di richiedere un incontro conoscitivo sempre in data 2024-03-12;
	\item Definizione delle domande da porre per entrambi i capitolati selezionati durante gli incontri conoscitivi.
\end{itemize}
\section{Attività da svolgere}
\begin{table}[ht!]
	\centering
	\begin{tabular}{lcl}
		\toprule
		\textbf{Titolo} & \textbf{\# Issue} & \textbf{Verificatori} \\
		\midrule
		Stesura lettera di presentazione & 8 & Pietro Busato, Linda Barbiero\\\\
		Stesura della dichiarazione impegni  &  6 & Davide Donanzan, Pietro Busato \\\\
		Stesura della valutazione dei capitolati & 7 & Davide Donanzan, Oscar Konieczny \\
		\bottomrule
	\end{tabular}
	\caption{Attività da svolgere}
	\label{table:Attivita da svolgere}
\end{table}
% togli il commento per la firma
%\signatureline{Padova, YYYY-MM-DD}
%\signatureline{Padova, YYYY-MM-DD}


