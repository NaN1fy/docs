\section{Contenuti del Verbale}
\subsection{Informazioni sulla riunione}
\begin{itemize}
	\setlength\itemsep{0em}
	\item\textbf{Luogo:} Chiamata Discord;
	\item\textbf{Ora di inizio:} 15:00;
	\item\textbf{Ora di fine:}  17:30.
\end{itemize}
\begin{table}[ht!]
	\begin{minipage}[t]{0.5\linewidth}
		\centering
		\begin{tabular}{p{3cm} p{3cm}}
			\toprule
			\textbf{Partecipante} & \textbf{Durata presenza} \\
			\midrule
			Guglielmo Barison & 2.5 h \\
			Linda Barbiero &  2.5 h \\
			Pietro Busato & 2.3 h \\
			Oscar Konieczny & 2.5 h \\
			Davide Donanzan & 2.5 h \\
			Veronica Tecchiati & 2.5 h \\
			\bottomrule
		\end{tabular}
		\caption{Partecipanti NaN1fy.}
		\label{table:Partecipanti NaN1fy}
	\end{minipage}
\end{table}
\subsection{Revisione verbale precedente}
Si evidenzia che alcune delle attività delineate nel verbale precedente non sono ancora state portate a termine.
Si intende continuare ad aderire alle decisioni precedentemente concordate e ci si impegna a completare le attività rimanenti nel minor tempo possibile, entro e non oltre la consegna della candidatura.
\subsection{Ordine del giorno}
\begin{itemize}
	\setlength\itemsep{0em}
	\item Scelta del capitolato finale per cui presentarsi;
	\item Spartizione di tutti i compiti da svolgere entro la candidatura;
	\item Allineamento sulla stesura dei documenti.
\end{itemize}
\subsection{Sintesi dell'incontro}
Si è iniziata la riunione con la decisione del capitolato da portare in candidatura, ogni membro ha posto le sue preferenze e motivazioni, arrivando così ad una conclusione chiara. In seguito si è passati a verificare gli obiettivi da svolgere per la candidatura, andando ad assegnare i verificatori per tutti i documenti. Successivamente, si è dedicato del tempo per la risoluzione di dubbi a riguardo di \LaTeX\: da parte di alcuni membri, cogliendo anche l'occasione per prendere decisioni a riguardo del versionamento dei documenti. Deciso il versionamento, si sono discussi i contenuti da scrivere nei documenti per la candidatura, principalmente è stata discussa la valutazione dei capitolati, ma è stato dedicato del tempo per calcolare il preventivo dei costi e la scadenza della consegna. Infine, ci si è organizzati per una revisione finale in vista della candidatura.
\subsection{Decisioni prese}
\begin{itemize}
	\setlength\itemsep{0em}
	\item Viene scelto il capitolato \texttt{SyncCity} (\texttt{C6}) per la candidatura;
	\item Assegnati verificatori per ogni documento creato;
	\item Definito il versionamento dei documenti, con \texttt{0.0.0} come versione di partenza;
	\item Calcolato il preventivo dei costi e la data di consegna prevista;
	\item Organizzata la revisione finale dei documenti in prossimità alla candidatura.
	\\\\
	% consigliata la forma \textit{Viene adottato} quando viene adottato un certo modo di fare/strumento
	% per nomi di aziene e capitolati usare \texttt{}, e.g \texttt{Easy meal} \texttt{C6} 
\end{itemize}
\section{Attività da svolgere}
Non sono state stabilite ulteriori attività da realizzare.
%\begin{table}[ht!]
%	\centering
%	\begin{tabular}{lcl}
%		\toprule
%		\textbf{Titolo} & \textbf{\# Issue} & \textbf{Verificatori} \\
%		\midrule
%		Stesura lettera di presentazione & 8 & Pietro Busato, Linda Barbiero\\\\
%		Stesura della dichiarazione impegni  &  6 & Davide Donanzan, Pietro Busato \\\\
%		Stesura della valutazione dei capitolati & 7 & Davide Donanzan, Oscar Konieczny \\
%		\bottomrule
%	\end{tabular}
%	\caption{Attività da svolgere}
%	\label{table:Attivita da svolgere}
%\end{table}

% togli il commento per la firma
%\signatureline{Padova, YYYY-MM-DD}
