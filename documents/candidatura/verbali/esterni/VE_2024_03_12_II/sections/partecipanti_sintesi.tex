\section{Contenuti del Verbale}
\subsection{Informazioni generali}
\begin{itemize}
	\setlength\itemsep{0em}
	\item\textbf{Luogo:} Chiamata tramite Google Meet;
	\item\textbf{Ora di inizio:} 17:00;
	\item\textbf{Ora di fine:}  18:00.
\end{itemize}
\begin{table}[ht!]
	\begin{minipage}[t]{0.5\linewidth}
		\centering
		\begin{tabular}{p{3cm} p{3cm}}
			\toprule
			\textbf{Partecipante} & \textbf{Durata presenza} \\
			\midrule
			Guglielmo Barison & 1.0 h \\
			Linda Barbiero &  1.0 h \\
			Pietro Busato & 1.0 h \\
			Oscar Konieczny & 1.0 h \\
			Davide Donanzan & 1.0 h \\
			Veronica Tecchiati & 1.0 h \\
			\bottomrule
		\end{tabular}
		\caption{Partecipanti NaN1fy.}
		\label{table:Partecipanti NaN1fy}
	\end{minipage} 
	\begin{minipage}[t]{0.5\linewidth} % -- COMMENTA/DECOMMENTA DA QUI
		\centering
		\begin{tabular}{p{3cm} p{3cm}}
			\toprule
			\textbf{Partecipante} & \textbf{Durata presenza} \\
			\midrule
			Daniele Zorzi & 1.0 h \\
			\bottomrule
		\end{tabular}
		\caption{Partecipanti SyncLab.}
		\label{table:Partecipanti XXXX}
	\end{minipage} % -- A QUI PER TOGLIERE AGGIUNGERE
\end{table}

\subsection{Sintesi dell'incontro}
Primo incontro conoscitivo con l'azienda proponente \texttt{SyncLab}. Sono seguite da parte del gruppo domande tecniche ed organizzative a scopo delucidativo (riportate di seguito più nel dettaglio), quali chiarimenti sulle tecnologie da implementare nel progetto, sull'organizzazione di eventuali incontri futuri (formativi e di revisione) e sulle modalità di firma dei verbali esterni.  
 %-- togliere, testo per riempiere il template

\section{Quesiti esposti}
Di seguito sono elencate le domande poste al referente di \texttt{SyncLab}.

\subsection{Dati \textit{mock} generati tramite \textit{Python}: mole, frequenza e tipologia}
La scelta del tipo di sensore e la frequenza di generazione dei dati è lasciata al fornitore, purché sia coerente e ragionevole. Ad esempio, per un sensore di temperatura si può generare un dato ogni minuto o ogni 10 minuti, con valori realistici. Per un sensore di parcheggio, invece, basta un semplice bit (0 occupato, 1 libero) che si aggiorna ogni 5-10 minuti. È libera anche la scelta del linguaggio utilizzato per generare i dati, bensì sia consigliato Python, in quanto presenta molte librerie utili a tale scopo.

\subsection{Quantità e tipologia di sensori}
Mentre per il \textit{PoC} è sufficiente simulare un paio di sensori, per il progetto completo è auspicabile l'implementazione di tutti quelli elencati al punto 1.3 della presentazione del capitolato.

\subsection{Strumenti forniti dall'azienda messi a nostra disposizione}
La proponente \texttt{SyncLab} non ritiene necessario per il progetto l'esposizione di server propri od eventuali repository proprietarie. Lo sviluppo del prodotto può essere eseguito in ambiente locale con codice condiviso su repository pubblica (es. \textit{Github}).

\subsection{Qualità dei dati ed eventuali rielaborazioni}
È preferibile non aggiungere dati sporchi, eventualmente si potrà riconsiderare questa decisione per la fase di test; eventuali statistiche e analisi dei dati sono consigliati dopo la consegna minima (\textit{PoC}).

\subsection{Altre tecnologie suggerite come possibile alternativa a quelle già proposte}
Per quanto riguarda la generazione dei dati, come già detto, è possibile usare anche altri linguaggi, e lo stesso vale per la rappresentazione tramite \textit{Grafana}. Le eventuali alternative trovate devono essere adeguatamente giustificate. \textit{Kafka} e \textit{ClickHouse} sono invece imprescindibili, dato che rappresentano l'aspetto più importante di questo progetto.

\subsection{Incontri formativi: organizzazione, modalità di erogazione, temi trattati}
Gli incontri formativi (o \textit{deep dive}), potranno essere pianificati una volta avviato il progetto, su richiesta del fornitore o meno, con almeno una settimana di preavviso. Avranno una durata di circa un'ora e seguiranno una struttura simile a quella delle \textit{flipped classroom}. Dopo che il gruppo si sarà autonomamente documentato sulle tecnologie necessarie, la proponente risponderà ad eventuali domande e si approfondiranno i temi trattati. I meeting introduttivi e pochi altri che saranno invece gestiti diversamente.

\subsection{Organizzazione degli incontri non formativi}
Verranno fissati dei \textit{SAL} ogni una o due settimane, su \textit{Google Meet}, in cui si discuterà lo stato di avanzamento del progetto e saranno fissate le successive \textit{milestone} da raggiungere per il \textit{SAL} seguente. I \textit{SAL} verteranno su:
\begin{itemize}
    \setlength\itemsep{0em}
    \item Generare dati da sensori;
    \item \textit{Kafka};
    \item Collegare \textit{Kafka} a \textit{ClickHouse} tramite connettori;
    \item Spostare e mostrare i dati da \textit{ClickHouse} a \textit{Grafana}.
\end{itemize}
Se alcune \textit{milestone} dovessero essere completate in anticipo si può proseguire immediatamente con le successive. Se dovessero verificarsi invece dei ritardi rispetto alla consegna prevista, la proponente si è mostrata estremamente disponibile ad aiutare il gruppo a risolvere le eventuali problematiche. È fondamentale, in entrambi i casi, avvisare preventivamente l'azienda, preferibilmente con almeno una settimana di anticipo.

\subsection{Preferenze sulle modalità di condivisione del lavoro svolto}
All'azienda interessa principalmente verificare che il lavoro soddisfi i requisiti prefissati, quindi basta condividere la repository su \textit{Github} o mostrare l'esecuzione locale del programma.

\subsection{Criteri di completamento del progetto: rapporto tra "consegna completa" menzionata nella presentazione del capitolato, \textit{MVP} e \textit{CA} indicati dal prof. Vardanega}
Il \textit{PoC} viene inteso come visualizzazione minima di almeno 1/2 sensori, poi diventerà la base per procedere con la parte successiva, in cui implementare tutti gli altri sensori.

\subsection{Firma dei verbali esterni}
Ad ogni \textit{SAL}, o comunque ad ogni incontro su cui verrà poi steso un verbale, basterà inviare quest'ultimo con data e luogo della firma. L'azienda procederà quindi a riconsegnarlo firmato.

\section{Conclusioni}
Essendo questo il primo incontro puramente informativo con l'azienda proponente \texttt{SyncLab}, non sono stati ancora definiti obiettivi futuri precisi. Si è concordato di organizzare un nuovo incontro in seguito per stabilire le prime \textit{milestone}, nel caso in cui il gruppo si candidi per il progetto e ottenga effettivamente l'appalto.\\\\
	% consigliata la forma \textit{Viene adottato} quando viene adottato un certo modo di fare/strumento
	% per nomi di aziende e capitolati usare \texttt{}, e.g \texttt{Easy meal} \texttt{C6} 

% togli il commento per la firma
\signatureline{Padova, 2024-03-22}
%\signatureline{Padova, YYYY-MM-DD}


