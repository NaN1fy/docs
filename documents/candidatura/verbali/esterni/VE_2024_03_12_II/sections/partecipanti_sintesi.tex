\section{Contenuti del Verbale}
\subsection{Informazioni sulla riunione}
\begin{itemize}
	\setlength\itemsep{0em}
	\item\textbf{Luogo:} Chiamata tramite Google Meet;
	\item\textbf{Ora di inizio:} 17:00;
	\item\textbf{Ora di fine:}  18:00.
\end{itemize}
\begin{table}[ht!]
	\begin{minipage}[t]{0.5\linewidth}
		\centering
		\begin{tabular}{p{3cm} p{3cm}}
			\toprule
			\textbf{Partecipante} & \textbf{Durata presenza} \\
			\midrule
			Guglielmo Barison & 1.0 h \\
			Linda Barbiero &  1.0 h \\
			Pietro Busato & 1.0 h \\
			Oscar Konieczny & 1.0 h \\
			Davide Donanzan & 1.0 h \\
			Veronica Tecchiati & 1.0 h \\
			\bottomrule
		\end{tabular}
		\caption{Partecipanti NaN1fy.}
		\label{table:Partecipanti NaN1fy}
	\end{minipage} 
	\begin{minipage}[t]{0.5\linewidth} % -- COMMENTA/DECOMMENTA DA QUI
		\centering
		\begin{tabular}{p{3cm} p{3cm}}
			\toprule
			\textbf{Partecipante} & \textbf{Durata presenza} \\
			\midrule
			Daniele Zorzi & 1.0 h \\
			\bottomrule
		\end{tabular}
		\caption{Partecipanti SyncLab.}
		\label{table:Partecipanti XXXX}
	\end{minipage} % -- A QUI PER TOGLIERE AGGIUNGERE
\end{table}

\subsection{Sintesi dell'incontro}
 Primo incontro formativo con l'azienda proponente \texttt{SyncLab}, in cui ci si è presentati a vicenda. Sono seguite da parte nostra domande di ambito tecnico/organizzativo a scopo delucidativo (riportate di seguito più in dettaglio), quali chiarimenti sulle tecnologie da implementare nel progetto, organizzazione di eventuali incontri futuri (didattici e di revisione) e firma dei verbali esterni.  
 %-- togliere, testo per riempiere il template

\section{Ordine del giorno}
Di seguito sono elencate le domande poste al relatore di \texttt{SyncLab}.
\subsection{Dati \textit{mock} generati tramite \textit{Python}: mole, frequenza e tipologia.}
La scelta del tipo di sensore e la frequenza di generazione dei dati è lasciata a voi, purché sia coerente e ragionevole. Ad esempio, per un sensore di temperatura si può generare un dato ogni minuto o ogni 10 minuti, con valori realistici. Per un sensore di parcheggio, invece, basta un semplice bit (0 occupato, 1 libero) che si aggiorna ogni 5-10 minuti. È libera anche la scelta del linguaggio da utilizzare per generare i dati, ma noi consigliamo Python, in quanto presenta molte librerie utili a tale scopo.
\subsection{E per quanto riguarda la quantitá e il tipo di sensori?}
Mentre per il \textit{PoC} basta averne un paio, per il progetto completo vorrremmo vedere implementati tutti quelli listati al punto 1.3 [della presentazione capitolato. n.d.r.].
\subsection{Strumenti forniti dall'azienda messi a nostra disposizione: strumenti utilizzabili anche da remoto, server con relativa modalità di accesso}
A noi interessa più la connessione e l'integrazione tra le parti che la codifica, per cui non servono server, database, accessi da remoto o altri strumenti del genere, si può fare tutto in locale.
\subsection{In riferimento a 1.7.3 [della presentazione capitolato. n.d.r], è sufficiente presentare i dati (con \textit{Grafana}) o serve anche rielaborarli, ad esempio con statistiche e/o pulizia degli stessi?}
Va da sè che bisogna generare dati con sensatezza e consapevolezza, quindi sarebbe preferibile non aggiungere dati sporchi, nel caso se ne può discutere più avanti per i test; eventuali statistiche/analisi dei dati sono consigliati dopo la consegna minima (\textit{PoC}).
\subsection{Vi sono altre tecnologie che potete consigliare come eventuale alternativa a quelle già proposte?}
Per quanto riguarda la generazione dati, come già detto, siete liberi di usare anche altri linguaggi, e lo stesso vale per la rappresentazione tramite \textit{Grafana}, potete scegliere quello che preferite, se trovate valide alternative e le giustificate. Per quanto riguarda invece \textit{Kafka} e \textit{ClickHouse}, dato che rappresentano l'aspetto più importante di questo progetto, siete tenuti ad utilizzarli.
\subsection{Incontri informativi, organizzazione, modalità di erogazione, temi trattati.}
Per quanto riguarda gli incontri formativi (o \textit{deep dive}), una volta avviato il progetto, ci coordineremo per pianificarli, su richiesta o meno, con almeno una settimana di preavviso. Avranno una durata di circa un'ora e seguiranno una struttura simile a quella delle \textit{flipped classroom}. Dopo che vi sarete documentati autonomamente sulle tecnologie necessarie, risponderemo alle vostre domande e approfondiremo i temi trattati, eccetto per i meeting introduttivi e pochi altri che saranno gestiti in modo diverso.
\subsection{E per quanto riguarda eventuai altri incontri, come saranno organizzati?}
Verranno fissati dei \textit{SAL}, ogni una, due settimane, su \textit{Google Meet}, in cui discuteremo dello stato di avanzamento del progetto e fisseremo le successive \textit{milestone} da raggiungere per il \textit{SAL} seguente. I \textit{SAL} saranno principalmente su:
\begin{itemize}
    \setlength\itemsep{0em}
    \item generare dati da sensori;
    \item \textit{Kafka};
    \item collegare \textit{Kafka} a \textit{ClickHouse} tramite connettori;
    \item spostare e mostrare i dati da \textit{ClickHouse} a \textit{Grafana}.
\end{itemize}
Se per caso completate alcune \textit{milestone} in anticipo, non c'è problema, potete iniziare immediatamente con quella successiva. Allo stesso modo, se siete in ritardo rispetto alla consegna, non dovete preoccuparvi, può capitare. È fondamentale in entrambi i casi che ci avvisiate per tempo, preferibilmente con almeno una settimana di anticipo.
\subsection{Avete preferenze su altre modalità di controllo dello stato di avanzamento del progetto? (\textit{Github}, incontri, revisione asincrona, ...).}
A noi interessa principalmente controllare che il lavoro svolto soddisfi i requisiti prefissati, quindi basta che condividiate la repository su \textit{Github} o ci mostriate una macchina con il lavoro svolto.
\subsection{In riferimento a 1.6 [della presentazione capitolato. n.d.r.], cosa intendete con consegna minima (\textit{PoC}) e consegna completa? Coincidono rispettivamente con \textit{PoC} e \textit{MVP} indicati dal Prof. Vardanega? La consegna completa corrisponde alla nostra \textit{CA} (opzionale)?}
Il \textit{PoC} viene inteso come visualizzazione minima di almeno 1/2 sensori, poi diventerà la base per procedere con la parte successiva, in cui  inserirete tutti gli altri sensori.\\\\
\emph{Nota: durante l'incontro si é visto da entrambe le parti che c'erano state alcune incomprensioni riguardo alla suddivisione del progetto pensata dall'azienda e quella strutturata invece dal Prof. Vardanega; si é quindi deciso, insieme al relatore di SyncLab, di discuterne meglio in eventuali riunioni future}.
\subsection{Come verrá organizzata invece il processo per la firma digitale?}
Ad ogni \textit{SAL} o comunque ad ogni incontro su cui verrà poi steso un verbale, basterà che ci inviate quest'ultimo, con data e luogo della firma, e noi ve lo riconsegneremo firmato.

\section{Decisioni prese}
Essendo questo il primo incontro puramente informativo con l'azienda proponente \texttt{SyncLab}, non sono stati ancora definiti obiettivi futuri precisi. Si è concordato di organizzare un nuovo incontro in seguito per stabilire le prime \textit{milestone}, nel caso in cui il nostro gruppo si candidi per il progetto e ottenga effettivamente l'appalto.\\\\
	% consigliata la forma \textit{Viene adottato} quando viene adottato un certo modo di fare/strumento
	% per nomi di aziene e capitolati usare \texttt{}, e.g \texttt{Easy meal} \texttt{C6} 

% togli il commento per la firma
\signatureline{Padova, 2024-03-12}
%\signatureline{Padova, YYYY-MM-DD}


