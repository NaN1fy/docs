\section{Contenuti del Verbale}
\subsection{Informazioni sulla riunione}
\begin{itemize}
	\setlength\itemsep{0em}
	\item\textbf{Luogo:} Chiamata Microsoft Teams;
	\item\textbf{Ora di inizio:} 15:00;
	\item\textbf{Ora di fine:}  15:45.
\end{itemize}
\begin{table}[ht!]
	\begin{minipage}[t]{0.5\linewidth}
		\centering
		\begin{tabular}{p{3cm} p{3cm}}
			\toprule
			\textbf{Partecipante} & \textbf{Durata presenza} \\
			\midrule
			Guglielmo Barison & 1.75 h \\
			Linda Barbiero &  1.75 h \\
			Pietro Busato & 1.75 h \\
			Oscar Konieczny & 1.75 h \\
			Davide Donanzan & 1.75 h \\
			Veronica Tecchiati & 1.75 h \\
			\bottomrule
		\end{tabular}
		\caption{Partecipanti NaN1fy.}
		\label{table:Partecipanti NaN1fy.}
	\end{minipage} 
	\begin{minipage}[t]{0.5\linewidth} % -- COMMENTA/DECOMMENTA DA QUI
		\centering
		\begin{tabular}{p{3cm} p{3cm}}
			\toprule
			\textbf{Partecipante} & \textbf{Durata presenza} \\
			\midrule
			Federico Bernacca & 1.75 h \\
			Stefan Glamotak &  1.75 h \\
			\bottomrule
		\end{tabular}
		\caption{Partecipanti Imola Informatica.}
		\label{table:Partecipanti Imola Informatica.}
	\end{minipage} % -- A QUI PER TOGLIERE AGGIUNGERE
\end{table}
\subsection{Sintesi dell'incontro}
La presente riunione consiste nel primo incontro ufficiale avvenuto con l'azienda \texttt{Imola Informatica} committende del capitolato \texttt{EasyMeal} e indetta su nostra richiesta. Questo incontro ha avuto come scopo l'instaurazione di una possibile collaborazione con l'azienda sopracitata e la chiarificazione di perplessità sorte nella presentazione del capitolato da loro proposto.
\section{Ordine del giorno}
Di seguito viene trascritto l'ordine del giorno composto dai dubbi esposti e dalle relative risposte.
\subsection{Modalità di comunicazione con il proponente e supporto tecnico}
    Per brevi comunicazioni verrà utilizzata la piattaforma \textit{Telegram}. Per supporto tecnico \texttt{Imola Informatica} si mette a disposizione per incontri attraverso la piattaforma \textit{Microsoft Teams}. Eventualmente, è stata resa disponibile la possbilità di indire un appuntamento fisso settimanalmente.
\subsection{Modalità di condivisione del lavoro svolto}
    Verrà condiviso l'accesso alla \textit{repository} del progetto affinché il proponente possa accedervi e osservarne il progredire. 
\subsection{Chiarimenti riguardanti i framework da adottare}
    Il dubbio deriva dalla necessità di comprendere se per l'azienda proponente esiste una preferenza nel \textit{linguaggio} e nel \textit{framework} da adottare nello sviluppo del capitolato. L'azienda ci ha reso noto che non vi è preferenza nel linguaggio scelto a patto che si utilizzi un \texttt{framework} che possa soddisfare le richieste del progetto stesso. Poiché il requisito minimo è lo sviluppo di una \textit{webapp} sono stati consigliati \textit{React}, \textit{Angular}, \textit{NodeJS} e \textit{Flutter} per il \textit{frontend} e \textit{Flask} e \textit{Sping} per il \textit{backend}.
\subsection{Accessibilità: è richiesto lo sviluppo di una webapp accessibile?}
    Il proponente ha specificato che non è richiesto ai fini del progetto proposto ma può essere introdotto successivamente come requisito opzionale.
\subsection{Restrizioni sulla progettazione della \textit{UX/UI}}
    Non vi è alcun tipo restrizione. A \texttt{NaN1fy} è riservata la completa libertà d'azione a riguardo.
\subsection{Servizio \textit{cloud}: Soluzione interna o database esterno?}
    L'azienda non mette a disposizione un servizio per l'implementazione della \textit{webapp} poiché il progetto deve essere funzionale e funzionante in locale. La scelta di un \textit{cloud service} è opzionale e la scelta del \textit{provider} è delegata a \texttt{NaN1fy}. Tra i servizi nominati citiamo \textit{Azure} e \textit{AWS}.
\subsection{Analisi del carico massimo supportato e analisi dei costi}
    Non è richiesta un'analisi approfondita del carico massimo supportato ma è richiesta la formulazione di un'ipotesi sviluppata previa documentazione e ricerca dei servizi che il \textit{cloud} fornisce.
    Lo scopo è testare la capacità di sostenere un numero cospicuo di richieste.
\subsection{Chiarimenti riguardanti lo sviluppo del sistema di notifca}
    Il requisito da soddisfare necessariamente è la creazione di un sistema di notifica basato sulle \textit{e-mail}. Opzionalmente, nel caso in cui il progetto si evolva nella creazione di una \textit{mobile app}, è consigliabile l'implementazione di un sistema che utilizzi \textit{push notifications}. A tal scopo è stato consigliato l'utilizzo di \textit{Firebase}.
\subsection{Modalità di pagamento alternative quali \textit{PayPal} e \textit{Satispay}}
    Nonostante l'esistenza di framework per il \textit{testing} di metodi di pagamento alternativi, vi è come unica necessità l'implementazione del metodo di pagamento attraverso carta di credito.
\subsection{Firma dei verbali esterni}
Ad ogni \textit{SAL}, o comunque ad ogni incontro su cui verrà poi steso un verbale, basterà inviare quest'ultimo con data e luogo della firma. L'azienda procederà quindi a riconsegnarlo firmato.
\section{Attività da svolgere}
Non sono state definite attività da svolgere. L'azienda proponente rimane in attesa di future notizie riguardanti l'effettiva adesione di \texttt{NaN1fy} al capitolato proposto e alla conseguente approvazione della candidatura da parte del destinatario \texttt{Prof.} \texttt{Tullio Vardanega}.
\\\\
% togli il commento per la firma
\signatureline{Padova, 2024-03-20}